\documentclass[aspectratio=169]{beamer}

\usepackage{graphicx}
\usepackage{amsmath, esint}

\usepackage{ragged2e}
\usepackage{tikz}
\usetikzlibrary{arrows,shapes}

\hypersetup{
  colorlinks=true,
  linkcolor=blue,
  filecolor=magenta,
  urlcolor=blue,
  pdfpagemode=FullScreen
}
\usepackage{algorithm}
\usepackage{algorithmic}

\usepackage{tabularx}
\newcolumntype{Y}{>{\centering\arraybackslash}X}

\usepackage{minted}
\usepackage{xcolor}
\definecolor{LightGray}{gray}{0.975}

\usepackage{amssymb}

\def\ojoin{\setbox0=\hbox{$\bowtie$}%
  \rule[-.02ex]{.25em}{.4pt}\llap{\rule[\ht0]{.25em}{.4pt}}}
\def\leftouterjoin{\mathbin{\ojoin\mkern-5.8mu\bowtie}}
\def\rightouterjoin{\mathbin{\bowtie\mkern-5.8mu\ojoin}}
\def\fullouterjoin{\mathbin{\ojoin\mkern-5.8mu\bowtie\mkern-5.8mu\ojoin}}

%\usetheme{Warsaw}
\usefonttheme{serif}

\title[MongoDB Geospatial]{MongoDB Geospatial Queries}
\author{mongodb.com}
\date{\today}

% Remove navigation symbols...
\setbeamertemplate{navigation symbols}{}

\defbeamertemplate*{footline}{shadow theme}{
    \leavevmode
    \hbox{
        \begin{beamercolorbox}[
                wd =        0.33\paperwidth,
                ht =        2.5ex,
                dp =        1.125ex,
                leftskip =  0.3cm plus1fil,
                rightskip = 0.3cm
            ]{author in head/foot}
            \flushleft DBA
        \end{beamercolorbox}
        \begin{beamercolorbox}[
                wd =        0.33\paperwidth,
                ht =        2.5ex,
                dp =        1.125ex,
                leftskip =  0.3cm plus1fil,
                rightskip = 0.3cm
            ]{author in head/foot}
            \insertshorttitle
        \end{beamercolorbox}
        \begin{beamercolorbox}[
                wd =        0.33\paperwidth,
                ht =        2.5ex,
                dp =        1.125ex,
                leftskip =  0.3cm plus1fil,
                rightskip = 0.3cm
            ]{title in head/foot}
            \hfill \insertframenumber\,/\,\inserttotalframenumber%
        \end{beamercolorbox}
    }
}

\AtBeginSection[]
{
     \begin{frame}<beamer>
     \frametitle{Plan}
     \tableofcontents[currentsection]
     \end{frame}
}

\newcommand{\toRight}[1]{
    \begin{FlushRight}
        {\tiny #1}
    \end{FlushRight}
} % Align to right

\begin{document}

\frame{\titlepage}

\begin{frame}{Overview}
  \begin{itemize}
    \item MongoDB supports geospatial queries on:
      \begin{itemize}
        \item GeoJSON objects (spherical geometry)
        \item Legacy coordinate pairs (planar geometry)
      \end{itemize}
    \item Geospatial indexes enhance query performance
    \item Applications include location-based services, mapping, and spatial analysis
  \end{itemize}
\end{frame}

\begin{frame}[fragile]{GeoJSON Format}
  \begin{itemize}
    \item Represents geospatial data on an earth-like sphere
    \item Structure:
  \end{itemize}
  \begin{minted}
  [tabsize=4, obeytabs, frame=lines, framesep=2mm, baselinestretch=1.2, bgcolor=LightGray, fontsize=\footnotesize]{json}
  {
    "type": "Point",
    "coordinates": [-73.856077, 40.848447]
  }
  \end{minted}
  \begin{itemize}
    \item Coordinates are in [longitude, latitude] order
  \end{itemize}
\end{frame}

\begin{frame}[fragile]{Legacy Coordinate Pairs}
  \begin{itemize}
    \item Represents geospatial data on a Euclidean plane
    \item Preferred format:
  \end{itemize}

  \begin{minted}
  [tabsize=4, obeytabs, frame=lines, framesep=2mm, baselinestretch=1.2, bgcolor=LightGray, fontsize=\footnotesize]{json}
    "location": [-73.856077, 40.848447]
  \end{minted}

  \begin{itemize}
    \item Also supports embedded documents:
  \end{itemize}

  \begin{minted}
  [tabsize=4, obeytabs, frame=lines, framesep=2mm, baselinestretch=1.2, bgcolor=LightGray, fontsize=\footnotesize]{json}
    "location": { "x": -73.856077, "y": 40.848447 }
  \end{minted}

  \begin{itemize}
    \item Use 2d indexes for planar geometry
  \end{itemize}
\end{frame}

\begin{frame}[fragile]{Geospatial Indexes}
  \begin{itemize}
    \item \textbf{2dsphere Index}:
      \begin{itemize}
        \item Supports spherical geometry
        \item Required for GeoJSON objects
        \item Created with:
          \begin{minted}
          [tabsize=4, obeytabs, frame=lines, framesep=2mm, baselinestretch=1.2, bgcolor=LightGray, fontsize=\footnotesize]{js}
  db.collection.createIndex({ location: "2dsphere" })
          \end{minted}
      \end{itemize}
    \item \textbf{2d Index}:
      \begin{itemize}
        \item Supports planar geometry
        \item Suitable for legacy coordinate pairs
        \item Created with:
          \begin{minted}
          [tabsize=4, obeytabs, frame=lines, framesep=2mm, baselinestretch=1.2, bgcolor=LightGray, fontsize=\footnotesize]{js}
  db.collection.createIndex({ location: "2d" })
          \end{minted}
      \end{itemize}
  \end{itemize}
\end{frame}

\begin{frame}{Geospatial Query Operators}
  \begin{itemize}
    \item \textbf{\texttt{\$near}}:
      \begin{itemize}
        \item Finds documents close to a point
        \item Requires a geospatial index
      \end{itemize}
    \item \textbf{\texttt{\$geoWithin}}:
      \begin{itemize}
        \item Finds documents within a specified geometry
      \end{itemize}
    \item \textbf{\texttt{\$geoIntersects}}:
      \begin{itemize}
        \item Finds documents that intersect with a geometry
      \end{itemize}
    \item \textbf{\texttt{\$nearSphere}}:
      \begin{itemize}
        \item Similar to \texttt{\$near} but calculates distances on a sphere
      \end{itemize}
  \end{itemize}
\end{frame}

\begin{frame}[fragile]{Example: \texttt{\$near} Query}
  \begin{minted}
  [tabsize=4, obeytabs, frame=lines, framesep=2mm, baselinestretch=1.2, bgcolor=LightGray, fontsize=\footnotesize, linenos]{js}
    db.places.find({
      location: {
        $near: {
          $geometry: {
            type: "Point",
            coordinates: [-73.856077, 40.848447]
          },
          $maxDistance: 1000
        }
      }
    })
  \end{minted}
  \begin{itemize}
    \item Finds places within 1 km of the specified point
  \end{itemize}
\end{frame}

\begin{frame}[fragile]{Example: \texttt{\$geoWithin} Query}
  \begin{minted}
  [tabsize=4, obeytabs, frame=lines, framesep=2mm, baselinestretch=1.2, bgcolor=LightGray, fontsize=\scriptsize, linenos]{js}
	db.places.find({
        location: {
            $geoWithin: {
                $geometry: {
                    type: "Polygon",
                    coordinates: [[
                        [-73.958, 40.800],
                        [-73.949, 40.796],
                        [-73.973, 40.764],
                        [-73.981, 40.768],
                        [-73.958, 40.800]
                    ]]
                }
            }
        }
	})
  \end{minted}
  \begin{itemize}
    \item Finds places within the specified polygon
  \end{itemize}
\end{frame}

\begin{frame}{Summary}
  \begin{itemize}
    \item MongoDB provides robust support for geospatial data
    \item Choose appropriate data formats and indexes based on your application's needs
    \item Utilize geospatial query operators to perform spatial searches
  \end{itemize}
\end{frame}

\begin{frame}{Resources}
  \begin{itemize}
    \item Official Documentation: \href{https://www.mongodb.com/docs/manual/geospatial-queries/}{MongoDB Geospatial Queries}
    \item GeoJSON Specification: \href{https://geojson.org/}{geojson.org}
  \end{itemize}
\end{frame}

\end{document}
