\documentclass[aspectratio=169]{beamer}

\usepackage{graphicx}
\usepackage{amsmath, esint}

\usepackage{ragged2e}
\usepackage{tikz}
\usetikzlibrary{arrows,shapes}

\hypersetup{
  colorlinks=true,
  linkcolor=blue,
  filecolor=magenta,
  urlcolor=blue,
  pdfpagemode=FullScreen
}
\usepackage{algorithm}
\usepackage{algorithmic}

\usepackage{tabularx}
\newcolumntype{Y}{>{\centering\arraybackslash}X}

\usepackage{minted}
\usepackage{xcolor}
\definecolor{LightGray}{gray}{0.975}

\usepackage{amssymb}

\def\ojoin{\setbox0=\hbox{$\bowtie$}%
  \rule[-.02ex]{.25em}{.4pt}\llap{\rule[\ht0]{.25em}{.4pt}}}
\def\leftouterjoin{\mathbin{\ojoin\mkern-5.8mu\bowtie}}
\def\rightouterjoin{\mathbin{\bowtie\mkern-5.8mu\ojoin}}
\def\fullouterjoin{\mathbin{\ojoin\mkern-5.8mu\bowtie\mkern-5.8mu\ojoin}}

%\usetheme{Warsaw}
\usefonttheme{serif}

\title[MongoDB Geospatial Tutorial]{MongoDB Geospatial Queries}
\subtitle{Find Restaurants with Geospatial Queries}
\author{MongoDB Documentation}
\date{\today}

% Remove navigation symbols...
\setbeamertemplate{navigation symbols}{}

\defbeamertemplate*{footline}{shadow theme}{
    \leavevmode
    \hbox{
        \begin{beamercolorbox}[
                wd =        0.33\paperwidth,
                ht =        2.5ex,
                dp =        1.125ex,
                leftskip =  0.3cm plus1fil,
                rightskip = 0.3cm
            ]{author in head/foot}
            \flushleft DBA
        \end{beamercolorbox}
        \begin{beamercolorbox}[
                wd =        0.33\paperwidth,
                ht =        2.5ex,
                dp =        1.125ex,
                leftskip =  0.3cm plus1fil,
                rightskip = 0.3cm
            ]{author in head/foot}
            \insertshorttitle
        \end{beamercolorbox}
        \begin{beamercolorbox}[
                wd =        0.33\paperwidth,
                ht =        2.5ex,
                dp =        1.125ex,
                leftskip =  0.3cm plus1fil,
                rightskip = 0.3cm
            ]{title in head/foot}
            \hfill \insertframenumber\,/\,\inserttotalframenumber%
        \end{beamercolorbox}
    }
}

\AtBeginSection[]
{
     \begin{frame}<beamer>
     \frametitle{Plan}
     \tableofcontents[currentsection]
     \end{frame}
}

\newcommand{\toRight}[1]{
    \begin{FlushRight}
        {\tiny #1}
    \end{FlushRight}
} % Align to right

\begin{document}

\frame{\titlepage}

\begin{frame}{Overview}
	\begin{itemize}
		\item Introduction to MongoDB's geospatial indexing capabilities.
		\item Demonstration of geospatial queries using:
			\begin{itemize}
				\item \texttt{\$geoWithin}
				\item \texttt{\$geoIntersects}
				\item \texttt{\$nearSphere}
			\end{itemize}
		\item Use case: Finding restaurants in New York City.
	\end{itemize}
\end{frame}

\begin{frame}{Geospatial Indexes}
	\begin{itemize}
		\item MongoDB supports geospatial indexes to efficiently execute spatial queries.
		\item Types of geospatial indexes:
			\begin{itemize}
				\item \texttt{2d} index: For planar geometry.
				\item \texttt{2dsphere} index: For spherical geometry (e.g., Earth).
			\end{itemize}
		\item This tutorial uses \texttt{2dsphere} indexes.
	\end{itemize}
\end{frame}

\begin{frame}[fragile]{GeoJSON Format}
	\begin{itemize}
		\item MongoDB uses GeoJSON objects to represent geospatial data.
		\item Common GeoJSON types:
		\begin{itemize}
			\item \texttt{Point}
			\item \texttt{LineString}
			\item \texttt{Polygon}
		\end{itemize}
		\item Example of a GeoJSON Point:
			\begin{minted}
			[tabsize=4, obeytabs, frame=lines, framesep=2mm, baselinestretch=1.2, bgcolor=LightGray, fontsize=\footnotesize]{json}
{
	"type": "Point",
	"coordinates": [-73.856077, 40.848447]
}
			\end{minted}
	\end{itemize}
\end{frame}

\begin{frame}[fragile]{Importing Data}
	\begin{itemize}
		\item Download datasets:
			\begin{itemize}
				\item \texttt{restaurants.json}
				\item \texttt{neighborhoods.json}
			\end{itemize}
		\item Import into MongoDB:
	\end{itemize}
	\begin{minted}
	[tabsize=4, obeytabs, frame=lines, framesep=2mm, baselinestretch=1.2, bgcolor=LightGray, fontsize=\footnotesize]{bash}
mongoimport <path to restaurants.json> -c=restaurants
mongoimport <path to neighborhoods.json> -c=neighborhoods
	\end{minted}
\end{frame}

\begin{frame}[fragile]{Creating Geospatial Indexes}
	\begin{itemize}
		\item Create \texttt{2dsphere} indexes on the collections:
	\end{itemize}
	\begin{minted}
	[tabsize=4, obeytabs, frame=lines, framesep=2mm, baselinestretch=1.2, bgcolor=LightGray, fontsize=\footnotesize]{js}
db.restaurants.createIndex({ location: "2dsphere" })
db.neighborhoods.createIndex({ geometry: "2dsphere" })
	\end{minted}
\end{frame}

\begin{frame}[fragile]{Sample Restaurant Document}
	\begin{minted}
	[tabsize=4, obeytabs, frame=lines, framesep=2mm, baselinestretch=1.2, bgcolor=LightGray, fontsize=\footnotesize]{json}
{
  "location": {
    "type": "Point",
    "coordinates": [-73.856077, 40.848447]
  },
  "name": "Morris Park Bake Shop"
}
	\end{minted}
\end{frame}

\begin{frame}[fragile]{Sample Neighborhood Document}
	\begin{minted}
	[tabsize=4, obeytabs, frame=lines, framesep=2mm, baselinestretch=1.2, bgcolor=LightGray, fontsize=\footnotesize]{json}
{
  "geometry": {
    "type": "Polygon",
    "coordinates": [[
      [-73.99, 40.75],
      [-73.98, 40.76],
      [-73.99, 40.75]
    ]]
  },
  "name": "Hell's Kitchen"
}
	\end{minted}
\end{frame}

\begin{frame}[fragile]{Finding Current Neighborhood}
	\begin{itemize}
		\item Determine the user's neighborhood using \texttt{\$geoIntersects}:
	\end{itemize}
	\begin{minted}
	[tabsize=4, obeytabs, frame=lines, framesep=2mm, baselinestretch=1.2, bgcolor=LightGray, fontsize=\footnotesize]{js}
db.neighborhoods.findOne({
  geometry: {
    $geoIntersects: {
      $geometry: {
        type: "Point",
        coordinates: [-73.93414657, 40.82302903]
      }
    }
  }
})
	\end{minted}
\end{frame}

\begin{frame}[fragile]{Counting Restaurants in Neighborhood}
	\begin{itemize}
		\item Find all restaurants within the user's neighborhood:
	\end{itemize}
	\begin{minted}
	[tabsize=4, obeytabs, frame=lines, framesep=2mm, baselinestretch=1.2, bgcolor=LightGray, fontsize=\scriptsize]{js}
var neighborhood = db.neighborhoods.findOne({
  geometry: {
    $geoIntersects: {
      $geometry: {
        type: "Point",
        coordinates: [-73.93414657, 40.82302903]
      }
    }
  }
})
db.restaurants.find({
  location: {
    $geoWithin: {
      $geometry: neighborhood.geometry
    }
  }
}).count()
	\end{minted}
\end{frame}

\begin{frame}[fragile]{Finding Nearby Restaurants (Unsorted)}
	\begin{itemize}
		\item Use \texttt{\$geoWithin} with \texttt{\$centerSphere} to find restaurants within 5 miles:
	\end{itemize}
	\begin{minted}
	[tabsize=4, obeytabs, frame=lines, framesep=2mm, baselinestretch=1.2, bgcolor=LightGray, fontsize=\footnotesize]{js}
db.restaurants.find({
  location: {
    $geoWithin: {
      $centerSphere: [[-73.93414657, 40.82302903], 5 / 3963.2]
    }
  }
})
	\end{minted}
	Note: \texttt{\$centerSphere}'s second argument accepts the radius in radians, so you must divide it by the radius of the earth in miles.
\end{frame}

\begin{frame}[fragile]{Finding Nearby Restaurants (Sorted)}
	\begin{itemize}
		\item Use \texttt{\$nearSphere} with \texttt{\$maxDistance} to find and sort restaurants by proximity:
	\end{itemize}
	\begin{minted}
	[tabsize=4, obeytabs, frame=lines, framesep=2mm, baselinestretch=1.2, bgcolor=LightGray, fontsize=\footnotesize]{js}
var METERS_PER_MILE = 1609.34
db.restaurants.find({
  location: {
    $nearSphere: {
      $geometry: {
        type: "Point",
        coordinates: [-73.93414657, 40.82302903]
      },
      $maxDistance: 5 * METERS_PER_MILE
    }
  }
})
	\end{minted}
\end{frame}

\begin{frame}{Conclusion}
	\begin{itemize}
		\item MongoDB's geospatial features enable efficient spatial queries.
		\item Utilizing \texttt{2dsphere} indexes and GeoJSON formats allows for complex geospatial operations.
		\item Operators like \texttt{\$geoWithin}, \texttt{\$geoIntersects}, and \texttt{\$nearSphere} provide powerful querying capabilities.
	\end{itemize}
\end{frame}

\begin{frame}{References}
	\begin{itemize}
		\item MongoDB Geospatial Tutorial: \url{https://www.mongodb.com/docs/manual/tutorial/geospatial-tutorial/}
		\item GeoJSON Specification: \url{https://geojson.org/}
	\end{itemize}
\end{frame}

\end{document}
