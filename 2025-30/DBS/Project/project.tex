\documentclass[11pt]{article}

\usepackage[margin=1in]{geometry}
\usepackage{booktabs}
\usepackage{array}
\usepackage{tabularx}
\usepackage{graphicx}
\usepackage{hyperref}
\usepackage{xcolor}
\usepackage{amsmath,amssymb}
\usepackage{enumitem}
\usepackage{multirow}
\usepackage{minted} % compile with -shell-escape
\usepackage{caption}
\usepackage{longtable}

\hypersetup{
  colorlinks=true,
  linkcolor=blue!60!black,
  urlcolor=blue,
  citecolor=blue!60!black
}

\title{\vspace{-1em}Project: Designing a Telemetry and UX Database for \textit{Chocolate-Doom} Research}
\author{DBS -- Semester Project Brief}
\date{\today}

\begin{document}
\maketitle
\tableofcontents

\section{Context and Motivation}
A research group is collecting gameplay telemetry from a \emph{hacked} version of \texttt{chocolate-doom}. This build emits per-tic data (on screen and/or to file) that includes the player position $(x,y,z)$, facing \emph{angle}, \emph{momentum} vector, \emph{point-of-view} (FOV/camera), and combat stats (health, armor, ammo). It also logs meta-information such as \emph{tic} number, \emph{episode}, \emph{map}, and \emph{sector}.

The group wants to aggregate multiple play sessions to detect movement trends and potential cooperation patterns among players. Student volunteers provide demographics (age, gender, experience) and complete \emph{one} of these UX instruments:
\begin{itemize}[leftmargin=*]
  \item \href{https://selfdeterminationtheory.org/player-experience-of-needs-satisfaction-pens}{\textbf{PENS}} (Player Experience of Need Satisfaction).
  \item \href{https://uxpajournal.org/wp-content/uploads/sites/7/pdf/JUS_Keebler_GUESS-18%20Scoring%20Guidelines.pdf}{\textbf{GUESS}} (Game User-Experience Satisfaction Scale).
  \item \href{https://www.sciencedirect.com/science/article/pii/S1071581924000739}{\textbf{BANGS}} (Basic Needs in Games; open-access).
\end{itemize}

\textbf{Goal:} Design and prototype a relational database that ingests telemetry and survey data, supports exploratory queries and analytics for movement/cooperation trends, and enforces data quality and research ethics.

\section{Learning Objectives}
By completing this project, you will:
\begin{enumerate}[leftmargin=*]
  \item Model a real-world domain into entities, attributes, and relationships (ER \& relational schema).
  \item Normalize tables to at least 3NF (justify any denormalizations for performance).
  \item Define keys, constraints, and reference integrity for high-frequency telemetry.
  \item Design an ingestion pipeline for semi-structured logs (TSV $\rightarrow$ staging $\rightarrow$ core).
  \item Implement indices and assess their impact with query plans and timings.
  \item Formulate SQL queries for trajectory, proximity, and cooperation analyses.
  \item Integrate user demographics and UX scales (PENS/GUESS/BANGS) with telemetry.
  \item Address privacy, consent, and research-ethics constraints in schema \& process.
\end{enumerate}

\section{Domain Overview and Core Concepts}
\textbf{Proposed high-level entities:}
\begin{itemize}[leftmargin=*]
  \item \textbf{User}: volunteer student providing consent and demographics.
  \item \textbf{Player}: in-game identity (may be linked 1:1 to a User or support multiple aliases).
  \item \textbf{Game}: a single gameplay session instance (start/end timestamps and settings).
  \item \textbf{Time/Tic}: per-tic (or per-frame) temporal index emitted by the engine.
  \item \textbf{Episode/Map/Sector}: level structure; sectors partition maps.
  \item \textbf{TelemetryEvent}: atomic record of state at a tic (position, momentum, stats, and more).
  \item \textbf{UXInstrument}: instrument metadata (PENS/GUESS/BANGS definitions).
  \item \textbf{UXResponse}: a user's instrument responses.
\end{itemize}

\paragraph{Movement \& Cooperation Signals (to inform schema/queries).}
\begin{itemize}[leftmargin=*]
  \item \emph{Trajectories}: sequence of $(x,y,z)$ ordered by tic per player per game.
  \item \emph{Proximity events}: players within a spatial threshold for $\geq k$ tics.
  \item \emph{Co-occurrence in sectors}: overlapping time in same sector (optional:
  adjacent sectors).
\end{itemize}

\section{Data Model (Conceptual $\rightarrow$ Logical)}

\subsection*{Conceptual ER (deliverable)}
Produce an ER diagram capturing main relationships.  For example:

  \begin{itemize}
    \item \texttt{User}--\texttt{Player},
    \item \texttt{Player}--\texttt{Game} (via \texttt{GameParticipant}),
    \item \texttt{Game}--\texttt{TelemetryEvent},
    \item \texttt{Map}--\texttt{Sector} (1:many),
    \item \texttt{User}--\texttt{UXResponse},
    \item \texttt{UXInstrument}--\texttt{UXItem}--\texttt{UXResponseItem} .
  \end{itemize}

% \subsection{Starter Logical Schema (proposed)}
% You may refine names/types; justify changes in your report.
% \begin{minted}[fontsize=\small, breaklines]{sql}
% -- Core identity
% CREATE TABLE "User" (
%   user_id        SERIAL PRIMARY KEY,
%   email          CITEXT UNIQUE NOT NULL,
%   nickname       TEXT,
%   age_years      INT CHECK (age_years BETWEEN 16 AND 99),
%   gender         TEXT CHECK (gender IN ('female','male','nonbinary','prefer_not')),
%   experience_yrs NUMERIC(4,1) CHECK (experience_yrs >= 0),
%   consent_signed BOOLEAN NOT NULL DEFAULT FALSE,
%   created_at     TIMESTAMPTZ NOT NULL DEFAULT now()
% );
%
% CREATE TABLE Player (
%   player_id      SERIAL PRIMARY KEY,
%   user_id        INT REFERENCES "User"(user_id) ON DELETE SET NULL,
%   handle         TEXT NOT NULL,
%   created_at     TIMESTAMPTZ NOT NULL DEFAULT now(),
%   UNIQUE (user_id, handle)
% );
%
% -- Game/session + roster
% CREATE TABLE Game (
%   game_id        UUID PRIMARY KEY DEFAULT gen_random_uuid(),
%   started_at     TIMESTAMPTZ NOT NULL,
%   ended_at       TIMESTAMPTZ,
%   engine_version TEXT NOT NULL,
%   wad            TEXT,          -- IWAD/PWAD name(s)
%   config_hash    TEXT,          -- settings hash
%   episode        INT NOT NULL,
%   map_code       TEXT NOT NULL  -- e.g., E1M1, MAP05
% );
%
% CREATE TABLE GameParticipant (
%   game_id   UUID REFERENCES Game(game_id) ON DELETE CASCADE,
%   player_id INT  REFERENCES Player(player_id) ON DELETE CASCADE,
%   team      TEXT,
%   role      TEXT,
%   PRIMARY KEY (game_id, player_id)
% );
%
% -- Level geometry
% CREATE TABLE Map (
%   episode     INT,
%   map_code    TEXT,
%   title       TEXT,
%   PRIMARY KEY (episode, map_code)
% );
%
% CREATE TABLE Sector (
%   episode     INT,
%   map_code    TEXT,
%   sector_id   INT,
%   floor_z     INT,
%   ceiling_z   INT,
%   tag         INT,
%   PRIMARY KEY (episode, map_code, sector_id),
%   FOREIGN KEY (episode, map_code) REFERENCES Map(episode, map_code) ON DELETE CASCADE
% );
%
% -- Time base
% CREATE TABLE TicTime (
%   tic              BIGINT PRIMARY KEY, -- global or per-game tick; choose one model
%   ms_from_start    INT NOT NULL
% );
%
% -- Telemetry at high frequency
% CREATE TABLE TelemetryEvent (
%   game_id     UUID NOT NULL REFERENCES Game(game_id) ON DELETE CASCADE,
%   tic         BIGINT NOT NULL, -- consider composite PK (game_id, tic, player_id)
%   player_id   INT NOT NULL REFERENCES Player(player_id) ON DELETE CASCADE,
%   pos_x       NUMERIC(10,3) NOT NULL,
%   pos_y       NUMERIC(10,3) NOT NULL,
%   pos_z       NUMERIC(10,3) NOT NULL,
%   angle_deg   NUMERIC(6,2)  NOT NULL CHECK (angle_deg >= 0 AND angle_deg < 360),
%   mom_x       NUMERIC(10,4),
%   mom_y       NUMERIC(10,4),
%   fov_deg     NUMERIC(5,2),
%   health      INT CHECK (health BETWEEN 0 AND 200),
%   armor       INT CHECK (armor BETWEEN 0 AND 200),
%   ammo_shell  INT CHECK (ammo_shell >= 0),
%   ammo_rocket INT CHECK (ammo_rocket >= 0),
%   sector_id   INT,
%   episode     INT,
%   map_code    TEXT,
%   PRIMARY KEY (game_id, tic, player_id),
%   FOREIGN KEY (episode, map_code, sector_id)
%     REFERENCES Sector(episode, map_code, sector_id) ON DELETE SET NULL
% );
%
% -- UX instruments and responses
% CREATE TABLE UXInstrument (
%   instrument_id  SERIAL PRIMARY KEY,
%   code           TEXT UNIQUE NOT NULL CHECK (code IN ('PENS','GUESS','BANGS')),
%   title          TEXT NOT NULL,
%   version        TEXT,
%   license_url    TEXT
% );
%
% CREATE TABLE UXItem (
%   item_id        SERIAL PRIMARY KEY,
%   instrument_id  INT NOT NULL REFERENCES UXInstrument(instrument_id) ON DELETE CASCADE,
%   item_code      TEXT NOT NULL,
%   prompt         TEXT NOT NULL,
%   min_val        INT NOT NULL,
%   max_val        INT NOT NULL,
%   reverse_scored BOOLEAN NOT NULL DEFAULT FALSE,
%   UNIQUE (instrument_id, item_code)
% );
%
% CREATE TABLE UXResponse (
%   response_id    UUID PRIMARY KEY DEFAULT gen_random_uuid(),
%   user_id        INT NOT NULL REFERENCES "User"(user_id) ON DELETE CASCADE,
%   instrument_id  INT NOT NULL REFERENCES UXInstrument(instrument_id),
%   taken_at       TIMESTAMPTZ NOT NULL DEFAULT now(),
%   notes          TEXT,
%   CHECK (taken_at <= now())
% );
%
% CREATE TABLE UXResponseItem (
%   response_id UUID NOT NULL REFERENCES UXResponse(response_id) ON DELETE CASCADE,
%   item_id     INT NOT NULL REFERENCES UXItem(item_id) ON DELETE CASCADE,
%   value_int   INT NOT NULL,
%   PRIMARY KEY (response_id, item_id)
% );
% \end{minted}

\paragraph{Indexing Suggestions (implement and evaluate).}
\begin{minted}[fontsize=\small]{sql}
  CREATE INDEX ON TelemetryEvent (game_id, player_id, tic);
  CREATE INDEX ON TelemetryEvent (episode, map_code, sector_id);
  CREATE INDEX ON TelemetryEvent USING gist ((pos_x, pos_y));
  CREATE INDEX ON GameParticipant (player_id, game_id);
\end{minted}

\section{Data Ingestion (ETL) Guidance}
Assume the hacked engine emits TSV lines.

% \begin{minted}[fontsize=\small]{text}
% {"game_id":"...","tic":12345,"player":"doomguy","pos":{"x":-102.5,"y":88.0,"z":0},
%  "angle":270.0,"mom":{"x":0.1,"y":-0.03},"fov":90.0,
%  "stats":{"hp":57,"armor":25,"shell":8,"rocket":0},
%  "episode":1,"map":"E1M1","sector":37,"ts":"2025-09-10T14:12:04.123Z"}
% \end{minted}

\noindent \textbf{Recommended pipeline:}
\begin{enumerate}[leftmargin=*]
  \item Load raw logs to a \emph{staging} table (text fields) with minimal constraints.
  \item Validate \& transform into typed core tables using \texttt{INSERT ... SELECT}.
  \item Deduplicate on \texttt{(game\_id, tic, player\_id)}; reject malformed records with an \emph{error log} table.
\end{enumerate}

\section{Analytics Queries (Examples to Implement)}

\subsection*{Movement Trends}
TBA

% \begin{minted}[fontsize=\small]{sql}
% -- Average speed per player per game (approx via finite differences)
% WITH step AS (
%   SELECT game_id, player_id, tic,
%          pos_x, pos_y,
%          LAG(pos_x) OVER (PARTITION BY game_id, player_id ORDER BY tic) AS px_prev,
%          LAG(pos_y) OVER (PARTITION BY game_id, player_id ORDER BY tic) AS py_prev
%   FROM TelemetryEvent
% )
% SELECT game_id, player_id,
%        AVG( sqrt( (pos_x - px_prev)^2 + (pos_y - py_prev)^2 ) ) AS mean_step
% FROM step
% WHERE px_prev IS NOT NULL
% GROUP BY game_id, player_id
% ORDER BY mean_step DESC;
% \end{minted}
%
% \subsection*{Heatmap by Sector}
% \begin{minted}[fontsize=\small]{sql}
% SELECT episode, map_code, sector_id, COUNT(*) AS ticks_in_sector
% FROM TelemetryEvent
% GROUP BY episode, map_code, sector_id
% ORDER BY ticks_in_sector DESC;
% \end{minted}
%
% \subsection*{Cooperation Proxy (Proximity)}
% \begin{minted}[fontsize=\small]{sql}
% -- Players within D units for >= K consecutive tics
% WITH pairs AS (
%   SELECT t1.game_id, t1.tic, t1.player_id AS p1, t2.player_id AS p2,
%          sqrt( (t1.pos_x - t2.pos_x)^2 + (t1.pos_y - t2.pos_y)^2 ) AS dist
%   FROM TelemetryEvent t1
%   JOIN TelemetryEvent t2
%     ON t1.game_id = t2.game_id AND t1.tic = t2.tic AND t1.player_id < t2.player_id
% )
% , close AS (
%   SELECT *, (dist <= 128.0) AS close_flag
%   FROM pairs
% )
% , runs AS (
%   SELECT game_id, p1, p2, tic,
%          close_flag,
%          tic - ROW_NUMBER() OVER (PARTITION BY game_id, p1, p2 ORDER BY tic) AS grp
%   FROM close
%   WHERE close_flag
% )
% SELECT game_id, p1, p2, MIN(tic) AS start_tic, MAX(tic) AS end_tic,
%        COUNT(*) AS consecutive_tics
% FROM runs
% GROUP BY game_id, p1, p2, grp
% HAVING COUNT(*) >= 20  -- K threshold
% ORDER BY consecutive_tics DESC;
% \end{minted}

\subsection*{Linking UX to Behavior}
TBA

% \begin{minted}[fontsize=\small]{sql}
% -- Example: correlate mean speed with BANGS competence subscale (precomputed)
% SELECT u.user_id, r.response_id, s.score AS bangs_competence, m.mean_step
% FROM "User" u
% JOIN UXResponse r ON r.user_id = u.user_id
% JOIN UXScore s   ON s.response_id = r.response_id AND s.subscale = 'competence'
% JOIN (
%   SELECT gp.player_id, AVG( sqrt( (pos_x - LAG(pos_x) OVER w)^2
%                                 + (pos_y - LAG(pos_y) OVER w)^2 ) ) AS mean_step
%   FROM TelemetryEvent
%   WINDOW w AS (PARTITION BY game_id, player_id ORDER BY tic)
%   GROUP BY gp.player_id
% ) m ON m.player_id IN (SELECT player_id FROM Player WHERE user_id = u.user_id);
% \end{minted}
%
% \noindent(\emph{Note}: define a small \texttt{UXScore} table or a view computing subscales per instrument.)

\section{Project Tasks \& Deliverables}

\subsection*{Part A: Conceptual and Logical Design (Week 1--3)}
\begin{enumerate}[leftmargin=*]
  \item Write assumptions and requirements (functional/non-functional, ethics).
  \item Produce an ER diagram with cardinalities and key attributes.
  \item Derive relational schema; list all FKs, PKs, and constraints (Data Dictionary); justify normalization.
\end{enumerate}

\subsection*{Part B: Implementation \& Ingestion (Week 4--6)}
\begin{enumerate}[leftmargin=*]
  \item Implement DDL in your DBMS (PostgreSQL recommended).
  \item Create staging tables and scripts to load sample telemetry logs (TSV).
  \item Populate \texttt{UXInstrument}, \texttt{UXItem} with at least one instrument (PENS/GUESS/BANGS).
  \item Insert synthetic sample data (at least 3 games, 6+ players, $\geq$ 20k telemetry rows).
\end{enumerate}

\subsection*{Part C: Queries, Indexing, and Reporting (Week 7--9)}
\begin{enumerate}[leftmargin=*]
  \item Implement at least 8 analytical queries including: trajectory steps, sector heatmap,
        proximity/cooperation runs, health under proximity, ammo usage patterns, player hotspots,
        per-player summary, and UX-behavior link.
  \item Create at least 3 indexes. Show \texttt{EXPLAIN(ANALYZE)} before/after and discuss.
  \item Provide 2 views and 1 materialized view for frequent analyses.
  \item Provide a \texttt{Makefile} or shell script to recreate the schema and load samples.
\end{enumerate}

\section{Submission Format}

Submit a single PDF report with:
\begin{itemize}[leftmargin=*]
  \item \textbf{ER diagram}, \textbf{relational schema}, and \textbf{rationale}.
  \item \textbf{DDL/constraints} (appendix with code snippets).
  \item \textbf{ETL description} + sample of raw telemetry.
  \item \textbf{Queries + results} (screenshots/tables) and index evaluation.
  \item \textbf{Ethics note} and \textbf{data dictionary}.
\end{itemize}

\section{Grading Rubric (100 pts)}
\begin{center}
\begin{tabular}{p{0.48\linewidth}r}
\toprule
\textbf{Criterion} & \textbf{Points} \\
\midrule
Problem framing, assumptions, requirements clearly stated & 10 \\
Conceptual ER correctness (entities, keys, cardinalities) \& data dictionary & 20 \\
Relational design \& normalization (3NF), constraints & 15 \\
Implementation quality (DDL, integrity, sample data) & 10 \\
ETL pipeline (staging $\rightarrow$ core, validation) & 10 \\
Analytics queries (8+) correctness \& insight & 15 \\
Indexing \& performance evaluation (EXPLAIN/ANALYZE) & 10 \\
Views/materialized view for reuse & 5 \\
Report quality (clarity, organization, reproducibility) & 5 \\
\midrule
\textbf{Total} & \textbf{100} \\
\bottomrule
\end{tabular}
\end{center}

% \section{Starter Inserts (Optional)}
% \begin{minted}[fontsize=\small]{sql}
% INSERT INTO UXInstrument (code, title, version, license_url)
% VALUES
%   ('BANGS','Basic Needs in Games','1.0','https://...'),
%   ('PENS','Player Experience of Need Satisfaction','1.0','https://...'),
%   ('GUESS','Game User-Experience Satisfaction','1.0','https://...');
%
% -- Example BANGS items (illustrative; replace with official wording/range)
% INSERT INTO UXItem (instrument_id, item_code, prompt, min_val, max_val, reverse_scored)
% SELECT instrument_id, 'B1', 'I felt competent while playing', 1, 7, FALSE
% FROM UXInstrument WHERE code='BANGS';
% \end{minted}

\section{Bonus}
\begin{itemize}[leftmargin=*]
  \item If available, enable extensions like \texttt{citext}, \texttt{uuid-ossp} or \texttt{pgcrypto} for authorization or \texttt{postgis} for spatial indexing.
\end{itemize}

\end{document}
