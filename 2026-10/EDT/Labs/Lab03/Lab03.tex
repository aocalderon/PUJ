\documentclass{article}

\usepackage[utf8]{inputenc}
\usepackage{amsmath, esint}
\usepackage{amsfonts}
\usepackage[printonlyused]{acronym}
\usepackage{wasysym}
\usepackage{qrcode}
\usepackage[colorlinks]{hyperref}
\usepackage{lmodern}
\usepackage{graphicx}
\usepackage{xcolor}
\usepackage[left=2cm, top=3cm, right=2cm]{geometry}
\usepackage{minted}
\usemintedstyle{tango}
\usepackage{booktabs}
\usepackage{svg}
\usepackage{xcolor}
\definecolor{LightGray}{gray}{0.975}
\hypersetup{
  urlcolor=blue,
  linkcolor=black,
}

\title{Lab 03: A `gentle' Introduction to Satellite Imagery Classification.}
\author{Andrés Oswaldo Calderón Romero, Ph.D.}
\date{\today}

\begin{document}

\maketitle

\section{Introduction}
In our previous sessions, we explored the capabilities of the Landsat program and its long-standing contribution to Earth observation. This lab shifts our focus to the European Space Agency's (ESA) Copernicus program, specifically working with \textbf{Sentinel-2} imagery. With its higher spatial resolution (up to 10 meters) and more frequent revisit times, Sentinel-2 provides a robust platform for detailed land-cover analysis.

The primary objective of this lab is to move beyond simple data visualization and into the realm of quantitative analysis through \textbf{Supervised Classification}. You will learn how to transform raw spectral bands into meaningful thematic maps by ``training'' the software to recognize specific land cover types. Furthermore, we will address a critical phase of any remote sensing project: \textbf{Accuracy Assessment}. Understanding how to validate your results using established metrics, such as the error matrix, is essential for ensuring the reliability of your geospatial products.

By the end of this session, you will have mastered the end-to-end workflow of data acquisition, plugin integration in QGIS, and the theoretical foundations of classification validation.

\section{Getting Data}
Now that we have covered Landsat data, we will move on to Sentinel data. Visit the \href{https://dataspace.copernicus.eu/}{Copernicus Data Space Ecosystem} to get started. Please create an account and use this \href{https://youtu.be/htF8JD7QEfM}{video} as a guide to downloading Sentinel-2 imagery. Although you do not need to submit anything yet, you will need to have the Blue, Green, Red, and NIR bands of a Sentinel scene ready for your specific study area.

\section{Performing Supervised Classification} \label{sec:class}
There is no need to reinvent the wheel when so many great resources exist for classifying satellite imagery. For QGIS, we will use the \href{https://semiautomaticclassificationmanual.readthedocs.io/en/latest/}{Semi-Automatic Classification Plugin}. Start by installing it according to the \href{https://semiautomaticclassificationmanual.readthedocs.io/en/latest/installation.html}{instructions here}. Then, watch this specific \href{https://youtu.be/7SZDCFXjIbA?si=XzImodNE6tb-9VD8}{tutorial} from the \href{https://www.youtube.com/@LucaCongedoGIS}{From GIS to Remote Sensing} channel. You may also consult other videos that suit your learning style. Your goal is to apply a supervised classification to the Sentinel bands you downloaded earlier. Be sure to document each step carefully, as this will be a required part of your report.

\section{Learning about Accuracy Assesment} \label{sec:asses}
As an additional activity, we will explore content from the textbook \textit{`Remote Sensing and Image Interpretation'} (7th ed.) by Lillesand, Chipman, and Kiefer (2015). Specifically, you are to read Section \textbf{7.17: Classification Accuracy Assessment}. After reading, please write a concise summary covering the primary concepts and methodologies discussed. You must include photos of your handwritten summary as figures within your final report.

\section{Autonomous Work}
You must submit a well-formatted report in \textbf{PDF} format detailing the methodology requested in Section \ref{sec:class}, including the final map layout of your classified image. Additionally, please include the summary requested in Section \ref{sec:asses}. The report must be uploaded to Brightspace\texttrademark\ no later than \textbf{March 12, 2026}, before the start of the next lab session.

\end{document}
