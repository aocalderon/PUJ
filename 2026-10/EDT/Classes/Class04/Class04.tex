\documentclass[aspectratio=169]{beamer}

\usepackage[utf8]{inputenc}
%\usepackage[spanish]{babel}
\usepackage{graphicx}
\usepackage{booktabs}
\usepackage{ragged2e}
\usepackage{minted}
\usepackage{xcolor}
\usepackage{algorithm}
\usepackage{algorithmic}
\usepackage{listings}
\usepackage{tikz}
\usepackage[style=authoryear, backend=biber]{biblatex}
\addbibresource{Class02.bib}
\usetikzlibrary{arrows.meta,positioning,fit,shapes.symbols}
\usetikzlibrary{arrows,shapes}
\definecolor{LightGray}{gray}{0.975}
\definecolor{links}{HTML}{2A1B81}
\hypersetup{colorlinks,linkcolor=,urlcolor=blue}
\AtBeginEnvironment{minted}{%
  \renewcommand{\fcolorbox}[4][]{#4}}

\usefonttheme{serif}

\usepackage{listings}
\lstdefinestyle{myCustomSQLStyle}{
  language=SQL,
  numbers=left,
  stepnumber=1,
  numbersep=10pt,
  tabsize=4,
  frame=single,
  backgroundcolor=\color{LightGray},
  breaklines=true,
  showspaces=false,
  showstringspaces=false
}

\newcommand\Wider[2][18mm]{%
    \makebox[\linewidth][c]{%
        \begin{minipage}{\dimexpr\textwidth+#1\relax}
            \raggedright#2
        \end{minipage}%
    }%
}
\newcommand\userinput[1]{\textbf{#1}}

\title[Class 04]{Spectral Signatures and Indices}
\author{\includegraphics[width=0.35\textwidth]{figures/G4S} \\ \url{https://gis4schools.readthedocs.io/en/latest/credits.html}}
\date{\today}

% Remove navigation symbols...
\setbeamertemplate{navigation symbols}{}

\defbeamertemplate*{footline}{shadow theme}{
    \leavevmode
    \hbox{
        \begin{beamercolorbox}[
                wd =        0.33\paperwidth,
                ht =        2.5ex,
                dp =        1.125ex,
                leftskip =  0.3cm plus1fil,
                rightskip = 0.3cm
            ]{author in head/foot}
            \flushleft EDT
        \end{beamercolorbox}
        \begin{beamercolorbox}[
                wd =        0.33\paperwidth,
                ht =        2.5ex,
                dp =        1.125ex,
                leftskip =  0.3cm plus1fil,
                rightskip = 0.3cm
            ]{author in head/foot}
            \insertshorttitle
        \end{beamercolorbox}
        \begin{beamercolorbox}[
                wd =        0.33\paperwidth,
                ht =        2.5ex,
                dp =        1.125ex,
                leftskip =  0.3cm plus1fil,
                rightskip = 0.3cm
            ]{title in head/foot}
            \hfill \insertframenumber\,/\,\inserttotalframenumber%
        \end{beamercolorbox}
    }
}

\AtBeginSection[]
{
     \begin{frame}<beamer>
     \frametitle{Plan}
     \tableofcontents[currentsection]
     \end{frame}
}

\begin{document}

\frame{\titlepage}

\begin{frame}{Outline}
    \tableofcontents
\end{frame}

\section{The Spectral Signature}

\begin{frame}{What is a Spectral Signature?}
    \begin{itemize}
        \item Every material (natural or manmade) reflects sunlight uniquely.
        \item Reflection depends on:
        \begin{itemize}
            \item Chemical composition and physical properties.
            \item Texture and surface roughness.
            \item Moisture and degradation state.
        \end{itemize}
        \item This reflected sunlight is called \textbf{reflectance} ($\rho$).
    \end{itemize}
\end{frame}

\begin{frame}{Defining the ``Fingerprint''}
    \begin{itemize}
        \item Properties define the brightness (reflectance) of ``colors'' (wavelengths) in different ``lights'' (spectral bands).
        \item \textbf{Spectral Signature:} The variation of reflectance across different wavelengths.
        \item Because these are unique, they act like a \textbf{fingerprint} for material identification.
    \end{itemize}
\end{frame}

\begin{frame}{Spectral Signature Concept}
    \centering
    \includegraphics[width=0.75\textwidth]{figures/hyperspectral}
\end{frame}

\begin{frame}{Why do we use Spectral Signatures?}
    \begin{block}{1. Monitoring Health/Status}
        If the material is known, signatures reveal degradation or health (e.g., crop monitoring).
    \end{block}
    \begin{block}{2. Identification}
        If the material is unknown, the signature allows for automatic land cover mapping.
    \end{block}
\end{frame}

\section{Land Cover vs. Land Use}

\begin{frame}{Distribution of main land covers on Earth}
    \centering
    \includegraphics[width=\textwidth]{figures/Fig1_signature.png}
\end{frame}

\begin{frame}{Land Cover}
    \begin{itemize}
        \item Describes the \textbf{physical coverage} of the Earth's surface.
        \item \textbf{Examples:}
        \begin{itemize}
            \item Farmlands, Glaciers, Forests, Lakes.
        \end{itemize}
        \item Satellite imagery is a highly efficient method for determining land cover.
    \end{itemize}
\end{frame}

\begin{frame}{Land Use}
    \begin{itemize}
        \item Describes \textbf{how humans use} the land and the activities performed there.
        \item \textbf{Examples:}
        \begin{itemize}
            \item Recreational, Residential, Commercial, Industrial.
        \end{itemize}
        \item \textbf{Warning:} Remote sensing systems provide info \textit{only} on physical coverage. Land use \textbf{cannot} be directly determined by satellite images alone.
    \end{itemize}
\end{frame}

\section{Macro Land Cover Classes}

\begin{frame}{Monitoring Global Ecosystems}
    \begin{itemize}
        \item Satellites track how land cover signatures change over time to monitor planetary health.
        \item We focus on "Macro" classes that dominate the Earth's surface.
    \end{itemize}
\end{frame}

\begin{frame}{Spectral Signature: Water}
    \begin{itemize}
        \item \textbf{Clear Water:} High absorption in the NIR/SWIR; reflects primarily in blue-green visible light.
        \item \textbf{Polluted Water:} Presence of chlorophyll or sediment shifts the signature, increasing reflectance in specific bands.
    \end{itemize}
\end{frame}

\begin{frame}{Spectral Signature: Water}
    \centering
    \includegraphics[width=0.8\textwidth]{figures/Fig2_signature.png}
\end{frame}

\begin{frame}{Spectral Signature: Water}
    \centering
    \includegraphics[width=0.8\textwidth]{figures/Fig7_signature.png}
\end{frame}

\begin{frame}{Spectral Signature: Snow vs. Clouds}
    \begin{itemize}
        \item \textbf{Snow/Ice:} High visible reflectance, but very low reflectance in Short-Wave Infrared (SWIR).
        \item \textbf{Clouds:} High reflectance across both visible and SWIR bands.
        \item This difference is key for cloud masking in satellite imagery.
    \end{itemize}
\end{frame}

\begin{frame}{Spectral Signature: Snow vs. Clouds}
    \centering
    \includegraphics[width=0.8\textwidth]{figures/Fig3_signature.png}
\end{frame}

\begin{frame}{Spectral Signature: Snow vs. Clouds}
    \centering
    \includegraphics[width=0.8\textwidth]{figures/Fig4_signature.png}
\end{frame}

% Slide 11: Soil and Vegetation
\begin{frame}{Spectral Signature: Soil and Vegetation}
    \begin{itemize}
        \item \textbf{Bare Soil:} Generally shows a steady increase in reflectance as wavelength increases.
        \item \textbf{Healthy Vegetation:} Low red reflectance (chlorophyll absorption) and very high NIR reflectance (leaf structure).
    \end{itemize}
\end{frame}

\begin{frame}{Spectral Signature: Soil and Vegetation}
    \centering
    \includegraphics[width=0.8\textwidth]{figures/Fig5_signature.png}
\end{frame}

\begin{frame}{Spectral Signature: Soil and Vegetation}
    \centering
    \includegraphics[width=0.8\textwidth]{figures/Fig6_signature.png}
\end{frame}

\section{Measuring with Satellites}

\begin{frame}{Satellite Measurements}
    \begin{itemize}
        \item Satellites record reflectance in discrete \textbf{spectral bands}.
        \item They produce multiband grayscale images.
        \item We approximate the continuous signature curve by plotting values as a \textbf{polyline}.
    \end{itemize}
\end{frame}

\begin{frame}{Satellite Measurements}
    \centering
    \includegraphics[height=0.925\textheight]{figures/Fig8tris_signature.png}
\end{frame}

\begin{frame}{The Necessity of Atmospheric Correction}
    \begin{alertblock}{Warning}
        Use ONLY atmospherically corrected images!
    \end{alertblock}
    \begin{itemize}
        \item Satellites capture both surface reflectance and atmospheric scattering (noise).
        \item Scattering must be removed to analyze the true signal of the Earth's surface.
    \end{itemize}
\end{frame}

\section{Comparing Signatures}

\begin{frame}{The Feature Space}
    \begin{itemize}
        \item Comparing ``curves'' visually is difficult.
        \item We use a reference system of orthogonal axes where \textbf{each axis is a spectral band}.
        \item In this space, a spectral signature becomes a single \textbf{point}.
    \end{itemize}
\end{frame}

\begin{frame}{The Feature Space}
    \centering
    \includegraphics[height=0.925\textheight]{figures/Fig8bis1_signature.png}
\end{frame}

\begin{frame}{Similarity via Euclidean Distance}
    Similarity is determined by how close points are in the feature space.
    \begin{equation}
        D = \sqrt{(x_2 - x_1)^2 + (y_2 - y_1)^2 + (z_2 - z_1)^2}
    \end{equation}
    \begin{itemize}
        \item $x, y, z$ represent reflectances in different bands.
        \item Closer points = More similar materials.
    \end{itemize}
\end{frame}

\begin{frame}{Similarity via Euclidean Distance}
    \centering
    \includegraphics[width=0.75\textwidth]{figures/Fig8bis2_signature.png}
\end{frame}

\begin{frame}{Spectral Classes}
    \begin{itemize}
        \item Points that group together in the feature space form a \textbf{cluster}.
        \item These clusters are referred to as \textbf{spectral classes}.
        \item This is the mathematical basis for automatic land cover mapping.
    \end{itemize}
\end{frame}

\section{Spectral Indices}

\begin{frame}{What is a Spectral Index?}
    \begin{itemize}
        \item A mathematical expression applied to a multispectral image.
        \item \textbf{Goal:} Highlight specific properties (health, moisture, mineral abundance).
        \item Transforms qualitative images into \textbf{quantitative numerical tools}.
    \end{itemize}
\end{frame}

\begin{frame}{Mathematical Families of Indices}
    \begin{table}
        \begin{tabular}{lll}
            \toprule
            \textbf{Family} & \textbf{Example} & \textbf{Pros} \\
            \midrule
            Difference & CRI & Simple calculation \\
            Ratio & RVI & Reduces atmospheric effects \\
            Normalized & NDVI & Bounded range (-1 to 1) \\
            Complex & EVI & Highly accurate mapping \\
            \bottomrule
        \end{tabular}
    \end{table}
\end{frame}

\begin{frame}{Designing Indices: The Vegetation Gap}
    \begin{itemize}
        \item Healthy leaves have a massive gap between \textbf{Red} (absorption) and \textbf{NIR} (reflectance).
        \item As vigor or biomass increases, this gap widens.
        \item Most vegetation indices use the relationship between these two bands.
    \end{itemize}
\end{frame}

\begin{frame}{Designing Indices: The Vegetation Gap}
    \centering
    \includegraphics[width=0.85\textwidth]{figures/Fig1_SI.png}
\end{frame}

\begin{frame}{Designing Indices: The Vegetation Gap}
    \centering
    \includegraphics[width=0.9\textwidth]{figures/Fig2_SI.png}
\end{frame}


\begin{frame}{Ratio Vegetation Index (RVI)}
    The simplest greenness index:
    \begin{equation}
        RVI = \frac{NIR}{Red}
    \end{equation}
    \begin{itemize}
        \item Always positive ($RVI \ge 0$).
        \item Larger ratio = healthier/denser vegetation.
        \item \textbf{Limitation:} Not bounded from above, making comparisons difficult.
    \end{itemize}
\end{frame}

\begin{frame}{RVI Typical Values}
    \begin{itemize}
        \item \textbf{4 to 10:} General healthy vegetation.
        \item \textbf{Up to 30:} Very dense, very healthy vegetation.
        \item \textbf{Below 4:} Sparse or ``sick'' vegetation.
    \end{itemize}
\end{frame}

\begin{frame}{Normalized Difference Vegetation Index (NDVI)}
    The most widely used index in remote sensing:
    \begin{equation}
        NDVI = \frac{NIR - Red}{NIR + Red}
    \end{equation}
    \begin{itemize}
        \item Normalized range: \textbf{-1 to 1}.
        \item For vegetated land, values are always positive.
    \end{itemize}
\end{frame}

\begin{frame}{NDVI Thresholds for Land Cover}
    \begin{itemize}
        \item \textbf{NDVI $<$ 0.2:} Bare ground / Soil.
        \item \textbf{0.2 to 0.6:} Shrubs, grass, and crops.
        \item \textbf{NDVI $>$ 0.6:} Dense forests.
    \end{itemize}
\end{frame}

\begin{frame}{NDVI: Interpreting the Signal}
    \begin{itemize}
        \item \textbf{For Mixed Pixels (Amount):}
        \begin{itemize}
            \item Low (0.2 - 0.4): Sparse veg.
            \item High ($>$ 0.6): High-density veg.
        \end{itemize}
        \item \textbf{For Full Cover (Health):}
        \begin{itemize}
            \item Low (0 - 0.2): Very sick veg.
            \item High ($>$ 0.6): Very healthy veg.
        \end{itemize}
    \end{itemize}
\end{frame}

\begin{frame}{NDVI Beyond Vegetation}
    \begin{itemize}
        \item \textbf{Close to -1:} Clear water.
        \item \textbf{-1 to 0:} Polluted water, snow, or ice.
        \item \textbf{Close to 0:} Clouds.
        \item \textbf{0 to 0.2:} Bare soil.
    \end{itemize}
\end{frame}

\begin{frame}{Normalized Difference Water Index (NDWI)}
    Designed to highlight water bodies:
    \begin{equation}
        NDWI = \frac{Green - NIR}{Green + NIR}
    \end{equation}
    \begin{itemize}
        \item Positive for water.
        \item \textbf{Threshold 0.3:} Often used to separate flooded from non-flooded areas.
    \end{itemize}
\end{frame}

\begin{frame}{NDWI for Water Stress}
    \begin{alertblock}{Caution}
        A different index, also called NDWI, is used for vegetation.
    \end{alertblock}
    \begin{equation}
        NDWI_{stress} = \frac{NIR - SWIR}{NIR + SWIR}
    \end{equation}
    \begin{itemize}
        \item Used to detect drought or water stress in plants.
    \end{itemize}
\end{frame}

\begin{frame}{Normalized Difference Snow Index (NDSI)}
    Differentiates snow from clouds using the SWIR band:
    \begin{equation}
        NDSI = \frac{Green - SWIR}{Green + SWIR}
    \end{equation}
    \begin{itemize}
        \item \textbf{Snow:} High Green, Low SWIR.
        \item \textbf{Clouds:} High Green, High SWIR.
        \item \textbf{Threshold 0.4:} Differentiates snow cover.
    \end{itemize}
\end{frame}

\begin{frame}{Applications}
    \begin{itemize}
        \item \textbf{Agriculture:} Monitoring crop health and yield prediction.
        \item \textbf{Biodiversity:} Mapping tropical forest cover and detecting deforestation.
        \item \textbf{Geology:} Identifying mineral compositions through SWIR absorption bands.
    \end{itemize}
\end{frame}

\begin{frame}{Build Your Own Spectral Index}
    \begin{enumerate}
        \item Identify the spectral signature of the standard state.
        \item Determine how your target phenomenon affects reflectance.
        \item Create a math expression using specific bands.
        \item Ensure the output is proportional to the observed phenomenon.
    \end{enumerate}
\end{frame}

\begin{frame}{Summary}
    \begin{itemize}
        \item Spectral signatures are the ``fingerprints'' of the Earth's surface.
        \item Indices (NDVI, NDWI, NDSI) turn raw imagery into quantitative data.
        \item Understanding these principles is essential for environmental monitoring, agriculture, and climate change studies.
    \end{itemize}
\end{frame}

\end{document}
