\documentclass[aspectratio=169]{beamer}

\usepackage[utf8]{inputenc}
%\usepackage[spanish]{babel}
\usepackage{graphicx}
\usepackage{booktabs}
\usepackage{ragged2e}
\usepackage{minted}
\usepackage{xcolor}
\usepackage{algorithm}
\usepackage{algorithmic}
\usepackage{listings}
\usepackage{tikz}
\usepackage[style=authoryear, backend=biber]{biblatex}
\addbibresource{Class02.bib}
\usetikzlibrary{arrows.meta,positioning,fit,shapes.symbols}
\usetikzlibrary{arrows,shapes}
\definecolor{LightGray}{gray}{0.975}
\definecolor{links}{HTML}{2A1B81}
\hypersetup{colorlinks,linkcolor=,urlcolor=blue}
\AtBeginEnvironment{minted}{%
  \renewcommand{\fcolorbox}[4][]{#4}}

\usefonttheme{serif}

\usepackage{listings}
\lstdefinestyle{myCustomSQLStyle}{
  language=SQL,
  numbers=left,
  stepnumber=1,
  numbersep=10pt,
  tabsize=4,
  frame=single,
  backgroundcolor=\color{LightGray},
  breaklines=true,
  showspaces=false,
  showstringspaces=false
}

\newcommand\Wider[2][18mm]{%
    \makebox[\linewidth][c]{%
        \begin{minipage}{\dimexpr\textwidth+#1\relax}
            \raggedright#2
        \end{minipage}%
    }%
}
\newcommand\userinput[1]{\textbf{#1}}

\title[Class 02]{Introduction to Remote Sensing}
\author{The EO4GEO project and IGIK Poland}
\date{\today}

% Remove navigation symbols...
\setbeamertemplate{navigation symbols}{}

\defbeamertemplate*{footline}{shadow theme}{
    \leavevmode
    \hbox{
        \begin{beamercolorbox}[
                wd =        0.33\paperwidth,
                ht =        2.5ex,
                dp =        1.125ex,
                leftskip =  0.3cm plus1fil,
                rightskip = 0.3cm
            ]{author in head/foot}
            \flushleft EDT
        \end{beamercolorbox}
        \begin{beamercolorbox}[
                wd =        0.33\paperwidth,
                ht =        2.5ex,
                dp =        1.125ex,
                leftskip =  0.3cm plus1fil,
                rightskip = 0.3cm
            ]{author in head/foot}
            \insertshorttitle
        \end{beamercolorbox}
        \begin{beamercolorbox}[
                wd =        0.33\paperwidth,
                ht =        2.5ex,
                dp =        1.125ex,
                leftskip =  0.3cm plus1fil,
                rightskip = 0.3cm
            ]{title in head/foot}
            \hfill \insertframenumber\,/\,\inserttotalframenumber%
        \end{beamercolorbox}
    }
}

\AtBeginSection[]
{
     \begin{frame}<beamer>
     \frametitle{Plan}
     \tableofcontents[currentsection]
     \end{frame}
}

\begin{document}

\frame{\titlepage}

% \begin{frame}{}
%     \centering
%     \includegraphics[width=0.35\textwidth]{figures/book_cover.jpg} \\
%     \vspace{5mm}
%     {
%         \tiny
%         Content has been extracted from \textit{Database System Concepts}, Seventh Edition, by Silberschatz, Korth and Sudarshan. Mc Graw Hill Education. 2019.\\
%         Visit \url{https://db-book.com/}.\\
%     }
% \end{frame}

\begin{frame}{Introduction to Remote Sensing}
    Content from:
    \begin{itemize}
         \item The EO4GEO project \\ \href{http://www.eo4geo.eu/}{http://www.eo4geo.eu/}

         \item EO4GEO Course Material \\ \href{https://github.com/eo4geocourses}{https://github.com/eo4geocourses}

         \item The Institute of Geodesy and Cartography - Poland \\ \href{http://www.igik.edu.pl/en}{http://www.igik.edu.pl/en}

         \item Introduction to Remote Sensing Slides \\ \href{https://eo4geocourses.github.io/IGIK_Introduction-to-Remote-Sensing/}{https://eo4geocourses.github.io/IGIK\_Introduction-to-Remote-Sensing/}

         \item The Electromagnetic Song \\ \href{https://www.youtube.com/watch?v=bjOGNVH3D4Y}{https://www.youtube.com/watch?v=bjOGNVH3D4Y}
    \end{itemize}
\end{frame}

\begin{frame}{Additional Material}
    \begin{itemize}
        \item Imaging Sensors: Aerial Survey Cameras (end page 36) \parencite{lillesand_remote_2015}.
        \item The QGIS Project \parencite{graser_qgis_2025}.
        \item A (gentle) QGIS Tutorial \parencite{flenniken_quantum_2020}.

    \end{itemize}
\end{frame}

\begin{frame}[allowframebreaks]{References}
    \printbibliography
\end{frame}

\end{document}
