\documentclass[aspectratio=169]{beamer}

\usepackage[utf8]{inputenc}
%\usepackage[spanish]{babel}
\usepackage{graphicx}
\usepackage{booktabs}
\usepackage{ragged2e}
\usepackage{minted}
\usepackage{xcolor}
\usepackage{algorithm}
\usepackage{algorithmic}
\usepackage{listings}
\usepackage{tikz}
\usepackage[style=authoryear, backend=biber]{biblatex}
\addbibresource{Class02.bib}
\usetikzlibrary{arrows.meta,positioning,fit,shapes.symbols}
\usetikzlibrary{arrows,shapes}
\definecolor{LightGray}{gray}{0.975}
\definecolor{links}{HTML}{2A1B81}
\hypersetup{colorlinks,linkcolor=,urlcolor=blue}
\AtBeginEnvironment{minted}{%
  \renewcommand{\fcolorbox}[4][]{#4}}

\usefonttheme{serif}

\usepackage{listings}
\lstdefinestyle{myCustomSQLStyle}{
  language=SQL,
  numbers=left,
  stepnumber=1,
  numbersep=10pt,
  tabsize=4,
  frame=single,
  backgroundcolor=\color{LightGray},
  breaklines=true,
  showspaces=false,
  showstringspaces=false
}

\newcommand\Wider[2][18mm]{%
    \makebox[\linewidth][c]{%
        \begin{minipage}{\dimexpr\textwidth+#1\relax}
            \raggedright#2
        \end{minipage}%
    }%
}
\newcommand\userinput[1]{\textbf{#1}}

\title[Class 05]{Satellite Imagery Classification}
\author{Lillesand, Chipman \& Kiefer, 2015.}
\date{\today}

% Remove navigation symbols...
\setbeamertemplate{navigation symbols}{}

\defbeamertemplate*{footline}{shadow theme}{
    \leavevmode
    \hbox{
        \begin{beamercolorbox}[
                wd =        0.33\paperwidth,
                ht =        2.5ex,
                dp =        1.125ex,
                leftskip =  0.3cm plus1fil,
                rightskip = 0.3cm
            ]{author in head/foot}
            \flushleft EDT
        \end{beamercolorbox}
        \begin{beamercolorbox}[
                wd =        0.33\paperwidth,
                ht =        2.5ex,
                dp =        1.125ex,
                leftskip =  0.3cm plus1fil,
                rightskip = 0.3cm
            ]{author in head/foot}
            \insertshorttitle
        \end{beamercolorbox}
        \begin{beamercolorbox}[
                wd =        0.33\paperwidth,
                ht =        2.5ex,
                dp =        1.125ex,
                leftskip =  0.3cm plus1fil,
                rightskip = 0.3cm
            ]{title in head/foot}
            \hfill \insertframenumber\,/\,\inserttotalframenumber%
        \end{beamercolorbox}
    }
}

\AtBeginSection[]
{
     \begin{frame}<beamer>
     \frametitle{Plan}
     \tableofcontents[currentsection]
     \end{frame}
}

\begin{document}

\frame{\titlepage}

\begin{frame}{Outline}
    \tableofcontents
\end{frame}

\section{Image Classification}

\begin{frame}{Spectral Pattern Recognition}
    \begin{itemize}
        \item \textbf{The Core Objective:} Automatically categorize pixels into land cover classes or ``themes''.
        \item \textbf{The ``Individualist'' Approach:}
        \begin{itemize}
            \item Classification is performed \textbf{pixel-by-pixel}.
            \item Based on spectral reflectance or emissivity signatures.
            \item \alert{Crucial Note:} It ignores the pixel's neighbors or surroundings.
        \end{itemize}
        \item \textbf{Extending the Model:}
        \begin{itemize}
            \item \textbf{Polarization:} Radar imagery.
            \item \textbf{Temporal:} Multitemporal sequences.
            \item \textbf{Bidirectional:} Multi-angle imagery (e.g., MISR).
        \end{itemize}
    \end{itemize}
\end{frame}

\begin{frame}{Spatial Pattern Recognition}
    \begin{itemize}
        \item \textbf{The ``Contextual'' Approach:} Categorizes pixels based on their relationship with surrounding pixels.
        \item \textbf{Key Variables Considered:}
        \begin{itemize}
            \item Texture, proximity, size, and shape.
            \item Directionality, repetition, and context.
        \end{itemize}
        \item \textbf{Human Synthesis:} Attempts to replicate the visual interpretation process of a human analyst.
        \item \textbf{Complexity:} More computationally intensive than spectral methods.
        \item \textbf{Hybrid Models (OBIA):} Object-Based Image Analysis combines both spectral and spatial logic.
    \end{itemize}
\end{frame}

\begin{frame}{Supervised Classification}
    \textit{The analyst acts as the teacher for the algorithm.}
    \vspace{0.3cm}
    \begin{enumerate}
        \item \textbf{Training Stage:} The analyst identifies representative sample sites of known cover types (\textbf{Training Areas}).
        \item \textbf{Interpretation Key:} These areas are used to compile numerical descriptors of spectral attributes.
        \item \textbf{Classification:} The computer compares every unknown pixel to the key.
        \item \textbf{Labeling:} The pixel is assigned to the category it ``looks most like'' based on statistical strategies.
    \end{enumerate}
\end{frame}

\begin{frame}{Unsupervised and Hybrid Classification}
    \begin{block}{Unsupervised Classification}
        \begin{itemize}
            \item \textbf{Data-First:} Pixels are grouped into ``natural'' clusters based on spectral similarity.
            \item \textbf{Post-Classification Identity:} The analyst labels the clusters \textit{after} the grouping is done using ground reference data.
        \end{itemize}
    \end{block}

    \begin{block}{Hybrid \& Specialized Procedures}
        \begin{itemize}
            \item \textbf{Hybrid:} Combines supervised and unsupervised steps for better efficiency.
            \item \textbf{Advanced Topics:} Neural networks, mixed pixel analysis, and specialized hyperspectral procedures.
        \end{itemize}
    \end{block}

    \centering
    \small \textit{``There is no single `right' manner... approach depends on data nature and intended application.''}
\end{frame}

\section{Supervised Classification}

\begin{frame}{Data Input and Spectral Patterns}
    \begin{itemize}
        \item \textbf{Sensor Agnostic:} Procedures apply to airborne multispectral data and satellite platforms (Landsat, SPOT, WorldView-2).
        \item \textbf{Multichannel Framework:} Analysis typically involves multiple spectral bands, for example:
        \begin{itemize}
            \item Visible: Blue, Green, Red
            \item Infrared: Near-Infrared (NIR), Thermal Infrared
        \end{itemize}
        \item \textbf{Digital Numbers (DNs):} The sensor measures scene radiance recorded as DNs for each pixel across all bands.
        \item \textbf{Basis for Classification:} If spectral response patterns are sufficiently distinct for different terrain features, they form a ``signature'' for automated categorization.
    \end{itemize}
\end{frame}

\begin{frame}{The Three-Step Procedure}
    Supervised classification follows a structured three-stage workflow:
    \vspace{0.3cm}
    \begin{enumerate}
        \item \textbf{Training Stage:} The analyst identifies representative ``training areas'' to develop numerical descriptions of the spectral attributes for each land cover type.
        \item \textbf{Classification Stage:} Each pixel is compared to the training data and categorized into the class it most closely resembles.
        \item \textbf{Output Stage:} Results are compiled into final products for end-users.
    \end{enumerate}
    \vspace{0.2cm}
    \begin{quote}
        Pixels that do not sufficiently match any training set are labeled as \textbf{``unknown''.}
    \end{quote}
\end{frame}

\begin{frame}{Classification Stage and Output Products}
    \begin{itemize}
        \item \textbf{The ``Heart'' of the Process:} The classification stage uses computer-based \textbf{decision rules} to evaluate spectral patterns.
        \item \textbf{Versatile Output Formats:}
        \begin{itemize}
            \item \textbf{Thematic Maps:} Visual representations of land cover.
            \item \textbf{Statistical Tables:} Summary data for each class.
            \item \textbf{Digital Data Files:} Classified data can be exported directly into a \textbf{Geographic Information System (GIS)}.
        \end{itemize}
        \item \textbf{Strategic Integration:} Classification output often becomes GIS input, enabling complex spatial modeling and decision-making.
    \end{itemize}
\end{frame}

\begin{frame}{Multispectral Scan Line Profile}
    \centering
    \includegraphics[width=\textwidth]{figures/fig34}
\end{frame}

\begin{frame}{Basic steps}
    \centering
    \includegraphics[width=\textwidth]{figures/fig35}
\end{frame}

\section{The Classification Stage}

\begin{frame}{The Concept of Measurement Space}
    \begin{itemize}
        \item \textbf{Measurement Vectors:} Each pixel is represented as a point in a multi-dimensional space.
        \item \textbf{Scatter Plots:} For a 2-band subset, DNs from Band 3 ($y$-axis) and Band 4 ($x$-axis) define a pixel's coordinates.
        \item \textbf{Spectral Clusters:} Training data forms ``clouds of points'' rather than single values.
        \item \textbf{Natural Variability:} These clouds illustrate the centralizing tendency and inherent variance of specific land cover classes.
    \end{itemize}
\end{frame}

\begin{frame}{Pixel observations to scatter diagram}
    \centering
    \includegraphics[width=0.575\textwidth]{figures/fig36}
\end{frame}

\begin{frame}{Minimum-Distance-to-Means Classifier}
    \begin{itemize}
        \item \textbf{Mechanism:}
        \begin{enumerate}
            \item Calculate the \textbf{mean vector} (average spectral value) for each category.
            \item Compute the Euclidean distance from an unknown pixel to each mean.
            \item Assign pixel to the ``closest'' class.
        \end{enumerate}
        \item \textbf{Pros:} Mathematically simple and computationally efficient.
        \item \textbf{Cons:} Insensitive to different degrees of \textbf{variance}.
        \item \alert{Limitation:} A pixel may be mathematically closer to a ``tight'' cluster (like sand) even if it logically belongs to a ``wide'' cluster (like urban).
    \end{itemize}
\end{frame}

\begin{frame}{Minimum distance to means classification strategy}
    \centering
    \includegraphics[width=0.575\textwidth]{figures/fig37}
\end{frame}

\begin{frame}{Parallelepiped Classifier}
    \begin{itemize}
        \item \textbf{Mechanism:} Uses the range (min/max) of DNs in each band to create ``decision regions'' (rectangles in 2D, parallelepipeds in n-D).
        \item \textbf{Sensitivity:} Accounts for category variance by sizing the boxes differently.
        \item \textbf{The Overlap Problem:}
        \begin{itemize}
            \item If categories overlap, pixels are labeled ``not sure''.
            \item Fails to account for \textbf{covariance} (slanted clouds where bands vary together).
        \end{itemize}
        \item \textbf{Performance:} Very fast, but often poor for highly correlated spectral response patterns.
    \end{itemize}
\end{frame}

\begin{frame}{Parallelepiped classification strategy}
    \centering
    \includegraphics[width=0.575\textwidth]{figures/fig38}
\end{frame}

\begin{frame}{Gaussian Maximum Likelihood Classifier (GMLC)}
    \begin{itemize}
        \item \textbf{The Probabilistic Leap:} Evaluates both variance and covariance by assuming a \textbf{Gaussian (Normal)} distribution.
        \item \textbf{Parameters:} Completely described by the mean vector $\mu$ and the covariance matrix $\Sigma$.
        \item \textbf{Mechanism:} Calculates the statistical probability of a pixel belonging to each class; assigns it to the ``most likely'' one.
        \item \textbf{Equi-probability Contours:} Delineates ellipsoidal decision regions that follow the ``slant'' of the data.
    \end{itemize}
\end{frame}

\begin{frame}{Probability density by a maximum likelihood classifier.}
    \centering
    \includegraphics[width=0.7\textwidth]{figures/fig39}
\end{frame}

\begin{frame}{Equiprobability contours by a maximum likelihood classifier.}
    \centering
    \includegraphics[width=0.575\textwidth]{figures/fig40}
\end{frame}

\begin{frame}{Bayesian Classifier \& Cost Functions}
    The Bayesian approach refines GMLC by adding two weighting factors:
    \vspace{0.2cm}
    \begin{enumerate}
        \item \textbf{A Priori Probability:} The anticipated likelihood of a class occurring in the scene (e.g., urban is more likely than rare sand).
        \item \textbf{Cost of Misclassification:} A weight applied to minimize the ``damage'' of specific errors.
    \end{enumerate}
    \vspace{0.2cm}
    \begin{block}{The Ideal Goal}
        To minimize the overall ``cost'' of errors, resulting in a theoretically optimum classification.
    \end{block}
\end{frame}

\begin{frame}{Efficiency vs. Accuracy}
    \begin{itemize}
        \item \textbf{The Drawback of GMLC:} Large number of computations, especially with many bands.
        \item \textbf{Dimensionality Reduction:} Using Principal Component transformations to speed up the process.
        \item \textbf{Decision Tree (Layered) Classifiers:}
        \begin{itemize}
            \item A ``multi-step'' approach.
            \item Separate simple classes (like water) first using basic thresholds.
            \item Reserve complex GMLC logic only for ``ambiguous'' overlapping classes.
        \end{itemize}
    \end{itemize}
\end{frame}

\section{The Training Stage}

\begin{frame}{The Analyst's Critical Role}
    \begin{itemize}
        \item \textbf{Manual vs. Automated:} While classification is automated, training is a manual process requiring ``art and science''.
        \item \textbf{The Objective:} To define the location, size, and orientation of the ``clouds of points'' (spectral signatures) for each class.
        \item \textbf{Requirements:}
        \begin{itemize}
            \item Close interaction with image data.
            \item Substantial reference data.
            \item Thorough knowledge of the geographic area.
        \end{itemize}
        \item \alert{Success Factor:} The quality of training determines the value of the entire classification effort.
    \end{itemize}
\end{frame}

\begin{frame}{Information Classes vs. Spectral Classes}
    \begin{itemize}
        \item \textbf{Information Class:} The desired category (e.g., ``Water'' or ``Agriculture'').
        \item \textbf{Spectral Class:} The distinct spectral subgroups within an information class.
        \item \textbf{The ``Representative'' Rule:}
        \begin{itemize}
            \item One information class often requires \textbf{multiple} spectral classes.
            \item \textbf{Example:} ``Water'' might include clear water, turbid water, and deep water (each requiring its own training set).
            \item \textbf{Complexity:} Agriculture may vary by planting date, soil moisture, and crop variety.
        \end{itemize}
    \end{itemize}
\end{frame}

\begin{frame}{Methods of Delineation}
    \begin{columns}
        \begin{column}{0.5\textwidth}
            \textbf{Manual Polygons:}
            \begin{itemize}
                \item Analyst draws boundaries around known areas.
                \item \textbf{Edge Avoidance:} Avoid pixels on boundaries or ``rough'' areas to ensure purity.
            \end{itemize}
        \end{column}
        \begin{column}{0.5\textwidth}
            \textbf{Seed Pixel Approach:}
            \begin{itemize}
                \item Analyst selects a representative ``seed''.
                \item Algorithm grows the area by including contiguous pixels with similar spectral traits.
            \end{itemize}
        \end{column}
    \end{columns}
\end{frame}

\begin{frame}{Statistical Requirements}
    \begin{itemize}
        \item \textbf{Sample Size ($n$ = number of bands):}
        \begin{itemize}
            \item Theoretical minimum: $n + 1$ pixels.
            \item \textbf{Practical Rule:} Use $10n$ to $100n$ pixels per training set.
        \end{itemize}
        \item \textbf{Spatial Dispersion:}
        \begin{itemize}
            \item It is better to have many small sites (e.g., 20 sites of 40 pixels) than one giant site (1 site of 800 pixels).
            \item Dispersion captures scene-wide variations.
        \end{itemize}
        \item \textbf{Mathematical Basis:} These pixels generate the \textbf{mean vector} and \textbf{covariance matrix} for the classifier.
    \end{itemize}
\end{frame}

\begin{frame}{Training area polygons}
    \centering
    \includegraphics[width=\textwidth]{figures/fig41}
\end{frame}

\begin{frame}{The Refinement Process}
    \begin{itemize}
        \item \textbf{Statistical Testing:} Ensure data is \textbf{unimodal} and Gaussian (normally distributed).
        \item \textbf{Cleaning Data:}
        \begin{itemize}
            \item Delete ``edge pixels'' or outliers (e.g., bare soil found within a crop field).
            \item Split bimodal distributions into two distinct spectral classes.
        \end{itemize}
        \item \textbf{Spectral Separability:} Evaluate if training sets for different classes overlap too much.
        \item \textbf{Final Goal:} A clean, non-redundant set of ``interpretation keys'' for the computer to follow.
    \end{itemize}
\end{frame}

\begin{frame}{Graphical Analysis: Histograms}
    \begin{itemize}
        \item \textbf{Visual Normality Check:} Essential for Maximum Likelihood classifiers to ensure data follows a Gaussian distribution.
        \item \textbf{Identifying Subclasses:}
        \begin{itemize}
            \item A \textbf{bimodal distribution} (two peaks) suggests the training site contains two distinct subclasses.
            \item \textit{Example:} ``Hay'' might actually be two different varieties or have different illumination conditions.
        \end{itemize}
        \item \textbf{Action:} Split bimodal classes into separate spectral categories to improve overall accuracy.
    \end{itemize}
\end{frame}

\begin{frame}{Visualization of training plot data}
    \centering
    \includegraphics[width=0.85\textwidth]{figures/fig42}
\end{frame}

\begin{frame}{Coincident Spectral Plots}
    \begin{itemize}
        \item \textbf{Comparison Tool:} Unlike histograms, these allow for direct comparison between different category types.
        \item \textbf{Visualization:} Displays mean response and variance ($\pm 2$ standard deviations) for each band.
        \item \textbf{Key Insight:} Reveals spectral overlap between classes (e.g., Hay vs. Corn).
        \item \textbf{Band Selection:} Helps identify which bands offer the best ``reversals'' or gaps for class separation.
    \end{itemize}
\end{frame}

\begin{frame}{Multispectral Images}
    \centering
    \includegraphics[width=0.425\textwidth]{figures/fig43}
\end{frame}

\begin{frame}{Scatter Diagrams (Measurement Space)}
    \begin{itemize}
        \item \textbf{Multi-Band View:} Captures relationships that single-band histograms miss.
        \item \textbf{Correlation Analysis:}
        \begin{itemize}
            \item \textbf{High Correlation:} Near-linear ``cloud of points'' (e.g., Visible Green vs. Red). Harder to separate classes.
            \item \textbf{Low Correlation:} Expanded ``measurement space'' (e.g., Red vs. Near-IR). Much easier to separate land cover types.
        \end{itemize}
        \item \textbf{Practical Application:} Often, just two well-chosen, low-correlation bands are enough for generalized classification.
    \end{itemize}
\end{frame}


\begin{frame}{Histograms and two-dimensional scatter diagram}
    \centering
    \includegraphics[width=0.65\textwidth]{figures/fig44}
\end{frame}

\begin{frame}{Histograms and two-dimensional scatter diagram}
    \centering
    \includegraphics[width=0.65\textwidth]{figures/fig45}
\end{frame}

\begin{frame}{Quantitative Separability Measures}
    \begin{itemize}
        \item \textbf{Statistical Distance:} Computing a ``matrix of divergence'' between all class pairs.
        \item \textbf{Transformed Divergence (TD):} A covariance-weighted distance.
        \begin{itemize}
            \item Values $>1500$ generally indicate good separation.
            \item Values $<1500$ suggest problematic spectral overlap.
        \end{itemize}
        \item \textbf{Jeffries-Matusita (JM) Distance:} Similar to TD but with a fixed scale (maximum 1414).
        \item \textbf{Diagnosis:} If overlap occurs between \textit{different} information classes (e.g., Hay and Corn), refinement is needed.
    \end{itemize}
\end{frame}

\begin{frame}{Divergence Matrix}
    \centering
    \includegraphics[width=\textwidth]{figures/tab1}
\end{frame}

\begin{frame}{Self-Classification of Training Data}
    \begin{itemize}
        \item \textbf{The ``Preliminary Test'':} Classifying only the training pixels to see if the computer assigns them correctly.
        \item \textbf{The Error Matrix:} Displays percentages of ``as expected'' vs. ``misclassified'' training pixels.
        \item \alert{The Warning:} Do NOT confuse this with final map accuracy.
        \begin{itemize}
            \item Training areas are ``pure'' examples; the rest of the scene is usually much ``messier''.
            \item High training accuracy does not guarantee high scene-wide accuracy.
        \end{itemize}
    \end{itemize}
\end{frame}

\begin{frame}{Interactive Preliminary Classification}
    \begin{itemize}
        \item \textbf{Real-Time Feedback:} Using efficient algorithms (like parallelepiped) to highlight areas the current statistics would capture.
        \item \textbf{Visual Approximation:} Highlighting ``classified'' pixels in color over the raw imagery.
        \item \textbf{Representative Subsets:} Classifying a small, diverse section of the image first to verify the logic before committing to the full scene.
        \item \textbf{Iteration:} ``Trial and error'' testing of alternative deletions and poolings of training classes.
    \end{itemize}
\end{frame}

\begin{frame}{Tactics for Difficult Overlaps}
    When spectral classes refuse to separate, the analyst has several choices:
    \vspace{0.2cm}
    \begin{enumerate}
        \item \textbf{Merger/Aggregation:} Combine specific classes into broader categories (e.g., Merge ``Birch'' and ``Aspen'' into ``Deciduous'').
        \item \textbf{Deletion:} Eliminate rare problem classes to preserve the accuracy of extensive, similar classes.
        \item \textbf{Trial and Error:} Adjust sample sizes or re-evaluate the identity of training sites.
    \end{enumerate}
    \vfill
    \centering
    \textit{``If classes are inherently similar, no amount of retraining will make them separable!''}
\end{frame}

\begin{frame}{Multi-Image and Ancillary Data}
    \begin{itemize}
        \item \textbf{The Multi-Image Problem:} Atmospheric and sun-angle changes mean ``signatures'' from one date rarely work on another.
        \item \textbf{Radiometric Normalization:} Matching secondary images to a ``base'' image to allow signature extension.
        \item \textbf{The Hybrid Future:}
        \begin{itemize}
            \item Incorporating \textbf{GIS data} (elevation, soil type).
            \item Using \textbf{Multitemporal} data (seasonal changes).
            \item \textbf{Spatial Pattern Recognition} to mimic human visual interpretation.
        \end{itemize}
    \end{itemize}
\end{frame}

%\begin{frame}{}
%\end{frame}

\end{document}
