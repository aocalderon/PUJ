\documentclass[aspectratio=169]{beamer}

\definecolor{links}{HTML}{2A1B81}
\hypersetup{colorlinks,linkcolor=,urlcolor=blue}
\usepackage[utf8]{inputenc}
\usepackage{graphicx}
\usepackage{booktabs}
\usepackage{ragged2e}
\usepackage{xcolor}
\definecolor{LightGray}{gray}{0.975}

\usepackage{tikz}
\usetikzlibrary{arrows,shapes}

\usepackage{algorithm}
\usepackage{algorithmic}

\usepackage{minted}
\usepackage{xcolor}
\definecolor{LightGray}{gray}{0.975}

\usepackage{listings}

%\usetheme{Warsaw}
\usefonttheme{serif}

\title[Reverse Design]{Database Administration}
\subtitle{Lecture 03: ERD Reverse Design.}
\author{Udemy and Visual Paradigm.}
\date{\today}

% Remove navigation symbols...
\setbeamertemplate{navigation symbols}{}

\defbeamertemplate*{footline}{shadow theme}{
    \leavevmode
    \hbox{
        \begin{beamercolorbox}[
                wd =        0.33\paperwidth,
                ht =        2.5ex,
                dp =        1.125ex,
                leftskip =  0.3cm plus1fil,
                rightskip = 0.3cm
            ]{author in head/foot}
            \flushleft DBA
        \end{beamercolorbox}
        \begin{beamercolorbox}[
                wd =        0.33\paperwidth,
                ht =        2.5ex,
                dp =        1.125ex,
                leftskip =  0.3cm plus1fil,
                rightskip = 0.3cm
            ]{author in head/foot}
            \insertshorttitle
        \end{beamercolorbox}
        \begin{beamercolorbox}[
                wd =        0.33\paperwidth,
                ht =        2.5ex,
                dp =        1.125ex,
                leftskip =  0.3cm plus1fil,
                rightskip = 0.3cm
            ]{title in head/foot}
            \hfill \insertframenumber\,/\,\inserttotalframenumber%
        \end{beamercolorbox}
    }
}

\AtBeginSection[]
{
     \begin{frame}<beamer>
     \frametitle{Plan}
     \tableofcontents[currentsection]
     \end{frame}
}

\newcommand{\toRight}[1]{
    \begin{FlushRight}
        {\tiny #1}
    \end{FlushRight}
} % Align to right

\begin{document}

\frame{\titlepage}

\begin{frame}{Database Administration: Reverse Design.}
    \centering
    \includegraphics[width=0.8\textwidth]{figures/udemy2}\\
    \includegraphics[width=0.8\textwidth]{figures/udemy3}\\
    \vspace{2mm}
    {
        \scriptsize
        Content has been extracted from \textit{Database Design and Management.} Udemy Course, created by Visual Paradigm, 2025.  Visit \url{https://www.udemy.com/course/database-design-and-management/} and \url{https://www.visual-paradigm.com/}.\\
    }
\end{frame}

\section{Introduction}
\begin{frame}{ }
    \centering
    \includegraphics[width=\textwidth, trim={14mm 32mm 14mm 156mm}, clip]{resources/01/01_1}
\end{frame}
\begin{frame}{ }
    \centering
    \includegraphics[width=\textwidth, trim={14mm 156mm 14mm 32mm}, clip]{resources/01/01_2}
\end{frame}
\begin{frame}{ }
    \centering
    \includegraphics[width=\textwidth, trim={14mm 32mm 14mm 156mm}, clip]{resources/01/01_2}
\end{frame}
\begin{frame}{ }
    \centering
    \includegraphics[width=\textwidth, trim={14mm 156mm 14mm 32mm}, clip]{resources/01/01_3}
\end{frame}
\begin{frame}{ }
    \centering
    \includegraphics[width=\textwidth, trim={14mm 32mm 14mm 156mm}, clip]{resources/01/01_3}
\end{frame}
\begin{frame}{ }
    \centering
    \includegraphics[width=\textwidth, trim={14mm 156mm 14mm 32mm}, clip]{resources/01/01_4}
\end{frame}
\begin{frame}{ }
    \centering
    \includegraphics[width=\textwidth, trim={14mm 32mm 14mm 156mm}, clip]{resources/01/01_4}
\end{frame}
\begin{frame}{ }
    \centering
    \includegraphics[width=\textwidth, trim={14mm 156mm 14mm 32mm}, clip]{resources/01/01_5}
\end{frame}
\begin{frame}{ }
    \centering
    \includegraphics[width=\textwidth, trim={14mm 32mm 14mm 156mm}, clip]{resources/01/01_5}
\end{frame}
\begin{frame}{ }
    \centering
    \includegraphics[width=\textwidth, trim={14mm 156mm 14mm 32mm}, clip]{resources/01/01_6}
\end{frame}
\begin{frame}{ }
    \centering
    \includegraphics[width=\textwidth, trim={14mm 32mm 14mm 156mm}, clip]{resources/01/01_6}
\end{frame}
\begin{frame}{ }
    \centering
    \includegraphics[width=\textwidth, trim={14mm 156mm 14mm 32mm}, clip]{resources/01/01_7}
\end{frame}
\begin{frame}{ }
    \centering
    \includegraphics[width=\textwidth, trim={14mm 32mm 14mm 156mm}, clip]{resources/01/01_7}
\end{frame}
\begin{frame}{ }
    \centering
    \includegraphics[width=\textwidth, trim={14mm 156mm 14mm 32mm}, clip]{resources/01/01_8}
\end{frame}
\begin{frame}{ }
    \centering
    \includegraphics[width=\textwidth, trim={14mm 32mm 14mm 156mm}, clip]{resources/01/01_8}
\end{frame}
\begin{frame}{ }
    \centering
    \includegraphics[width=\textwidth, trim={14mm 156mm 14mm 32mm}, clip]{resources/01/01_9}
\end{frame}
\begin{frame}{ }
    \centering
    \includegraphics[width=\textwidth, trim={14mm 32mm 14mm 156mm}, clip]{resources/01/01_9}
\end{frame}
\begin{frame}{ }
    \centering
    \includegraphics[width=\textwidth, trim={14mm 156mm 14mm 32mm}, clip]{resources/01/01_10}
\end{frame}
\begin{frame}{ }
    \centering
    \includegraphics[width=\textwidth, trim={14mm 32mm 14mm 156mm}, clip]{resources/01/01_10}
\end{frame}
\begin{frame}{ }
    \centering
    \includegraphics[width=\textwidth, trim={14mm 156mm 14mm 32mm}, clip]{resources/01/01_11}
\end{frame}
\begin{frame}{ }
    \centering
    \includegraphics[width=\textwidth, trim={14mm 32mm 14mm 156mm}, clip]{resources/01/01_11}
\end{frame}
\begin{frame}{ }
    \centering
    \includegraphics[width=\textwidth, trim={14mm 156mm 14mm 32mm}, clip]{resources/01/01_12}
\end{frame}
\begin{frame}{ }
    \centering
    \includegraphics[width=\textwidth, trim={14mm 32mm 14mm 156mm}, clip]{resources/01/01_12}
\end{frame}
\begin{frame}{ }
    \centering
    \includegraphics[width=\textwidth, trim={14mm 156mm 14mm 32mm}, clip]{resources/01/01_13}
\end{frame}
\begin{frame}{ }
    \centering
    \includegraphics[width=\textwidth, trim={14mm 32mm 14mm 156mm}, clip]{resources/01/01_13}
\end{frame}



\section{Entities and Columns}
\begin{frame}{ }
    \centering
    \includegraphics[width=\textwidth, trim={10mm 134mm 10mm 22mm}, clip]{resources/02/02_2}
\end{frame}
\begin{frame}{ }
    \centering
    \includegraphics[width=\textwidth, trim={10mm 24mm 10mm 134mm}, clip]{resources/02/02_2}
\end{frame}
\begin{frame}{ }
    \centering
    \includegraphics[width=\textwidth, trim={10mm 134mm 10mm 22mm}, clip]{resources/02/02_3}
\end{frame}
\begin{frame}{ }
    \centering
    \includegraphics[width=\textwidth, trim={10mm 24mm 10mm 134mm}, clip]{resources/02/02_3}
\end{frame}
\begin{frame}{ }
    \centering
    \includegraphics[width=\textwidth, trim={10mm 134mm 10mm 22mm}, clip]{resources/02/02_4}
\end{frame}
\begin{frame}{ }
    \centering
    \includegraphics[width=\textwidth, trim={10mm 24mm 10mm 134mm}, clip]{resources/02/02_4}
\end{frame}
\begin{frame}{ }
    \centering
    \includegraphics[width=\textwidth, trim={10mm 134mm 10mm 22mm}, clip]{resources/02/02_5}
\end{frame}
\begin{frame}{ }
    \centering
    \includegraphics[width=\textwidth, trim={10mm 24mm 10mm 134mm}, clip]{resources/02/02_5}
\end{frame}
\begin{frame}{ }
    \centering
    \includegraphics[width=\textwidth, trim={10mm 134mm 10mm 22mm}, clip]{resources/02/02_6}
\end{frame}
\begin{frame}{ }
    \centering
    \includegraphics[width=\textwidth, trim={10mm 24mm 10mm 134mm}, clip]{resources/02/02_6}
\end{frame}
\begin{frame}{ }
    \centering
    \includegraphics[width=\textwidth, trim={10mm 134mm 10mm 22mm}, clip]{resources/02/02_7}
\end{frame}
\begin{frame}{ }
    \centering
    \includegraphics[width=\textwidth, trim={10mm 24mm 10mm 134mm}, clip]{resources/02/02_7}
\end{frame}
\begin{frame}{ }
    \centering
    \includegraphics[width=\textwidth, trim={10mm 134mm 10mm 22mm}, clip]{resources/02/02_8}
\end{frame}

\begin{frame}{ }
    \centering
    \includegraphics[width=\textwidth, trim={10mm 134mm 10mm 22mm}, clip]{resources/02/02_9}
\end{frame}
\begin{frame}{ }
    \centering
    \includegraphics[width=\textwidth, trim={10mm 24mm 10mm 134mm}, clip]{resources/02/02_9}
\end{frame}


\section{The Three Levels of E-R Models}
\begin{frame}{ }
    \centering
    \includegraphics[width=\textwidth, trim={10mm 24mm 10mm 134mm}, clip]{resources/05/05_1}
\end{frame}
\begin{frame}{ }
    \centering
    \includegraphics[width=\textwidth, trim={10mm 134mm 10mm 24mm}, clip]{resources/05/05_2}
\end{frame}
\begin{frame}{ }
    \centering
    \includegraphics[width=\textwidth, trim={10mm 24mm 10mm 134mm}, clip]{resources/05/05_2}
\end{frame}
\begin{frame}{ }
    \centering
    \includegraphics[width=\textwidth, trim={10mm 134mm 10mm 24mm}, clip]{resources/05/05_3}
\end{frame}
\begin{frame}{ }
    \centering
    \includegraphics[width=\textwidth, trim={10mm 24mm 10mm 134mm}, clip]{resources/05/05_3}
\end{frame}
\begin{frame}{ }
    \centering
    \includegraphics[width=\textwidth, trim={10mm 134mm 10mm 24mm}, clip]{resources/05/05_4}
\end{frame}
\begin{frame}{ }
    \centering
    \includegraphics[width=\textwidth, trim={10mm 24mm 10mm 134mm}, clip]{resources/05/05_4}
\end{frame}
\begin{frame}{ }
    \centering
    \includegraphics[width=\textwidth, trim={10mm 134mm 10mm 24mm}, clip]{resources/05/05_5}
\end{frame}
\begin{frame}{ }
    \centering
    \includegraphics[width=\textwidth, trim={10mm 24mm 10mm 134mm}, clip]{resources/05/05_5}
\end{frame}
\begin{frame}{ }
    \centering
    \includegraphics[width=\textwidth, trim={10mm 134mm 10mm 24mm}, clip]{resources/05/05_6}
\end{frame}
\begin{frame}{ }
    \centering
    \includegraphics[width=\textwidth, trim={10mm 24mm 10mm 134mm}, clip]{resources/05/05_6}
\end{frame}


\section{Key and Relationships}
\begin{frame}{ }
    \centering
    \includegraphics[width=\textwidth, trim={10mm 24mm 10mm 134mm}, clip]{resources/06/06_1}
\end{frame}
\begin{frame}{ }
    \centering
    \includegraphics[width=\textwidth, trim={10mm 134mm 10mm 24mm}, clip]{resources/06/06_2}
\end{frame}
\begin{frame}{ }
    \centering
    \includegraphics[width=\textwidth, trim={10mm 24mm 10mm 134mm}, clip]{resources/06/06_2}
\end{frame}
\begin{frame}{ }
    \centering
    \includegraphics[width=\textwidth, trim={10mm 134mm 10mm 24mm}, clip]{resources/06/06_3}
\end{frame}
\begin{frame}{ }
    \centering
    \includegraphics[width=\textwidth, trim={10mm 24mm 10mm 134mm}, clip]{resources/06/06_3}
\end{frame}
\begin{frame}{ }
    \centering
    \includegraphics[width=\textwidth, trim={10mm 134mm 10mm 24mm}, clip]{resources/06/06_4}
\end{frame}
\begin{frame}{ }
    \centering
    \includegraphics[width=\textwidth, trim={10mm 24mm 10mm 134mm}, clip]{resources/06/06_4}
\end{frame}
\begin{frame}{ }
    \centering
    \includegraphics[width=\textwidth, trim={10mm 134mm 10mm 24mm}, clip]{resources/06/06_5}
\end{frame}
\begin{frame}{ }
    \centering
    \includegraphics[width=\textwidth, trim={10mm 24mm 10mm 134mm}, clip]{resources/06/06_5}
\end{frame}
\begin{frame}{ }
    \centering
    \includegraphics[width=\textwidth, trim={10mm 134mm 10mm 24mm}, clip]{resources/06/06_6}
\end{frame}
\begin{frame}{ }
    \centering
    \includegraphics[width=\textwidth, trim={10mm 24mm 10mm 134mm}, clip]{resources/06/06_6}
\end{frame}
\begin{frame}{ }
    \centering
    \includegraphics[width=\textwidth, trim={10mm 134mm 10mm 24mm}, clip]{resources/06/06_7}
\end{frame}
\begin{frame}{ }
    \centering
    \includegraphics[width=\textwidth, trim={10mm 24mm 10mm 134mm}, clip]{resources/06/06_7}
\end{frame}
\begin{frame}{ }
    \centering
    \includegraphics[width=\textwidth, trim={10mm 134mm 10mm 24mm}, clip]{resources/06/06_8}
\end{frame}
\begin{frame}{ }
    \centering
    \includegraphics[width=\textwidth, trim={10mm 24mm 10mm 134mm}, clip]{resources/06/06_8}
\end{frame}
\begin{frame}{ }
    \centering
    \includegraphics[width=\textwidth, trim={10mm 134mm 10mm 24mm}, clip]{resources/06/06_9}
\end{frame}
\begin{frame}{ }
    \centering
    \includegraphics[width=\textwidth, trim={10mm 24mm 10mm 134mm}, clip]{resources/06/06_9}
\end{frame}
\begin{frame}{ }
    \centering
    \includegraphics[width=\textwidth, trim={10mm 134mm 10mm 24mm}, clip]{resources/06/06_10}
\end{frame}
\begin{frame}{ }
    \centering
    \includegraphics[width=\textwidth, trim={10mm 24mm 10mm 134mm}, clip]{resources/06/06_10}
\end{frame}
\begin{frame}{ }
    \centering
    \includegraphics[width=\textwidth, trim={10mm 134mm 10mm 24mm}, clip]{resources/06/06_11}
\end{frame}
\begin{frame}{ }
    \centering
    \includegraphics[width=\textwidth, trim={10mm 24mm 10mm 134mm}, clip]{resources/06/06_11}
\end{frame}
\begin{frame}{ }
    \centering
    \includegraphics[width=\textwidth, trim={10mm 134mm 10mm 24mm}, clip]{resources/06/06_12}
\end{frame}
\begin{frame}{ }
    \centering
    \includegraphics[width=\textwidth, trim={10mm 24mm 10mm 134mm}, clip]{resources/06/06_12}
\end{frame}
\begin{frame}{ }
    \centering
    \includegraphics[width=\textwidth, trim={10mm 134mm 10mm 24mm}, clip]{resources/06/06_13}
\end{frame}
\begin{frame}{ }
    \centering
    \includegraphics[width=\textwidth, trim={10mm 24mm 10mm 134mm}, clip]{resources/06/06_13}
\end{frame}
\begin{frame}{ }
    \centering
    \includegraphics[width=\textwidth, trim={10mm 134mm 10mm 24mm}, clip]{resources/06/06_14}
\end{frame}
\begin{frame}{ }
    \centering
    \includegraphics[width=\textwidth, trim={10mm 134mm 10mm 24mm}, clip]{resources/06/06_18}
\end{frame}
\begin{frame}{ }
    \centering
    \includegraphics[width=\textwidth, trim={10mm 24mm 10mm 134mm}, clip]{resources/06/06_18}
\end{frame}
\begin{frame}{ }
    \centering
    \includegraphics[width=\textwidth, trim={10mm 134mm 10mm 24mm}, clip]{resources/06/06_19}
\end{frame}
\begin{frame}{ }
    \centering
    \includegraphics[width=\textwidth, trim={10mm 24mm 10mm 134mm}, clip]{resources/06/06_19}
\end{frame}
\begin{frame}{ }
    \centering
    \includegraphics[width=\textwidth, trim={10mm 134mm 10mm 24mm}, clip]{resources/06/06_20}
\end{frame}
\begin{frame}{ }
    \centering
    \includegraphics[width=\textwidth, trim={10mm 24mm 10mm 134mm}, clip]{resources/06/06_20}
\end{frame}
\begin{frame}{ }
    \centering
    \includegraphics[width=\textwidth, trim={10mm 134mm 10mm 24mm}, clip]{resources/06/06_21}
\end{frame}
\begin{frame}{ }
    \centering
    \includegraphics[width=\textwidth, trim={10mm 24mm 10mm 134mm}, clip]{resources/06/06_21}
\end{frame}
\begin{frame}{ }
    \centering
    \includegraphics[width=\textwidth, trim={10mm 134mm 10mm 24mm}, clip]{resources/06/06_22}
\end{frame}
\begin{frame}{ }
    \centering
    \includegraphics[width=\textwidth, trim={10mm 24mm 10mm 134mm}, clip]{resources/06/06_22}
\end{frame}
\begin{frame}{ }
    \centering
    \includegraphics[width=\textwidth, trim={10mm 134mm 10mm 24mm}, clip]{resources/06/06_23}
\end{frame}
\begin{frame}{ }
    \centering
    \includegraphics[width=\textwidth, trim={10mm 24mm 10mm 134mm}, clip]{resources/06/06_23}
\end{frame}
\begin{frame}{ }
    \centering
    \includegraphics[width=\textwidth, trim={10mm 134mm 10mm 24mm}, clip]{resources/06/06_24}
\end{frame}
\begin{frame}{ }
    \centering
    \includegraphics[width=\textwidth, trim={10mm 24mm 10mm 134mm}, clip]{resources/06/06_24}
\end{frame}
\begin{frame}{ }
    \centering
    \includegraphics[width=\textwidth, trim={10mm 134mm 10mm 24mm}, clip]{resources/06/06_25}
\end{frame}
\begin{frame}{ }
    \centering
    \includegraphics[width=\textwidth, trim={10mm 24mm 10mm 134mm}, clip]{resources/06/06_25}
\end{frame}
\begin{frame}{ }
    \centering
    \includegraphics[width=\textwidth, trim={10mm 134mm 10mm 24mm}, clip]{resources/06/06_30}
\end{frame}
\begin{frame}{ }
    \centering
    \includegraphics[width=\textwidth, trim={10mm 24mm 10mm 134mm}, clip]{resources/06/06_30}
\end{frame}
\begin{frame}{ }
    \centering
    \includegraphics[width=\textwidth, trim={10mm 134mm 10mm 24mm}, clip]{resources/06/06_35}
\end{frame}
\begin{frame}{ }
    \centering
    \includegraphics[width=\textwidth, trim={10mm 24mm 10mm 134mm}, clip]{resources/06/06_35}
\end{frame}


\section{Database Views}
\begin{frame}{ }
    \centering
    \includegraphics[width=\textwidth, trim={10mm 24mm 10mm 134mm}, clip]{resources/07/07_1}
\end{frame}
\begin{frame}{ }
    \centering
    \includegraphics[width=\textwidth, trim={10mm 134mm 10mm 24mm}, clip]{resources/07/07_2}
\end{frame}
\begin{frame}{ }
    \centering
    \includegraphics[width=\textwidth, trim={10mm 24mm 10mm 134mm}, clip]{resources/07/07_2}
\end{frame}
\begin{frame}{ }
    \centering
    \includegraphics[width=\textwidth, trim={10mm 134mm 10mm 24mm}, clip]{resources/07/07_3}
\end{frame}
\begin{frame}{ }
    \centering
    \includegraphics[width=\textwidth, trim={10mm 24mm 10mm 134mm}, clip]{resources/07/07_3}
\end{frame}
\begin{frame}{ }
    \centering
    \includegraphics[width=\textwidth, trim={10mm 134mm 10mm 24mm}, clip]{resources/07/07_4}
\end{frame}
\begin{frame}{ }
    \centering
    \includegraphics[width=\textwidth, trim={10mm 24mm 10mm 134mm}, clip]{resources/07/07_7}
\end{frame}
\begin{frame}{ }
    \centering
    \includegraphics[width=\textwidth, trim={10mm 134mm 10mm 24mm}, clip]{resources/07/07_8}
\end{frame}
\begin{frame}{ }
    \centering
    \includegraphics[width=\textwidth, trim={10mm 24mm 10mm 134mm}, clip]{resources/07/07_8}
\end{frame}
\begin{frame}{ }
    \centering
    \includegraphics[width=\textwidth, trim={10mm 134mm 10mm 24mm}, clip]{resources/07/07_9}
\end{frame}
\begin{frame}{ }
    \centering
    \includegraphics[width=\textwidth, trim={10mm 24mm 10mm 134mm}, clip]{resources/07/07_9}
\end{frame}
\begin{frame}{ }
    \centering
    \includegraphics[width=\textwidth, trim={10mm 134mm 10mm 24mm}, clip]{resources/07/07_10}
\end{frame}
\begin{frame}{ }
    \centering
    \includegraphics[width=\textwidth, trim={10mm 134mm 10mm 24mm}, clip]{resources/07/07_13}
\end{frame}
\begin{frame}{ }
    \centering
    \includegraphics[width=\textwidth, trim={10mm 24mm 10mm 134mm}, clip]{resources/07/07_13}
\end{frame}
\begin{frame}{ }
    \centering
    \includegraphics[width=\textwidth, trim={10mm 134mm 10mm 24mm}, clip]{resources/07/07_14}
\end{frame}
\begin{frame}{ }
    \centering
    \includegraphics[width=\textwidth, trim={10mm 24mm 10mm 134mm}, clip]{resources/07/07_14}
\end{frame}
\begin{frame}{ }
    \centering
    \includegraphics[width=\textwidth, trim={10mm 134mm 10mm 24mm}, clip]{resources/07/07_15}
\end{frame}


\section{Triggers and Stored Procedures}
\begin{frame}{ }
    \centering
    \includegraphics[width=\textwidth, trim={10mm 24mm 10mm 134mm}, clip]{resources/08/08_1}
\end{frame}
\begin{frame}{ }
    \centering
    \includegraphics[width=\textwidth, trim={10mm 134mm 10mm 24mm}, clip]{resources/08/08_2}
\end{frame}
\begin{frame}{ }
    \centering
    \includegraphics[width=\textwidth, trim={10mm 24mm 10mm 134mm}, clip]{resources/08/08_2}
\end{frame}
\begin{frame}{ }
    \centering
    \includegraphics[width=\textwidth, trim={10mm 134mm 10mm 24mm}, clip]{resources/08/08_3}
\end{frame}
\begin{frame}{ }
    \centering
    \includegraphics[width=\textwidth, trim={10mm 24mm 10mm 134mm}, clip]{resources/08/08_3}
\end{frame}
\begin{frame}{ }
    \centering
    \includegraphics[width=\textwidth, trim={10mm 24mm 10mm 134mm}, clip]{resources/08/08_7}
\end{frame}
\begin{frame}{ }
    \centering
    \includegraphics[width=\textwidth, trim={10mm 134mm 10mm 24mm}, clip]{resources/08/08_8}
\end{frame}
\begin{frame}{ }
    \centering
    \includegraphics[width=\textwidth, trim={10mm 24mm 10mm 134mm}, clip]{resources/08/08_8}
\end{frame}
\begin{frame}{ }
    \centering
    \includegraphics[width=\textwidth, trim={10mm 134mm 10mm 24mm}, clip]{resources/08/08_9}
\end{frame}
\begin{frame}{ }
    \centering
    \includegraphics[width=\textwidth, trim={10mm 24mm 10mm 134mm}, clip]{resources/08/08_9}
\end{frame}
\begin{frame}{ }
    \centering
    \includegraphics[width=\textwidth, trim={10mm 134mm 10mm 24mm}, clip]{resources/08/08_10}
\end{frame}
\begin{frame}{ }
    \centering
    \includegraphics[width=\textwidth, trim={10mm 24mm 10mm 134mm}, clip]{resources/08/08_10}
\end{frame}
\begin{frame}{ }
    \centering
    \includegraphics[width=\textwidth, trim={10mm 134mm 10mm 24mm}, clip]{resources/08/08_14}
\end{frame}


\begin{frame}{}
    \centering
    \Huge End of Lecture 4.
\end{frame}

\section*{Takeaways}

% Tim Duncan's Top 5 Fundamental Takeaways of the Today's Class
\begin{frame}{TDT5FTOTTC}
    \centering
    \includegraphics[width=0.55\textwidth]{figures/tim.png}
\end{frame}

\begin{frame}{Top 5 Fundamental Takeaways}
    \small
    \begin{enumerate} \pause
        \item[5] \textbf{Database Views, Triggers, and Stored Procedures} enhance database security, automation, and efficiency by providing pre-queried data, event-based actions, and reusable SQL routines. \pause

        \item[4] \textbf{Keys and Relationships} define how data is uniquely identified (Primary Keys), linked across tables (Foreign Keys), and structured using Composite Keys for complex relationships. \pause

        \item[3] \textbf{Normalization} and Database Optimization help minimize data redundancy and balance between normalized (integrity-focused) and denormalized (performance-oriented) structures. \pause

        \item[2] \textbf{Three Levels of ER Modeling} include the conceptual (high-level overview), logical (detailed structure with normalization), and physical (implementation-specific design). \pause

        \item[1] \textbf{ERDs (Entity-Relationship Diagrams)} provide a visual representation of database structures and relationships to improve data organization and communication.
        \end{enumerate}
\end{frame}

\begin{frame}{Database Administration: Reverse Design.}
    \centering
    \includegraphics[width=0.8\textwidth]{figures/udemy2}\\
    \includegraphics[width=0.8\textwidth]{figures/udemy3}\\
    \vspace{2mm}
    {
        \scriptsize
        Content has been extracted from \textit{Database Design and Management.} Udemy Course, created by Visual Paradigm, 2025.  Visit \url{https://www.udemy.com/course/database-design-and-management/} and \url{https://www.visual-paradigm.com/}.\\
    }
\end{frame}

\end{document}
