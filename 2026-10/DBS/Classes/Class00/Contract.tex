\documentclass[11pt]{article}

% --- Packages ---
\usepackage[spanish]{babel}
\usepackage[utf8]{inputenc} % Ensures accents like á, é, ñ work
\usepackage[T1]{fontenc}    % Proper font encoding for European languages
\usepackage[margin=1in]{geometry}
\usepackage{enumitem}
\usepackage{setspace}
\usepackage{titlesec}
\usepackage{tikz}
\usepackage{hyperref}
\hypersetup{
    colorlinks=true,
    linkcolor=blue,
    filecolor=blue,
    urlcolor=blue,
    pdftitle={Overleaf Example},
    pdfpagemode=FullScreen,
}

% --- Formatting ---
\setstretch{1.2} % Better readability
\titleformat{\section}{\normalfont\large\bfseries}{}{0pt}{}
\setlength{\parindent}{0pt}
\setlength{\parskip}{1em}

\begin{document}

% --- Header ---
{\centering
    \LARGE \textbf{KNOWLEDGE TRANSFER AGREEMENT} \\
    \vspace{0.5em}
    \small This agreement is effective as of \textbf{\today}
\par}

\vspace{2em}

\section{Administración de Bases de Datos.}\label{contrato---administraciuxf3n-de-bases-de-datos.}

\subsection{Horario de Clases:}\label{horario-de-clases}
\begin{itemize}
    \item Las clases están programadas los días Lunes de 02:00pm a 5:00pm, cada semana durante el primer semestre académico del 2026.
    \item El inicio de la clase será 15 minutos después de la hora programada con el fin de facilitar el desplazamiento y llegada de todos los estudiantes al aula de clase. Se distribuirán 30 minutos de descanso a lo largo de la clase si hubiese lugar.
    \item En los días asignados para parciales, el inicio y finalización de la clase será en la hora en punto programada (\textbf{2:00PM}) con el fin de que los estudiantes tengan el tiempo reglamentario para el desarrollo de sus actividades evaluativas.
\end{itemize}

\subsection{Asistencia a clase:}\label{asistencia-a-clase}
\begin{itemize}
    \item La clase se desarrolla bajo el principio de responsabilidad del estudiante universitario, por lo tanto no se realiza un control de asistencia a las clases, sin embargo es responsabilidad del estudiante actualizarse de los temas vistos en clase a causa de su insistencia.
    \item La inasistencia a las actividades evaluativas de quices o talleres será responsabilidad del estudiante asumiendo la calificación mínima posible de acuerdo con la escala de calificación de la universidad. No se realizará ningún tipo de actividad de recuperación de quices sin una justificación válida. La evaluación de los talleres puede ser más flexible teniendo en cuenta que se evaluará de acuerdo al material entregado dentro de las franjas estableciadas para dicho fin.
    \item La inasistencia a las actividades evaluativas de parciales será responsabilidad del estudiante asumiendo la calificación mínima posible de acuerdo con la escala de calificación de la universidad, salvo los casos debidamente justificados ante la Dirección de Carrera que permita y avale la realización de un supletorio.
\end{itemize}

\subsection{Plataformas utilizadas para el desarrollo del curso:}\label{plataformas-utilizadas-para-el-desarrollo-del-curso}
\begin{itemize}
    \item Los anuncios y comunicados del curso serán publicados a través del email masivo por medio de BrightSpace (BS).
    \item Los archivos entregables (informes, trabajos, tareas, proyectos, etc) también serán gestionados a través de BS.
    \item El material de clase como dispositivas y lecturas serán gestionados a través GitHub (GH) \href{dd}{https://github.com/aocalderon/PUJ}.
\end{itemize}

\subsection{Metodología:}\label{metodologuxeda}
\begin{itemize}
    \item Las clases serán realizadas en modalidad \textbf{presencial} de acuerdo con los lineamientos de la universidad.
    \item Para el desarrollo del \textbf{Proyecto Final}, los estudiantes estarán organizados en grupos de trabajo de máximo 4 integrantes.
    \item Todos los archivos calificables \textbf{grupales} deben ser oportunamente cargados en la plataforma de BS por alguno de los integrantes del grupo haciendo uso de los links destinados para tal fin. \textbf{Nota:} La plataforma solo mostrará el ultimo archivo cargado, tengan cuidado con las versiones que suben.
    \item La calificación de talleres y del proyecto \textbf{es de manera grupal}.
    \item La calificación de quices y parciales \textbf{es de manera individual}.
\end{itemize}

\subsection{Archivos:}\label{archivos}
\begin{itemize}
    \item Las diapositivas presentadas en clase serán cargadas después de finalizar la temática a GH para que el estudiante pueda revisar su contenido posteriormente.
    \item Los archivos de códigos de ejemplo o desarrollo de ejercicios serán cargados a GH para que el estudiante pueda revisar su contenido posteriormente.
    \item El orden de los archivos publicados en GH, estará asociado a la semana académica en el cual se desarrolla su contenido.
    \item Todos los archivos entregables de texto que los estudiantes carguen a la plataforma (BS) para ser calificados deben estar en formato \textbf{PDF}.
    \item Los archivos especiales como ejercicios de código y similares deberán ser cargados en el formato que indique el profesor.
    \item Todos los archivos deben ser cargados a la plataforma de BS en los \textbf{horarios habilitados} para las entregas. Entregas posteriores \textbf{no serán} tenidas en cuenta y serán asumidas por el estudiante.
\end{itemize}

\subsection{Desarrollo y entrega de talleres:}\label{desarrollo-y-entrega-de-talleres}
\begin{itemize}
\item El desarrollo de los talleres se realizará durante la sesión de clase  (generalmente).
\item Durante las sesiones de clase los estudiantes, de manera grupal, deberán descargar el archivo del taller y desarrollar los ejercicios propuestos.
\item El link de carga para desarrollo estará en BS.
\end{itemize}

\subsection{Desarrollo y entrega de Evaluaciones:}\label{desarrollo-y-entrega-de-evaluaciones}
\begin{itemize}
\item Los quizzes y parciales se desarrollarán durante las jornadas de clase   establecidas para tal fin.
\item \textbf{El uso de celular durante las evaluaciones está restringido}.
\end{itemize}

\subsection{Calificación:}\label{calificacion}
\begin{itemize}
    \item Talleres, Quizzes y Tareas  $\rightarrow$ 30\%
    \item Proyecto Midterm  $\rightarrow$ 20\%
    \item Proyecto Final  $\rightarrow$ 20\%
    \item Parciales Presenciales  $\rightarrow$ 30\% de la siguiente manera:
    \begin{itemize}
        \item Parcial teórico-practico 1  $\rightarrow$ 15\%
        \item Parcial teórico-practico 2  $\rightarrow$ 15\%
    \end{itemize}
\end{itemize}

\subsection{Contacto y horarios de atención fuera de clase:}\label{contacto-y-horarios-de-atenciuxf3n-fuera-de-clase}
\begin{itemize}
\item La comunicación Estudiantes-Profesor se hará por medio de correo electrónico: \\ andrescalderonr@javeriana.edu.co (Andrés Oswaldo Calderón Romero) o por el chat de Teams.
\item Cualquier correo al profesor debe incluir al inicio del asunto la siguiente combinación de caracteres: \textbf{{[}DBA{]}}.
\item Todo correo o mensaje de Teams será contestado en horario laboral, es decir, de 8:00am a 6:00pm de Lunes a Viernes.
\item El horario de atención a estudiantes en oficina (Office Hours) será los días Lunes de 5:00 a 6:00pm durante el trascurso del semestre académico. La atención se realizará por orden de llegada.
\end{itemize}

\vspace{4em}

% --- Signatures ---
\begin{minipage}[t]{0.45\textwidth}

    
\definecolor{signature_color}{RGB}{50,50,50}

\def \globalscale {0.5}
\begin{tikzpicture}[y=1cm, x=1cm, yscale=\globalscale,xscale=\globalscale, every node/.append style={scale=\globalscale}, inner sep=0pt, outer sep=0pt]
  \path[fill=signature_color,shift={(14.7373, -0.0265)}] (0.0, 6.006).. controls (0.082, 5.9481) and (0.1006, 5.9158) .. (0.1323, 5.8208).. controls (0.1706, 5.0587) and (-0.4639, 4.1775) .. (-0.9475, 3.638).. controls (-0.9666, 3.6176) and (-0.9856, 3.5972) .. (-1.0052, 3.5761).. controls (-1.055, 3.5225) and (-1.1022, 3.4674) .. (-1.1493, 3.4115).. controls (-1.4965, 3.0157) and (-1.9616, 2.7296) .. (-2.4408, 2.5202).. controls (-2.4675, 2.5085) and (-2.4675, 2.5085) .. (-2.4948, 2.4965).. controls (-2.5722, 2.4633) and (-2.648, 2.4335) .. (-2.7286, 2.4084).. controls (-2.797, 2.387) and (-2.8504, 2.3673) .. (-2.9104, 2.3283).. controls (-2.9387, 2.2756) and (-2.9387, 2.2756) .. (-2.9529, 2.2124).. controls (-2.9592, 2.1886) and (-2.9654, 2.1648) .. (-2.9719, 2.1403).. controls (-3.0027, 2.0088) and (-3.0311, 1.8773) .. (-3.0559, 1.7446).. controls (-3.0645, 1.6992) and (-3.0645, 1.6992) .. (-3.0732, 1.6528).. controls (-3.1251, 1.3605) and (-3.1528, 1.0673) .. (-3.1753, 0.7714).. controls (-3.2125, 0.2889) and (-3.2125, 0.2889) .. (-3.3073, 0.1323).. controls (-3.4057, 0.0755) and (-3.4634, 0.0484) .. (-3.5768, 0.0661).. controls (-3.6648, 0.1131) and (-3.705, 0.1532) .. (-3.7571, 0.2381).. controls (-3.9026, 0.8668) and (-3.711, 1.5819) .. (-3.5719, 2.196).. controls (-3.5881, 2.1962) and (-3.6043, 2.1963) .. (-3.621, 2.1964).. controls (-4.1421, 2.2025) and (-4.625, 2.2582) .. (-5.0238, 2.6326).. controls (-5.0425, 2.6527) and (-5.0425, 2.6527) .. (-5.0616, 2.6731).. controls (-5.1355, 2.7153) and (-5.1847, 2.6941) .. (-5.2652, 2.6723).. controls (-5.3369, 2.6419) and (-5.4058, 2.6067) .. (-5.4752, 2.5714).. controls (-5.6836, 2.4697) and (-5.8841, 2.4026) .. (-6.1119, 2.3548).. controls (-6.1537, 2.3459) and (-6.1537, 2.3459) .. (-6.1963, 2.3368).. controls (-6.6341, 2.2568) and (-7.1662, 2.2544) .. (-7.5655, 2.4746).. controls (-7.6403, 2.528) and (-7.6957, 2.5897) .. (-7.7507, 2.6629).. controls (-7.76, 2.6747) and (-7.7692, 2.6866) .. (-7.7788, 2.6988).. controls (-7.7875, 2.6988) and (-7.7962, 2.6988) .. (-7.8052, 2.6988).. controls (-7.8445, 2.463) and (-7.8445, 2.463) .. (-7.8846, 2.2225).. controls (-7.8497, 2.2312) and (-7.8147, 2.24) .. (-7.7788, 2.249).. controls (-7.7066, 2.2548) and (-7.6399, 2.2525) .. (-7.5671, 2.249).. controls (-7.5142, 2.196) and (-7.5142, 2.196) .. (-7.5042, 2.1183).. controls (-7.5142, 2.0373) and (-7.5142, 2.0373) .. (-7.5591, 1.9944).. controls (-7.6267, 1.9539) and (-7.6798, 1.9407) .. (-7.7573, 1.9282).. controls (-7.8421, 1.9197) and (-7.8421, 1.9197) .. (-7.911, 1.8785).. controls (-7.9174, 1.8053) and (-7.922, 1.7327) .. (-7.9254, 1.6593).. controls (-7.9265, 1.6369) and (-7.9276, 1.6144) .. (-7.9288, 1.5913).. controls (-7.9363, 1.4291) and (-7.9401, 1.2671) .. (-7.9422, 1.1047).. controls (-7.9425, 1.0825) and (-7.9428, 1.0603) .. (-7.9431, 1.0374).. controls (-7.9444, 0.9467) and (-7.9457, 0.8559) .. (-7.9461, 0.7652).. controls (-7.9465, 0.6976) and (-7.9476, 0.63) .. (-7.9488, 0.5623).. controls (-7.9487, 0.5419) and (-7.9487, 0.5214) .. (-7.9486, 0.5004).. controls (-7.9507, 0.4079) and (-7.9573, 0.3533) .. (-8.0118, 0.2772).. controls (-8.0698, 0.2381) and (-8.0698, 0.2381) .. (-8.1657, 0.2414).. controls (-8.255, 0.2646) and (-8.255, 0.2646) .. (-8.3079, 0.3175).. controls (-8.4508, 0.7956) and (-8.351, 1.3664) .. (-8.3079, 1.8521).. controls (-8.3244, 1.8539) and (-8.3409, 1.8558) .. (-8.3579, 1.8577).. controls (-9.225, 1.9571) and (-9.225, 1.9571) .. (-9.4985, 2.196).. controls (-9.516, 2.2048) and (-9.5335, 2.2135) .. (-9.5515, 2.2225).. controls (-9.6562, 2.133) and (-9.7592, 2.043) .. (-9.8557, 1.9447).. controls (-9.9357, 1.8632) and (-10.0202, 1.7889) .. (-10.1068, 1.7145).. controls (-10.1592, 1.6676) and (-10.2074, 1.6185) .. (-10.2559, 1.5677).. controls (-10.4045, 1.417) and (-10.5736, 1.2916) .. (-10.7421, 1.1642).. controls (-10.7775, 1.1369) and (-10.7775, 1.1369) .. (-10.8136, 1.1091).. controls (-11.025, 0.9526) and (-11.262, 0.8186) .. (-11.5325, 0.8359).. controls (-11.6238, 0.8534) and (-11.6627, 0.88) .. (-11.721, 0.9525).. controls (-11.8364, 1.1914) and (-11.8318, 1.4603) .. (-11.8269, 1.7198).. controls (-11.8443, 1.7198) and (-11.8618, 1.7198) .. (-11.8798, 1.7198).. controls (-11.8861, 1.705) and (-11.8925, 1.6903) .. (-11.899, 1.675).. controls (-11.94, 1.6007) and (-11.9907, 1.5456) .. (-12.0485, 1.4833).. controls (-12.1092, 1.4176) and (-12.1682, 1.3528) .. (-12.223, 1.2821).. controls (-12.5237, 0.9177) and (-13.0316, 0.6397) .. (-13.4987, 0.587).. controls (-13.8183, 0.5673) and (-14.0689, 0.6706) .. (-14.318, 0.8645).. controls (-14.5301, 1.0547) and (-14.681, 1.3565) .. (-14.7108, 1.6404).. controls (-14.713, 1.6577) and (-14.7151, 1.6751) .. (-14.7173, 1.6929).. controls (-14.7484, 2.1942) and (-14.4572, 2.6094) .. (-14.1436, 2.9666).. controls (-14.0582, 3.0602) and (-13.9703, 3.1454) .. (-13.8735, 3.227).. controls (-13.8071, 3.2845) and (-13.7432, 3.3441) .. (-13.6792, 3.4042).. controls (-13.5083, 3.5648) and (-13.3334, 3.7169) .. (-13.1498, 3.8629).. controls (-13.1196, 3.8876) and (-13.0895, 3.9122) .. (-13.0594, 3.9369).. controls (-12.832, 4.1224) and (-12.5969, 4.2937) .. (-12.3559, 4.461).. controls (-12.3078, 4.4947) and (-12.2602, 4.5289) .. (-12.2128, 4.5636).. controls (-11.8602, 4.8219) and (-11.4942, 5.0729) .. (-11.0993, 5.2619).. controls (-11.0814, 5.2706) and (-11.0635, 5.2792) .. (-11.0451, 5.2882).. controls (-10.9405, 5.3357) and (-10.8568, 5.356) .. (-10.7421, 5.3446).. controls (-10.7156, 5.3181) and (-10.7156, 5.3181) .. (-10.704, 5.2569).. controls (-10.7156, 5.1858) and (-10.7156, 5.1858) .. (-10.7752, 5.1346).. controls (-10.8544, 5.0878) and (-10.9377, 5.0496) .. (-11.0206, 5.0098).. controls (-11.1879, 4.9263) and (-11.339, 4.8276) .. (-11.4912, 4.7195).. controls (-11.5307, 4.6918) and (-11.5307, 4.6918) .. (-11.5709, 4.6634).. controls (-11.7724, 4.5206) and (-11.967, 4.371) .. (-12.1571, 4.2135).. controls (-12.215, 4.1658) and (-12.274, 4.1194) .. (-12.3329, 4.0729).. controls (-12.5624, 3.8892) and (-12.7861, 3.6984) .. (-13.0034, 3.5004).. controls (-13.0451, 3.4626) and (-13.0869, 3.4249) .. (-13.129, 3.3876).. controls (-13.2991, 3.2365) and (-13.4631, 3.08) .. (-13.6245, 2.9198).. controls (-13.653, 2.8916) and (-13.6817, 2.8635) .. (-13.7104, 2.8355).. controls (-14.0355, 2.5192) and (-14.3343, 2.1822) .. (-14.357, 1.7082).. controls (-14.3504, 1.4848) and (-14.2824, 1.309) .. (-14.1373, 1.1385).. controls (-13.9501, 0.9629) and (-13.743, 0.934) .. (-13.4939, 0.9402).. controls (-13.0598, 0.9783) and (-12.6918, 1.2936) .. (-12.4188, 1.6111).. controls (-12.2087, 1.8695) and (-12.03, 2.1372) .. (-11.8798, 2.4342).. controls (-11.8559, 2.4796) and (-11.8559, 2.4796) .. (-11.8315, 2.5259).. controls (-11.6984, 2.8012) and (-11.6283, 3.0927) .. (-11.558, 3.3889).. controls (-11.4573, 3.8091) and (-11.4573, 3.8091) .. (-11.3242, 3.9423).. controls (-11.2597, 3.9522) and (-11.2597, 3.9522) .. (-11.1919, 3.9423).. controls (-11.1343, 3.8843) and (-11.1137, 3.8506) .. (-11.1068, 3.7686).. controls (-11.1398, 3.4122) and (-11.2152, 3.0695) .. (-11.3275, 2.7302).. controls (-11.3946, 2.527) and (-11.437, 2.3305) .. (-11.4565, 2.1167).. controls (-11.4587, 2.0939) and (-11.461, 2.071) .. (-11.4634, 2.0475).. controls (-11.4821, 1.8474) and (-11.4861, 1.6495) .. (-11.4846, 1.4486).. controls (-11.4845, 1.4185) and (-11.4843, 1.3884) .. (-11.4842, 1.3573).. controls (-11.4839, 1.2841) and (-11.4835, 1.2109) .. (-11.4829, 1.1377).. controls (-10.9971, 1.2885) and (-10.6514, 1.6475) .. (-10.3022, 2.0026).. controls (-9.8277, 2.4843) and (-9.8277, 2.4843) .. (-9.525, 2.5929).. controls (-9.4144, 2.5517) and (-9.3532, 2.4771) .. (-9.2869, 2.3813).. controls (-9.2869, 2.3638) and (-9.2869, 2.3463) .. (-9.2869, 2.3283).. controls (-9.0749, 2.2464) and (-8.8611, 2.2184) .. (-8.637, 2.1911).. controls (-8.5998, 2.1864) and (-8.5627, 2.1816) .. (-8.5255, 2.1769).. controls (-8.4354, 2.1654) and (-8.3452, 2.1542) .. (-8.255, 2.1431).. controls (-8.2519, 2.1665) and (-8.2488, 2.1899) .. (-8.2456, 2.214).. controls (-8.2185, 2.4128) and (-8.1864, 2.6079) .. (-8.1392, 2.8029).. controls (-8.1294, 2.8438) and (-8.1294, 2.8438) .. (-8.1193, 2.8854).. controls (-7.7382, 4.4095) and (-7.7382, 4.4095) .. (-7.4083, 5.0271).. controls (-7.3643, 5.0282) and (-7.3201, 5.0282) .. (-7.276, 5.0271).. controls (-7.2496, 5.0006) and (-7.2496, 5.0006) .. (-7.2439, 4.9193).. controls (-7.25, 4.807) and (-7.2772, 4.7137) .. (-7.3141, 4.6087).. controls (-7.3914, 4.3766) and (-7.4568, 4.1418) .. (-7.5196, 3.9054).. controls (-7.5302, 3.8652) and (-7.5409, 3.8251) .. (-7.5517, 3.7849).. controls (-7.5671, 3.7276) and (-7.5823, 3.6703) .. (-7.5974, 3.6129).. controls (-7.6062, 3.5795) and (-7.615, 3.5461) .. (-7.6241, 3.5118).. controls (-7.6476, 3.4079) and (-7.6614, 3.3072) .. (-7.6729, 3.2015).. controls (-7.6587, 3.2143) and (-7.6445, 3.2271) .. (-7.6299, 3.2403).. controls (-7.1587, 3.6434) and (-6.5184, 3.8194) .. (-5.9304, 3.9734).. controls (-5.8779, 3.9872) and (-5.8255, 4.0011) .. (-5.7732, 4.0153).. controls (-5.3977, 4.117) and (-5.017, 4.188) .. (-4.6341, 4.2551).. controls (-4.5793, 4.2647) and (-4.5246, 4.2744) .. (-4.4699, 4.2843).. controls (-3.6287, 4.4344) and (-3.6287, 4.4344) .. (-3.4158, 4.3269).. controls (-3.3602, 4.2863) and (-3.3602, 4.2863) .. (-3.3338, 4.2598).. controls (-3.3327, 4.2069) and (-3.3326, 4.1539) .. (-3.3338, 4.101).. controls (-3.4169, 4.0742) and (-3.4658, 4.0765) .. (-3.549, 4.101).. controls (-3.6853, 4.1339) and (-3.8, 4.1228) .. (-3.9373, 4.101).. controls (-3.9623, 4.0973) and (-3.9872, 4.0936) .. (-4.0129, 4.0898).. controls (-5.885, 3.8035) and (-5.885, 3.8035) .. (-7.5142, 2.9104).. controls (-7.5154, 2.8475) and (-7.5154, 2.8475) .. (-7.4877, 2.7781).. controls (-7.2711, 2.6018) and (-6.9361, 2.6032) .. (-6.6675, 2.6194).. controls (-6.6421, 2.6206) and (-6.6421, 2.6206) .. (-6.6162, 2.6218).. controls (-6.192, 2.6443) and (-5.5291, 2.7523) .. (-5.2123, 3.0692).. controls (-5.1412, 3.084) and (-5.1412, 3.084) .. (-5.0535, 3.0692).. controls (-5.0198, 3.0294) and (-5.0198, 3.0294) .. (-4.9857, 2.9766).. controls (-4.8236, 2.7363) and (-4.5268, 2.6153) .. (-4.2525, 2.5527).. controls (-4.123, 2.5299) and (-3.9924, 2.5351) .. (-3.8613, 2.535).. controls (-3.8325, 2.5347) and (-3.8037, 2.5344) .. (-3.774, 2.5341).. controls (-3.7463, 2.534) and (-3.7186, 2.534) .. (-3.69, 2.5339).. controls (-3.6648, 2.5338) and (-3.6396, 2.5337) .. (-3.6136, 2.5336).. controls (-3.5414, 2.5404) and (-3.4993, 2.552) .. (-3.4396, 2.5929).. controls (-3.4119, 2.6618) and (-3.3903, 2.7269) .. (-3.3701, 2.798).. controls (-3.1462, 3.5235) and (-2.7627, 4.1773) .. (-2.2348, 4.7222).. controls (-2.1938, 4.7648) and (-2.1535, 4.8081) .. (-2.1132, 4.8514).. controls (-1.6975, 5.2898) and (-1.2012, 5.5999) .. (-0.6598, 5.8589).. controls (-0.6324, 5.872) and (-0.6324, 5.872) .. (-0.6044, 5.8854).. controls (-0.4094, 5.976) and (-0.2169, 6.0374) .. (0.0, 6.006) -- cycle(-0.5143, 5.543).. controls (-0.5333, 5.5334) and (-0.5522, 5.5238) .. (-0.5718, 5.5139).. controls (-1.0837, 5.2497) and (-1.5996, 4.8464) .. (-1.9579, 4.3921).. controls (-1.9695, 4.3775) and (-1.981, 4.3629) .. (-1.9929, 4.3479).. controls (-2.3078, 3.9475) and (-2.8046, 3.2915) .. (-2.8046, 2.7517).. controls (-2.6492, 2.7645) and (-2.5384, 2.8106) .. (-2.4027, 2.8856).. controls (-2.383, 2.8964) and (-2.3632, 2.9072) .. (-2.3429, 2.9182).. controls (-1.8803, 3.1737) and (-1.4857, 3.4961) .. (-1.1642, 3.9158).. controls (-1.1475, 3.9376) and (-1.1475, 3.9376) .. (-1.1304, 3.9597).. controls (-0.7589, 4.4487) and (-0.3954, 5.0391) .. (-0.2381, 5.6356).. controls (-0.3448, 5.6356) and (-0.4213, 5.5902) .. (-0.5143, 5.543) -- cycle;



  \path[fill=signature_color,shift={(8.2285, -2.3019)}] (0.0, 6.006).. controls (0.0489, 5.9326) and (0.0794, 5.8833) .. (0.0794, 5.7944).. controls (0.0619, 5.7944) and (0.0445, 5.7944) .. (0.0265, 5.7944).. controls (0.0177, 5.7682) and (0.009, 5.742) .. (0.0, 5.715).. controls (-0.0437, 5.6895) and (-0.0437, 5.6895) .. (-0.1005, 5.6675).. controls (-0.1213, 5.6591) and (-0.1422, 5.6508) .. (-0.1636, 5.6422).. controls (-0.186, 5.6335) and (-0.2084, 5.6248) .. (-0.2315, 5.6158).. controls (-0.6093, 5.4673) and (-0.6093, 5.4673) .. (-0.9409, 5.2388).. controls (-1.0054, 5.1858) and (-1.0054, 5.1858) .. (-1.0683, 5.1908).. controls (-1.0824, 5.1979) and (-1.0966, 5.205) .. (-1.1113, 5.2123).. controls (-1.1113, 5.1948) and (-1.1113, 5.1774) .. (-1.1113, 5.1594).. controls (-1.0938, 5.1594) and (-1.0763, 5.1594) .. (-1.0583, 5.1594).. controls (-1.0671, 5.1332) and (-1.0758, 5.107) .. (-1.0848, 5.08).. controls (-1.1085, 5.0874) and (-1.1085, 5.0874) .. (-1.1327, 5.0949).. controls (-1.195, 5.1123) and (-1.195, 5.1123) .. (-1.27, 5.08).. controls (-1.2764, 5.1881) and (-1.2547, 5.2699) .. (-1.2171, 5.371).. controls (-1.1996, 5.371) and (-1.1822, 5.371) .. (-1.1642, 5.371).. controls (-1.1593, 5.4046) and (-1.1593, 5.4046) .. (-1.1542, 5.4388).. controls (-1.0338, 5.6936) and (-0.6667, 5.8101) .. (-0.4233, 5.9267).. controls (-0.4059, 5.9354) and (-0.3884, 5.9441) .. (-0.3704, 5.9531).. controls (-0.3096, 5.9613) and (-0.2488, 5.9666) .. (-0.1877, 5.9721).. controls (-0.1151, 5.9819) and (-0.0764, 6.006) .. (0.0, 6.006) -- cycle;



  \path[fill=signature_color,shift={(11.0473, -1.6561)}] (0.0, 6.006).. controls (0.0181, 6.0064) and (0.0363, 6.0067) .. (0.055, 6.007).. controls (0.1782, 6.0053) and (0.2679, 5.9778) .. (0.3563, 5.8895).. controls (0.3573, 5.8366) and (0.3574, 5.7836) .. (0.3563, 5.7307).. controls (0.2733, 5.7039) and (0.2248, 5.7062) .. (0.1419, 5.7309).. controls (0.0008, 5.7651) and (-0.1202, 5.7502) .. (-0.2622, 5.7274).. controls (-0.2998, 5.7219) and (-0.2998, 5.7219) .. (-0.3381, 5.7162).. controls (-0.5316, 5.6866) and (-0.7231, 5.6489) .. (-0.914, 5.6056).. controls (-1.0053, 5.5849) and (-1.0841, 5.572) .. (-1.1783, 5.572).. controls (-1.1783, 5.5894) and (-1.1783, 5.6069) .. (-1.1783, 5.6249).. controls (-1.1609, 5.6249) and (-1.1434, 5.6249) .. (-1.1254, 5.6249).. controls (-1.1254, 5.6773) and (-1.1254, 5.7297) .. (-1.1254, 5.7836).. controls (-1.1865, 5.7749) and (-1.2476, 5.7662) .. (-1.3106, 5.7572).. controls (-1.3193, 5.7746) and (-1.3281, 5.7921) .. (-1.3371, 5.8101).. controls (-1.1909, 5.837) and (-1.0447, 5.8633) .. (-0.8983, 5.8892).. controls (-0.8487, 5.8981) and (-0.7992, 5.907) .. (-0.7497, 5.9162).. controls (-0.5003, 5.962) and (-0.2541, 6.0036) .. (0.0, 6.006) -- cycle;



  \path[fill=signature_color,shift={(0.1588, -3.6777)}] (0.0, 6.006).. controls (0.0087, 6.006) and (0.0175, 6.006) .. (0.0265, 6.006).. controls (0.0352, 5.9449) and (0.0439, 5.8838) .. (0.0529, 5.8208).. controls (0.0005, 5.7946) and (0.0005, 5.7946) .. (-0.0529, 5.7679).. controls (-0.0355, 5.7505) and (-0.018, 5.733) .. (0.0, 5.715).. controls (0.0, 5.6801) and (0.0, 5.6452) .. (0.0, 5.6092).. controls (-0.0175, 5.6092) and (-0.0349, 5.6092) .. (-0.0529, 5.6092).. controls (-0.0442, 5.5742) and (-0.0355, 5.5393) .. (-0.0265, 5.5033).. controls (-0.0439, 5.4946) and (-0.0614, 5.4859) .. (-0.0794, 5.4769).. controls (-0.0968, 5.3967) and (-0.0968, 5.3967) .. (-0.1058, 5.3181).. controls (-0.0884, 5.3181) and (-0.0709, 5.3181) .. (-0.0529, 5.3181).. controls (-0.0355, 5.3705) and (-0.018, 5.4229) .. (0.0, 5.4769).. controls (0.0262, 5.4681) and (0.0524, 5.4594) .. (0.0794, 5.4504).. controls (0.0976, 5.3958) and (0.0976, 5.3958) .. (0.1058, 5.3446).. controls (0.1662, 5.4351) and (0.1646, 5.4685) .. (0.1621, 5.5744).. controls (0.1613, 5.6139) and (0.1613, 5.6139) .. (0.1606, 5.6541).. controls (0.16, 5.6742) and (0.1594, 5.6943) .. (0.1588, 5.715).. controls (0.0933, 5.6757) and (0.0933, 5.6757) .. (0.0265, 5.6356).. controls (0.0376, 5.6793) and (0.0493, 5.7228) .. (0.0612, 5.7663).. controls (0.0676, 5.7905) and (0.0741, 5.8148) .. (0.0807, 5.8397).. controls (0.089, 5.8597) and (0.0973, 5.8797) .. (0.1058, 5.9002).. controls (0.132, 5.9089) and (0.1582, 5.9177) .. (0.1852, 5.9267).. controls (0.215, 5.8274) and (0.215, 5.8274) .. (0.1852, 5.7679).. controls (0.2201, 5.7766) and (0.2551, 5.7854) .. (0.291, 5.7944).. controls (0.2823, 5.7633) and (0.2736, 5.7322) .. (0.2646, 5.7001).. controls (0.2044, 5.4187) and (0.2046, 5.1623) .. (0.3501, 4.9088).. controls (0.4463, 4.7712) and (0.5616, 4.6709) .. (0.7233, 4.6158).. controls (0.8941, 4.5865) and (1.071, 4.5985) .. (1.2435, 4.6038).. controls (1.2304, 4.5776) and (1.2304, 4.5776) .. (1.2171, 4.5508).. controls (0.9944, 4.5377) and (0.9944, 4.5377) .. (0.7673, 4.5244).. controls (0.7673, 4.5418) and (0.7673, 4.5593) .. (0.7673, 4.5773).. controls (0.7324, 4.5773) and (0.6974, 4.5773) .. (0.6615, 4.5773).. controls (0.6615, 4.5511) and (0.6615, 4.5249) .. (0.6615, 4.4979).. controls (0.644, 4.5066) and (0.6265, 4.5154) .. (0.6085, 4.5244).. controls (0.6173, 4.5506) and (0.626, 4.5768) .. (0.635, 4.6038).. controls (0.6088, 4.6038) and (0.5826, 4.6038) .. (0.5556, 4.6038).. controls (0.5469, 4.5863) and (0.5382, 4.5688) .. (0.5292, 4.5508).. controls (0.503, 4.577) and (0.4768, 4.6032) .. (0.4498, 4.6302).. controls (0.4411, 4.6127) and (0.4323, 4.5953) .. (0.4233, 4.5773).. controls (0.3969, 4.6798) and (0.3969, 4.6798) .. (0.3969, 4.7096).. controls (0.3445, 4.7183) and (0.2921, 4.727) .. (0.2381, 4.736).. controls (0.225, 4.6706) and (0.225, 4.6706) .. (0.2117, 4.6038).. controls (0.0112, 4.7689) and (-0.1022, 5.0369) .. (-0.1323, 5.2917).. controls (-0.1345, 5.3091) and (-0.1367, 5.3265) .. (-0.1389, 5.3444).. controls (-0.1514, 5.5519) and (-0.0929, 5.8203) .. (0.0, 6.006) -- cycle;



  \path[fill=signature_color,shift={(5.2487, -3.4545)}] (0.0, 6.006).. controls (0.0943, 5.9544) and (0.1384, 5.8961) .. (0.2017, 5.8093).. controls (0.2017, 5.7918) and (0.2017, 5.7743) .. (0.2017, 5.7563).. controls (0.3769, 5.6902) and (0.5472, 5.6559) .. (0.7326, 5.6323).. controls (0.758, 5.6289) and (0.7834, 5.6255) .. (0.8096, 5.6219).. controls (0.8716, 5.6136) and (0.9335, 5.6055) .. (0.9955, 5.5976).. controls (0.9426, 5.5447) and (0.9426, 5.5447) .. (0.8731, 5.5414).. controls (0.8524, 5.5425) and (0.8317, 5.5436) .. (0.8103, 5.5447).. controls (0.7928, 5.4923) and (0.7754, 5.4399) .. (0.7574, 5.3859).. controls (0.8272, 5.3859) and (0.8971, 5.3859) .. (0.969, 5.3859).. controls (0.969, 5.3685) and (0.969, 5.351) .. (0.969, 5.333).. controls (0.8035, 5.3287) and (0.6543, 5.3398) .. (0.4928, 5.376).. controls (0.472, 5.3806) and (0.4513, 5.3852) .. (0.4299, 5.3899).. controls (0.2679, 5.4291) and (0.1448, 5.4899) .. (0.0165, 5.5976).. controls (-0.0097, 5.6151) and (-0.0359, 5.6325) .. (-0.0628, 5.6505).. controls (-0.089, 5.6418) and (-0.1152, 5.633) .. (-0.1422, 5.624).. controls (-0.1684, 5.6939) and (-0.1946, 5.7637) .. (-0.2216, 5.8357).. controls (-0.2391, 5.8357) and (-0.2565, 5.8357) .. (-0.2745, 5.8357).. controls (-0.2832, 5.8183) and (-0.292, 5.8008) .. (-0.301, 5.7828).. controls (-0.3184, 5.7828) and (-0.3359, 5.7828) .. (-0.3539, 5.7828).. controls (-0.3626, 5.7479) and (-0.3713, 5.7129) .. (-0.3803, 5.677).. controls (-0.424, 5.677) and (-0.4677, 5.677) .. (-0.5126, 5.677).. controls (-0.4182, 5.8186) and (-0.1853, 6.0499) .. (0.0, 6.006) -- cycle;



  \path[fill=signature_color,shift={(7.329, -1.0054)}] (0.0, 6.006).. controls (0.0441, 6.0072) and (0.0882, 6.0071) .. (0.1323, 6.006).. controls (0.1588, 5.9796) and (0.1588, 5.9796) .. (0.1648, 5.8999).. controls (0.1577, 5.7769) and (0.1274, 5.677) .. (0.0876, 5.5612).. controls (0.0731, 5.5171) and (0.0586, 5.4729) .. (0.0441, 5.4287).. controls (0.0331, 5.395) and (0.0331, 5.395) .. (0.0218, 5.3606).. controls (-0.0137, 5.2486) and (-0.045, 5.1355) .. (-0.0761, 5.0221).. controls (-0.0818, 5.0014) and (-0.0874, 4.9806) .. (-0.0933, 4.9593).. controls (-0.1064, 4.9113) and (-0.1193, 4.8634) .. (-0.1323, 4.8154).. controls (-0.1864, 4.9138) and (-0.1864, 4.9138) .. (-0.1753, 4.9808).. controls (-0.1698, 4.9961) and (-0.1644, 5.0113) .. (-0.1588, 5.0271).. controls (-0.1762, 5.0271) and (-0.1937, 5.0271) .. (-0.2117, 5.0271).. controls (-0.2242, 5.0145) and (-0.2368, 5.002) .. (-0.2497, 4.989).. controls (-0.291, 4.9477) and (-0.291, 4.9477) .. (-0.3704, 4.9213).. controls (-0.3555, 5.0483) and (-0.3295, 5.1607) .. (-0.2863, 5.281).. controls (-0.2748, 5.3131) and (-0.2633, 5.3452) .. (-0.2515, 5.3783).. controls (-0.2455, 5.3948) and (-0.2395, 5.4113) .. (-0.2333, 5.4282).. controls (-0.215, 5.4784) and (-0.197, 5.5287) .. (-0.1791, 5.579).. controls (-0.1261, 5.726) and (-0.0721, 5.8672) .. (0.0, 6.006) -- cycle;



  \path[fill=signature_color,shift={(7.0908, -1.9315)}] (0.0, 6.006).. controls (0.0087, 5.9886) and (0.0175, 5.9711) .. (0.0265, 5.9531).. controls (0.0439, 5.9531) and (0.0614, 5.9531) .. (0.0794, 5.9531).. controls (0.0934, 5.6594) and (0.0346, 5.4099) .. (-0.061, 5.1342).. controls (-0.0878, 5.0552) and (-0.1101, 4.9752) .. (-0.1323, 4.8948).. controls (-0.2216, 4.9093) and (-0.2668, 4.9227) .. (-0.344, 4.9742).. controls (-0.3396, 4.99) and (-0.3352, 5.0058) .. (-0.3307, 5.0221).. controls (-0.3146, 5.083) and (-0.3146, 5.083) .. (-0.3175, 5.1594).. controls (-0.3137, 5.1779) and (-0.3099, 5.1965) .. (-0.3059, 5.2156).. controls (-0.2937, 5.2782) and (-0.2876, 5.3359) .. (-0.2828, 5.3992).. controls (-0.2669, 5.556) and (-0.2124, 5.6999) .. (-0.1588, 5.8473).. controls (-0.1413, 5.8473) and (-0.1238, 5.8473) .. (-0.1058, 5.8473).. controls (-0.0971, 5.8648) and (-0.0884, 5.8822) .. (-0.0794, 5.9002).. controls (-0.0619, 5.9089) and (-0.0445, 5.9177) .. (-0.0265, 5.9267).. controls (-0.0177, 5.9529) and (-0.009, 5.9791) .. (0.0, 6.006) -- cycle;



  \path[fill=signature_color,shift={(6.7733, -2.5929)}] (0.0, 6.006).. controls (0.0175, 5.9886) and (0.0349, 5.9711) .. (0.0529, 5.9531).. controls (0.043, 5.8886) and (0.043, 5.8886) .. (0.0265, 5.8208).. controls (0.0265, 5.7946) and (0.0265, 5.7684) .. (0.0265, 5.7415).. controls (0.0184, 5.706) and (0.0097, 5.6707) .. (0.0, 5.6356).. controls (0.0175, 5.6356) and (0.0349, 5.6356) .. (0.0529, 5.6356).. controls (0.0616, 5.6182) and (0.0704, 5.6007) .. (0.0794, 5.5827).. controls (0.0968, 5.5914) and (0.1143, 5.6002) .. (0.1323, 5.6092).. controls (0.1323, 5.5917) and (0.1323, 5.5742) .. (0.1323, 5.5563).. controls (0.1498, 5.5563) and (0.1672, 5.5563) .. (0.1852, 5.5563).. controls (0.2114, 5.6872) and (0.2376, 5.8182) .. (0.2646, 5.9531).. controls (0.2733, 5.9531) and (0.282, 5.9531) .. (0.291, 5.9531).. controls (0.2905, 5.9237) and (0.29, 5.8942) .. (0.2894, 5.8638).. controls (0.2841, 5.7724) and (0.2841, 5.7724) .. (0.3175, 5.715).. controls (0.3088, 5.6975) and (0.3, 5.6801) .. (0.291, 5.6621).. controls (0.2736, 5.6621) and (0.2561, 5.6621) .. (0.2381, 5.6621).. controls (0.2398, 5.6343) and (0.2398, 5.6343) .. (0.2414, 5.6059).. controls (0.2377, 5.5196) and (0.2152, 5.4516) .. (0.1852, 5.371).. controls (0.2027, 5.3798) and (0.2201, 5.3885) .. (0.2381, 5.3975).. controls (0.2469, 5.4237) and (0.2556, 5.4499) .. (0.2646, 5.4769).. controls (0.2646, 5.4594) and (0.2646, 5.442) .. (0.2646, 5.424).. controls (0.2995, 5.4152) and (0.3344, 5.4065) .. (0.3704, 5.3975).. controls (0.3791, 5.415) and (0.3879, 5.4324) .. (0.3969, 5.4504).. controls (0.3794, 5.4504) and (0.362, 5.4504) .. (0.344, 5.4504).. controls (0.344, 5.4679) and (0.344, 5.4853) .. (0.344, 5.5033).. controls (0.3702, 5.4946) and (0.3963, 5.4859) .. (0.4233, 5.4769).. controls (0.431, 5.4567) and (0.4386, 5.4365) .. (0.4465, 5.4157).. controls (0.4833, 5.3278) and (0.5319, 5.2932) .. (0.6085, 5.2388).. controls (0.5562, 5.2388) and (0.5038, 5.2388) .. (0.4498, 5.2388).. controls (0.4411, 5.2213) and (0.4323, 5.2038) .. (0.4233, 5.1858).. controls (0.4059, 5.1858) and (0.3884, 5.1858) .. (0.3704, 5.1858).. controls (0.3791, 5.1596) and (0.3879, 5.1334) .. (0.3969, 5.1065).. controls (0.3707, 5.1065) and (0.3445, 5.1065) .. (0.3175, 5.1065).. controls (0.2728, 5.1588) and (0.2286, 5.2117) .. (0.1852, 5.2652).. controls (0.1677, 5.2652) and (0.1503, 5.2652) .. (0.1323, 5.2652).. controls (0.141, 5.3263) and (0.1498, 5.3874) .. (0.1588, 5.4504).. controls (0.1238, 5.4504) and (0.0889, 5.4504) .. (0.0529, 5.4504).. controls (0.0529, 5.4679) and (0.0529, 5.4853) .. (0.0529, 5.5033).. controls (0.0267, 5.5033) and (0.0005, 5.5033) .. (-0.0265, 5.5033).. controls (-0.0439, 5.4859) and (-0.0614, 5.4684) .. (-0.0794, 5.4504).. controls (-0.1089, 5.5948) and (-0.0986, 5.6923) .. (-0.0546, 5.8324).. controls (-0.0445, 5.8651) and (-0.0344, 5.8978) .. (-0.0241, 5.9315).. controls (-0.0161, 5.9561) and (-0.0082, 5.9807) .. (0.0, 6.006) -- cycle;



  \path[fill=signature_color,shift={(2.5135, -4.154)}] (0.0, 6.006).. controls (0.0175, 5.9886) and (0.0349, 5.9711) .. (0.0529, 5.9531).. controls (0.0791, 5.9531) and (0.1053, 5.9531) .. (0.1323, 5.9531).. controls (0.141, 5.9706) and (0.1498, 5.9881) .. (0.1588, 6.006).. controls (0.2024, 5.9973) and (0.2461, 5.9886) .. (0.291, 5.9796).. controls (0.291, 5.9621) and (0.291, 5.9447) .. (0.291, 5.9267).. controls (0.3347, 5.9354) and (0.3784, 5.9441) .. (0.4233, 5.9531).. controls (0.4233, 5.8309) and (0.4233, 5.7087) .. (0.4233, 5.5827).. controls (0.4146, 5.5827) and (0.4059, 5.5827) .. (0.3969, 5.5827).. controls (0.3969, 5.67) and (0.3969, 5.7573) .. (0.3969, 5.8473).. controls (0.3794, 5.8473) and (0.362, 5.8473) .. (0.344, 5.8473).. controls (0.3344, 5.8251) and (0.3344, 5.8251) .. (0.3247, 5.8025).. controls (0.2837, 5.7282) and (0.233, 5.6731) .. (0.1753, 5.6108).. controls (0.1124, 5.5428) and (0.0521, 5.4758) .. (-0.0043, 5.4024).. controls (-0.0204, 5.3833) and (-0.0364, 5.3642) .. (-0.0529, 5.3446).. controls (-0.0704, 5.3446) and (-0.0878, 5.3446) .. (-0.1058, 5.3446).. controls (-0.1058, 5.362) and (-0.1058, 5.3795) .. (-0.1058, 5.3975).. controls (-0.0884, 5.3975) and (-0.0709, 5.3975) .. (-0.0529, 5.3975).. controls (-0.0442, 5.4324) and (-0.0355, 5.4674) .. (-0.0265, 5.5033).. controls (-0.009, 5.5033) and (0.0085, 5.5033) .. (0.0265, 5.5033).. controls (0.0265, 5.547) and (0.0265, 5.5906) .. (0.0265, 5.6356).. controls (0.0439, 5.6269) and (0.0614, 5.6182) .. (0.0794, 5.6092).. controls (0.0706, 5.6354) and (0.0619, 5.6616) .. (0.0529, 5.6885).. controls (0.0267, 5.6754) and (0.0267, 5.6754) .. (0.0, 5.6621).. controls (-0.0175, 5.6795) and (-0.0349, 5.697) .. (-0.0529, 5.715).. controls (-0.0878, 5.7412) and (-0.1228, 5.7674) .. (-0.1588, 5.7944).. controls (-0.1064, 5.8642) and (-0.054, 5.9341) .. (0.0, 6.006) -- cycle;



  \path[fill=signature_color,shift={(10.2129, -3.4131)}] (0.0, 6.006).. controls (0.0181, 6.0006) and (0.0362, 5.9952) .. (0.0548, 5.9896).. controls (0.3423, 5.9128) and (0.6299, 5.9074) .. (0.926, 5.9002).. controls (0.926, 5.8915) and (0.926, 5.8827) .. (0.926, 5.8738).. controls (0.7863, 5.8563) and (0.6466, 5.8388) .. (0.5027, 5.8208).. controls (0.5114, 5.7859) and (0.5202, 5.751) .. (0.5292, 5.715).. controls (0.5466, 5.7063) and (0.5641, 5.6975) .. (0.5821, 5.6885).. controls (0.5734, 5.6623) and (0.5646, 5.6362) .. (0.5556, 5.6092).. controls (0.3971, 5.6024) and (0.2723, 5.6219) .. (0.1224, 5.672).. controls (0.1036, 5.6779) and (0.0848, 5.6837) .. (0.0654, 5.6898).. controls (-0.072, 5.7341) and (-0.072, 5.7341) .. (-0.1323, 5.7944).. controls (-0.0624, 5.7944) and (0.0074, 5.7944) .. (0.0794, 5.7944).. controls (0.0881, 5.7769) and (0.0968, 5.7595) .. (0.1058, 5.7415).. controls (0.1408, 5.7415) and (0.1757, 5.7415) .. (0.2117, 5.7415).. controls (0.2029, 5.7589) and (0.1942, 5.7764) .. (0.1852, 5.7944).. controls (0.2027, 5.7944) and (0.2201, 5.7944) .. (0.2381, 5.7944).. controls (0.2381, 5.8293) and (0.2381, 5.8642) .. (0.2381, 5.9002).. controls (0.2179, 5.9002) and (0.1977, 5.9002) .. (0.1769, 5.9002).. controls (0.1268, 5.9002) and (0.0766, 5.9002) .. (0.0265, 5.9002).. controls (0.0177, 5.9351) and (0.009, 5.9701) .. (0.0, 6.006) -- cycle;



  \path[fill=signature_color,shift={(0.979, -2.7517)}] (0.0, 6.006).. controls (0.0087, 5.9886) and (0.0175, 5.9711) .. (0.0265, 5.9531).. controls (0.0527, 5.9619) and (0.0788, 5.9706) .. (0.1058, 5.9796).. controls (0.0562, 5.8771) and (0.0562, 5.8771) .. (0.0265, 5.8473).. controls (0.0614, 5.8648) and (0.0963, 5.8822) .. (0.1323, 5.9002).. controls (0.1323, 5.8391) and (0.1323, 5.778) .. (0.1323, 5.715).. controls (0.0974, 5.7237) and (0.0624, 5.7325) .. (0.0265, 5.7415).. controls (0.0352, 5.7153) and (0.0439, 5.6891) .. (0.0529, 5.6621).. controls (0.0267, 5.6534) and (0.0005, 5.6446) .. (-0.0265, 5.6356).. controls (-0.0177, 5.6007) and (-0.009, 5.5658) .. (0.0, 5.5298).. controls (-0.0349, 5.5385) and (-0.0699, 5.5473) .. (-0.1058, 5.5563).. controls (-0.1058, 5.5737) and (-0.1058, 5.5912) .. (-0.1058, 5.6092).. controls (-0.1584, 5.6021) and (-0.1584, 5.6021) .. (-0.2117, 5.5827).. controls (-0.2204, 5.5565) and (-0.2291, 5.5303) .. (-0.2381, 5.5033).. controls (-0.2556, 5.4946) and (-0.2731, 5.4859) .. (-0.291, 5.4769).. controls (-0.3085, 5.4507) and (-0.326, 5.4245) .. (-0.344, 5.3975).. controls (-0.344, 5.415) and (-0.344, 5.4324) .. (-0.344, 5.4504).. controls (-0.3614, 5.4504) and (-0.3789, 5.4504) .. (-0.3969, 5.4504).. controls (-0.4167, 5.4967) and (-0.4167, 5.4967) .. (-0.4233, 5.5563).. controls (-0.3892, 5.5926) and (-0.3538, 5.6279) .. (-0.3175, 5.6621).. controls (-0.2967, 5.7187) and (-0.2967, 5.7187) .. (-0.291, 5.7679).. controls (-0.2665, 5.7712) and (-0.2665, 5.7712) .. (-0.2414, 5.7745).. controls (-0.1803, 5.7894) and (-0.1803, 5.7894) .. (-0.1521, 5.8489).. controls (-0.1423, 5.8743) and (-0.1423, 5.8743) .. (-0.1323, 5.9002).. controls (-0.0974, 5.9002) and (-0.0624, 5.9002) .. (-0.0265, 5.9002).. controls (-0.0177, 5.9351) and (-0.009, 5.9701) .. (0.0, 6.006) -- cycle;



  \path[fill=signature_color,shift={(4.4185, -4.2863)}] (0.0, 6.006).. controls (0.0, 5.9886) and (0.0, 5.9711) .. (0.0, 5.9531).. controls (0.0175, 5.9444) and (0.0349, 5.9357) .. (0.0529, 5.9267).. controls (0.0258, 5.7719) and (-0.0637, 5.6979) .. (-0.175, 5.597).. controls (-0.2472, 5.5504) and (-0.2861, 5.5476) .. (-0.3704, 5.5563).. controls (-0.3989, 5.6264) and (-0.4258, 5.696) .. (-0.4498, 5.7679).. controls (-0.402, 5.8232) and (-0.3703, 5.8473) .. (-0.301, 5.8738).. controls (-0.2699, 5.8868) and (-0.2699, 5.8868) .. (-0.2381, 5.9002).. controls (-0.2294, 5.9264) and (-0.2207, 5.9526) .. (-0.2117, 5.9796).. controls (-0.137, 6.0079) and (-0.0819, 6.006) .. (0.0, 6.006) -- cycle(-0.0265, 5.9531).. controls (-0.0265, 5.9095) and (-0.0265, 5.8658) .. (-0.0265, 5.8208).. controls (-0.0177, 5.8208) and (-0.009, 5.8208) .. (0.0, 5.8208).. controls (0.0, 5.8645) and (0.0, 5.9081) .. (0.0, 5.9531).. controls (-0.0087, 5.9531) and (-0.0175, 5.9531) .. (-0.0265, 5.9531) -- cycle;



  \path[fill=signature_color,shift={(1.1906, -2.5929)}] (0.0, 6.006).. controls (0.0, 5.9886) and (0.0, 5.9711) .. (0.0, 5.9531).. controls (-0.0175, 5.9531) and (-0.0349, 5.9531) .. (-0.0529, 5.9531).. controls (-0.0355, 5.9182) and (-0.018, 5.8833) .. (0.0, 5.8473).. controls (0.0175, 5.8648) and (0.0349, 5.8822) .. (0.0529, 5.9002).. controls (0.0704, 5.8915) and (0.0878, 5.8827) .. (0.1058, 5.8738).. controls (0.1241, 5.7859) and (0.1305, 5.7469) .. (0.1058, 5.6621).. controls (0.0176, 5.5547) and (-0.0812, 5.4629) .. (-0.1852, 5.371).. controls (-0.2027, 5.406) and (-0.2201, 5.4409) .. (-0.2381, 5.4769).. controls (-0.2207, 5.4943) and (-0.2032, 5.5118) .. (-0.1852, 5.5298).. controls (-0.1852, 5.5473) and (-0.1852, 5.5647) .. (-0.1852, 5.5827).. controls (-0.1503, 5.574) and (-0.1154, 5.5652) .. (-0.0794, 5.5563).. controls (-0.0881, 5.6086) and (-0.0968, 5.661) .. (-0.1058, 5.715).. controls (-0.132, 5.6888) and (-0.1582, 5.6626) .. (-0.1852, 5.6356).. controls (-0.1852, 5.6531) and (-0.1852, 5.6706) .. (-0.1852, 5.6885).. controls (-0.2027, 5.6973) and (-0.2201, 5.706) .. (-0.2381, 5.715).. controls (-0.2032, 5.7325) and (-0.1683, 5.7499) .. (-0.1323, 5.7679).. controls (-0.1585, 5.7766) and (-0.1847, 5.7854) .. (-0.2117, 5.7944).. controls (-0.2204, 5.7769) and (-0.2291, 5.7595) .. (-0.2381, 5.7415).. controls (-0.2731, 5.7415) and (-0.308, 5.7415) .. (-0.344, 5.7415).. controls (-0.3614, 5.6978) and (-0.3789, 5.6541) .. (-0.3969, 5.6092).. controls (-0.4318, 5.6179) and (-0.4667, 5.6266) .. (-0.5027, 5.6356).. controls (-0.4851, 5.6513) and (-0.4676, 5.6669) .. (-0.4495, 5.6831).. controls (-0.4261, 5.704) and (-0.4028, 5.7249) .. (-0.3787, 5.7464).. controls (-0.3424, 5.7789) and (-0.3424, 5.7789) .. (-0.3054, 5.8119).. controls (-0.2551, 5.8582) and (-0.2081, 5.9045) .. (-0.1621, 5.9548).. controls (-0.1058, 6.006) and (-0.1058, 6.006) .. (0.0, 6.006) -- cycle;



  \path[fill=signature_color,shift={(9.869, -3.2808)}] (0.0, 6.006).. controls (0.0278, 5.9935) and (0.0557, 5.9809) .. (0.0843, 5.968).. controls (0.1458, 5.9411) and (0.1907, 5.9259) .. (0.258, 5.9167).. controls (0.3175, 5.9002) and (0.3175, 5.9002) .. (0.3655, 5.8324).. controls (0.3758, 5.8111) and (0.3862, 5.7898) .. (0.3969, 5.7679).. controls (0.4143, 5.7766) and (0.4318, 5.7854) .. (0.4498, 5.7944).. controls (0.4942, 5.787) and (0.5383, 5.7783) .. (0.5821, 5.7679).. controls (0.5772, 5.7472) and (0.5723, 5.7264) .. (0.5672, 5.7051).. controls (0.5488, 5.6364) and (0.5488, 5.6364) .. (0.5821, 5.5827).. controls (0.5192, 5.5794) and (0.5192, 5.5794) .. (0.4498, 5.5827).. controls (0.4323, 5.6002) and (0.4149, 5.6176) .. (0.3969, 5.6356).. controls (0.343, 5.6449) and (0.343, 5.6449) .. (0.2828, 5.6472).. controls (0.179, 5.6571) and (0.1073, 5.6707) .. (0.0265, 5.7415).. controls (0.005, 5.8126) and (0.005, 5.8126) .. (0.0, 5.8738).. controls (-0.0262, 5.8738) and (-0.0524, 5.8738) .. (-0.0794, 5.8738).. controls (-0.0794, 5.8912) and (-0.0794, 5.9087) .. (-0.0794, 5.9267).. controls (-0.0619, 5.9267) and (-0.0445, 5.9267) .. (-0.0265, 5.9267).. controls (-0.0177, 5.9529) and (-0.009, 5.9791) .. (0.0, 6.006) -- cycle;



  \path[fill=signature_color,shift={(6.6675, -3.0427)}] (0.0, 6.006).. controls (0.131, 5.9537) and (0.131, 5.9537) .. (0.2646, 5.9002).. controls (0.2564, 5.8522) and (0.2481, 5.8043) .. (0.2398, 5.7563).. controls (0.2352, 5.7296) and (0.2306, 5.7029) .. (0.2258, 5.6754).. controls (0.2117, 5.6092) and (0.2117, 5.6092) .. (0.1852, 5.5827).. controls (0.1841, 5.5298) and (0.1841, 5.4769) .. (0.1852, 5.424).. controls (0.1677, 5.424) and (0.1503, 5.424) .. (0.1323, 5.424).. controls (0.1323, 5.4414) and (0.1323, 5.4589) .. (0.1323, 5.4769).. controls (0.1148, 5.4769) and (0.0974, 5.4769) .. (0.0794, 5.4769).. controls (0.0513, 5.5265) and (0.0513, 5.5265) .. (0.0265, 5.5827).. controls (0.0352, 5.6002) and (0.0439, 5.6176) .. (0.0529, 5.6356).. controls (0.0275, 5.6283) and (0.0275, 5.6283) .. (0.0017, 5.6207).. controls (-0.0164, 5.6169) and (-0.0344, 5.6131) .. (-0.0529, 5.6092).. controls (-0.0616, 5.6179) and (-0.0704, 5.6266) .. (-0.0794, 5.6356).. controls (-0.078, 5.6856) and (-0.078, 5.6856) .. (-0.0711, 5.7481).. controls (-0.069, 5.7684) and (-0.0669, 5.7887) .. (-0.0648, 5.8097).. controls (-0.0529, 5.8738) and (-0.0529, 5.8738) .. (0.0, 6.006) -- cycle;



  \path[fill=signature_color,shift={(8.2285, -2.3019)}] (0.0, 6.006).. controls (0.0489, 5.9326) and (0.0794, 5.8833) .. (0.0794, 5.7944).. controls (0.0619, 5.7944) and (0.0445, 5.7944) .. (0.0265, 5.7944).. controls (0.0265, 5.7769) and (0.0265, 5.7595) .. (0.0265, 5.7415).. controls (-0.0261, 5.7423) and (-0.0786, 5.7432) .. (-0.1312, 5.744).. controls (-0.1906, 5.7436) and (-0.1906, 5.7436) .. (-0.2646, 5.715).. controls (-0.2646, 5.6975) and (-0.2646, 5.6801) .. (-0.2646, 5.6621).. controls (-0.282, 5.6621) and (-0.2995, 5.6621) .. (-0.3175, 5.6621).. controls (-0.3175, 5.6795) and (-0.3175, 5.697) .. (-0.3175, 5.715).. controls (-0.335, 5.715) and (-0.3524, 5.715) .. (-0.3704, 5.715).. controls (-0.3617, 5.7499) and (-0.353, 5.7849) .. (-0.344, 5.8208).. controls (-0.3702, 5.8296) and (-0.3963, 5.8383) .. (-0.4233, 5.8473).. controls (-0.4233, 5.8735) and (-0.4233, 5.8997) .. (-0.4233, 5.9267).. controls (-0.3474, 5.9646) and (-0.2715, 5.9645) .. (-0.1877, 5.9721).. controls (-0.1151, 5.9819) and (-0.0764, 6.006) .. (0.0, 6.006) -- cycle;



  \path[fill=signature_color,shift={(7.1438, -1.4552)}] (0.0, 6.006).. controls (0.0296, 5.9076) and (0.0296, 5.9076) .. (0.0017, 5.8407).. controls (-0.0123, 5.8178) and (-0.0123, 5.8178) .. (-0.0265, 5.7944).. controls (-0.0003, 5.7944) and (0.0259, 5.7944) .. (0.0529, 5.7944).. controls (0.0704, 5.7769) and (0.0878, 5.7595) .. (0.1058, 5.7415).. controls (0.1233, 5.7415) and (0.1408, 5.7415) .. (0.1588, 5.7415).. controls (0.1433, 5.5747) and (0.1048, 5.4243) .. (0.0529, 5.2652).. controls (-0.0012, 5.3636) and (-0.0012, 5.3636) .. (0.0099, 5.4306).. controls (0.0154, 5.4459) and (0.0208, 5.4611) .. (0.0265, 5.4769).. controls (0.009, 5.4769) and (-0.0085, 5.4769) .. (-0.0265, 5.4769).. controls (-0.039, 5.4643) and (-0.0516, 5.4518) .. (-0.0645, 5.4388).. controls (-0.1058, 5.3975) and (-0.1058, 5.3975) .. (-0.1852, 5.371).. controls (-0.1695, 5.521) and (-0.1298, 5.6516) .. (-0.0777, 5.7927).. controls (-0.0703, 5.8133) and (-0.0628, 5.834) .. (-0.0551, 5.8552).. controls (-0.0369, 5.9055) and (-0.0185, 5.9558) .. (0.0, 6.006) -- cycle;



  \path[fill=signature_color,shift={(11.0596, -1.6669)}] (0.0, 6.006).. controls (0.1329, 6.0139) and (0.2093, 6.0062) .. (0.3175, 5.9267).. controls (0.344, 5.9002) and (0.344, 5.9002) .. (0.3456, 5.8192).. controls (0.3451, 5.7935) and (0.3445, 5.7679) .. (0.344, 5.7415).. controls (0.2447, 5.7117) and (0.2447, 5.7117) .. (0.1852, 5.7415).. controls (0.1852, 5.7589) and (0.1852, 5.7764) .. (0.1852, 5.7944).. controls (0.1676, 5.7941) and (0.1501, 5.7938) .. (0.132, 5.7934).. controls (-0.1001, 5.7911) and (-0.1001, 5.7911) .. (-0.2117, 5.8208).. controls (-0.2117, 5.8645) and (-0.2117, 5.9081) .. (-0.2117, 5.9531).. controls (-0.1359, 5.991) and (-0.0811, 5.969) .. (0.0, 5.9531).. controls (0.0, 5.9706) and (0.0, 5.9881) .. (0.0, 6.006) -- cycle;



  \path[fill=signature_color,shift={(11.0414, -3.5438)}] (0.0, 6.006).. controls (0.0196, 6.0059) and (0.0392, 6.0058) .. (0.0594, 6.0057).. controls (0.1074, 6.0054) and (0.1554, 6.0049) .. (0.2034, 6.0044).. controls (0.2034, 5.9869) and (0.2034, 5.9695) .. (0.2034, 5.9515).. controls (0.1685, 5.9427) and (0.1335, 5.934) .. (0.0976, 5.925).. controls (0.0943, 5.9043) and (0.091, 5.8835) .. (0.0876, 5.8622).. controls (0.0711, 5.7927) and (0.0711, 5.7927) .. (0.0182, 5.7398).. controls (-0.0506, 5.7346) and (-0.0506, 5.7346) .. (-0.1306, 5.7365).. controls (-0.1704, 5.7372) and (-0.1704, 5.7372) .. (-0.211, 5.7379).. controls (-0.2314, 5.7386) and (-0.2518, 5.7392) .. (-0.2729, 5.7398).. controls (-0.2766, 5.7558) and (-0.2804, 5.7717) .. (-0.2843, 5.7882).. controls (-0.3015, 5.8541) and (-0.3261, 5.9151) .. (-0.3522, 5.9779).. controls (-0.2348, 6.0073) and (-0.1202, 6.0074) .. (0.0, 6.006) -- cycle;



  \path[fill=signature_color,shift={(7.2496, -2.7252)}] (0.0, 6.006).. controls (0.0262, 5.9973) and (0.0524, 5.9886) .. (0.0794, 5.9796).. controls (0.1119, 5.8626) and (0.1119, 5.8626) .. (0.0843, 5.7927).. controls (0.074, 5.7758) and (0.0636, 5.7589) .. (0.0529, 5.7415).. controls (0.0447, 5.7224) and (0.0365, 5.7033) .. (0.0281, 5.6836).. controls (0.0, 5.6356) and (0.0, 5.6356) .. (-0.0794, 5.6092).. controls (-0.0968, 5.6179) and (-0.1143, 5.6266) .. (-0.1323, 5.6356).. controls (-0.1323, 5.6182) and (-0.1323, 5.6007) .. (-0.1323, 5.5827).. controls (-0.1148, 5.5827) and (-0.0974, 5.5827) .. (-0.0794, 5.5827).. controls (-0.0881, 5.5565) and (-0.0968, 5.5303) .. (-0.1058, 5.5033).. controls (-0.1217, 5.5082) and (-0.1375, 5.5132) .. (-0.1538, 5.5182).. controls (-0.216, 5.5356) and (-0.216, 5.5356) .. (-0.291, 5.5033).. controls (-0.2974, 5.6115) and (-0.2758, 5.6932) .. (-0.2381, 5.7944).. controls (-0.2207, 5.7944) and (-0.2032, 5.7944) .. (-0.1852, 5.7944).. controls (-0.1841, 5.8097) and (-0.183, 5.8249) .. (-0.1819, 5.8407).. controls (-0.1463, 5.9322) and (-0.0842, 5.9606) .. (0.0, 6.006) -- cycle;



  \path[fill=signature_color,shift={(9.5994, -2.9451)}] (0.0, 6.006).. controls (0.1144, 5.9814) and (0.1347, 5.9279) .. (0.1975, 5.8339).. controls (0.274, 5.7371) and (0.3759, 5.6792) .. (0.4812, 5.6174).. controls (0.4548, 5.5645) and (0.4548, 5.5645) .. (0.4018, 5.5381).. controls (0.4018, 5.5643) and (0.4018, 5.5904) .. (0.4018, 5.6174).. controls (0.3364, 5.6305) and (0.3364, 5.6305) .. (0.2695, 5.6439).. controls (0.2608, 5.6264) and (0.2521, 5.609) .. (0.2431, 5.591).. controls (0.2518, 5.5822) and (0.2605, 5.5735) .. (0.2695, 5.5645).. controls (0.2792, 5.5206) and (0.2881, 5.4765) .. (0.296, 5.4322).. controls (0.1948, 5.4876) and (0.1112, 5.5339) .. (0.0314, 5.6174).. controls (-0.0241, 5.6485) and (-0.0241, 5.6485) .. (-0.0744, 5.6704).. controls (-0.0531, 5.6769) and (-0.0318, 5.6834) .. (-0.0099, 5.6902).. controls (0.0624, 5.7159) and (0.0624, 5.7159) .. (0.0794, 5.7795).. controls (0.081, 5.7959) and (0.0826, 5.8122) .. (0.0843, 5.8291).. controls (0.0931, 5.8204) and (0.1018, 5.8116) .. (0.1108, 5.8026).. controls (0.1021, 5.8376) and (0.0933, 5.8725) .. (0.0843, 5.9085).. controls (0.0058, 5.9085) and (-0.0728, 5.9085) .. (-0.1538, 5.9085).. controls (-0.0744, 5.9879) and (-0.0744, 5.9879) .. (0.0, 6.006) -- cycle;



  \path[fill=signature_color,shift={(7.2231, -3.3338)}] (0.0, 6.006).. controls (0.1572, 5.9798) and (0.1572, 5.9798) .. (0.3175, 5.9531).. controls (0.3262, 5.9269) and (0.335, 5.9007) .. (0.344, 5.8738).. controls (0.3614, 5.865) and (0.3789, 5.8563) .. (0.3969, 5.8473).. controls (0.3881, 5.8036) and (0.3794, 5.76) .. (0.3704, 5.715).. controls (0.353, 5.7325) and (0.3355, 5.7499) .. (0.3175, 5.7679).. controls (0.3088, 5.7505) and (0.3, 5.733) .. (0.291, 5.715).. controls (0.1539, 5.7068) and (0.0618, 5.7152) .. (-0.0529, 5.7944).. controls (-0.0761, 5.8655) and (-0.0761, 5.8655) .. (-0.0794, 5.9267).. controls (-0.0619, 5.9267) and (-0.0445, 5.9267) .. (-0.0265, 5.9267).. controls (-0.0177, 5.9529) and (-0.009, 5.9791) .. (0.0, 6.006) -- cycle;



  \path[fill=signature_color,shift={(0.4763, -4.7625)}] (0.0, 6.006).. controls (0.0169, 5.9875) and (0.0338, 5.9689) .. (0.0513, 5.9498).. controls (0.1816, 5.8194) and (0.3404, 5.6962) .. (0.5296, 5.6855).. controls (0.557, 5.6857) and (0.557, 5.6857) .. (0.585, 5.686).. controls (0.6149, 5.6861) and (0.6149, 5.6861) .. (0.6454, 5.6862).. controls (0.666, 5.6864) and (0.6866, 5.6867) .. (0.7078, 5.6869).. controls (0.7287, 5.687) and (0.7497, 5.6871) .. (0.7713, 5.6872).. controls (0.8229, 5.6876) and (0.8745, 5.688) .. (0.926, 5.6885).. controls (0.9173, 5.6711) and (0.9086, 5.6536) .. (0.8996, 5.6356).. controls (0.6769, 5.6225) and (0.6769, 5.6225) .. (0.4498, 5.6092).. controls (0.4498, 5.6266) and (0.4498, 5.6441) .. (0.4498, 5.6621).. controls (0.4149, 5.6621) and (0.3799, 5.6621) .. (0.344, 5.6621).. controls (0.344, 5.6359) and (0.344, 5.6097) .. (0.344, 5.5827).. controls (0.3265, 5.5914) and (0.309, 5.6002) .. (0.291, 5.6092).. controls (0.2998, 5.6354) and (0.3085, 5.6616) .. (0.3175, 5.6885).. controls (0.2913, 5.6885) and (0.2651, 5.6885) .. (0.2381, 5.6885).. controls (0.2294, 5.6711) and (0.2207, 5.6536) .. (0.2117, 5.6356).. controls (0.1855, 5.6618) and (0.1593, 5.688) .. (0.1323, 5.715).. controls (0.1236, 5.6975) and (0.1148, 5.6801) .. (0.1058, 5.6621).. controls (0.0794, 5.7646) and (0.0794, 5.7646) .. (0.0794, 5.7944).. controls (0.0008, 5.8075) and (0.0008, 5.8075) .. (-0.0794, 5.8208).. controls (-0.0881, 5.7772) and (-0.0968, 5.7335) .. (-0.1058, 5.6885).. controls (-0.2348, 5.8142) and (-0.2348, 5.8142) .. (-0.2646, 5.8738).. controls (-0.1158, 5.93) and (-0.1158, 5.93) .. (-0.0265, 5.9002).. controls (-0.0177, 5.9351) and (-0.009, 5.9701) .. (0.0, 6.006) -- cycle;



  \path[fill=signature_color,shift={(0.2117, -3.8365)}] (0.0, 6.006).. controls (0.0247, 5.9264) and (0.0276, 5.9037) .. (0.0, 5.8208).. controls (0.0175, 5.8208) and (0.0349, 5.8208) .. (0.0529, 5.8208).. controls (0.0529, 5.8645) and (0.0529, 5.9081) .. (0.0529, 5.9531).. controls (0.0791, 5.9357) and (0.1053, 5.9182) .. (0.1323, 5.9002).. controls (0.1438, 5.8293) and (0.1438, 5.8293) .. (0.1439, 5.7448).. controls (0.1444, 5.717) and (0.1449, 5.6892) .. (0.1454, 5.6606).. controls (0.1296, 5.5668) and (0.0989, 5.5362) .. (0.0265, 5.4769).. controls (0.0177, 5.5205) and (0.009, 5.5642) .. (0.0, 5.6092).. controls (-0.0175, 5.6092) and (-0.0349, 5.6092) .. (-0.0529, 5.6092).. controls (-0.0616, 5.5568) and (-0.0704, 5.5044) .. (-0.0794, 5.4504).. controls (-0.1056, 5.4591) and (-0.1318, 5.4679) .. (-0.1588, 5.4769).. controls (-0.1621, 5.5529) and (-0.1621, 5.5529) .. (-0.1588, 5.6356).. controls (-0.1413, 5.6531) and (-0.1238, 5.6706) .. (-0.1058, 5.6885).. controls (-0.1058, 5.7147) and (-0.1058, 5.7409) .. (-0.1058, 5.7679).. controls (-0.0884, 5.7679) and (-0.0709, 5.7679) .. (-0.0529, 5.7679).. controls (-0.0704, 5.7854) and (-0.0878, 5.8028) .. (-0.1058, 5.8208).. controls (-0.1058, 5.8558) and (-0.1058, 5.8907) .. (-0.1058, 5.9267).. controls (-0.0529, 5.973) and (-0.0529, 5.973) .. (0.0, 6.006) -- cycle;



  \path[fill=signature_color,shift={(2.3548, -1.5875)}] (0.0, 6.006).. controls (0.0087, 5.9798) and (0.0175, 5.9537) .. (0.0265, 5.9267).. controls (0.0003, 5.9092) and (-0.0259, 5.8917) .. (-0.0529, 5.8738).. controls (-0.0724, 5.7794) and (-0.0724, 5.7794) .. (-0.0794, 5.6885).. controls (-0.0881, 5.706) and (-0.0968, 5.7235) .. (-0.1058, 5.7415).. controls (-0.1233, 5.7415) and (-0.1408, 5.7415) .. (-0.1588, 5.7415).. controls (-0.1675, 5.724) and (-0.1762, 5.7065) .. (-0.1852, 5.6885).. controls (-0.1852, 5.706) and (-0.1852, 5.7235) .. (-0.1852, 5.7415).. controls (-0.2027, 5.7415) and (-0.2201, 5.7415) .. (-0.2381, 5.7415).. controls (-0.2349, 5.7076) and (-0.2316, 5.6738) .. (-0.2282, 5.6389).. controls (-0.2264, 5.6199) and (-0.2245, 5.6009) .. (-0.2226, 5.5813).. controls (-0.2153, 5.5238) and (-0.2153, 5.5238) .. (-0.1588, 5.4769).. controls (-0.189, 5.4503) and (-0.2193, 5.4239) .. (-0.2497, 5.3975).. controls (-0.2666, 5.3828) and (-0.2835, 5.368) .. (-0.3009, 5.3529).. controls (-0.344, 5.3181) and (-0.344, 5.3181) .. (-0.3704, 5.3181).. controls (-0.3704, 5.5563) and (-0.3704, 5.5563) .. (-0.344, 5.6092).. controls (-0.3352, 5.6354) and (-0.3265, 5.6616) .. (-0.3175, 5.6885).. controls (-0.3524, 5.6798) and (-0.3874, 5.6711) .. (-0.4233, 5.6621).. controls (-0.2993, 5.7989) and (-0.1504, 5.8999) .. (0.0, 6.006) -- cycle;



  \path[fill=signature_color,shift={(3.3338, -0.979)}] (0.0, 6.006).. controls (0.0467, 5.9243) and (0.0529, 5.894) .. (0.0529, 5.7944).. controls (0.0387, 5.7916) and (0.0245, 5.7889) .. (0.0099, 5.7861).. controls (-0.1107, 5.7512) and (-0.1942, 5.7148) .. (-0.2646, 5.6092).. controls (-0.2908, 5.6092) and (-0.317, 5.6092) .. (-0.344, 5.6092).. controls (-0.3527, 5.6266) and (-0.3614, 5.6441) .. (-0.3704, 5.6621).. controls (-0.3879, 5.6708) and (-0.4053, 5.6795) .. (-0.4233, 5.6885).. controls (-0.4233, 5.706) and (-0.4233, 5.7235) .. (-0.4233, 5.7415).. controls (-0.3884, 5.7502) and (-0.3535, 5.7589) .. (-0.3175, 5.7679).. controls (-0.3152, 5.7839) and (-0.313, 5.7998) .. (-0.3107, 5.8163).. controls (-0.2853, 5.8906) and (-0.2552, 5.9069) .. (-0.1885, 5.9465).. controls (-0.1608, 5.9635) and (-0.1608, 5.9635) .. (-0.1325, 5.9808).. controls (-0.0794, 6.006) and (-0.0794, 6.006) .. (0.0, 6.006) -- cycle;



  \path[fill=signature_color,shift={(13.1763, -2.54)}] (0.0, 6.006).. controls (0.036, 5.934) and (0.0186, 5.8954) .. (0.0017, 5.8175).. controls (-0.0035, 5.7934) and (-0.0086, 5.7694) .. (-0.014, 5.7446).. controls (-0.0181, 5.7261) and (-0.0222, 5.7076) .. (-0.0265, 5.6885).. controls (-0.0527, 5.6885) and (-0.0788, 5.6885) .. (-0.1058, 5.6885).. controls (-0.0907, 5.6127) and (-0.083, 5.5864) .. (-0.0265, 5.5298).. controls (-0.0614, 5.5211) and (-0.0963, 5.5123) .. (-0.1323, 5.5033).. controls (-0.141, 5.547) and (-0.1498, 5.5906) .. (-0.1588, 5.6356).. controls (-0.1849, 5.6269) and (-0.2111, 5.6182) .. (-0.2381, 5.6092).. controls (-0.2556, 5.6266) and (-0.273, 5.6441) .. (-0.291, 5.6621).. controls (-0.3085, 5.6446) and (-0.326, 5.6272) .. (-0.344, 5.6092).. controls (-0.4054, 5.5986) and (-0.4672, 5.5896) .. (-0.5292, 5.5827).. controls (-0.5104, 5.5977) and (-0.4916, 5.6127) .. (-0.4722, 5.6281).. controls (-0.3725, 5.7081) and (-0.2739, 5.7894) .. (-0.1752, 5.8705).. controls (-0.1177, 5.9171) and (-0.0596, 5.9622) .. (0.0, 6.006) -- cycle;



  \path[fill=signature_color,shift={(6.8263, -3.7042)}] (0.0, 6.006).. controls (0.0087, 5.9711) and (0.0175, 5.9362) .. (0.0265, 5.9002).. controls (0.0396, 5.9046) and (0.0527, 5.9089) .. (0.0661, 5.9134).. controls (0.1613, 5.9325) and (0.2467, 5.9314) .. (0.344, 5.9267).. controls (0.3969, 5.8738) and (0.3969, 5.8738) .. (0.4068, 5.796).. controls (0.3969, 5.715) and (0.3969, 5.715) .. (0.3518, 5.6721).. controls (0.2847, 5.6318) and (0.2323, 5.6185) .. (0.1554, 5.6059).. controls (0.0614, 5.5906) and (0.0614, 5.5906) .. (-0.0265, 5.5563).. controls (-0.0265, 5.5912) and (-0.0265, 5.6261) .. (-0.0265, 5.6621).. controls (0.0085, 5.6621) and (0.0434, 5.6621) .. (0.0794, 5.6621).. controls (0.0881, 5.6795) and (0.0968, 5.697) .. (0.1058, 5.715).. controls (0.1495, 5.7063) and (0.1931, 5.6975) .. (0.2381, 5.6885).. controls (0.2248, 5.7418) and (0.2248, 5.7418) .. (0.1852, 5.7944).. controls (0.0914, 5.8287) and (0.0914, 5.8287) .. (0.0, 5.8473).. controls (0.0, 5.8997) and (0.0, 5.9521) .. (0.0, 6.006) -- cycle;



  \path[fill=signature_color,shift={(6.2442, -3.8894)}] (0.0, 6.006).. controls (0.0, 5.9886) and (0.0, 5.9711) .. (0.0, 5.9531).. controls (0.1441, 5.9662) and (0.1441, 5.9662) .. (0.291, 5.9796).. controls (0.2646, 5.9002) and (0.2646, 5.9002) .. (0.2117, 5.8738).. controls (0.2204, 5.8388) and (0.2291, 5.8039) .. (0.2381, 5.7679).. controls (0.1508, 5.7679) and (0.0635, 5.7679) .. (-0.0265, 5.7679).. controls (-0.0265, 5.7854) and (-0.0265, 5.8028) .. (-0.0265, 5.8208).. controls (-0.0963, 5.8208) and (-0.1662, 5.8208) .. (-0.2381, 5.8208).. controls (-0.2207, 5.8732) and (-0.2032, 5.9256) .. (-0.1852, 5.9796).. controls (-0.0595, 6.006) and (-0.0595, 6.006) .. (0.0, 6.006) -- cycle;



  \path[fill=signature_color,shift={(6.985, -2.5665)}] (0.0, 6.006).. controls (0.0529, 5.9531) and (0.0529, 5.9531) .. (0.0529, 5.8473).. controls (0.0409, 5.7919) and (0.0275, 5.7368) .. (0.0132, 5.6819).. controls (0.0059, 5.6534) and (-0.0015, 5.6249) .. (-0.0091, 5.5955).. controls (-0.0148, 5.5738) and (-0.0206, 5.5521) .. (-0.0265, 5.5298).. controls (-0.1157, 5.5443) and (-0.1609, 5.5577) .. (-0.2381, 5.6092).. controls (-0.2338, 5.625) and (-0.2294, 5.6408) .. (-0.2249, 5.6571).. controls (-0.2087, 5.718) and (-0.2087, 5.718) .. (-0.2117, 5.7944).. controls (-0.1753, 5.839) and (-0.1753, 5.839) .. (-0.1323, 5.8738).. controls (-0.1148, 5.8738) and (-0.0974, 5.8738) .. (-0.0794, 5.8738).. controls (-0.0712, 5.8907) and (-0.063, 5.9076) .. (-0.0546, 5.925).. controls (-0.0265, 5.9796) and (-0.0265, 5.9796) .. (0.0, 6.006) -- cycle;



  \path[fill=signature_color,shift={(2.4077, -4.3656)}] (0.0, 6.006).. controls (0.0794, 5.9796) and (0.0794, 5.9796) .. (0.1323, 5.9002).. controls (0.1498, 5.9089) and (0.1672, 5.9177) .. (0.1852, 5.9267).. controls (0.2034, 5.8804) and (0.2034, 5.8804) .. (0.2117, 5.8208).. controls (0.1769, 5.7613) and (0.1769, 5.7613) .. (0.1323, 5.715).. controls (0.1148, 5.715) and (0.0974, 5.715) .. (0.0794, 5.715).. controls (0.0706, 5.6801) and (0.0619, 5.6452) .. (0.0529, 5.6092).. controls (0.0093, 5.6092) and (-0.0344, 5.6092) .. (-0.0794, 5.6092).. controls (-0.116, 5.8168) and (-0.116, 5.8168) .. (-0.0794, 5.9267).. controls (-0.038, 5.9746) and (-0.038, 5.9746) .. (0.0, 6.006) -- cycle;



  \path[fill=signature_color,shift={(8.4931, -2.1696)}] (0.0, 6.006).. controls (0.0, 5.9886) and (0.0, 5.9711) .. (0.0, 5.9531).. controls (-0.0175, 5.9531) and (-0.0349, 5.9531) .. (-0.0529, 5.9531).. controls (-0.0442, 5.9269) and (-0.0355, 5.9007) .. (-0.0265, 5.8738).. controls (-0.0527, 5.8607) and (-0.0527, 5.8607) .. (-0.0794, 5.8473).. controls (-0.0706, 5.8124) and (-0.0619, 5.7774) .. (-0.0529, 5.7415).. controls (-0.018, 5.7327) and (0.0169, 5.724) .. (0.0529, 5.715).. controls (-0.0446, 5.65) and (-0.1243, 5.6353) .. (-0.2381, 5.6092).. controls (-0.2381, 5.6266) and (-0.2381, 5.6441) .. (-0.2381, 5.6621).. controls (-0.2207, 5.6621) and (-0.2032, 5.6621) .. (-0.1852, 5.6621).. controls (-0.1934, 5.6774) and (-0.2016, 5.6926) .. (-0.21, 5.7084).. controls (-0.2362, 5.7638) and (-0.2518, 5.8141) .. (-0.2646, 5.8738).. controls (-0.2831, 5.865) and (-0.3017, 5.8563) .. (-0.3208, 5.8473).. controls (-0.4038, 5.8184) and (-0.4685, 5.8168) .. (-0.5556, 5.8208).. controls (-0.4671, 5.8799) and (-0.3779, 5.9104) .. (-0.2778, 5.9432).. controls (-0.2599, 5.9493) and (-0.242, 5.9553) .. (-0.2236, 5.9616).. controls (-0.1459, 5.9873) and (-0.0824, 6.006) .. (0.0, 6.006) -- cycle;



  \path[fill=signature_color,shift={(12.3296, -3.1485)}] (0.0, 6.006).. controls (0.0, 5.9886) and (0.0, 5.9711) .. (0.0, 5.9531).. controls (-0.0175, 5.9531) and (-0.0349, 5.9531) .. (-0.0529, 5.9531).. controls (-0.0616, 5.8658) and (-0.0704, 5.7785) .. (-0.0794, 5.6885).. controls (-0.027, 5.6973) and (0.0254, 5.706) .. (0.0794, 5.715).. controls (-0.0107, 5.6378) and (-0.1036, 5.6032) .. (-0.215, 5.5645).. controls (-0.2478, 5.553) and (-0.2806, 5.5414) .. (-0.3144, 5.5295).. controls (-0.3969, 5.5033) and (-0.3969, 5.5033) .. (-0.4498, 5.5033).. controls (-0.4498, 5.5208) and (-0.4498, 5.5383) .. (-0.4498, 5.5563).. controls (-0.4323, 5.5563) and (-0.4149, 5.5563) .. (-0.3969, 5.5563).. controls (-0.3969, 5.5824) and (-0.3969, 5.6086) .. (-0.3969, 5.6356).. controls (-0.3794, 5.6356) and (-0.362, 5.6356) .. (-0.344, 5.6356).. controls (-0.344, 5.6182) and (-0.344, 5.6007) .. (-0.344, 5.5827).. controls (-0.2342, 5.6456) and (-0.2342, 5.6456) .. (-0.1588, 5.7415).. controls (-0.2024, 5.7502) and (-0.2461, 5.7589) .. (-0.291, 5.7679).. controls (-0.2813, 5.8381) and (-0.2729, 5.8668) .. (-0.2176, 5.913).. controls (-0.1987, 5.9241) and (-0.1798, 5.9351) .. (-0.1604, 5.9465).. controls (-0.1416, 5.9578) and (-0.1228, 5.9692) .. (-0.1035, 5.9808).. controls (-0.0529, 6.006) and (-0.0529, 6.006) .. (0.0, 6.006) -- cycle;



  \path[fill=signature_color,shift={(2.884, -3.5454)}] (0.0, 6.006).. controls (0.0262, 5.9711) and (0.0524, 5.9362) .. (0.0794, 5.9002).. controls (0.0619, 5.8827) and (0.0445, 5.8653) .. (0.0265, 5.8473).. controls (0.009, 5.856) and (-0.0085, 5.8648) .. (-0.0265, 5.8738).. controls (-0.0236, 5.8584) and (-0.0208, 5.8431) .. (-0.0179, 5.8272).. controls (-0.0017, 5.7217) and (0.0037, 5.6485) .. (-0.0265, 5.5447).. controls (-0.0967, 5.4547) and (-0.154, 5.4437) .. (-0.2646, 5.424).. controls (-0.2908, 5.424) and (-0.317, 5.424) .. (-0.344, 5.424).. controls (-0.3258, 5.4527) and (-0.3076, 5.4814) .. (-0.2889, 5.511).. controls (-0.2648, 5.5492) and (-0.2407, 5.5875) .. (-0.2166, 5.6257).. controls (-0.2047, 5.6446) and (-0.1927, 5.6634) .. (-0.1804, 5.6829).. controls (-0.1142, 5.7882) and (-0.0535, 5.8935) .. (0.0, 6.006) -- cycle;



  \path[fill=signature_color,shift={(7.329, -1.0054)}] (0.0, 6.006).. controls (0.0441, 6.0072) and (0.0882, 6.0071) .. (0.1323, 6.006).. controls (0.1588, 5.9796) and (0.1588, 5.9796) .. (0.1654, 5.9018).. controls (0.1635, 5.7712) and (0.1208, 5.6541) .. (0.0794, 5.5314).. controls (0.0718, 5.5083) and (0.0642, 5.4852) .. (0.0564, 5.4613).. controls (0.0379, 5.4047) and (0.019, 5.3481) .. (0.0, 5.2917).. controls (-0.0087, 5.2917) and (-0.0175, 5.2917) .. (-0.0265, 5.2917).. controls (-0.0298, 5.4967) and (-0.0298, 5.4967) .. (0.0, 5.5563).. controls (0.0175, 5.5563) and (0.0349, 5.5563) .. (0.0529, 5.5563).. controls (0.0529, 5.6174) and (0.0529, 5.6785) .. (0.0529, 5.7415).. controls (0.0093, 5.7502) and (-0.0344, 5.7589) .. (-0.0794, 5.7679).. controls (-0.0646, 5.8564) and (-0.0436, 5.9275) .. (0.0, 6.006) -- cycle;



  \path[fill=signature_color,shift={(11.0596, -1.6669)}] (0.0, 6.006).. controls (0.0, 5.9886) and (0.0, 5.9711) .. (0.0, 5.9531).. controls (-0.0611, 5.9531) and (-0.1222, 5.9531) .. (-0.1852, 5.9531).. controls (-0.1939, 5.9095) and (-0.2027, 5.8658) .. (-0.2117, 5.8208).. controls (-0.1897, 5.8214) and (-0.1677, 5.8221) .. (-0.145, 5.8227).. controls (-0.1163, 5.8232) and (-0.0875, 5.8236) .. (-0.0579, 5.8241).. controls (-0.0293, 5.8248) and (-0.0008, 5.8254) .. (0.0286, 5.826).. controls (0.1058, 5.8208) and (0.1058, 5.8208) .. (0.1852, 5.7679).. controls (0.0368, 5.7679) and (-0.1117, 5.7679) .. (-0.2646, 5.7679).. controls (-0.2733, 5.8116) and (-0.282, 5.8552) .. (-0.291, 5.9002).. controls (-0.3784, 5.9002) and (-0.4657, 5.9002) .. (-0.5556, 5.9002).. controls (-0.5644, 5.8827) and (-0.5731, 5.8653) .. (-0.5821, 5.8473).. controls (-0.6345, 5.8473) and (-0.6869, 5.8473) .. (-0.7408, 5.8473).. controls (-0.7408, 5.8735) and (-0.7408, 5.8997) .. (-0.7408, 5.9267).. controls (-0.2429, 6.0178) and (-0.2429, 6.0178) .. (0.0, 6.006) -- cycle;



  \path[fill=signature_color,shift={(6.694, -4.6831)}] (0.0, 6.006).. controls (0.0529, 5.9531) and (0.0529, 5.9531) .. (0.0589, 5.8573).. controls (0.0589, 5.8161) and (0.0586, 5.775) .. (0.0581, 5.7338).. controls (0.058, 5.7122) and (0.0579, 5.6905) .. (0.0579, 5.6682).. controls (0.0576, 5.5989) and (0.0569, 5.5296) .. (0.0562, 5.4603).. controls (0.056, 5.4134) and (0.0557, 5.3665) .. (0.0555, 5.3196).. controls (0.0549, 5.2044) and (0.054, 5.0893) .. (0.0529, 4.9742).. controls (0.0355, 4.9742) and (0.018, 4.9742) .. (0.0, 4.9742).. controls (-0.028, 5.0582) and (-0.0295, 5.1177) .. (-0.029, 5.206).. controls (-0.029, 5.2356) and (-0.0289, 5.2652) .. (-0.0288, 5.2956).. controls (-0.0286, 5.3265) and (-0.0283, 5.3574) .. (-0.0281, 5.3892).. controls (-0.0279, 5.436) and (-0.0279, 5.436) .. (-0.0278, 5.4837).. controls (-0.0274, 5.5608) and (-0.027, 5.6379) .. (-0.0265, 5.715).. controls (-0.0177, 5.715) and (-0.009, 5.715) .. (0.0, 5.715).. controls (0.0, 5.7434) and (0.0, 5.7718) .. (0.0, 5.801).. controls (0.0, 5.8693) and (0.0, 5.9377) .. (0.0, 6.006) -- cycle;



  \path[fill=signature_color,shift={(1.9315, -4.789)}] (0.0, 6.006).. controls (0.0349, 5.9624) and (0.0699, 5.9187) .. (0.1058, 5.8738).. controls (0.086, 5.7861) and (0.086, 5.7861) .. (0.0529, 5.715).. controls (0.0878, 5.7063) and (0.1228, 5.6975) .. (0.1588, 5.6885).. controls (0.1247, 5.6663) and (0.0905, 5.6443) .. (0.0562, 5.6224).. controls (0.0372, 5.6101) and (0.0182, 5.5978) .. (-0.0014, 5.5852).. controls (-0.0529, 5.5563) and (-0.0529, 5.5563) .. (-0.1058, 5.5563).. controls (-0.1058, 5.5737) and (-0.1058, 5.5912) .. (-0.1058, 5.6092).. controls (-0.0884, 5.6092) and (-0.0709, 5.6092) .. (-0.0529, 5.6092).. controls (-0.0616, 5.6877) and (-0.0704, 5.7663) .. (-0.0794, 5.8473).. controls (-0.1318, 5.8211) and (-0.1318, 5.8211) .. (-0.1852, 5.7944).. controls (-0.2201, 5.8031) and (-0.2551, 5.8118) .. (-0.291, 5.8208).. controls (-0.195, 5.882) and (-0.099, 5.9431) .. (0.0, 6.006) -- cycle;



  \path[fill=signature_color,shift={(13.7054, -2.0902)}] (0.0, 6.006).. controls (0.0175, 5.9973) and (0.0349, 5.9886) .. (0.0529, 5.9796).. controls (0.0529, 5.9534) and (0.0529, 5.9272) .. (0.0529, 5.9002).. controls (0.0878, 5.9177) and (0.1228, 5.9351) .. (0.1588, 5.9531).. controls (0.1781, 5.8949) and (0.1781, 5.8949) .. (0.1852, 5.8208).. controls (0.1452, 5.7558) and (0.1452, 5.7558) .. (0.0876, 5.6935).. controls (0.069, 5.6726) and (0.0503, 5.6517) .. (0.0311, 5.6301).. controls (-0.0265, 5.5827) and (-0.0265, 5.5827) .. (-0.1323, 5.5827).. controls (-0.1255, 5.6247) and (-0.1182, 5.6665) .. (-0.1108, 5.7084).. controls (-0.1048, 5.7434) and (-0.1048, 5.7434) .. (-0.0987, 5.7791).. controls (-0.0779, 5.8526) and (-0.047, 5.8939) .. (0.0, 5.9531).. controls (0.0, 5.9706) and (0.0, 5.9881) .. (0.0, 6.006) -- cycle;



  \path[fill=signature_color,shift={(9.6838, -3.0163)}] (0.0, 6.006).. controls (0.0529, 5.9531) and (0.0529, 5.9531) .. (0.0529, 5.9035).. controls (0.0132, 5.819) and (-0.0543, 5.7898) .. (-0.1323, 5.7415).. controls (-0.1798, 5.6985) and (-0.1798, 5.6985) .. (-0.2117, 5.6621).. controls (-0.254, 5.768) and (-0.2194, 5.8485) .. (-0.1852, 5.9531).. controls (-0.2027, 5.9531) and (-0.2201, 5.9531) .. (-0.2381, 5.9531).. controls (-0.2381, 5.9357) and (-0.2381, 5.9182) .. (-0.2381, 5.9002).. controls (-0.2992, 5.8915) and (-0.3604, 5.8827) .. (-0.4233, 5.8738).. controls (-0.2941, 5.9915) and (-0.1731, 6.0205) .. (0.0, 6.006) -- cycle;



  \path[fill=signature_color,shift={(13.335, -2.4606)}] (0.0, 6.006).. controls (0.0342, 5.9376) and (0.0216, 5.9079) .. (0.005, 5.8341).. controls (0.0004, 5.8131) and (-0.0041, 5.792) .. (-0.0088, 5.7704).. controls (-0.0265, 5.715) and (-0.0265, 5.715) .. (-0.0794, 5.6621).. controls (-0.0788, 5.6403) and (-0.0783, 5.6184) .. (-0.0777, 5.5959).. controls (-0.0783, 5.5741) and (-0.0788, 5.5523) .. (-0.0794, 5.5298).. controls (-0.1588, 5.4769) and (-0.1588, 5.4769) .. (-0.2166, 5.4868).. controls (-0.2404, 5.495) and (-0.2404, 5.495) .. (-0.2646, 5.5033).. controls (-0.2646, 5.5383) and (-0.2646, 5.5732) .. (-0.2646, 5.6092).. controls (-0.2384, 5.6092) and (-0.2122, 5.6092) .. (-0.1852, 5.6092).. controls (-0.1837, 5.6364) and (-0.1837, 5.6364) .. (-0.1822, 5.6643).. controls (-0.1796, 5.7) and (-0.1796, 5.7) .. (-0.1769, 5.7365).. controls (-0.1754, 5.7601) and (-0.1739, 5.7838) .. (-0.1723, 5.8081).. controls (-0.15, 5.9164) and (-0.0895, 5.9463) .. (0.0, 6.006) -- cycle;



  \path[fill=signature_color,shift={(4.9742, -3.5983)}] (0.0, 6.006).. controls (0.0262, 5.9973) and (0.0524, 5.9886) .. (0.0794, 5.9796).. controls (0.0987, 5.9307) and (0.0987, 5.9307) .. (0.1108, 5.8721).. controls (0.115, 5.8527) and (0.1192, 5.8333) .. (0.1235, 5.8133).. controls (0.1264, 5.7983) and (0.1293, 5.7833) .. (0.1323, 5.7679).. controls (0.1148, 5.7679) and (0.0974, 5.7679) .. (0.0794, 5.7679).. controls (0.0619, 5.7854) and (0.0445, 5.8028) .. (0.0265, 5.8208).. controls (0.0224, 5.7979) and (0.0224, 5.7979) .. (0.0182, 5.7745).. controls (-0.0026, 5.7065) and (-0.0345, 5.6638) .. (-0.0794, 5.6092).. controls (-0.2084, 5.7613) and (-0.2084, 5.7613) .. (-0.2381, 5.8208).. controls (-0.1945, 5.8296) and (-0.1508, 5.8383) .. (-0.1058, 5.8473).. controls (-0.0971, 5.8735) and (-0.0884, 5.8997) .. (-0.0794, 5.9267).. controls (-0.0619, 5.9267) and (-0.0445, 5.9267) .. (-0.0265, 5.9267).. controls (-0.0177, 5.9529) and (-0.009, 5.9791) .. (0.0, 6.006) -- cycle;



  \path[fill=signature_color,shift={(10.5569, -3.519)}] (0.0, 6.006).. controls (0.1921, 6.006) and (0.3842, 6.006) .. (0.5821, 6.006).. controls (0.5821, 5.9973) and (0.5821, 5.9886) .. (0.5821, 5.9796).. controls (0.4424, 5.9621) and (0.3027, 5.9447) .. (0.1588, 5.9267).. controls (0.1675, 5.8917) and (0.1762, 5.8568) .. (0.1852, 5.8208).. controls (0.2027, 5.8121) and (0.2201, 5.8034) .. (0.2381, 5.7944).. controls (0.2294, 5.7682) and (0.2207, 5.742) .. (0.2117, 5.715).. controls (-0.0151, 5.7074) and (-0.0151, 5.7074) .. (-0.1058, 5.7679).. controls (-0.0622, 5.7766) and (-0.0185, 5.7854) .. (0.0265, 5.7944).. controls (0.0265, 5.8118) and (0.0265, 5.8293) .. (0.0265, 5.8473).. controls (0.0439, 5.8473) and (0.0614, 5.8473) .. (0.0794, 5.8473).. controls (0.0881, 5.8822) and (0.0968, 5.9171) .. (0.1058, 5.9531).. controls (0.0709, 5.9531) and (0.036, 5.9531) .. (0.0, 5.9531).. controls (0.0, 5.9706) and (0.0, 5.9881) .. (0.0, 6.006) -- cycle;



  \path[fill=signature_color,shift={(5.8208, -3.8365)}] (0.0, 6.006).. controls (0.1397, 5.9973) and (0.2794, 5.9886) .. (0.4233, 5.9796).. controls (0.3704, 5.9267) and (0.3704, 5.9267) .. (0.301, 5.9234).. controls (0.2802, 5.9245) and (0.2595, 5.9255) .. (0.2381, 5.9267).. controls (0.2207, 5.8743) and (0.2032, 5.8219) .. (0.1852, 5.7679).. controls (0.2551, 5.7679) and (0.3249, 5.7679) .. (0.3969, 5.7679).. controls (0.3969, 5.7505) and (0.3969, 5.733) .. (0.3969, 5.715).. controls (0.2134, 5.709) and (0.0472, 5.7303) .. (-0.1323, 5.7679).. controls (-0.1236, 5.7854) and (-0.1148, 5.8028) .. (-0.1058, 5.8208).. controls (-0.0273, 5.8208) and (0.0513, 5.8208) .. (0.1323, 5.8208).. controls (0.1498, 5.8558) and (0.1672, 5.8907) .. (0.1852, 5.9267).. controls (0.1939, 5.9354) and (0.2027, 5.9441) .. (0.2117, 5.9531).. controls (0.1418, 5.9531) and (0.072, 5.9531) .. (0.0, 5.9531).. controls (0.0, 5.9706) and (0.0, 5.9881) .. (0.0, 6.006) -- cycle;



  \path[fill=signature_color,shift={(8.4931, -2.1696)}] (0.0, 6.006).. controls (0.0202, 6.0017) and (0.0404, 5.9973) .. (0.0612, 5.9928).. controls (0.1332, 5.9772) and (0.1332, 5.9772) .. (0.2117, 5.9796).. controls (0.2029, 5.9185) and (0.1942, 5.8573) .. (0.1852, 5.7944).. controls (0.1639, 5.7911) and (0.1426, 5.7878) .. (0.1207, 5.7845).. controls (0.0983, 5.779) and (0.076, 5.7735) .. (0.0529, 5.7679).. controls (0.0442, 5.7505) and (0.0355, 5.733) .. (0.0265, 5.715).. controls (-0.0172, 5.7237) and (-0.0609, 5.7325) .. (-0.1058, 5.7415).. controls (-0.092, 5.8146) and (-0.0765, 5.8824) .. (-0.0529, 5.9531).. controls (-0.0355, 5.9531) and (-0.018, 5.9531) .. (0.0, 5.9531).. controls (0.0, 5.9706) and (0.0, 5.9881) .. (0.0, 6.006) -- cycle(0.0794, 5.9267).. controls (0.1058, 5.8738) and (0.1058, 5.8738) .. (0.1058, 5.8738) -- cycle;



  \path[fill=signature_color,shift={(0.6085, -4.9477)}] (0.0, 6.006).. controls (-0.0393, 5.9406) and (-0.0393, 5.9406) .. (-0.0794, 5.8738).. controls (-0.027, 5.8126) and (0.0254, 5.7515) .. (0.0794, 5.6885).. controls (0.0968, 5.6973) and (0.1143, 5.706) .. (0.1323, 5.715).. controls (0.1323, 5.6801) and (0.1323, 5.6452) .. (0.1323, 5.6092).. controls (-0.021, 5.6437) and (-0.1406, 5.7249) .. (-0.2381, 5.8473).. controls (-0.2381, 5.9382) and (-0.2381, 5.9382) .. (-0.2117, 6.006).. controls (-0.1309, 6.032) and (-0.0808, 6.032) .. (0.0, 6.006) -- cycle;



  \path[fill=signature_color,shift={(0.253, -4.6021)}] (0.0, 6.006).. controls (0.0319, 6.0052) and (0.0319, 6.0052) .. (0.0645, 6.0044).. controls (0.0732, 5.9782) and (0.082, 5.952) .. (0.091, 5.925).. controls (0.1171, 5.9163) and (0.1433, 5.9076) .. (0.1703, 5.8986).. controls (0.1703, 5.8811) and (0.1703, 5.8636) .. (0.1703, 5.8456).. controls (0.1878, 5.8456) and (0.2053, 5.8456) .. (0.2232, 5.8456).. controls (0.2145, 5.8107) and (0.2058, 5.7758) .. (0.1968, 5.7398).. controls (0.1673, 5.7406) and (0.1673, 5.7406) .. (0.1373, 5.7415).. controls (0.0723, 5.74) and (0.0205, 5.7317) .. (-0.0413, 5.7133).. controls (-0.0402, 5.7374) and (-0.0392, 5.7614) .. (-0.038, 5.7861).. controls (-0.0412, 5.8689) and (-0.0517, 5.9088) .. (-0.0943, 5.9779).. controls (-0.0678, 6.0044) and (-0.0678, 6.0044) .. (0.0, 6.006) -- cycle;



  \path[fill=signature_color,shift={(0.1588, -3.6777)}] (0.0, 6.006).. controls (0.0087, 6.006) and (0.0175, 6.006) .. (0.0265, 6.006).. controls (0.0352, 5.9449) and (0.0439, 5.8838) .. (0.0529, 5.8208).. controls (0.0005, 5.7946) and (0.0005, 5.7946) .. (-0.0529, 5.7679).. controls (-0.0355, 5.7505) and (-0.018, 5.733) .. (0.0, 5.715).. controls (0.0, 5.6801) and (0.0, 5.6452) .. (0.0, 5.6092).. controls (-0.0175, 5.6092) and (-0.0349, 5.6092) .. (-0.0529, 5.6092).. controls (-0.0442, 5.5742) and (-0.0355, 5.5393) .. (-0.0265, 5.5033).. controls (-0.0439, 5.4946) and (-0.0614, 5.4859) .. (-0.0794, 5.4769).. controls (-0.0968, 5.3967) and (-0.0968, 5.3967) .. (-0.1058, 5.3181).. controls (-0.0796, 5.3094) and (-0.0534, 5.3007) .. (-0.0265, 5.2917).. controls (0.0071, 5.2194) and (0.0011, 5.1892) .. (-0.0248, 5.1131).. controls (-0.0341, 5.0934) and (-0.0434, 5.0738) .. (-0.0529, 5.0535).. controls (-0.1989, 5.3163) and (-0.1244, 5.5913) .. (-0.0522, 5.8637).. controls (-0.0265, 5.9531) and (-0.0265, 5.9531) .. (0.0, 6.006) -- cycle;



  \path[fill=signature_color,shift={(7.0908, -1.9315)}] (0.0, 6.006).. controls (0.0087, 5.9886) and (0.0175, 5.9711) .. (0.0265, 5.9531).. controls (0.0439, 5.9531) and (0.0614, 5.9531) .. (0.0794, 5.9531).. controls (0.0801, 5.9096) and (0.0806, 5.866) .. (0.081, 5.8225).. controls (0.0813, 5.7982) and (0.0816, 5.774) .. (0.082, 5.749).. controls (0.0794, 5.6885) and (0.0794, 5.6885) .. (0.0529, 5.6621).. controls (-0.0177, 5.6583) and (-0.088, 5.661) .. (-0.1588, 5.6621).. controls (-0.152, 5.7013) and (-0.1448, 5.7404) .. (-0.1373, 5.7795).. controls (-0.1333, 5.8013) and (-0.1293, 5.8231) .. (-0.1252, 5.8455).. controls (-0.1188, 5.8636) and (-0.1124, 5.8816) .. (-0.1058, 5.9002).. controls (-0.0796, 5.9089) and (-0.0534, 5.9177) .. (-0.0265, 5.9267).. controls (-0.0177, 5.9529) and (-0.009, 5.9791) .. (0.0, 6.006) -- cycle;



  \path[fill=signature_color,shift={(1.8521, -4.9477)}] (0.0, 6.006).. controls (0.0562, 5.8572) and (0.0562, 5.8572) .. (0.0265, 5.7679).. controls (0.009, 5.7679) and (-0.0085, 5.7679) .. (-0.0265, 5.7679).. controls (-0.0265, 5.7417) and (-0.0265, 5.7155) .. (-0.0265, 5.6885).. controls (-0.0981, 5.6527) and (-0.159, 5.6604) .. (-0.2381, 5.6621).. controls (-0.2185, 5.6858) and (-0.2185, 5.6858) .. (-0.1984, 5.71).. controls (-0.1513, 5.7671) and (-0.1513, 5.7671) .. (-0.1588, 5.8473).. controls (-0.1356, 5.9167) and (-0.1356, 5.9167) .. (-0.1058, 5.9796).. controls (-0.0529, 6.006) and (-0.0529, 6.006) .. (0.0, 6.006) -- cycle;



  \path[fill=signature_color,shift={(4.2069, -4.3127)}] (0.0, 6.006).. controls (0.0175, 5.9711) and (0.0349, 5.9362) .. (0.0529, 5.9002).. controls (0.0267, 5.8827) and (0.0005, 5.8653) .. (-0.0265, 5.8473).. controls (-0.0418, 5.7892) and (-0.0418, 5.7892) .. (-0.0446, 5.7249).. controls (-0.0464, 5.7033) and (-0.0481, 5.6817) .. (-0.0499, 5.6594).. controls (-0.0509, 5.6428) and (-0.0519, 5.6262) .. (-0.0529, 5.6092).. controls (-0.0878, 5.6004) and (-0.1228, 5.5917) .. (-0.1588, 5.5827).. controls (-0.1872, 5.6529) and (-0.2142, 5.7225) .. (-0.2381, 5.7944).. controls (-0.1904, 5.8497) and (-0.1587, 5.8738) .. (-0.0893, 5.9002).. controls (-0.0582, 5.9133) and (-0.0582, 5.9133) .. (-0.0265, 5.9267).. controls (-0.0177, 5.9529) and (-0.009, 5.9791) .. (0.0, 6.006) -- cycle;



  \path[fill=signature_color,shift={(5.3975, -3.6248)}] (0.0, 6.006).. controls (0.0175, 5.9973) and (0.0349, 5.9886) .. (0.0529, 5.9796).. controls (0.0529, 5.9621) and (0.0529, 5.9447) .. (0.0529, 5.9267).. controls (0.2352, 5.8629) and (0.2352, 5.8629) .. (0.4233, 5.8208).. controls (0.4233, 5.8121) and (0.4233, 5.8034) .. (0.4233, 5.7944).. controls (0.2183, 5.7911) and (0.2183, 5.7911) .. (0.1588, 5.8208).. controls (0.1631, 5.8056) and (0.1675, 5.7903) .. (0.172, 5.7745).. controls (0.1852, 5.715) and (0.1852, 5.715) .. (0.1852, 5.6092).. controls (0.0658, 5.6232) and (-0.0092, 5.6718) .. (-0.1058, 5.7415).. controls (-0.0884, 5.7502) and (-0.0709, 5.7589) .. (-0.0529, 5.7679).. controls (-0.0442, 5.7854) and (-0.0355, 5.8028) .. (-0.0265, 5.8208).. controls (-0.0177, 5.7946) and (-0.009, 5.7684) .. (0.0, 5.7415).. controls (0.0349, 5.7415) and (0.0698, 5.7415) .. (0.1058, 5.7415).. controls (0.0749, 5.8424) and (0.0749, 5.8424) .. (-0.0033, 5.9085).. controls (-0.0279, 5.9175) and (-0.0279, 5.9175) .. (-0.0529, 5.9267).. controls (-0.0355, 5.9529) and (-0.018, 5.9791) .. (0.0, 6.006) -- cycle;



  \path[fill=signature_color,shift={(0.3175, -3.7571)}] (0.0, 6.006).. controls (0.0175, 5.9973) and (0.0349, 5.9886) .. (0.0529, 5.9796).. controls (0.0442, 5.9359) and (0.0355, 5.8923) .. (0.0265, 5.8473).. controls (0.0614, 5.856) and (0.0963, 5.8648) .. (0.1323, 5.8738).. controls (0.1236, 5.8437) and (0.1148, 5.8137) .. (0.1058, 5.7828).. controls (0.0612, 5.5874) and (0.0753, 5.3849) .. (0.0794, 5.1858).. controls (0.0619, 5.1946) and (0.0444, 5.2033) .. (0.0265, 5.2123).. controls (0.0177, 5.4044) and (0.009, 5.5965) .. (0.0, 5.7944).. controls (-0.0655, 5.7551) and (-0.0655, 5.7551) .. (-0.1323, 5.715).. controls (-0.1204, 5.7586) and (-0.1082, 5.8021) .. (-0.0959, 5.8456).. controls (-0.0892, 5.8699) and (-0.0824, 5.8941) .. (-0.0754, 5.9191).. controls (-0.0529, 5.9796) and (-0.0529, 5.9796) .. (0.0, 6.006) -- cycle;



  \path[fill=signature_color,shift={(11.1125, -1.6669)}] (0.0, 6.006).. controls (0.1939, 5.9902) and (0.1939, 5.9902) .. (0.2646, 5.9482).. controls (0.3029, 5.8787) and (0.2962, 5.8192) .. (0.291, 5.7415).. controls (0.1918, 5.7117) and (0.1918, 5.7117) .. (0.1323, 5.7415).. controls (0.1323, 5.7589) and (0.1323, 5.7764) .. (0.1323, 5.7944).. controls (0.1061, 5.8031) and (0.0799, 5.8118) .. (0.0529, 5.8208).. controls (0.0704, 5.8208) and (0.0878, 5.8208) .. (0.1058, 5.8208).. controls (0.0971, 5.8645) and (0.0884, 5.9081) .. (0.0794, 5.9531).. controls (0.0532, 5.9531) and (0.027, 5.9531) .. (0.0, 5.9531).. controls (0.0, 5.9706) and (0.0, 5.9881) .. (0.0, 6.006) -- cycle;



  \path[fill=signature_color,shift={(10.2923, -1.7727)}] (0.0, 6.006).. controls (0.0, 5.9886) and (0.0, 5.9711) .. (0.0, 5.9531).. controls (-0.0175, 5.9531) and (-0.0349, 5.9531) .. (-0.0529, 5.9531).. controls (-0.037, 5.8997) and (-0.0192, 5.8467) .. (0.0, 5.7944).. controls (0.0175, 5.7856) and (0.0349, 5.7769) .. (0.0529, 5.7679).. controls (-0.0169, 5.7592) and (-0.0868, 5.7505) .. (-0.1588, 5.7415).. controls (-0.15, 5.7677) and (-0.1413, 5.7938) .. (-0.1323, 5.8208).. controls (-0.1672, 5.8383) and (-0.2021, 5.8558) .. (-0.2381, 5.8738).. controls (-0.2381, 5.8912) and (-0.2381, 5.9087) .. (-0.2381, 5.9267).. controls (-0.308, 5.9267) and (-0.3778, 5.9267) .. (-0.4498, 5.9267).. controls (-0.4498, 5.9354) and (-0.4498, 5.9441) .. (-0.4498, 5.9531).. controls (-0.1449, 6.0139) and (-0.1449, 6.0139) .. (0.0, 6.006) -- cycle;



  \path[fill=signature_color,shift={(5.2917, -3.5719)}] (0.0, 6.006).. controls (0.0087, 5.9886) and (0.0175, 5.9711) .. (0.0265, 5.9531).. controls (0.0527, 5.9531) and (0.0788, 5.9531) .. (0.1058, 5.9531).. controls (0.0971, 5.9269) and (0.0884, 5.9007) .. (0.0794, 5.8738).. controls (0.1007, 5.8628) and (0.1219, 5.8519) .. (0.1439, 5.8407).. controls (0.2117, 5.7944) and (0.2117, 5.7944) .. (0.2365, 5.7398).. controls (0.237, 5.7229) and (0.2376, 5.706) .. (0.2381, 5.6885).. controls (0.2207, 5.6711) and (0.2032, 5.6536) .. (0.1852, 5.6356).. controls (0.1645, 5.6487) and (0.1437, 5.6618) .. (0.1224, 5.6753).. controls (0.0586, 5.7194) and (0.0586, 5.7194) .. (0.0, 5.715).. controls (0.0, 5.7412) and (0.0, 5.7674) .. (0.0, 5.7944).. controls (-0.0513, 5.8523) and (-0.0513, 5.8523) .. (-0.1058, 5.9002).. controls (-0.0796, 5.9089) and (-0.0534, 5.9177) .. (-0.0265, 5.9267).. controls (-0.0177, 5.9529) and (-0.009, 5.9791) .. (0.0, 6.006) -- cycle;



  \path[fill=signature_color,shift={(2.0373, -1.905)}] (0.0, 6.006).. controls (0.0011, 5.9531) and (0.0011, 5.9002) .. (0.0, 5.8473).. controls (-0.0087, 5.8386) and (-0.0175, 5.8298) .. (-0.0265, 5.8208).. controls (-0.0362, 5.7592) and (-0.045, 5.6975) .. (-0.0529, 5.6356).. controls (-0.0704, 5.6356) and (-0.0878, 5.6356) .. (-0.1058, 5.6356).. controls (-0.1058, 5.6531) and (-0.1058, 5.6706) .. (-0.1058, 5.6885).. controls (-0.1408, 5.6885) and (-0.1757, 5.6885) .. (-0.2117, 5.6885).. controls (-0.2204, 5.7409) and (-0.2291, 5.7933) .. (-0.2381, 5.8473).. controls (-0.2212, 5.8544) and (-0.2043, 5.8615) .. (-0.1869, 5.8688).. controls (-0.1235, 5.8972) and (-0.1235, 5.8972) .. (-0.1058, 5.9796).. controls (-0.0529, 6.006) and (-0.0529, 6.006) .. (0.0, 6.006) -- cycle;



  \path[fill=signature_color,shift={(3.519, -2.0902)}] (0.0, 6.006).. controls (0.0531, 5.9632) and (0.0531, 5.9632) .. (0.1058, 5.9002).. controls (0.1122, 5.8249) and (0.1122, 5.8249) .. (0.1025, 5.7398).. controls (0.0992, 5.7097) and (0.0959, 5.6795) .. (0.0925, 5.6484).. controls (0.0742, 5.5199) and (0.0527, 5.3924) .. (0.0265, 5.2652).. controls (-0.0502, 5.3578) and (-0.0612, 5.4089) .. (-0.0529, 5.5298).. controls (-0.0355, 5.5298) and (-0.018, 5.5298) .. (0.0, 5.5298).. controls (0.0087, 5.5036) and (0.0175, 5.4774) .. (0.0265, 5.4504).. controls (0.0265, 5.4941) and (0.0265, 5.5377) .. (0.0265, 5.5827).. controls (0.0003, 5.5914) and (-0.0259, 5.6002) .. (-0.0529, 5.6092).. controls (-0.018, 5.6179) and (0.0169, 5.6266) .. (0.0529, 5.6356).. controls (0.0492, 5.6775) and (0.0453, 5.7194) .. (0.0413, 5.7613).. controls (0.0392, 5.7846) and (0.037, 5.808) .. (0.0348, 5.832).. controls (0.0275, 5.8921) and (0.0162, 5.9478) .. (0.0, 6.006) -- cycle(-0.1058, 5.6356).. controls (-0.0884, 5.6182) and (-0.0709, 5.6007) .. (-0.0529, 5.5827).. controls (-0.0704, 5.5827) and (-0.0878, 5.5827) .. (-0.1058, 5.5827).. controls (-0.1058, 5.6002) and (-0.1058, 5.6176) .. (-0.1058, 5.6356) -- cycle;



  \path[fill=signature_color,shift={(0.3704, -3.3338)}] (0.0, 6.006).. controls (0.0421, 5.852) and (-0.0167, 5.7508) .. (-0.0794, 5.6092).. controls (-0.0968, 5.6092) and (-0.1143, 5.6092) .. (-0.1323, 5.6092).. controls (-0.1236, 5.5742) and (-0.1148, 5.5393) .. (-0.1058, 5.5033).. controls (-0.132, 5.5121) and (-0.1582, 5.5208) .. (-0.1852, 5.5298).. controls (-0.2159, 5.7272) and (-0.1108, 5.8497) .. (0.0, 6.006) -- cycle;



  \path[fill=signature_color,shift={(3.2279, -4.789)}] (0.0, 6.006).. controls (0.0087, 5.9711) and (0.0175, 5.9362) .. (0.0265, 5.9002).. controls (0.105, 5.9177) and (0.1836, 5.9351) .. (0.2646, 5.9531).. controls (0.2381, 5.8473) and (0.2381, 5.8473) .. (0.21, 5.7811).. controls (0.179, 5.7103) and (0.179, 5.7103) .. (0.1852, 5.6092).. controls (0.1416, 5.6092) and (0.0979, 5.6092) .. (0.0529, 5.6092).. controls (0.0529, 5.6441) and (0.0529, 5.679) .. (0.0529, 5.715).. controls (0.0704, 5.715) and (0.0878, 5.715) .. (0.1058, 5.715).. controls (0.1058, 5.7587) and (0.1058, 5.8023) .. (0.1058, 5.8473).. controls (0.0709, 5.8473) and (0.036, 5.8473) .. (0.0, 5.8473).. controls (0.0, 5.8997) and (0.0, 5.9521) .. (0.0, 6.006) -- cycle;



  \path[fill=signature_color,shift={(9.1281, -2.0373)}] (0.0, 6.006).. controls (0.0, 5.9886) and (0.0, 5.9711) .. (0.0, 5.9531).. controls (-0.0262, 5.9531) and (-0.0524, 5.9531) .. (-0.0794, 5.9531).. controls (-0.0881, 5.8833) and (-0.0968, 5.8134) .. (-0.1058, 5.7415).. controls (-0.1582, 5.7415) and (-0.2106, 5.7415) .. (-0.2646, 5.7415).. controls (-0.2901, 5.8103) and (-0.2914, 5.8459) .. (-0.2745, 5.9184).. controls (-0.2381, 5.9796) and (-0.2381, 5.9796) .. (-0.1753, 6.0011).. controls (-0.117, 6.0052) and (-0.0585, 6.006) .. (0.0, 6.006) -- cycle(-0.2381, 5.8473).. controls (-0.2117, 5.7944) and (-0.2117, 5.7944) .. (-0.2117, 5.7944) -- cycle;



  \path[fill=signature_color,shift={(2.3548, -1.5875)}] (0.0, 6.006).. controls (0.0087, 5.9798) and (0.0175, 5.9537) .. (0.0265, 5.9267).. controls (0.0003, 5.9092) and (-0.0259, 5.8917) .. (-0.0529, 5.8738).. controls (-0.0724, 5.7794) and (-0.0724, 5.7794) .. (-0.0794, 5.6885).. controls (-0.0881, 5.706) and (-0.0968, 5.7235) .. (-0.1058, 5.7415).. controls (-0.1233, 5.7415) and (-0.1408, 5.7415) .. (-0.1588, 5.7415).. controls (-0.1675, 5.724) and (-0.1762, 5.7065) .. (-0.1852, 5.6885).. controls (-0.1939, 5.706) and (-0.2027, 5.7235) .. (-0.2117, 5.7415).. controls (-0.2379, 5.7284) and (-0.2379, 5.7284) .. (-0.2646, 5.715).. controls (-0.2733, 5.6888) and (-0.282, 5.6626) .. (-0.291, 5.6356).. controls (-0.2998, 5.6531) and (-0.3085, 5.6706) .. (-0.3175, 5.6885).. controls (-0.3524, 5.6798) and (-0.3874, 5.6711) .. (-0.4233, 5.6621).. controls (-0.2993, 5.7989) and (-0.1504, 5.8999) .. (0.0, 6.006) -- cycle;



  \path[fill=signature_color,shift={(3.4131, -2.0902)}] (0.0, 6.006).. controls (0.0, 5.9798) and (0.0, 5.9537) .. (0.0, 5.9267).. controls (0.0262, 5.9092) and (0.0524, 5.8917) .. (0.0794, 5.8738).. controls (0.0747, 5.8108) and (0.0609, 5.7758) .. (0.0158, 5.7314).. controls (-0.024, 5.6986) and (-0.0649, 5.667) .. (-0.1058, 5.6356).. controls (-0.1233, 5.6444) and (-0.1408, 5.6531) .. (-0.1588, 5.6621).. controls (-0.1455, 5.8013) and (-0.1003, 5.9058) .. (0.0, 6.006) -- cycle;



  \path[fill=signature_color,shift={(9.869, -3.2808)}] (0.0, 6.006).. controls (0.0442, 5.9864) and (0.0442, 5.9864) .. (0.0893, 5.9664).. controls (0.1761, 5.9233) and (0.1761, 5.9233) .. (0.2381, 5.9267).. controls (0.2381, 5.8917) and (0.2381, 5.8568) .. (0.2381, 5.8208).. controls (0.2119, 5.8208) and (0.1857, 5.8208) .. (0.1588, 5.8208).. controls (0.1588, 5.7946) and (0.1588, 5.7684) .. (0.1588, 5.7415).. controls (0.1151, 5.7502) and (0.0714, 5.7589) .. (0.0265, 5.7679).. controls (0.0177, 5.8028) and (0.009, 5.8378) .. (0.0, 5.8738).. controls (-0.0262, 5.8738) and (-0.0524, 5.8738) .. (-0.0794, 5.8738).. controls (-0.0794, 5.8912) and (-0.0794, 5.9087) .. (-0.0794, 5.9267).. controls (-0.0619, 5.9267) and (-0.0445, 5.9267) .. (-0.0265, 5.9267).. controls (-0.0177, 5.9529) and (-0.009, 5.9791) .. (0.0, 6.006) -- cycle;



  \path[fill=signature_color,shift={(0.3969, -3.5719)}] (0.0, 6.006).. controls (0.0262, 5.9973) and (0.0524, 5.9886) .. (0.0794, 5.9796).. controls (0.0881, 5.9534) and (0.0968, 5.9272) .. (0.1058, 5.9002).. controls (0.0884, 5.8827) and (0.0709, 5.8653) .. (0.0529, 5.8473).. controls (0.0529, 5.8298) and (0.0529, 5.8124) .. (0.0529, 5.7944).. controls (0.0445, 5.7502) and (0.0357, 5.7061) .. (0.0265, 5.6621).. controls (0.0003, 5.6621) and (-0.0259, 5.6621) .. (-0.0529, 5.6621).. controls (-0.0529, 5.7057) and (-0.0529, 5.7494) .. (-0.0529, 5.7944).. controls (-0.0878, 5.7856) and (-0.1228, 5.7769) .. (-0.1588, 5.7679).. controls (-0.1588, 5.7941) and (-0.1588, 5.8203) .. (-0.1588, 5.8473).. controls (-0.1413, 5.8473) and (-0.1238, 5.8473) .. (-0.1058, 5.8473).. controls (-0.1058, 5.8648) and (-0.1058, 5.8822) .. (-0.1058, 5.9002).. controls (-0.0884, 5.9002) and (-0.0709, 5.9002) .. (-0.0529, 5.9002).. controls (-0.0442, 5.9264) and (-0.0355, 5.9526) .. (-0.0265, 5.9796).. controls (-0.0177, 5.9883) and (-0.009, 5.997) .. (0.0, 6.006) -- cycle;



  \path[fill=signature_color,shift={(8.4138, -3.4131)}] (0.0, 6.006).. controls (0.0349, 6.006) and (0.0698, 6.006) .. (0.1058, 6.006).. controls (0.1146, 5.9886) and (0.1233, 5.9711) .. (0.1323, 5.9531).. controls (0.1585, 5.9444) and (0.1847, 5.9357) .. (0.2117, 5.9267).. controls (0.2166, 5.8638) and (0.2166, 5.8638) .. (0.2117, 5.7944).. controls (0.1323, 5.7415) and (0.1323, 5.7415) .. (0.0612, 5.7514).. controls (0.041, 5.7568) and (0.0208, 5.7623) .. (0.0, 5.7679).. controls (0.0049, 5.7876) and (0.0098, 5.8072) .. (0.0149, 5.8274).. controls (0.0262, 5.8989) and (0.0225, 5.9385) .. (0.0, 6.006) -- cycle;



  \path[fill=signature_color,shift={(3.9158, -0.6879)}] (0.0, 6.006).. controls (0.0349, 5.9973) and (0.0698, 5.9886) .. (0.1058, 5.9796).. controls (0.1058, 5.9359) and (0.1058, 5.8923) .. (0.1058, 5.8473).. controls (0.0182, 5.8174) and (-0.0685, 5.7887) .. (-0.1588, 5.7679).. controls (-0.1675, 5.8028) and (-0.1762, 5.8378) .. (-0.1852, 5.8738).. controls (-0.0893, 5.9531) and (-0.0893, 5.9531) .. (0.0, 5.9531).. controls (0.0, 5.9706) and (0.0, 5.9881) .. (0.0, 6.006) -- cycle;



  \path[fill=signature_color,shift={(6.7733, -3.2808)}] (0.0, 6.006).. controls (0.0529, 5.9796) and (0.0529, 5.9796) .. (0.0529, 5.9796) -- cycle(-0.1588, 5.9531).. controls (-0.1413, 5.9269) and (-0.1238, 5.9007) .. (-0.1058, 5.8738).. controls (-0.132, 5.8738) and (-0.1582, 5.8738) .. (-0.1852, 5.8738).. controls (-0.1765, 5.8999) and (-0.1677, 5.9261) .. (-0.1588, 5.9531) -- cycle(-0.1058, 5.9531).. controls (-0.0813, 5.9458) and (-0.0813, 5.9458) .. (-0.0562, 5.9382).. controls (0.0011, 5.9204) and (0.0011, 5.9204) .. (0.0529, 5.9531).. controls (0.0616, 5.8571) and (0.0704, 5.761) .. (0.0794, 5.6621).. controls (0.0619, 5.6621) and (0.0445, 5.6621) .. (0.0265, 5.6621).. controls (0.0265, 5.6795) and (0.0265, 5.697) .. (0.0265, 5.715).. controls (0.009, 5.715) and (-0.0085, 5.715) .. (-0.0265, 5.715).. controls (-0.0546, 5.7646) and (-0.0546, 5.7646) .. (-0.0794, 5.8208).. controls (-0.0706, 5.8383) and (-0.0619, 5.8558) .. (-0.0529, 5.8738).. controls (-0.0704, 5.8825) and (-0.0878, 5.8912) .. (-0.1058, 5.9002).. controls (-0.1058, 5.9177) and (-0.1058, 5.9351) .. (-0.1058, 5.9531) -- cycle;



  \path[fill=signature_color,shift={(0.6615, -2.9898)}] (0.0, 6.006).. controls (0.0165, 5.9564) and (0.0165, 5.9564) .. (0.0265, 5.9002).. controls (0.009, 5.8827) and (-0.0085, 5.8653) .. (-0.0265, 5.8473).. controls (-0.0691, 5.7724) and (-0.0691, 5.7724) .. (-0.0794, 5.6885).. controls (-0.0619, 5.6885) and (-0.0444, 5.6885) .. (-0.0265, 5.6885).. controls (-0.0352, 5.6623) and (-0.0439, 5.6362) .. (-0.0529, 5.6092).. controls (-0.0616, 5.6266) and (-0.0704, 5.6441) .. (-0.0794, 5.6621).. controls (-0.0968, 5.6621) and (-0.1143, 5.6621) .. (-0.1323, 5.6621).. controls (-0.141, 5.6446) and (-0.1498, 5.6272) .. (-0.1588, 5.6092).. controls (-0.1762, 5.6266) and (-0.1937, 5.6441) .. (-0.2117, 5.6621).. controls (-0.2291, 5.6621) and (-0.2466, 5.6621) .. (-0.2646, 5.6621).. controls (-0.2245, 5.7541) and (-0.1699, 5.8197) .. (-0.1025, 5.8936).. controls (-0.0834, 5.9147) and (-0.0642, 5.9359) .. (-0.0444, 5.9577).. controls (-0.0224, 5.9816) and (-0.0224, 5.9816) .. (0.0, 6.006) -- cycle;



  \path[fill=signature_color,shift={(6.3765, -4.8683)}] (0.0, 6.006).. controls (0.0087, 6.006) and (0.0175, 6.006) .. (0.0265, 6.006).. controls (0.0352, 5.91) and (0.0439, 5.814) .. (0.0529, 5.715).. controls (0.0616, 5.715) and (0.0704, 5.715) .. (0.0794, 5.715).. controls (0.0794, 5.584) and (0.0794, 5.4531) .. (0.0794, 5.3181).. controls (0.0968, 5.3181) and (0.1143, 5.3181) .. (0.1323, 5.3181).. controls (0.1323, 5.3007) and (0.1323, 5.2832) .. (0.1323, 5.2652).. controls (0.1148, 5.2652) and (0.0974, 5.2652) .. (0.0794, 5.2652).. controls (0.0706, 5.2303) and (0.0619, 5.1954) .. (0.0529, 5.1594).. controls (-0.0113, 5.4413) and (-0.0051, 5.7186) .. (0.0, 6.006) -- cycle;



  \path[fill=signature_color,shift={(11.1125, -4.3921)}] (0.0, 6.006).. controls (0.0262, 6.006) and (0.0524, 6.006) .. (0.0794, 6.006).. controls (0.1111, 5.876) and (0.1019, 5.7869) .. (0.0529, 5.6621).. controls (0.0267, 5.6534) and (0.0005, 5.6446) .. (-0.0265, 5.6356).. controls (-0.0265, 5.6531) and (-0.0265, 5.6706) .. (-0.0265, 5.6885).. controls (-0.0439, 5.6885) and (-0.0614, 5.6885) .. (-0.0794, 5.6885).. controls (-0.0794, 5.706) and (-0.0794, 5.7235) .. (-0.0794, 5.7415).. controls (-0.0619, 5.7415) and (-0.0445, 5.7415) .. (-0.0265, 5.7415).. controls (-0.0018, 5.8104) and (0.0039, 5.8378) .. (-0.0248, 5.9068).. controls (-0.0341, 5.9221) and (-0.0434, 5.9374) .. (-0.0529, 5.9531).. controls (-0.0355, 5.9531) and (-0.018, 5.9531) .. (0.0, 5.9531).. controls (0.0, 5.9706) and (0.0, 5.9881) .. (0.0, 6.006) -- cycle;



  \path[fill=signature_color,shift={(1.1642, -2.9104)}] (0.0, 6.006).. controls (0.0349, 5.9886) and (0.0699, 5.9711) .. (0.1058, 5.9531).. controls (0.0185, 5.8658) and (-0.0688, 5.7785) .. (-0.1588, 5.6885).. controls (-0.1762, 5.7235) and (-0.1937, 5.7584) .. (-0.2117, 5.7944).. controls (-0.1942, 5.8118) and (-0.1767, 5.8293) .. (-0.1588, 5.8473).. controls (-0.1588, 5.8648) and (-0.1588, 5.8822) .. (-0.1588, 5.9002).. controls (-0.1238, 5.8915) and (-0.0889, 5.8827) .. (-0.0529, 5.8738).. controls (-0.0529, 5.8999) and (-0.0529, 5.9261) .. (-0.0529, 5.9531).. controls (-0.0355, 5.9531) and (-0.018, 5.9531) .. (0.0, 5.9531).. controls (0.0, 5.9706) and (0.0, 5.9881) .. (0.0, 6.006) -- cycle;



  \path[fill=signature_color,shift={(6.641, -3.0692)}] (0.0, 6.006).. controls (0.0087, 6.006) and (0.0175, 6.006) .. (0.0265, 6.006).. controls (0.0314, 5.9741) and (0.0314, 5.9741) .. (0.0364, 5.9415).. controls (0.0418, 5.9192) and (0.0473, 5.8968) .. (0.0529, 5.8738).. controls (0.0704, 5.865) and (0.0878, 5.8563) .. (0.1058, 5.8473).. controls (0.1058, 5.8648) and (0.1058, 5.8822) .. (0.1058, 5.9002).. controls (0.132, 5.9002) and (0.1582, 5.9002) .. (0.1852, 5.9002).. controls (0.1654, 5.8357) and (0.1654, 5.8357) .. (0.1323, 5.7679).. controls (0.0777, 5.7497) and (0.0777, 5.7497) .. (0.0265, 5.7415).. controls (0.0177, 5.724) and (0.009, 5.7065) .. (0.0, 5.6885).. controls (-0.0175, 5.6973) and (-0.0349, 5.706) .. (-0.0529, 5.715).. controls (-0.0265, 5.9115) and (-0.0265, 5.9115) .. (0.0, 6.006) -- cycle;



  \path[fill=signature_color,shift={(7.3819, -2.8046)}] (0.0, 6.006).. controls (0.0262, 6.006) and (0.0524, 6.006) .. (0.0794, 6.006).. controls (0.0794, 5.9886) and (0.0794, 5.9711) .. (0.0794, 5.9531).. controls (0.1318, 5.9619) and (0.1841, 5.9706) .. (0.2381, 5.9796).. controls (0.1598, 5.9186) and (0.0788, 5.8665) .. (-0.0066, 5.8159).. controls (-0.0302, 5.8019) and (-0.0538, 5.7878) .. (-0.0781, 5.7734).. controls (-0.096, 5.7629) and (-0.1139, 5.7523) .. (-0.1323, 5.7415).. controls (-0.1104, 5.8473) and (-0.0598, 5.9163) .. (0.0, 6.006) -- cycle;



  \path[fill=signature_color,shift={(3.2808, -1.1642)}] (0.0, 6.006).. controls (0.0349, 5.9886) and (0.0699, 5.9711) .. (0.1058, 5.9531).. controls (-0.0164, 5.8658) and (-0.1386, 5.7785) .. (-0.2646, 5.6885).. controls (-0.291, 5.7944) and (-0.291, 5.7944) .. (-0.2563, 5.8539).. controls (-0.2342, 5.8768) and (-0.2342, 5.8768) .. (-0.2117, 5.9002).. controls (-0.1942, 5.9002) and (-0.1767, 5.9002) .. (-0.1588, 5.9002).. controls (-0.0596, 5.9291) and (-0.0596, 5.9291) .. (0.0, 6.006) -- cycle;



  \path[fill=signature_color,shift={(10.2923, -3.4925)}] (0.0, 6.006).. controls (0.0786, 5.9929) and (0.0786, 5.9929) .. (0.1588, 5.9796).. controls (0.1538, 5.9588) and (0.1489, 5.9381) .. (0.1439, 5.9167).. controls (0.1254, 5.8481) and (0.1254, 5.8481) .. (0.1588, 5.7944).. controls (0.0604, 5.7891) and (0.0325, 5.7904) .. (-0.0529, 5.8473).. controls (-0.0395, 5.9047) and (-0.0265, 5.9531) .. (0.0, 6.006) -- cycle;



  \path[fill=signature_color,shift={(2.3548, -1.7727)}] (0.0, 6.006).. controls (0.0349, 5.9973) and (0.0699, 5.9886) .. (0.1058, 5.9796).. controls (0.0971, 5.9621) and (0.0884, 5.9447) .. (0.0794, 5.9267).. controls (0.0968, 5.9179) and (0.1143, 5.9092) .. (0.1323, 5.9002).. controls (0.0144, 5.7954) and (0.0144, 5.7954) .. (-0.1058, 5.6885).. controls (-0.1146, 5.7147) and (-0.1233, 5.7409) .. (-0.1323, 5.7679).. controls (-0.1236, 5.7854) and (-0.1148, 5.8028) .. (-0.1058, 5.8208).. controls (-0.0796, 5.8208) and (-0.0534, 5.8208) .. (-0.0265, 5.8208).. controls (-0.0352, 5.8645) and (-0.0439, 5.9081) .. (-0.0529, 5.9531).. controls (-0.0355, 5.9531) and (-0.018, 5.9531) .. (0.0, 5.9531).. controls (0.0, 5.9706) and (0.0, 5.9881) .. (0.0, 6.006) -- cycle;



  \path[fill=signature_color,shift={(2.5665, -1.4552)}] (0.0, 6.006).. controls (0.0761, 6.011) and (0.0761, 6.011) .. (0.1588, 6.006).. controls (0.2117, 5.9267) and (0.2117, 5.9267) .. (0.2117, 5.8208).. controls (0.1331, 5.8339) and (0.1331, 5.8339) .. (0.0529, 5.8473).. controls (0.0529, 5.8648) and (0.0529, 5.8822) .. (0.0529, 5.9002).. controls (0.0267, 5.8915) and (0.0005, 5.8827) .. (-0.0265, 5.8738).. controls (-0.0177, 5.9174) and (-0.009, 5.9611) .. (0.0, 6.006) -- cycle;



  \path[fill=signature_color,shift={(3.9688, -4.4715)}] (0.0, 6.006).. controls (0.0498, 5.9196) and (0.066, 5.8399) .. (0.0794, 5.7415).. controls (0.1056, 5.7327) and (0.1318, 5.724) .. (0.1588, 5.715).. controls (0.0933, 5.6757) and (0.0933, 5.6757) .. (0.0265, 5.6356).. controls (0.0177, 5.6967) and (0.009, 5.7579) .. (0.0, 5.8208).. controls (-0.0087, 5.8121) and (-0.0175, 5.8034) .. (-0.0265, 5.7944).. controls (-0.0919, 5.8206) and (-0.0919, 5.8206) .. (-0.1588, 5.8473).. controls (-0.1058, 5.9002) and (-0.0529, 5.9531) .. (0.0, 6.006) -- cycle;



  \path[fill=signature_color,shift={(3.6777, -4.7096)}] (0.0, 6.006).. controls (0.0146, 5.9511) and (0.0265, 5.9043) .. (0.0265, 5.8473).. controls (0.0439, 5.8473) and (0.0614, 5.8473) .. (0.0794, 5.8473).. controls (0.0968, 5.8298) and (0.1143, 5.8124) .. (0.1323, 5.7944).. controls (0.1585, 5.7856) and (0.1847, 5.7769) .. (0.2117, 5.7679).. controls (0.168, 5.7505) and (0.1244, 5.733) .. (0.0794, 5.715).. controls (0.0794, 5.7325) and (0.0794, 5.7499) .. (0.0794, 5.7679).. controls (0.0532, 5.7592) and (0.027, 5.7505) .. (0.0, 5.7415).. controls (-0.0529, 5.8208) and (-0.0529, 5.8208) .. (-0.0529, 5.9002).. controls (-0.0966, 5.8915) and (-0.1402, 5.8827) .. (-0.1852, 5.8738).. controls (-0.0595, 5.9763) and (-0.0595, 5.9763) .. (0.0, 6.006) -- cycle;



  \path[fill=signature_color,shift={(12.065, -2.6723)}] (0.0, 6.006).. controls (0.0175, 5.9973) and (0.0349, 5.9886) .. (0.0529, 5.9796).. controls (0.0341, 5.8662) and (-0.0018, 5.7613) .. (-0.0413, 5.6538).. controls (-0.0475, 5.6367) and (-0.0537, 5.6197) .. (-0.0601, 5.6021).. controls (-0.0753, 5.5603) and (-0.0905, 5.5186) .. (-0.1058, 5.4769).. controls (-0.1146, 5.4769) and (-0.1233, 5.4769) .. (-0.1323, 5.4769).. controls (-0.1307, 5.5364) and (-0.1283, 5.596) .. (-0.1257, 5.6555).. controls (-0.1244, 5.6886) and (-0.1232, 5.7218) .. (-0.122, 5.7559).. controls (-0.1037, 5.8593) and (-0.0778, 5.8866) .. (0.0, 5.9531).. controls (0.0, 5.9706) and (0.0, 5.9881) .. (0.0, 6.006) -- cycle;



  \path[fill=signature_color,shift={(10.8215, -1.7463)}] (0.0, 6.006).. controls (-0.0262, 5.9929) and (-0.0262, 5.9929) .. (-0.0529, 5.9796).. controls (-0.0529, 5.9621) and (-0.0529, 5.9447) .. (-0.0529, 5.9267).. controls (-0.0355, 5.9179) and (-0.018, 5.9092) .. (0.0, 5.9002).. controls (-0.0529, 5.8473) and (-0.0529, 5.8473) .. (-0.1356, 5.844).. controls (-0.1607, 5.8451) and (-0.1858, 5.8462) .. (-0.2117, 5.8473).. controls (-0.2204, 5.8648) and (-0.2291, 5.8822) .. (-0.2381, 5.9002).. controls (-0.2643, 5.8871) and (-0.2643, 5.8871) .. (-0.291, 5.8738).. controls (-0.291, 5.9087) and (-0.291, 5.9436) .. (-0.291, 5.9796).. controls (-0.1938, 6.0107) and (-0.1014, 6.008) .. (0.0, 6.006) -- cycle;



  \path[fill=signature_color,shift={(14.6116, -0.0232)}] (0.0, 6.006).. controls (0.0236, 6.0056) and (0.0472, 6.0051) .. (0.0715, 6.0046).. controls (0.0983, 6.0037) and (0.0983, 6.0037) .. (0.1257, 6.0027).. controls (0.1257, 5.9853) and (0.1257, 5.9678) .. (0.1257, 5.9498).. controls (0.0823, 5.9392) and (0.0823, 5.9392) .. (0.038, 5.9283).. controls (-0.0494, 5.9002) and (-0.0847, 5.8877) .. (-0.1389, 5.8175).. controls (-0.1416, 5.8344) and (-0.1444, 5.8514) .. (-0.1472, 5.8688).. controls (-0.1532, 5.8868) and (-0.1592, 5.9048) .. (-0.1654, 5.9234).. controls (-0.1916, 5.9321) and (-0.2178, 5.9408) .. (-0.2447, 5.9498).. controls (-0.1598, 6.0002) and (-0.0982, 6.0086) .. (0.0, 6.006) -- cycle;



  \path[fill=signature_color,shift={(10.2923, -1.7727)}] (0.0, 6.006).. controls (0.0611, 5.9973) and (0.1222, 5.9886) .. (0.1852, 5.9796).. controls (0.1765, 5.9621) and (0.1677, 5.9447) .. (0.1588, 5.9267).. controls (0.1675, 5.9005) and (0.1762, 5.8743) .. (0.1852, 5.8473).. controls (0.1416, 5.8473) and (0.0979, 5.8473) .. (0.0529, 5.8473).. controls (0.0529, 5.8298) and (0.0529, 5.8124) .. (0.0529, 5.7944).. controls (0.0355, 5.7944) and (0.018, 5.7944) .. (0.0, 5.7944).. controls (-0.0175, 5.8468) and (-0.0349, 5.8992) .. (-0.0529, 5.9531).. controls (-0.0355, 5.9531) and (-0.018, 5.9531) .. (0.0, 5.9531).. controls (0.0, 5.9706) and (0.0, 5.9881) .. (0.0, 6.006) -- cycle;



  \path[fill=signature_color,shift={(3.1221, -1.1377)}] (0.0, 6.006).. controls (0.0175, 5.9886) and (0.0349, 5.9711) .. (0.0529, 5.9531).. controls (0.0355, 5.9357) and (0.018, 5.9182) .. (0.0, 5.9002).. controls (-0.0171, 5.8322) and (-0.0171, 5.8322) .. (-0.0265, 5.7679).. controls (-0.0527, 5.7592) and (-0.0788, 5.7505) .. (-0.1058, 5.7415).. controls (-0.1233, 5.7677) and (-0.1408, 5.7938) .. (-0.1588, 5.8208).. controls (-0.1762, 5.8296) and (-0.1937, 5.8383) .. (-0.2117, 5.8473).. controls (-0.2117, 5.8648) and (-0.2117, 5.8822) .. (-0.2117, 5.9002).. controls (-0.1767, 5.9089) and (-0.1418, 5.9177) .. (-0.1058, 5.9267).. controls (-0.1058, 5.9441) and (-0.1058, 5.9616) .. (-0.1058, 5.9796).. controls (-0.0562, 5.9961) and (-0.0562, 5.9961) .. (0.0, 6.006) -- cycle;



  \path[fill=signature_color,shift={(0.3704, -5.08)}] (0.0, 6.006).. controls (0.1462, 5.9745) and (0.2164, 5.9373) .. (0.3175, 5.8208).. controls (0.335, 5.8296) and (0.3524, 5.8383) .. (0.3704, 5.8473).. controls (0.3704, 5.8124) and (0.3704, 5.7774) .. (0.3704, 5.7415).. controls (0.2127, 5.7769) and (0.1111, 5.8671) .. (0.0, 5.9796).. controls (0.0, 5.9883) and (0.0, 5.997) .. (0.0, 6.006) -- cycle;



  \path[fill=signature_color,shift={(2.6723, -4.445)}] (0.0, 6.006).. controls (0.0175, 6.006) and (0.0349, 6.006) .. (0.0529, 6.006).. controls (0.0385, 5.8977) and (-0.0187, 5.8403) .. (-0.091, 5.7613).. controls (-0.1122, 5.7377) and (-0.1334, 5.7141) .. (-0.1553, 5.6898).. controls (-0.2117, 5.6356) and (-0.2117, 5.6356) .. (-0.2646, 5.6356).. controls (-0.2646, 5.6531) and (-0.2646, 5.6706) .. (-0.2646, 5.6885).. controls (-0.2471, 5.6885) and (-0.2297, 5.6885) .. (-0.2117, 5.6885).. controls (-0.2029, 5.7235) and (-0.1942, 5.7584) .. (-0.1852, 5.7944).. controls (-0.1677, 5.7944) and (-0.1503, 5.7944) .. (-0.1323, 5.7944).. controls (-0.1323, 5.838) and (-0.1323, 5.8817) .. (-0.1323, 5.9267).. controls (-0.1148, 5.9179) and (-0.0974, 5.9092) .. (-0.0794, 5.9002).. controls (-0.0674, 5.9177) and (-0.0554, 5.9351) .. (-0.043, 5.9531).. controls (-0.0288, 5.9706) and (-0.0146, 5.9881) .. (0.0, 6.006) -- cycle;



  \path[fill=signature_color,shift={(6.9056, -3.8629)}] (0.0, 6.006).. controls (0.0645, 5.9862) and (0.0645, 5.9862) .. (0.1323, 5.9531).. controls (0.1538, 5.8969) and (0.1538, 5.8969) .. (0.1588, 5.8473).. controls (0.0822, 5.8254) and (0.0294, 5.8208) .. (-0.0529, 5.8208).. controls (-0.0298, 5.9465) and (-0.0298, 5.9465) .. (0.0, 6.006) -- cycle;



  \path[fill=signature_color,shift={(10.4775, -3.519)}] (0.0, 6.006).. controls (0.0262, 6.006) and (0.0524, 6.006) .. (0.0794, 6.006).. controls (0.0794, 5.9886) and (0.0794, 5.9711) .. (0.0794, 5.9531).. controls (0.1143, 5.9531) and (0.1492, 5.9531) .. (0.1852, 5.9531).. controls (0.1677, 5.892) and (0.1503, 5.8309) .. (0.1323, 5.7679).. controls (0.0886, 5.7679) and (0.045, 5.7679) .. (0.0, 5.7679).. controls (0.0049, 5.7876) and (0.0098, 5.8072) .. (0.0149, 5.8274).. controls (0.0262, 5.8989) and (0.0225, 5.9385) .. (0.0, 6.006) -- cycle;



  \path[fill=signature_color,shift={(5.0006, -3.7571)}] (0.0, 6.006).. controls (0.0175, 5.9973) and (0.0349, 5.9886) .. (0.0529, 5.9796).. controls (0.0529, 5.9621) and (0.0529, 5.9447) .. (0.0529, 5.9267).. controls (0.0704, 5.9179) and (0.0878, 5.9092) .. (0.1058, 5.9002).. controls (0.036, 5.8304) and (-0.0339, 5.7605) .. (-0.1058, 5.6885).. controls (-0.1233, 5.7147) and (-0.1408, 5.7409) .. (-0.1588, 5.7679).. controls (-0.1058, 5.8208) and (-0.1058, 5.8208) .. (-0.0529, 5.8473).. controls (-0.0337, 5.8996) and (-0.0159, 5.9526) .. (0.0, 6.006) -- cycle;



  \path[fill=signature_color,shift={(0.7673, -3.2279)}] (0.0, 6.006).. controls (0.0786, 5.9929) and (0.0786, 5.9929) .. (0.1588, 5.9796).. controls (0.1169, 5.8665) and (0.0395, 5.8138) .. (-0.0529, 5.7415).. controls (-0.0529, 5.7764) and (-0.0529, 5.8113) .. (-0.0529, 5.8473).. controls (-0.0267, 5.8473) and (-0.0005, 5.8473) .. (0.0265, 5.8473).. controls (0.0265, 5.8735) and (0.0265, 5.8997) .. (0.0265, 5.9267).. controls (0.009, 5.9354) and (-0.0085, 5.9441) .. (-0.0265, 5.9531).. controls (-0.0177, 5.9706) and (-0.009, 5.9881) .. (0.0, 6.006) -- cycle;



  \path[fill=signature_color,shift={(7.1438, -1.4552)}] (0.0, 6.006).. controls (0.0296, 5.9076) and (0.0296, 5.9076) .. (0.0017, 5.8407).. controls (-0.0123, 5.8178) and (-0.0123, 5.8178) .. (-0.0265, 5.7944).. controls (-0.0003, 5.7944) and (0.0259, 5.7944) .. (0.0529, 5.7944).. controls (0.0442, 5.7682) and (0.0355, 5.742) .. (0.0265, 5.715).. controls (0.009, 5.715) and (-0.0085, 5.715) .. (-0.0265, 5.715).. controls (-0.0265, 5.6975) and (-0.0265, 5.6801) .. (-0.0265, 5.6621).. controls (-0.0439, 5.6621) and (-0.0614, 5.6621) .. (-0.0794, 5.6621).. controls (-0.0968, 5.6272) and (-0.1143, 5.5922) .. (-0.1323, 5.5563).. controls (-0.1014, 5.7131) and (-0.0594, 5.8576) .. (0.0, 6.006) -- cycle;



  \path[fill=signature_color,shift={(7.2231, -1.5081)}] (0.0, 6.006).. controls (0.0349, 5.9973) and (0.0698, 5.9886) .. (0.1058, 5.9796).. controls (0.0971, 5.9185) and (0.0884, 5.8573) .. (0.0794, 5.7944).. controls (0.0532, 5.7856) and (0.027, 5.7769) .. (0.0, 5.7679).. controls (-0.0298, 5.8126) and (-0.0298, 5.8126) .. (-0.0529, 5.8738).. controls (-0.0298, 5.9465) and (-0.0298, 5.9465) .. (0.0, 6.006) -- cycle;



  \path[fill=signature_color,shift={(4.3392, -4.1275)}] (0.0, 6.006).. controls (0.0175, 5.9711) and (0.0349, 5.9362) .. (0.0529, 5.9002).. controls (0.0327, 5.8942) and (0.0125, 5.8882) .. (-0.0083, 5.882).. controls (-0.0843, 5.8533) and (-0.0843, 5.8533) .. (-0.1158, 5.7811).. controls (-0.1588, 5.715) and (-0.1588, 5.715) .. (-0.2563, 5.6803).. controls (-0.2852, 5.6743) and (-0.3142, 5.6683) .. (-0.344, 5.6621).. controls (-0.2293, 5.7767) and (-0.1147, 5.8914) .. (0.0, 6.006) -- cycle;



  \path[fill=signature_color,shift={(8.3079, -2.2225)}] (0.0, 6.006).. controls (-0.0087, 5.9449) and (-0.0175, 5.8838) .. (-0.0265, 5.8208).. controls (-0.0439, 5.8296) and (-0.0614, 5.8383) .. (-0.0794, 5.8473).. controls (-0.0794, 5.8735) and (-0.0794, 5.8997) .. (-0.0794, 5.9267).. controls (-0.0979, 5.9179) and (-0.1165, 5.9092) .. (-0.1356, 5.9002).. controls (-0.2185, 5.8714) and (-0.2833, 5.8697) .. (-0.3704, 5.8738).. controls (-0.306, 5.9167) and (-0.2563, 5.943) .. (-0.1852, 5.9697).. controls (-0.1672, 5.9766) and (-0.1492, 5.9834) .. (-0.1306, 5.9905).. controls (-0.0794, 6.006) and (-0.0794, 6.006) .. (0.0, 6.006) -- cycle;



  \path[fill=signature_color,shift={(11.6681, -4.2333)}] (0.0, 6.006).. controls (-0.0255, 5.7012) and (-0.0255, 5.7012) .. (-0.0529, 5.5563).. controls (-0.0616, 5.5563) and (-0.0704, 5.5563) .. (-0.0794, 5.5563).. controls (-0.0881, 5.6872) and (-0.0968, 5.8182) .. (-0.1058, 5.9531).. controls (-0.0265, 6.006) and (-0.0265, 6.006) .. (0.0, 6.006) -- cycle;



  \path[fill=signature_color,shift={(6.5881, -5.5827)}] (0.0, 6.006).. controls (0.0529, 5.9531) and (0.1058, 5.9002) .. (0.1588, 5.8473).. controls (0.0733, 5.7904) and (0.0454, 5.7891) .. (-0.0529, 5.7944).. controls (-0.0529, 5.8555) and (-0.0529, 5.9166) .. (-0.0529, 5.9796).. controls (-0.0355, 5.9883) and (-0.018, 5.997) .. (0.0, 6.006) -- cycle;



  \path[fill=signature_color,shift={(10.1071, -3.3602)}] (0.0, 6.006).. controls (0.0262, 5.9973) and (0.0524, 5.9886) .. (0.0794, 5.9796).. controls (0.0706, 5.9359) and (0.0619, 5.8923) .. (0.0529, 5.8473).. controls (0.0267, 5.8473) and (0.0005, 5.8473) .. (-0.0265, 5.8473).. controls (-0.0265, 5.8124) and (-0.0265, 5.7774) .. (-0.0265, 5.7415).. controls (-0.1036, 5.7699) and (-0.1529, 5.7885) .. (-0.2117, 5.8473).. controls (-0.1462, 5.8342) and (-0.1462, 5.8342) .. (-0.0794, 5.8208).. controls (-0.0794, 5.847) and (-0.0794, 5.8732) .. (-0.0794, 5.9002).. controls (-0.0532, 5.9002) and (-0.027, 5.9002) .. (0.0, 5.9002).. controls (0.0, 5.9351) and (0.0, 5.9701) .. (0.0, 6.006) -- cycle;



  \path[fill=signature_color,shift={(9.6011, -2.9485)}] (0.0, 6.006).. controls (0.0893, 5.99) and (0.1024, 5.9805) .. (0.1588, 5.9151).. controls (0.215, 5.8324) and (0.215, 5.8324) .. (0.215, 5.7795).. controls (0.18, 5.7882) and (0.1451, 5.797) .. (0.1091, 5.806).. controls (0.1004, 5.8409) and (0.0917, 5.8758) .. (0.0827, 5.9118).. controls (0.0041, 5.9118) and (-0.0745, 5.9118) .. (-0.1554, 5.9118).. controls (-0.0761, 5.9912) and (-0.0761, 5.9912) .. (0.0, 6.006) -- cycle;



  \path[fill=signature_color,shift={(7.1438, -2.249)}] (0.0, 6.006).. controls (0.0395, 5.927) and (0.0108, 5.875) .. (-0.0116, 5.7911).. controls (-0.0192, 5.7621) and (-0.0268, 5.7331) .. (-0.0346, 5.7032).. controls (-0.0407, 5.6809) and (-0.0467, 5.6586) .. (-0.0529, 5.6356).. controls (-0.0616, 5.6356) and (-0.0704, 5.6356) .. (-0.0794, 5.6356).. controls (-0.0794, 5.7142) and (-0.0794, 5.7928) .. (-0.0794, 5.8738).. controls (-0.0968, 5.8563) and (-0.1143, 5.8388) .. (-0.1323, 5.8208).. controls (-0.1189, 5.8782) and (-0.1058, 5.9267) .. (-0.0794, 5.9796).. controls (-0.0532, 5.9883) and (-0.027, 5.997) .. (0.0, 6.006) -- cycle;



  \path[fill=signature_color,shift={(13.0175, -0.8202)}] (0.0, 6.006).. controls (0.0087, 5.9711) and (0.0175, 5.9362) .. (0.0265, 5.9002).. controls (0.0095, 5.8931) and (-0.0074, 5.886) .. (-0.0248, 5.8787).. controls (-0.0881, 5.8503) and (-0.0881, 5.8503) .. (-0.1058, 5.7679).. controls (-0.1408, 5.7854) and (-0.1757, 5.8028) .. (-0.2117, 5.8208).. controls (-0.1418, 5.882) and (-0.072, 5.9431) .. (0.0, 6.006) -- cycle;



  \path[fill=signature_color,shift={(2.3283, -4.3921)}] (0.0, 6.006).. controls (0.0423, 5.9214) and (0.0008, 5.8546) .. (-0.0265, 5.7679).. controls (-0.0963, 5.7505) and (-0.1662, 5.733) .. (-0.2381, 5.715).. controls (-0.2276, 5.7271) and (-0.217, 5.7393) .. (-0.2062, 5.7518).. controls (-0.1346, 5.8347) and (-0.064, 5.9171) .. (0.0, 6.006) -- cycle;



  \path[fill=signature_color,shift={(0.1588, -4.5244)}] (0.0, 6.006).. controls (0.0049, 5.9839) and (0.0049, 5.9839) .. (0.0099, 5.9614).. controls (0.0251, 5.898) and (0.0251, 5.898) .. (0.0579, 5.8341).. controls (0.0794, 5.7679) and (0.0794, 5.7679) .. (0.0562, 5.7067).. controls (0.0415, 5.6846) and (0.0415, 5.6846) .. (0.0265, 5.6621).. controls (-0.0296, 5.7626) and (-0.0691, 5.8367) .. (-0.0529, 5.9531).. controls (-0.0355, 5.9706) and (-0.018, 5.9881) .. (0.0, 6.006) -- cycle;



  \path[fill=signature_color,shift={(0.0265, -4.154)}] (0.0, 6.006).. controls (0.0087, 6.006) and (0.0175, 6.006) .. (0.0265, 6.006).. controls (0.0265, 5.9362) and (0.0265, 5.8663) .. (0.0265, 5.7944).. controls (0.0527, 5.7856) and (0.0788, 5.7769) .. (0.1058, 5.7679).. controls (0.1394, 5.6957) and (0.1334, 5.6654) .. (0.1075, 5.5893).. controls (0.0982, 5.5697) and (0.0889, 5.55) .. (0.0794, 5.5298).. controls (-0.0066, 5.6846) and (-0.0034, 5.8334) .. (0.0, 6.006) -- cycle;



  \path[fill=signature_color,shift={(5.1329, -3.6777)}] (0.0, 6.006).. controls (0.0655, 5.9766) and (0.1238, 5.9377) .. (0.1852, 5.9002).. controls (0.159, 5.8609) and (0.159, 5.8609) .. (0.1323, 5.8208).. controls (0.1061, 5.8383) and (0.0799, 5.8558) .. (0.0529, 5.8738).. controls (0.0267, 5.865) and (0.0005, 5.8563) .. (-0.0265, 5.8473).. controls (-0.0352, 5.8909) and (-0.0439, 5.9346) .. (-0.0529, 5.9796).. controls (-0.0355, 5.9883) and (-0.018, 5.997) .. (0.0, 6.006) -- cycle;



  \path[fill=signature_color,shift={(7.2231, -3.3338)}] (0.0, 6.006).. controls (0.0786, 5.9929) and (0.0786, 5.9929) .. (0.1588, 5.9796).. controls (0.1326, 5.9665) and (0.1326, 5.9665) .. (0.1058, 5.9531).. controls (0.1058, 5.9357) and (0.1058, 5.9182) .. (0.1058, 5.9002).. controls (0.0709, 5.8915) and (0.036, 5.8827) .. (0.0, 5.8738).. controls (0.0, 5.8563) and (0.0, 5.8388) .. (0.0, 5.8208).. controls (-0.0262, 5.8296) and (-0.0524, 5.8383) .. (-0.0794, 5.8473).. controls (-0.0794, 5.8735) and (-0.0794, 5.8997) .. (-0.0794, 5.9267).. controls (-0.0619, 5.9267) and (-0.0445, 5.9267) .. (-0.0265, 5.9267).. controls (-0.0177, 5.9529) and (-0.009, 5.9791) .. (0.0, 6.006) -- cycle;



  \path[fill=signature_color,shift={(1.0054, -2.6723)}] (0.0, 6.006).. controls (0.0, 5.9239) and (-0.0143, 5.8909) .. (-0.0529, 5.8208).. controls (-0.0878, 5.8208) and (-0.1228, 5.8208) .. (-0.1588, 5.8208).. controls (-0.1762, 5.7772) and (-0.1937, 5.7335) .. (-0.2117, 5.6885).. controls (-0.2466, 5.6973) and (-0.2815, 5.706) .. (-0.3175, 5.715).. controls (-0.299, 5.7316) and (-0.2805, 5.7482) .. (-0.2615, 5.7652).. controls (-0.2374, 5.7869) and (-0.2133, 5.8085) .. (-0.1885, 5.8308).. controls (-0.1526, 5.863) and (-0.1526, 5.863) .. (-0.116, 5.8959).. controls (-0.0765, 5.9317) and (-0.0377, 5.9683) .. (0.0, 6.006) -- cycle;



  \path[fill=signature_color,shift={(7.3819, -2.5929)}] (0.0, 6.006).. controls (0.0087, 5.9886) and (0.0175, 5.9711) .. (0.0265, 5.9531).. controls (0.009, 5.9531) and (-0.0085, 5.9531) .. (-0.0265, 5.9531).. controls (-0.0265, 5.892) and (-0.0265, 5.8309) .. (-0.0265, 5.7679).. controls (-0.0794, 5.7944) and (-0.0794, 5.7944) .. (-0.1323, 5.8473).. controls (-0.1672, 5.8211) and (-0.2021, 5.7949) .. (-0.2381, 5.7679).. controls (-0.2556, 5.7766) and (-0.273, 5.7854) .. (-0.291, 5.7944).. controls (-0.195, 5.8642) and (-0.099, 5.9341) .. (0.0, 6.006) -- cycle;



  \path[fill=signature_color,shift={(6.9321, -2.0902)}] (0.0, 6.006).. controls (0.0256, 5.9234) and (0.0262, 5.86) .. (0.0149, 5.7745).. controls (0.0122, 5.7534) and (0.0095, 5.7322) .. (0.0067, 5.7105).. controls (0.0034, 5.6865) and (0.0034, 5.6865) .. (0.0, 5.6621).. controls (-0.0262, 5.6534) and (-0.0524, 5.6446) .. (-0.0794, 5.6356).. controls (-0.1114, 5.5827) and (-0.1114, 5.5827) .. (-0.1323, 5.5298).. controls (-0.0995, 5.6931) and (-0.0557, 5.8491) .. (0.0, 6.006) -- cycle;



  \path[fill=signature_color,shift={(6.8263, -3.7042)}] (0.0, 6.006).. controls (0.0087, 5.9711) and (0.0175, 5.9362) .. (0.0265, 5.9002).. controls (0.0396, 5.9046) and (0.0527, 5.9089) .. (0.0661, 5.9134).. controls (0.1597, 5.9321) and (0.2486, 5.9283) .. (0.344, 5.9267).. controls (0.3527, 5.9092) and (0.3614, 5.8917) .. (0.3704, 5.8738).. controls (0.2482, 5.865) and (0.1259, 5.8563) .. (0.0, 5.8473).. controls (0.0, 5.8997) and (0.0, 5.9521) .. (0.0, 6.006) -- cycle;



  \path[fill=signature_color,shift={(9.3133, -1.9579)}] (0.0, 6.006).. controls (0.0, 5.9886) and (0.0, 5.9711) .. (0.0, 5.9531).. controls (-0.0262, 5.9531) and (-0.0524, 5.9531) .. (-0.0794, 5.9531).. controls (-0.0881, 5.9269) and (-0.0968, 5.9007) .. (-0.1058, 5.8738).. controls (-0.1146, 5.8912) and (-0.1233, 5.9087) .. (-0.1323, 5.9267).. controls (-0.2283, 5.9179) and (-0.3244, 5.9092) .. (-0.4233, 5.9002).. controls (-0.3466, 5.9386) and (-0.2921, 5.9631) .. (-0.2117, 5.9829).. controls (-0.1942, 5.9873) and (-0.1767, 5.9918) .. (-0.1588, 5.9963).. controls (-0.1058, 6.006) and (-0.1058, 6.006) .. (0.0, 6.006) -- cycle;



  \path[fill=signature_color,shift={(13.6525, -0.5556)}] (0.0, 6.006).. controls (0.0262, 5.9973) and (0.0524, 5.9886) .. (0.0794, 5.9796).. controls (0.0728, 5.8903) and (0.0728, 5.8903) .. (0.0529, 5.7944).. controls (-0.0017, 5.7563) and (-0.0017, 5.7563) .. (-0.0529, 5.7415).. controls (-0.0529, 5.7677) and (-0.0529, 5.7938) .. (-0.0529, 5.8208).. controls (-0.0267, 5.8208) and (-0.0005, 5.8208) .. (0.0265, 5.8208).. controls (0.0003, 5.8383) and (-0.0259, 5.8558) .. (-0.0529, 5.8738).. controls (-0.0355, 5.9174) and (-0.018, 5.9611) .. (0.0, 6.006) -- cycle;



  \path[fill=signature_color,shift={(3.519, -2.0902)}] (0.0, 6.006).. controls (0.0533, 5.9643) and (0.0533, 5.9643) .. (0.1058, 5.9002).. controls (0.1104, 5.8146) and (0.1104, 5.8146) .. (0.0992, 5.7216).. controls (0.0958, 5.6906) and (0.0924, 5.6597) .. (0.0889, 5.6278).. controls (0.0857, 5.6042) and (0.0826, 5.5806) .. (0.0794, 5.5563).. controls (0.0706, 5.5563) and (0.0619, 5.5563) .. (0.0529, 5.5563).. controls (0.0513, 5.5799) and (0.0497, 5.6035) .. (0.0481, 5.6278).. controls (0.0458, 5.6587) and (0.0436, 5.6897) .. (0.0413, 5.7216).. controls (0.0392, 5.7523) and (0.037, 5.783) .. (0.0348, 5.8146).. controls (0.0282, 5.8819) and (0.019, 5.9415) .. (0.0, 6.006) -- cycle;



  \path[fill=signature_color,shift={(3.2279, -4.789)}] (0.0, 6.006).. controls (0.0087, 5.9711) and (0.0175, 5.9362) .. (0.0265, 5.9002).. controls (0.105, 5.9177) and (0.1836, 5.9351) .. (0.2646, 5.9531).. controls (0.2559, 5.9182) and (0.2471, 5.8833) .. (0.2381, 5.8473).. controls (0.1595, 5.8473) and (0.081, 5.8473) .. (0.0, 5.8473).. controls (0.0, 5.8997) and (0.0, 5.9521) .. (0.0, 6.006) -- cycle;



  \path[fill=signature_color,shift={(4.736, -3.7835)}] (0.0, 6.006).. controls (0.0, 5.9886) and (0.0, 5.9711) .. (0.0, 5.9531).. controls (-0.0262, 5.94) and (-0.0262, 5.94) .. (-0.0529, 5.9267).. controls (-0.0703, 5.8598) and (-0.0703, 5.8598) .. (-0.0794, 5.7944).. controls (-0.1525, 5.7786) and (-0.2159, 5.7679) .. (-0.291, 5.7679).. controls (-0.2526, 5.8078) and (-0.214, 5.8474) .. (-0.1753, 5.887).. controls (-0.1431, 5.9201) and (-0.1431, 5.9201) .. (-0.1102, 5.954).. controls (-0.0529, 6.006) and (-0.0529, 6.006) .. (0.0, 6.006) -- cycle;



  \path[fill=signature_color,shift={(5.0271, -3.5454)}] (0.0, 6.006).. controls (0.0, 5.9531) and (0.0, 5.9531) .. (-0.0364, 5.9101).. controls (-0.0794, 5.8738) and (-0.0794, 5.8738) .. (-0.1323, 5.8738).. controls (-0.141, 5.8388) and (-0.1498, 5.8039) .. (-0.1588, 5.7679).. controls (-0.2024, 5.7679) and (-0.2461, 5.7679) .. (-0.291, 5.7679).. controls (-0.1124, 6.006) and (-0.1124, 6.006) .. (0.0, 6.006) -- cycle;



  \path[fill=signature_color,shift={(8.2285, -2.5665)}] (0.0, 6.006).. controls (-0.0622, 5.9439) and (-0.1047, 5.9338) .. (-0.1885, 5.9101).. controls (-0.2126, 5.9032) and (-0.2367, 5.8964) .. (-0.2615, 5.8893).. controls (-0.28, 5.8841) and (-0.2985, 5.879) .. (-0.3175, 5.8738).. controls (-0.3175, 5.8912) and (-0.3175, 5.9087) .. (-0.3175, 5.9267).. controls (-0.3, 5.9267) and (-0.2826, 5.9267) .. (-0.2646, 5.9267).. controls (-0.2646, 5.9441) and (-0.2646, 5.9616) .. (-0.2646, 5.9796).. controls (-0.1627, 6.0475) and (-0.1168, 6.0281) .. (0.0, 6.006) -- cycle;



  \path[fill=signature_color,shift={(13.3085, -0.6085)}] (0.0, 6.006).. controls (0.0087, 5.9886) and (0.0175, 5.9711) .. (0.0265, 5.9531).. controls (-0.0066, 5.9002) and (-0.0066, 5.9002) .. (-0.0529, 5.8473).. controls (-0.0878, 5.8473) and (-0.1228, 5.8473) .. (-0.1588, 5.8473).. controls (-0.1762, 5.8124) and (-0.1937, 5.7774) .. (-0.2117, 5.7415).. controls (-0.2117, 5.7589) and (-0.2117, 5.7764) .. (-0.2117, 5.7944).. controls (-0.2291, 5.8031) and (-0.2466, 5.8118) .. (-0.2646, 5.8208).. controls (-0.1773, 5.882) and (-0.09, 5.9431) .. (0.0, 6.006) -- cycle;



  \path[fill=signature_color,shift={(6.9056, -3.9952)}] (0.0, 6.006).. controls (0.062, 5.9993) and (0.1238, 5.9904) .. (0.1852, 5.9796).. controls (0.1939, 5.9621) and (0.2027, 5.9447) .. (0.2117, 5.9267).. controls (0.1932, 5.9232) and (0.1747, 5.9196) .. (0.1556, 5.916).. controls (0.1316, 5.9113) and (0.1075, 5.9067) .. (0.0827, 5.9019).. controls (0.0587, 5.8973) and (0.0348, 5.8927) .. (0.0101, 5.8879).. controls (-0.0529, 5.8764) and (-0.0529, 5.8764) .. (-0.1058, 5.8473).. controls (-0.1058, 5.8822) and (-0.1058, 5.9171) .. (-0.1058, 5.9531).. controls (-0.0709, 5.9531) and (-0.036, 5.9531) .. (0.0, 5.9531).. controls (0.0, 5.9706) and (0.0, 5.9881) .. (0.0, 6.006) -- cycle;



  \path[fill=signature_color,shift={(3.4131, -3.175)}] (0.0, 6.006).. controls (0.0265, 5.9531) and (0.0265, 5.9531) .. (0.0059, 5.8785).. controls (-0.0042, 5.8497) and (-0.0144, 5.8208) .. (-0.0248, 5.7911).. controls (-0.0349, 5.7621) and (-0.0449, 5.7331) .. (-0.0553, 5.7032).. controls (-0.0632, 5.6809) and (-0.0712, 5.6586) .. (-0.0794, 5.6356).. controls (-0.0881, 5.6356) and (-0.0968, 5.6356) .. (-0.1058, 5.6356).. controls (-0.1146, 5.688) and (-0.1233, 5.7404) .. (-0.1323, 5.7944).. controls (-0.1148, 5.7944) and (-0.0974, 5.7944) .. (-0.0794, 5.7944).. controls (-0.0761, 5.8244) and (-0.0728, 5.8544) .. (-0.0695, 5.8853).. controls (-0.064, 5.9164) and (-0.0585, 5.9475) .. (-0.0529, 5.9796).. controls (-0.0355, 5.9883) and (-0.018, 5.997) .. (0.0, 6.006) -- cycle;



  \path[fill=signature_color,shift={(7.4083, -2.5929)}] (0.0, 6.006).. controls (0.0349, 5.9886) and (0.0698, 5.9711) .. (0.1058, 5.9531).. controls (0.0927, 5.9368) and (0.0796, 5.9204) .. (0.0661, 5.9035).. controls (0.0214, 5.8516) and (0.0214, 5.8516) .. (0.0265, 5.7944).. controls (0.0003, 5.7856) and (-0.0259, 5.7769) .. (-0.0529, 5.7679).. controls (-0.0827, 5.8936) and (-0.0827, 5.8936) .. (-0.0529, 5.9531).. controls (-0.0355, 5.9531) and (-0.018, 5.9531) .. (0.0, 5.9531).. controls (0.0, 5.9706) and (0.0, 5.9881) .. (0.0, 6.006) -- cycle;



  \path[fill=signature_color,shift={(8.7577, -2.0902)}] (0.0, 6.006).. controls (-0.0087, 5.9798) and (-0.0175, 5.9537) .. (-0.0265, 5.9267).. controls (-0.0527, 5.9179) and (-0.0788, 5.9092) .. (-0.1058, 5.9002).. controls (-0.1233, 5.8827) and (-0.1408, 5.8653) .. (-0.1588, 5.8473).. controls (-0.1588, 5.8648) and (-0.1588, 5.8822) .. (-0.1588, 5.9002).. controls (-0.1937, 5.9177) and (-0.2286, 5.9351) .. (-0.2646, 5.9531).. controls (-0.0893, 6.006) and (-0.0893, 6.006) .. (0.0, 6.006) -- cycle;



  \path[fill=signature_color,shift={(1.9315, -1.9315)}] (0.0, 6.006).. controls (-0.0083, 5.9548) and (-0.0083, 5.9548) .. (-0.0265, 5.9002).. controls (-0.0527, 5.8915) and (-0.0788, 5.8827) .. (-0.1058, 5.8738).. controls (-0.1252, 5.8204) and (-0.1252, 5.8204) .. (-0.1323, 5.7679).. controls (-0.1498, 5.7766) and (-0.1672, 5.7854) .. (-0.1852, 5.7944).. controls (-0.2114, 5.8031) and (-0.2376, 5.8118) .. (-0.2646, 5.8208).. controls (-0.2306, 5.8519) and (-0.1963, 5.8827) .. (-0.1621, 5.9134).. controls (-0.1335, 5.9392) and (-0.1335, 5.9392) .. (-0.1044, 5.9655).. controls (-0.0529, 6.006) and (-0.0529, 6.006) .. (0.0, 6.006) -- cycle;



  \path[fill=signature_color,shift={(7.3819, -1.2965)}] (0.0, 6.006).. controls (0.0175, 6.006) and (0.0349, 6.006) .. (0.0529, 6.006).. controls (0.026, 5.8614) and (-0.0019, 5.7212) .. (-0.0529, 5.5827).. controls (-0.0616, 5.5827) and (-0.0704, 5.5827) .. (-0.0794, 5.5827).. controls (-0.0827, 5.7878) and (-0.0827, 5.7878) .. (-0.0529, 5.8473).. controls (-0.0355, 5.8473) and (-0.018, 5.8473) .. (0.0, 5.8473).. controls (0.0, 5.8997) and (0.0, 5.9521) .. (0.0, 6.006) -- cycle;



  \path[fill=signature_color,shift={(10.0277, -1.8521)}] (0.0, 6.006).. controls (0.0262, 5.9973) and (0.0524, 5.9886) .. (0.0794, 5.9796).. controls (0.0794, 5.9621) and (0.0794, 5.9447) .. (0.0794, 5.9267).. controls (0.1056, 5.9179) and (0.1318, 5.9092) .. (0.1588, 5.9002).. controls (0.1326, 5.8609) and (0.1326, 5.8609) .. (0.1058, 5.8208).. controls (0.0273, 5.8339) and (0.0273, 5.8339) .. (-0.0529, 5.8473).. controls (-0.0355, 5.856) and (-0.018, 5.8648) .. (0.0, 5.8738).. controls (0.0, 5.9174) and (0.0, 5.9611) .. (0.0, 6.006) -- cycle;



  \path[fill=signature_color,shift={(13.9171, -1.5875)}] (0.0, 6.006).. controls (0.0268, 5.9256) and (0.0439, 5.8523) .. (0.0529, 5.7679).. controls (0.0355, 5.7505) and (0.018, 5.733) .. (0.0, 5.715).. controls (-0.0076, 5.7319) and (-0.0153, 5.7488) .. (-0.0232, 5.7663).. controls (-0.0484, 5.8249) and (-0.0484, 5.8249) .. (-0.1058, 5.8473).. controls (-0.0709, 5.8997) and (-0.036, 5.9521) .. (0.0, 6.006) -- cycle;



  \path[fill=signature_color,shift={(3.2544, -0.979)}] (0.0, 6.006).. controls (0.0175, 5.9973) and (0.0349, 5.9886) .. (0.0529, 5.9796).. controls (0.0355, 5.9272) and (0.018, 5.8748) .. (0.0, 5.8208).. controls (-0.0087, 5.847) and (-0.0175, 5.8732) .. (-0.0265, 5.9002).. controls (-0.0527, 5.9002) and (-0.0788, 5.9002) .. (-0.1058, 5.9002).. controls (-0.1146, 5.874) and (-0.1233, 5.8478) .. (-0.1323, 5.8208).. controls (-0.1672, 5.8296) and (-0.2021, 5.8383) .. (-0.2381, 5.8473).. controls (-0.1595, 5.8997) and (-0.081, 5.9521) .. (0.0, 6.006) -- cycle;



  \path[fill=signature_color,shift={(14.2875, -0.8731)}] (0.0, 6.006).. controls (0.0175, 5.9973) and (0.0349, 5.9886) .. (0.0529, 5.9796).. controls (0.039, 5.8794) and (0.0164, 5.8163) .. (-0.0529, 5.7415).. controls (-0.0704, 5.7415) and (-0.0878, 5.7415) .. (-0.1058, 5.7415).. controls (-0.0765, 5.8337) and (-0.0459, 5.9207) .. (0.0, 6.006) -- cycle;



  \path[fill=signature_color,shift={(3.6248, -4.789)}] (0.0, 6.006).. controls (0.0146, 5.9511) and (0.0265, 5.9043) .. (0.0265, 5.8473).. controls (-0.0728, 5.8175) and (-0.0728, 5.8175) .. (-0.1323, 5.8473).. controls (-0.1323, 5.8822) and (-0.1323, 5.9171) .. (-0.1323, 5.9531).. controls (-0.0886, 5.9706) and (-0.045, 5.9881) .. (0.0, 6.006) -- cycle;



  \path[fill=signature_color,shift={(10.2129, -3.4131)}] (0.0, 6.006).. controls (0.1135, 5.9798) and (0.227, 5.9537) .. (0.344, 5.9267).. controls (0.344, 5.9179) and (0.344, 5.9092) .. (0.344, 5.9002).. controls (0.2381, 5.9002) and (0.1323, 5.9002) .. (0.0265, 5.9002).. controls (0.0177, 5.9351) and (0.009, 5.9701) .. (0.0, 6.006) -- cycle;



  \path[fill=signature_color,shift={(12.5942, -2.9633)}] (0.0, 6.006).. controls (0.0175, 5.9973) and (0.0349, 5.9886) .. (0.0529, 5.9796).. controls (-0.0234, 5.8894) and (-0.0627, 5.8508) .. (-0.1802, 5.8258).. controls (-0.222, 5.8233) and (-0.222, 5.8233) .. (-0.2646, 5.8208).. controls (-0.1803, 5.8907) and (-0.0976, 5.9559) .. (0.0, 6.006) -- cycle;



  \path[fill=signature_color,shift={(11.8798, -2.0902)}] (0.0, 6.006).. controls (0.0731, 5.933) and (0.0682, 5.8683) .. (0.0794, 5.7679).. controls (0.0532, 5.7941) and (0.027, 5.8203) .. (0.0, 5.8473).. controls (-0.0262, 5.808) and (-0.0262, 5.808) .. (-0.0529, 5.7679).. controls (-0.0704, 5.7766) and (-0.0878, 5.7854) .. (-0.1058, 5.7944).. controls (-0.0709, 5.8642) and (-0.036, 5.9341) .. (0.0, 6.006) -- cycle;



  \path[fill=signature_color,shift={(14.4727, -0.4233)}] (0.0, 6.006).. controls (0.0349, 5.9711) and (0.0698, 5.9362) .. (0.1058, 5.9002).. controls (0.0884, 5.8566) and (0.0709, 5.8129) .. (0.0529, 5.7679).. controls (0.018, 5.7766) and (-0.0169, 5.7854) .. (-0.0529, 5.7944).. controls (-0.0355, 5.8642) and (-0.018, 5.9341) .. (0.0, 6.006) -- cycle;



  \path[fill=signature_color,shift={(3.8365, -0.7408)}] (0.0, 6.006).. controls (0.0175, 5.9798) and (0.0349, 5.9537) .. (0.0529, 5.9267).. controls (0.0442, 5.9005) and (0.0355, 5.8743) .. (0.0265, 5.8473).. controls (-0.0281, 5.8291) and (-0.0281, 5.8291) .. (-0.0794, 5.8208).. controls (-0.0881, 5.8558) and (-0.0968, 5.8907) .. (-0.1058, 5.9267).. controls (-0.0709, 5.9529) and (-0.036, 5.9791) .. (0.0, 6.006) -- cycle;



  \path[fill=signature_color,shift={(11.0331, -4.4715)}] (0.0, 6.006).. controls (0.0175, 6.006) and (0.0349, 6.006) .. (0.0529, 6.006).. controls (0.0529, 5.9886) and (0.0529, 5.9711) .. (0.0529, 5.9531).. controls (0.0704, 5.9444) and (0.0878, 5.9357) .. (0.1058, 5.9267).. controls (0.0884, 5.8917) and (0.0709, 5.8568) .. (0.0529, 5.8208).. controls (0.0355, 5.8208) and (0.018, 5.8208) .. (0.0, 5.8208).. controls (-0.0087, 5.7859) and (-0.0175, 5.751) .. (-0.0265, 5.715).. controls (-0.0177, 5.811) and (-0.009, 5.9071) .. (0.0, 6.006) -- cycle;



  \path[fill=signature_color,shift={(3.1221, -3.0427)}] (0.0, 6.006).. controls (0.0175, 5.9798) and (0.0349, 5.9537) .. (0.0529, 5.9267).. controls (0.0355, 5.9267) and (0.018, 5.9267) .. (0.0, 5.9267).. controls (0.0, 5.8917) and (0.0, 5.8568) .. (0.0, 5.8208).. controls (-0.0175, 5.8208) and (-0.0349, 5.8208) .. (-0.0529, 5.8208).. controls (-0.0529, 5.7946) and (-0.0529, 5.7684) .. (-0.0529, 5.7415).. controls (-0.0704, 5.7415) and (-0.0878, 5.7415) .. (-0.1058, 5.7415).. controls (-0.0984, 5.7806) and (-0.0906, 5.8198) .. (-0.0827, 5.8589).. controls (-0.0784, 5.8807) and (-0.0741, 5.9025) .. (-0.0697, 5.9249).. controls (-0.0641, 5.943) and (-0.0586, 5.961) .. (-0.0529, 5.9796).. controls (-0.0355, 5.9883) and (-0.018, 5.997) .. (0.0, 6.006) -- cycle;



  \path[fill=signature_color,shift={(13.5996, -0.4498)}] (0.0, 6.006).. controls (-0.0087, 5.9711) and (-0.0175, 5.9362) .. (-0.0265, 5.9002).. controls (-0.1054, 5.8798) and (-0.1846, 5.8629) .. (-0.2646, 5.8473).. controls (-0.1946, 5.9246) and (-0.1086, 6.006) .. (0.0, 6.006) -- cycle;



  \path[fill=signature_color,shift={(0.4233, -3.6777)}] (0.0, 6.006).. controls (0.0524, 5.9929) and (0.0524, 5.9929) .. (0.1058, 5.9796).. controls (0.086, 5.8539) and (0.086, 5.8539) .. (0.0265, 5.7944).. controls (-0.0265, 5.8738) and (-0.0265, 5.8738) .. (-0.0165, 5.9449).. controls (-0.0111, 5.965) and (-0.0056, 5.9852) .. (0.0, 6.006) -- cycle;



  \path[fill=signature_color,shift={(10.0806, -3.6248)}] (0.0, 6.006).. controls (0.1398, 6.0177) and (0.1398, 6.0177) .. (0.2117, 6.006).. controls (0.2778, 5.9399) and (0.2778, 5.9399) .. (0.3175, 5.8738).. controls (0.2744, 5.8857) and (0.2315, 5.8978) .. (0.1885, 5.9101).. controls (0.1526, 5.9203) and (0.1526, 5.9203) .. (0.116, 5.9306).. controls (0.0529, 5.9531) and (0.0529, 5.9531) .. (0.0, 6.006) -- cycle;



  \path[fill=signature_color,shift={(12.8852, -2.7781)}] (0.0, 6.006).. controls (0.0087, 5.9886) and (0.0175, 5.9711) .. (0.0265, 5.9531).. controls (-0.0017, 5.9019) and (-0.0017, 5.9019) .. (-0.0529, 5.8473).. controls (-0.1505, 5.8274) and (-0.1505, 5.8274) .. (-0.2381, 5.8208).. controls (-0.1597, 5.884) and (-0.0809, 5.9461) .. (0.0, 6.006) -- cycle;



  \path[fill=signature_color,shift={(7.0379, -2.6458)}] (0.0, 6.006).. controls (0.0087, 6.006) and (0.0175, 6.006) .. (0.0265, 6.006).. controls (0.0259, 5.9766) and (0.0254, 5.9471) .. (0.0248, 5.9167).. controls (0.0195, 5.8253) and (0.0195, 5.8253) .. (0.0529, 5.7679).. controls (0.0442, 5.7505) and (0.0355, 5.733) .. (0.0265, 5.715).. controls (0.009, 5.715) and (-0.0085, 5.715) .. (-0.0265, 5.715).. controls (-0.0352, 5.6975) and (-0.0439, 5.6801) .. (-0.0529, 5.6621).. controls (-0.0492, 5.6996) and (-0.0453, 5.7371) .. (-0.0413, 5.7745).. controls (-0.0392, 5.7954) and (-0.037, 5.8163) .. (-0.0348, 5.8378).. controls (-0.0271, 5.8955) and (-0.0153, 5.9499) .. (0.0, 6.006) -- cycle;



  \path[fill=signature_color,shift={(3.3073, -2.2225)}] (0.0, 6.006).. controls (0.0428, 5.938) and (0.0541, 5.9088) .. (0.043, 5.8274).. controls (0.0375, 5.8078) and (0.0321, 5.7882) .. (0.0265, 5.7679).. controls (0.0003, 5.7766) and (-0.0259, 5.7854) .. (-0.0529, 5.7944).. controls (-0.0298, 5.9465) and (-0.0298, 5.9465) .. (0.0, 6.006) -- cycle;



  \path[fill=signature_color,shift={(14.5521, -1.1113)}] (0.0, 6.006).. controls (0.0175, 6.006) and (0.0349, 6.006) .. (0.0529, 6.006).. controls (0.0066, 5.8539) and (0.0066, 5.8539) .. (-0.0529, 5.7944).. controls (-0.0704, 5.8031) and (-0.0878, 5.8118) .. (-0.1058, 5.8208).. controls (-0.1058, 5.847) and (-0.1058, 5.8732) .. (-0.1058, 5.9002).. controls (-0.0884, 5.9002) and (-0.0709, 5.9002) .. (-0.0529, 5.9002).. controls (-0.0529, 5.9177) and (-0.0529, 5.9351) .. (-0.0529, 5.9531).. controls (-0.0355, 5.9531) and (-0.018, 5.9531) .. (0.0, 5.9531).. controls (0.0, 5.9706) and (0.0, 5.9881) .. (0.0, 6.006) -- cycle;



  \path[fill=signature_color,shift={(0.344, -4.5508)}] (0.0, 6.006).. controls (0.0262, 5.9973) and (0.0524, 5.9886) .. (0.0794, 5.9796).. controls (0.1108, 5.8853) and (0.1108, 5.8853) .. (0.1323, 5.7944).. controls (0.1148, 5.7944) and (0.0974, 5.7944) .. (0.0794, 5.7944).. controls (0.0215, 5.8556) and (0.0215, 5.8556) .. (-0.0265, 5.9267).. controls (-0.0177, 5.9529) and (-0.009, 5.9791) .. (0.0, 6.006) -- cycle;



  \path[fill=signature_color,shift={(11.5094, -2.8575)}] (0.0, 6.006).. controls (0.0298, 5.9068) and (0.0298, 5.9068) .. (0.0, 5.8473).. controls (0.0175, 5.8473) and (0.0349, 5.8473) .. (0.0529, 5.8473).. controls (0.0529, 5.8298) and (0.0529, 5.8124) .. (0.0529, 5.7944).. controls (0.018, 5.7944) and (-0.0169, 5.7944) .. (-0.0529, 5.7944).. controls (-0.0616, 5.7769) and (-0.0704, 5.7595) .. (-0.0794, 5.7415).. controls (-0.066, 5.8399) and (-0.0498, 5.9196) .. (0.0, 6.006) -- cycle;



  \path[fill=signature_color,shift={(1.2965, -2.7517)}] (0.0, 6.006).. controls (0.0524, 6.006) and (0.1048, 6.006) .. (0.1588, 6.006).. controls (0.1153, 5.9526) and (0.0712, 5.8997) .. (0.0265, 5.8473).. controls (0.0177, 5.8473) and (0.009, 5.8473) .. (0.0, 5.8473).. controls (0.0, 5.8997) and (0.0, 5.9521) .. (0.0, 6.006) -- cycle;



  \path[fill=signature_color,shift={(13.3879, -2.3019)}] (0.0, 6.006).. controls (0.0, 5.9531) and (0.0, 5.9002) .. (0.0, 5.8473).. controls (-0.0628, 5.801) and (-0.0628, 5.801) .. (-0.1323, 5.7679).. controls (-0.1498, 5.7766) and (-0.1672, 5.7854) .. (-0.1852, 5.7944).. controls (-0.1241, 5.8642) and (-0.063, 5.9341) .. (0.0, 6.006) -- cycle;



  \path[fill=signature_color,shift={(11.3506, -1.7992)}] (0.0, 6.006).. controls (0.0175, 5.9973) and (0.0349, 5.9886) .. (0.0529, 5.9796).. controls (0.0442, 5.9359) and (0.0355, 5.8923) .. (0.0265, 5.8473).. controls (-0.0085, 5.856) and (-0.0434, 5.8648) .. (-0.0794, 5.8738).. controls (-0.0298, 5.9763) and (-0.0298, 5.9763) .. (0.0, 6.006) -- cycle;



  \path[fill=signature_color,shift={(5.3975, -3.6248)}] (0.0, 6.006).. controls (0.0175, 5.9973) and (0.0349, 5.9886) .. (0.0529, 5.9796).. controls (0.0529, 5.9621) and (0.0529, 5.9447) .. (0.0529, 5.9267).. controls (0.1053, 5.9092) and (0.1577, 5.8917) .. (0.2117, 5.8738).. controls (0.2117, 5.865) and (0.2117, 5.8563) .. (0.2117, 5.8473).. controls (0.1096, 5.8415) and (0.0399, 5.8578) .. (-0.0529, 5.9002).. controls (-0.0355, 5.9351) and (-0.018, 5.9701) .. (0.0, 6.006) -- cycle;



  \path[fill=signature_color,shift={(11.9327, -3.519)}] (0.0, 6.006).. controls (0.0175, 6.006) and (0.0349, 6.006) .. (0.0529, 6.006).. controls (0.0529, 5.9886) and (0.0529, 5.9711) .. (0.0529, 5.9531).. controls (0.0878, 5.9706) and (0.1228, 5.9881) .. (0.1588, 6.006).. controls (0.1588, 5.9798) and (0.1588, 5.9537) .. (0.1588, 5.9267).. controls (0.0889, 5.9092) and (0.0191, 5.8917) .. (-0.0529, 5.8738).. controls (-0.0529, 5.8912) and (-0.0529, 5.9087) .. (-0.0529, 5.9267).. controls (-0.0355, 5.9267) and (-0.018, 5.9267) .. (0.0, 5.9267).. controls (0.0, 5.9529) and (0.0, 5.9791) .. (0.0, 6.006) -- cycle;



  \path[fill=signature_color,shift={(8.6519, -3.3602)}] (0.0, 6.006).. controls (0.0, 5.9886) and (0.0, 5.9711) .. (0.0, 5.9531).. controls (-0.0175, 5.9531) and (-0.0349, 5.9531) .. (-0.0529, 5.9531).. controls (-0.0529, 5.9269) and (-0.0529, 5.9007) .. (-0.0529, 5.8738).. controls (-0.0791, 5.8825) and (-0.1053, 5.8912) .. (-0.1323, 5.9002).. controls (-0.1323, 5.9177) and (-0.1323, 5.9351) .. (-0.1323, 5.9531).. controls (-0.1672, 5.9619) and (-0.2021, 5.9706) .. (-0.2381, 5.9796).. controls (-0.1572, 5.994) and (-0.0823, 6.006) .. (0.0, 6.006) -- cycle;



  \path[fill=signature_color,shift={(2.6458, -1.5875)}] (0.0, 6.006).. controls (0.0655, 5.9798) and (0.0655, 5.9798) .. (0.1323, 5.9531).. controls (0.0888, 5.9085) and (0.0446, 5.8644) .. (0.0, 5.8208).. controls (-0.0087, 5.8208) and (-0.0175, 5.8208) .. (-0.0265, 5.8208).. controls (-0.0276, 5.8737) and (-0.0275, 5.9267) .. (-0.0265, 5.9796).. controls (-0.0177, 5.9883) and (-0.009, 5.997) .. (0.0, 6.006) -- cycle;



  \path[fill=signature_color,shift={(14.4727, -0.0265)}] (0.0, 6.006).. controls (0.0873, 6.006) and (0.1746, 6.006) .. (0.2646, 6.006).. controls (0.2646, 5.9886) and (0.2646, 5.9711) .. (0.2646, 5.9531).. controls (0.1636, 5.9143) and (0.0982, 5.9417) .. (0.0, 5.9796).. controls (0.0, 5.9883) and (0.0, 5.997) .. (0.0, 6.006) -- cycle;



  \path[fill=signature_color,shift={(3.0163, -5.0271)}] (0.0, 6.006).. controls (0.0246, 5.9987) and (0.0246, 5.9987) .. (0.0496, 5.9912).. controls (0.1069, 5.9733) and (0.1069, 5.9733) .. (0.1588, 6.006).. controls (0.1588, 5.9537) and (0.1588, 5.9013) .. (0.1588, 5.8473).. controls (0.0298, 5.9167) and (0.0298, 5.9167) .. (0.0, 6.006) -- cycle;



  \path[fill=signature_color,shift={(4.7096, -4.0481)}] (0.0, 6.006).. controls (0.0087, 5.9798) and (0.0175, 5.9537) .. (0.0265, 5.9267).. controls (0.0439, 5.9267) and (0.0614, 5.9267) .. (0.0794, 5.9267).. controls (0.0794, 5.9441) and (0.0794, 5.9616) .. (0.0794, 5.9796).. controls (0.1056, 5.9709) and (0.1318, 5.9621) .. (0.1588, 5.9531).. controls (0.0802, 5.8876) and (0.0802, 5.8876) .. (0.0, 5.8208).. controls (-0.0276, 5.9037) and (-0.0247, 5.9264) .. (0.0, 6.006) -- cycle;



  \path[fill=signature_color,shift={(7.1967, -3.8365)}] (0.0, 6.006).. controls (0.0087, 6.006) and (0.0175, 6.006) .. (0.0265, 6.006).. controls (0.0265, 5.9537) and (0.0265, 5.9013) .. (0.0265, 5.8473).. controls (-0.0259, 5.8211) and (-0.0259, 5.8211) .. (-0.0794, 5.7944).. controls (-0.0649, 5.8837) and (-0.0515, 5.9288) .. (0.0, 6.006) -- cycle;



  \path[fill=signature_color,shift={(2.884, -3.5454)}] (0.0, 6.006).. controls (0.0262, 5.9711) and (0.0524, 5.9362) .. (0.0794, 5.9002).. controls (0.0619, 5.8827) and (0.0445, 5.8653) .. (0.0265, 5.8473).. controls (0.009, 5.856) and (-0.0085, 5.8648) .. (-0.0265, 5.8738).. controls (-0.0265, 5.8476) and (-0.0265, 5.8214) .. (-0.0265, 5.7944).. controls (-0.0439, 5.7944) and (-0.0614, 5.7944) .. (-0.0794, 5.7944).. controls (-0.0298, 5.9465) and (-0.0298, 5.9465) .. (0.0, 6.006) -- cycle;



  \path[fill=signature_color,shift={(7.5142, -3.519)}] (0.0, 6.006).. controls (0.0794, 5.9068) and (0.0794, 5.9068) .. (0.0794, 5.8473).. controls (0.0095, 5.856) and (-0.0603, 5.8648) .. (-0.1323, 5.8738).. controls (-0.1323, 5.8825) and (-0.1323, 5.8912) .. (-0.1323, 5.9002).. controls (-0.0886, 5.9002) and (-0.045, 5.9002) .. (0.0, 5.9002).. controls (0.0, 5.9351) and (0.0, 5.9701) .. (0.0, 6.006) -- cycle;



  \path[fill=signature_color,shift={(1.0848, -2.5929)}] (0.0, 6.006).. controls (0.0175, 5.9973) and (0.0349, 5.9886) .. (0.0529, 5.9796).. controls (0.0136, 5.8879) and (0.0136, 5.8879) .. (-0.0265, 5.7944).. controls (-0.0439, 5.7944) and (-0.0614, 5.7944) .. (-0.0794, 5.7944).. controls (-0.0595, 5.9465) and (-0.0595, 5.9465) .. (0.0, 6.006) -- cycle;



  \path[fill=signature_color,shift={(7.6465, -2.6194)}] (0.0, 6.006).. controls (0.0513, 5.9978) and (0.0513, 5.9978) .. (0.1058, 5.9796).. controls (0.1146, 5.9534) and (0.1233, 5.9272) .. (0.1323, 5.9002).. controls (0.0886, 5.8915) and (0.045, 5.8827) .. (0.0, 5.8738).. controls (0.0, 5.9174) and (0.0, 5.9611) .. (0.0, 6.006) -- cycle;



  \path[fill=signature_color,shift={(8.599, -2.3548)}] (0.0, 6.006).. controls (0.0524, 6.006) and (0.1048, 6.006) .. (0.1588, 6.006).. controls (0.1675, 5.9886) and (0.1762, 5.9711) .. (0.1852, 5.9531).. controls (0.1066, 5.9357) and (0.028, 5.9182) .. (-0.0529, 5.9002).. controls (-0.0529, 5.9177) and (-0.0529, 5.9351) .. (-0.0529, 5.9531).. controls (-0.0355, 5.9531) and (-0.018, 5.9531) .. (0.0, 5.9531).. controls (0.0, 5.9706) and (0.0, 5.9881) .. (0.0, 6.006) -- cycle;



  \path[fill=signature_color,shift={(11.43, -5.715)}] (0.0, 6.006).. controls (0.0175, 6.006) and (0.0349, 6.006) .. (0.0529, 6.006).. controls (0.0331, 5.8804) and (0.0331, 5.8804) .. (-0.0265, 5.8208).. controls (-0.0527, 5.8558) and (-0.0788, 5.8907) .. (-0.1058, 5.9267).. controls (-0.0796, 5.9354) and (-0.0534, 5.9441) .. (-0.0265, 5.9531).. controls (-0.0177, 5.9706) and (-0.009, 5.9881) .. (0.0, 6.006) -- cycle;



  \path[fill=signature_color,shift={(0.3175, -4.9213)}] (0.0, 6.006).. controls (0.0262, 6.006) and (0.0524, 6.006) .. (0.0794, 6.006).. controls (0.0706, 5.9537) and (0.0619, 5.9013) .. (0.0529, 5.8473).. controls (0.0267, 5.8648) and (0.0005, 5.8822) .. (-0.0265, 5.9002).. controls (-0.0177, 5.9351) and (-0.009, 5.9701) .. (0.0, 6.006) -- cycle;



  \path[fill=signature_color,shift={(0.4233, -3.8894)}] (0.0, 6.006).. controls (0.0087, 5.9886) and (0.0175, 5.9711) .. (0.0265, 5.9531).. controls (0.0177, 5.9269) and (0.009, 5.9007) .. (0.0, 5.8738).. controls (-0.0175, 5.8738) and (-0.0349, 5.8738) .. (-0.0529, 5.8738).. controls (-0.0616, 5.8476) and (-0.0704, 5.8214) .. (-0.0794, 5.7944).. controls (-0.1041, 5.874) and (-0.107, 5.8967) .. (-0.0794, 5.9796).. controls (-0.0532, 5.9883) and (-0.027, 5.997) .. (0.0, 6.006) -- cycle;



  \path[fill=signature_color,shift={(7.6729, -2.4606)}] (0.0, 6.006).. controls (0.0, 5.9886) and (0.0, 5.9711) .. (0.0, 5.9531).. controls (-0.0609, 5.8945) and (-0.0933, 5.8746) .. (-0.1786, 5.8688).. controls (-0.2081, 5.8712) and (-0.2081, 5.8712) .. (-0.2381, 5.8738).. controls (-0.1708, 5.9478) and (-0.1038, 6.006) .. (0.0, 6.006) -- cycle;



  \path[fill=signature_color,shift={(12.2238, -2.3813)}] (0.0, 6.006).. controls (0.0175, 6.006) and (0.0349, 6.006) .. (0.0529, 6.006).. controls (0.0066, 5.8539) and (0.0066, 5.8539) .. (-0.0529, 5.7944).. controls (-0.0616, 5.838) and (-0.0704, 5.8817) .. (-0.0794, 5.9267).. controls (-0.0532, 5.9354) and (-0.027, 5.9441) .. (0.0, 5.9531).. controls (0.0, 5.9706) and (0.0, 5.9881) .. (0.0, 6.006) -- cycle;



  \path[fill=signature_color,shift={(1.7727, -2.249)}] (0.0, 6.006).. controls (0.0175, 6.006) and (0.0349, 6.006) .. (0.0529, 6.006).. controls (0.0704, 5.9624) and (0.0878, 5.9187) .. (0.1058, 5.8738).. controls (0.0273, 5.8868) and (0.0273, 5.8868) .. (-0.0529, 5.9002).. controls (-0.0355, 5.9351) and (-0.018, 5.9701) .. (0.0, 6.006) -- cycle;



  \path[fill=signature_color,shift={(13.8642, -1.7198)}] (0.0, 6.006).. controls (0.0281, 5.9564) and (0.0281, 5.9564) .. (0.0529, 5.9002).. controls (0.0442, 5.8827) and (0.0355, 5.8653) .. (0.0265, 5.8473).. controls (-0.0281, 5.8308) and (-0.0281, 5.8308) .. (-0.0794, 5.8208).. controls (-0.0706, 5.8645) and (-0.0619, 5.9081) .. (-0.0529, 5.9531).. controls (-0.0355, 5.9531) and (-0.018, 5.9531) .. (0.0, 5.9531).. controls (0.0, 5.9706) and (0.0, 5.9881) .. (0.0, 6.006) -- cycle;



  \path[fill=signature_color,shift={(3.519, -5.0006)}] (0.0, 6.006).. controls (0.0087, 5.9798) and (0.0175, 5.9537) .. (0.0265, 5.9267).. controls (0.0527, 5.9179) and (0.0788, 5.9092) .. (0.1058, 5.9002).. controls (0.0447, 5.8827) and (-0.0164, 5.8653) .. (-0.0794, 5.8473).. controls (-0.0794, 5.8648) and (-0.0794, 5.8822) .. (-0.0794, 5.9002).. controls (-0.0619, 5.9002) and (-0.0445, 5.9002) .. (-0.0265, 5.9002).. controls (-0.0177, 5.9351) and (-0.009, 5.9701) .. (0.0, 6.006) -- cycle;



  \path[fill=signature_color,shift={(3.4396, -4.9742)}] (0.0, 6.006).. controls (0.0175, 5.9973) and (0.0349, 5.9886) .. (0.0529, 5.9796).. controls (0.0529, 5.9447) and (0.0529, 5.9097) .. (0.0529, 5.8738).. controls (0.0355, 5.8738) and (0.018, 5.8738) .. (0.0, 5.8738).. controls (-0.0087, 5.8563) and (-0.0175, 5.8388) .. (-0.0265, 5.8208).. controls (-0.043, 5.8837) and (-0.043, 5.8837) .. (-0.0529, 5.9531).. controls (-0.0355, 5.9706) and (-0.018, 5.9881) .. (0.0, 6.006) -- cycle;



  \path[fill=signature_color,shift={(2.6458, -3.9688)}] (0.0, 6.006).. controls (0.0262, 5.9973) and (0.0524, 5.9886) .. (0.0794, 5.9796).. controls (0.0761, 5.93) and (0.0761, 5.93) .. (0.0529, 5.8738).. controls (-0.0281, 5.8407) and (-0.0281, 5.8407) .. (-0.1058, 5.8208).. controls (-0.0709, 5.882) and (-0.036, 5.9431) .. (0.0, 6.006) -- cycle;



  \path[fill=signature_color,shift={(0.3704, -3.3338)}] (0.0, 6.006).. controls (0.0182, 5.9401) and (0.0265, 5.8902) .. (0.0265, 5.8208).. controls (-0.0172, 5.8208) and (-0.0609, 5.8208) .. (-0.1058, 5.8208).. controls (-0.0709, 5.882) and (-0.036, 5.9431) .. (0.0, 6.006) -- cycle;



  \path[fill=signature_color,shift={(12.4354, -3.3073)}] (0.0, 6.006).. controls (0.0087, 5.9886) and (0.0175, 5.9711) .. (0.0265, 5.9531).. controls (0.0701, 5.9706) and (0.1138, 5.9881) .. (0.1588, 6.006).. controls (0.1675, 5.9886) and (0.1762, 5.9711) .. (0.1852, 5.9531).. controls (0.0935, 5.9138) and (0.0935, 5.9138) .. (0.0, 5.8738).. controls (0.0, 5.9174) and (0.0, 5.9611) .. (0.0, 6.006) -- cycle;



  \path[fill=signature_color,shift={(12.9117, -3.0163)}] (0.0, 6.006).. controls (0.0262, 5.9973) and (0.0524, 5.9886) .. (0.0794, 5.9796).. controls (0.0881, 5.9534) and (0.0968, 5.9272) .. (0.1058, 5.9002).. controls (0.0622, 5.8827) and (0.0185, 5.8653) .. (-0.0265, 5.8473).. controls (-0.0177, 5.8997) and (-0.009, 5.9521) .. (0.0, 6.006) -- cycle;



  \path[fill=signature_color,shift={(1.2171, -2.8046)}] (0.0, 6.006).. controls (0.0087, 5.9886) and (0.0175, 5.9711) .. (0.0265, 5.9531).. controls (0.0439, 5.9531) and (0.0614, 5.9531) .. (0.0794, 5.9531).. controls (0.0794, 5.9269) and (0.0794, 5.9007) .. (0.0794, 5.8738).. controls (0.0008, 5.8868) and (0.0008, 5.8868) .. (-0.0794, 5.9002).. controls (-0.0794, 5.9177) and (-0.0794, 5.9351) .. (-0.0794, 5.9531).. controls (-0.0532, 5.9531) and (-0.027, 5.9531) .. (0.0, 5.9531).. controls (0.0, 5.9706) and (0.0, 5.9881) .. (0.0, 6.006) -- cycle;



  \path[fill=signature_color,shift={(12.3296, -1.4817)}] (0.0, 6.006).. controls (0.0087, 6.006) and (0.0175, 6.006) .. (0.0265, 6.006).. controls (0.0352, 5.9537) and (0.0439, 5.9013) .. (0.0529, 5.8473).. controls (-0.0082, 5.8386) and (-0.0693, 5.8298) .. (-0.1323, 5.8208).. controls (-0.1109, 5.8518) and (-0.0894, 5.8826) .. (-0.0678, 5.9134).. controls (-0.0558, 5.9306) and (-0.0439, 5.9478) .. (-0.0315, 5.9655).. controls (-0.0211, 5.9789) and (-0.0107, 5.9923) .. (0.0, 6.006) -- cycle;



  \path[fill=signature_color,shift={(3.2808, -1.1642)}] (0.0, 6.006).. controls (0.0349, 5.9886) and (0.0699, 5.9711) .. (0.1058, 5.9531).. controls (0.0447, 5.9182) and (-0.0164, 5.8833) .. (-0.0794, 5.8473).. controls (-0.0794, 5.9267) and (-0.0794, 5.9267) .. (-0.0397, 5.9713).. controls (-0.02, 5.9885) and (-0.02, 5.9885) .. (0.0, 6.006) -- cycle;




\end{tikzpicture}


    \vspace{-17mm}
    \noindent \rule{\textwidth}{0.4pt} \\ % Línea de firma
    \textbf{Provider Signature} \\
    Date: \makebox[1.5in]{\textbf{\today}}
\end{minipage}
\hfill
\begin{minipage}[t]{0.45\textwidth}
    \vspace{-6mm}
    \noindent \rule{\textwidth}{0.4pt} \\
    \textbf{Client Signature} \\
    Date: \makebox[1.5in]{\hrulefill}
\end{minipage}

\end{document}
