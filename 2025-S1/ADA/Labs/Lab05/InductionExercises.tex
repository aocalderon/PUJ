\documentclass{beamer}
\usepackage{graphicx}
\usepackage{booktabs}
\usepackage{tcolorbox}
\usepackage{amsmath, esint}
\usepackage{tikz}
\usetikzlibrary{backgrounds, positioning, arrows, scopes, shapes, shapes.misc, shapes.multipart}
\tikzset{
    cross/.style={cross out, draw=black, minimum size=2*(#1-\pgflinewidth), inner sep=0pt, outer sep=0pt},
    cross/.default={10pt},
    split/.style={rectangle split, rectangle split parts=7, draw, inner sep=0ex, rectangle split horizontal, minimum size=4ex},
    textstyle/.style={text height=1.5ex, text depth=.25ex}
}

\usepackage{amsthm}
\usepackage{amstext}
\usepackage{amssymb}
\usepackage[plain]{algorithm}
\usepackage{algpseudocode}
%setup new colors
\hypersetup{
%linkcolor=blue
%,citecolor=
%,filecolor=
urlcolor=blue
%,menucolor=
%,runcolor=
%,linkbordercolor=
%,citebordercolor=
%,filebordercolor=
%,urlbordercolor=
%,menubordercolor=
%,runbordercolor=
}

\usepackage{xcolor} 
\definecolor{LightGray}{gray}{0.975}

\renewcommand{\qed}{\\ \hfill $\blacksquare$}

%\usetheme{Warsaw}
\usefonttheme{serif} 

\title[Induction]{Introduction to Algorithms \\ Bonus Lecture: Proof by Induction}
\author{et al.}
\date{\today}

\setbeamertemplate{navigation symbols}{}%remove navigation symbols

\defbeamertemplate*{footline}{shadow theme}{%
    \leavevmode%
    \hbox{
        \begin{beamercolorbox}[wd=.5\paperwidth,ht=2.5ex,dp=1.125ex,leftskip=.3cm plus1fil,rightskip=.3cm]{author in head/foot}%
            \usebeamerfont{author in head/foot} Introduction to Algorithms: \hfill \insertshorttitle
        \end{beamercolorbox}%
        \begin{beamercolorbox}[wd=.5\paperwidth,ht=2.5ex,dp=1.125ex,leftskip=.3cm,rightskip=.3cm plus1fil]{title in head/foot}%
            \usebeamerfont{title in head/foot} \hfill \insertframenumber\,/\,\inserttotalframenumber%
        \end{beamercolorbox}
    }%
    \vskip0pt%
}

\AtBeginSection[]{
    \begin{frame}<beamer>
    \frametitle{Plan}
    \tableofcontents[currentsection]
    \end{frame}
}

\begin{document}

\frame{\titlepage}

\begin{frame}{Proof by Induction}
    \begin{itemize}
        \item A powerful mathematical technique.
        \item Prove that a statement is true for all natural numbers (or some sequence of numbers). \item It's like knocking over a line of dominoes...
    \end{itemize}
\end{frame}

\begin{frame}{How Induction Works}
    \textbf{Principle of Mathematical Induction:}
    \begin{itemize}
        \item \textbf{Base Case:} Show the statement holds for the first value (usually $n = 1$).
        \item \textbf{Inductive Hypothesis:} Assume the statement holds for some arbitrary $n = k$.
        \item \textbf{Inductive Step:} Prove it holds for $n = k+1$.
    \end{itemize}
    \pause
    \textbf{If those steps hold, the statement is true for all $n$.}
\end{frame}

\begin{frame}{Example}
    \begin{exampleblock}{ }
        Prove that:
        $$
            1 + 2 + 3 + \cdots + n = \frac{n(n + 1)}{2}
        $$
        for all $n \geq 1$.
    \end{exampleblock}
\end{frame}

\begin{frame}{Example}
    \begin{exampleblock}{Step 1: Base Case}
        For $n = 1$:
        $$
            1 = \frac{1(1 + 1)}{2} = \frac{2}{2} = 1
        $$ \checkmark
        \textcolor{olive}{\textbf{True!}}
    \end{exampleblock}
\end{frame}

\begin{frame}{Example}
    \begin{exampleblock}{Step 2: Inductive Hypothesis}
        Assume that for some $n = k$, the formula holds:
        $$
            1 + 2 + 3 + \cdots + k = \frac{k(k + 1)}{2}
        $$
        (This is our assumption or ``\textit{inductive hypothesis}''.)
    \end{exampleblock}
\end{frame}

\begin{frame}{Example}
    \begin{exampleblock}{Step 3: Inductive Step}
        We must prove it holds for $n = k + 1$, meaning:
        $$
            1 + 2 + 3 + \cdots + k + (k + 1) = \frac{(k + 1)(k + 2)}{2}
        $$ \pause
        Using the inductive hypothesis:
        $$
            \left( \frac{k (k + 1)}{2} \right) + (k + 1)
        $$ \pause
        Factor $k + 1$ out:
        $$
            \frac{k (k + 1) + 2(k + 1)}{2} = \frac{(k + 1)(k + 2)}{2}
        $$ \pause
    \end{exampleblock}
    This matches the formula for $n = k + 1$, so the statement holds!
    \qed
\end{frame}

\begin{frame}{Conclusion}
    By induction, the formula is true for all natural numbers n.\\
    \vspace{5mm}

    \textsc{Why Does This Work?}\\
    \vspace{2mm}
    Think of induction like climbing an infinite ladder:

    \begin{itemize}
        \item The \textbf{base case} puts your foot on the first rung.
        \item The \textbf{inductive hypothesis} and the \textbf{inductive step} shows that if you can reach one step, you can reach the next.
    \end{itemize}

    Since they are true, you can climb forever!
    \qed \footnote{$\blacksquare =$ Q.E.D. which means ``\textit{quod erat demonstrandum}''.}
\end{frame}


\begin{frame}{Example}
    \begin{exampleblock}{ }
        Proof by Induction:
        $$
            2^n \geq n + 1
        $$
        for all $n \geq 1$.
    \end{exampleblock}
\end{frame}

\begin{frame}{Step 1: Base Case}
    For $n = 1$:
    \begin{align*}
        2^1 &= 2,   \text{\hspace{7mm} left side} \\
        1 + 1 &= 2. \text{\hspace{8mm} right side}
    \end{align*}
    Since $2 \geq 2$, the base case holds. \checkmark
\end{frame}

\begin{frame}{Step 2: Inductive Hypothesis}
    Assume the statement is true for $n = k$:
    \begin{align*}
        2^k \geq k + 1.
    \end{align*}
    This assumption is the \textit{inductive hypothesis}.
\end{frame}

% Inductive Step
\begin{frame}{Step 3: Inductive Step}
    We need to prove the statement holds for $n = k+1$:
    \begin{align*}
        2^{k+1} &\geq (k+1) + 1.
    \end{align*}
    \pause
    Start with the left-hand side:
    \begin{align*}
        2^{k+1} &= 2 \cdot 2^k.
    \end{align*}
    \pause
    Using the inductive hypothesis $2^k \geq k+1$:
    \begin{align*}
        2^{k+1} &\geq 2 \cdot (k+1) = 2k + 2.
    \end{align*}
    \pause
    Since $2k + 2 \geq k + 2$, the statement holds for $n = k+1$.
    \qed
\end{frame}

% Conclusion
\begin{frame}{Conclusion}
    By mathematical induction, we have proven that:
    \begin{align*}
        2^n \geq n + 1 \quad \text{for all } n \geq 1.
    \end{align*}
    \textbf{Induction helps us prove statements for infinitely many cases!}
\end{frame}

\begin{frame}{Another Example}
    We will prove that the sum of the first $n$ odd numbers is given by:
    \begin{align*}
        1 + 3 + 5 + \dots + (2n - 1) = n^2.
    \end{align*}
\end{frame}

\begin{frame}{Step 1: Base Case}
    For $n = 1$:
    \begin{align*}
        1 = 1^2.
    \end{align*}
    Since both sides are equal, the base case holds. \checkmark
\end{frame}

\begin{frame}{Step 2: Inductive Hypothesis}
  Assume the statement is true for $n = k$:
  \begin{align*}
    1 + 3 + 5 + \dots + (2k - 1) = k^2.
  \end{align*}
  The \textit{inductive hypothesis}.
\end{frame}

\begin{frame}{Step 3: Inductive Step}
    We need to prove the statement holds for $n = k+1$:
    \begin{align*}
        1 + 3 + 5 + \dots + (2k - 1) + (2(k+1) - 1) = (k+1)^2.
    \end{align*}
    \pause
    Using the \textit{inductive hypothesis}:
    \begin{align*}
        k^2 + (2(k+1) - 1).
    \end{align*}
    \pause
    Expanding the term:
    \begin{align*}
        k^2 + (2k + 1) = (k+1)^2.
    \end{align*}
    \pause
    Since both sides match, the statement holds for $n = k+1$.
    \qed
\end{frame}

\begin{frame}{One More}
    Prove that:
    \begin{align*}
        \sum_{i = 0}^n 3^i &= \frac{3^{n + 1} - 1}{2} \\
    \end{align*}
    for all non negative integer $n$.
\end{frame}

\begin{frame}{Step 1: Base Case}
    For $n = 0$:
    \begin{align*}
        \sum_{i = 0}^0 3^i &= \frac{3^{0 + 1} - 1}{2} \\
                       3^0 &= \frac{3^1 - 1}{2} \\
                         1 &= \frac{3 - 1}{2} \\
                         1 &= \frac{2}{2} \\
                         1 &= 1 \\
    \end{align*}
    Since both sides are equal, the base case holds. \checkmark
\end{frame}

\begin{frame}{Step 1: Base Case}
    For $n = 1$:
    \begin{align*}
        \sum_{i = 0}^1 3^i &= \frac{3^{1 + 1} - 1}{2} \\
                 3^0 + 3^1 &= \frac{3^2 - 1}{2} \\
                     1 + 3 &= \frac{9 - 1}{2} \\
                         4 &= \frac{8}{2} \\
                         4 &= 4 \\
    \end{align*}
    Since both sides are equal, the base case holds. \checkmark
\end{frame}

\begin{frame}{Step 2: Inductive Hypothesis}
    Assume the statement is true for $n = k$:
    \begin{align*}
        \sum_{i = 0}^k 3^i &= \frac{3^{k + 1} - 1}{2} \\
    \end{align*}
    The \textit{inductive hypothesis}.
\end{frame}

\begin{frame}{Step 3: Inductive Step}
    \footnotesize
    We need to prove the statement holds for $n = k + 1$:
    \begin{align*}
        \sum_{i = 0}^{k + 1} 3^i &= \frac{3^{(k + 1) + 1} - 1}{2} \\
    \end{align*}
    \pause
    By $\sum$ definition:
    \begin{align*}
        \sum_{i = 0}^k 3^i + 3^{k + 1} &= \frac{3^{(k + 1) + 1} - 1}{2} \\
    \end{align*}
    \pause
    Using the \textit{inductive hypothesis}:
    \begin{align*}
        \frac{3^{k + 1} - 1}{2} + 3^{k + 1} &= \frac{3^{(k + 1) + 1} - 1}{2} \\
    \end{align*}
    \pause
    Expanding the term:
    \begin{align*}
        \frac{3^{k + 1}}{2} - \frac{1}{2} + \frac{2 \cdot 3^{k + 1}}{2} &= \frac{3^{(k + 1) + 1} - 1}{2} \\
        \frac{3^{k + 1} - 1 + 2 \cdot 3^{k + 1}}{2} &= \frac{3^{(k + 1) + 1} - 1}{2} \\
    \end{align*}
\end{frame}

\begin{frame}{Step 3: Inductive Step}
    \footnotesize
    Using the \textit{inductive hypothesis}:
    \begin{align*}
        \frac{3^{k + 1} - 1}{2} + 3^{k + 1} &= \frac{3^{(k + 1) + 1} - 1}{2} \\
    \end{align*}
    Expanding the term:
    \begin{align*}
        \frac{3^{k + 1}}{2} - \frac{1}{2} + \frac{2 \cdot 3^{k + 1}}{2} &= \frac{3^{(k + 1) + 1} - 1}{2} \\
        \frac{3^{k + 1} - 1 + 2 \cdot 3^{k + 1}}{2} &= \frac{3^{(k + 1) + 1} - 1}{2} \\
        \frac{3 \cdot 3^{k + 1} - 1}{2} &= \frac{3^{k + 1 + 1} - 1}{2} \\
        \frac{3^{k + 1 + 1} - 1}{2} &= \frac{3^{k + 1 + 1} - 1}{2} \\
        \frac{3^{k + 2} - 1}{2} &= \frac{3^{k + 2} - 1}{2} \\
    \end{align*}
    \pause
    Since both sides match, the statement holds for $n = k + 1$.
    \qed
\end{frame}

\begin{frame}{More Examples}
    \footnotesize
    Use mathematical induction to show that:
    $$
        1 + 2 + 2^2 + \cdots + 2^n = 2^{n + 1} - 1
    $$
    for all non negative integers $n$.

   \begin{enumerate}[<+->]
        \item \textbf{Basis case:} For $n = 0$, $2^0 = 1 = 2^1 - 1$ \checkmark
        \item \textbf{Inductive hypothesis:} Let $n = k$, so $$ 1 + 2 + 2^2 + \cdots + 2^k = 2^{k + 1} - 1 $$ holds...
        \item \textbf{Inductive step:} Let's solve for $n = k + 1$,
        $$
            1 + 2 + 2^2 + \cdots + 2^{k + 1} = 2^{(k + 1) + 1} - 1
        $$
   \end{enumerate}
\end{frame}

\begin{frame}{More Examples}
    \footnotesize
    \begin{enumerate}
        \item[3] \textbf{Inductive step:} Let's solve for $n = k + 1$, \pause
            $$ 1 + 2 + 2^2 + \cdots + 2^{k + 1} \stackrel{?}{=} 2^{(k + 1) + 1} - 1$$ \pause
            $$ 1 + 2 + 2^2 + \cdots + 2^k + 2^{k + 1} \stackrel{?}{=} 2^{k + 2} - 1$$ \pause
            $$ 2^{k + 1} - 1 + 2^{k + 1} \stackrel{?}{=} 2^{k + 2} - 1 $$ \pause
            $$ 2 \cdot 2^{k + 1} - 1 \stackrel{?}{=} 2^{k + 2} - 1 $$ \pause
            $$ 2^{k + 2} - 1 = 2^{k + 2} - 1 $$ \qed
    \end{enumerate}
\end{frame}

\begin{frame}{Another One}
    \footnotesize
    Prove the following statement by induction:
    $$
        1 + 2^2 + 3^2 + \cdots + n^2 = \frac{n \cdot (n + 1) \cdot (2n + 1)}{6}
    $$
    \begin{enumerate}[<+->]
        \item \textbf{Basis step:} For $n = 1$, $1 = \frac {1 \times 2 \times 3}{6}$ is true!
        \item \textbf{Assumption step:} Let $n = k$, so $$ 1 + 2^2 + 3^2 + \cdots + k^2 = \frac{k \cdot (k + 1) \cdot (2k + 1)}{6} $$ holds...
        \item \textbf{Inductive step:} Let's solve for $n = k + 1$,
        $$ 1 + 2^2 + 3^2 + \cdots + (k + 1)^2 = \frac{(k + 1) \cdot ((k + 1) + 1) \cdot (2 \cdot (k + 1) + 1)}{6} $$
    \end{enumerate}
\end{frame}

\begin{frame}{Another One}
    \begin{enumerate}
        \item[3] \textbf{Inductive step:} Let's solve for $n = k + 1$,
        { \scriptsize
            \uncover<+->{$$ 1 + 2^2 + 3^2 + \cdots + (k + 1)^2 \stackrel{?}{=} \frac{(k + 1) \cdot ((k + 1) + 1) \cdot (2 \cdot (k + 1) + 1)}{6} $$ }
            \uncover<+->{$$ 1 + 2^2 + 3^2 + \cdots + k^2 + (k + 1)^2 \stackrel{?}{=} \frac{(k + 1) \cdot (k + 2) \cdot (2k + 2 + 1)}{6} $$ }
            \uncover<+->{$$ \frac{k \cdot (k + 1) \cdot (2k + 1)}{6} + (k + 1)^2 \stackrel{?}{=} \frac{(k + 1) \cdot (k + 2) \cdot (2k + 3)}{6} $$ }
            \uncover<+->{$$ \frac{k \cdot (k + 1) \cdot (2k + 1) + 6(k + 1)^2}{6} \stackrel{?}{=} \frac{(k + 1) \cdot (k + 2) \cdot (2k + 3)}{6} $$ }
            \uncover<+->{$$ (k + 1) \cdot (k \cdot (2k + 1) + 6(k + 1)) \stackrel{?}{=} (k + 1) \cdot (k + 2) \cdot (2k + 3) $$ }
            \uncover<+->{$$ (k + 1) \cdot (2k^2 + 7k + 6) \stackrel{?}{=} (k + 1) \cdot (k + 2) \cdot (2k + 3) $$ }
            \uncover<+->{$$ (k + 1) \cdot (k + 2) \cdot (2k + 3) = (k + 1) \cdot (k + 2) \cdot (2k + 3) $$ \qed }
        }
    \end{enumerate}
\end{frame}

\begin{frame}{Theorems}
    \begin{theorem}\label{theo:one}
        Let $b$ be a positive real number and $x$ and $y$ real numbers. Then,
        \begin{enumerate}
         \item $b^{x+y} = b^x \cdot b^y$, and
         \item $(b^x)^y = b^{x \cdot y}$.
        \end{enumerate}
    \end{theorem}

\end{frame}

\begin{frame}{Theorems}
    \begin{theorem}\label{theo:two}
        Let $b$ be a real number greater than 1. Then,
        \begin{enumerate}
         \item $\log_b (xy) = \log_b x + \log_b y$ whenever $x$ and $y$ are positive real numbers, and
         \item $\log_b (x^y) = y \log_b x$ whenever $x$ is a positive real number and $y$ is a real number.
        \end{enumerate}
    \end{theorem}
\end{frame}

\begin{frame}{Theorems}
    \begin{theorem}\label{theo:three}
        Let $a$ and $b$ be real numbers greater than 1, and let $x$ be a positive real number. Then,
        \begin{enumerate}
         \item $\log_a x = \frac{\log_b x}{\log_b a}$.
        \end{enumerate}
    \end{theorem}
\end{frame}

\section*{Substitution Method}

\begin{frame}{Substitution Method}{Example: Solving $T(n) = 2T(\frac{n}{2}) + O(n)$}
    The \textbf{substitution method} is used to solve recurrence relations by guessing a solution and then proving it using mathematical induction.

    Let's solve the recurrence:
    \begin{equation*}
        T(n) = 2T \left(\frac{n}{2}\right) + O(n)
    \end{equation*}
\end{frame}

\begin{frame}{Substitution Method}{Step 1: Guess the solution}
    We assume $T(n) = O(n \log n)$. We will prove this by induction.
\end{frame}

\begin{frame}{Substitution Method}{Step 2: Expand the recurrence}
    Expand the recurrence for a few levels:
    \begin{align*}
    T(n) &= 2T\left(\frac{n}{2}\right) + cn \\
    T\left(\frac{n}{2}\right) &= 2T\left(\frac{n}{4}\right) + c\left(\frac{n}{2}\right)
    \end{align*}

    Substituting $T\left(\frac{n}{2}\right)$ into $T(n)$:
    \begin{align*}
    T(n) &= 2\left[ 2T\left(\frac{n}{4}\right) + c\left(\frac{n}{2}\right) \right] + cn \\
         &= 4T\left(\frac{n}{4}\right) + 2cn + cn \\
         &= 4T\left(\frac{n}{4}\right) + 3cn
    \end{align*}

    Continuing this process for $k$ steps:
    \begin{equation*}
    T(n) = 2^k T\left(\frac{n}{2^k}\right) + kcn
    \end{equation*}
\end{frame}

\begin{frame}{Substitution Method}{Step 3: Base Case}
    The recursion stops when $\frac{n}{2^k} = 1$, so $k = \lg n$. At this point, $T(1) = O(1)$, so:
    \begin{align*}
    T(n) &= 2^{\lg n} T(1) + (\lg n)cn
    \end{align*}

    Since $2^{\lg n} = n$, we get:
    \begin{align*}
    T(n) &= n O(1) + cn \lg n \\
        &= O(n \lg n)
    \end{align*}
\end{frame}

\begin{frame}{Substitution Method}{Step 4: Inductive Proof}
    \footnotesize
    We assume $T(n) \leq dn \lg n$ holds for all smaller values and prove it for $n$:
    \begin{align*}
    T(n) &= 2T\left(\frac{n}{2}\right) + cn
    \end{align*}

    Using the inductive hypothesis:
    \begin{align*}
    T(n) &\leq 2 \left[ d \left(\frac{n}{2}\right) \lg \left(\frac{n}{2}\right) \right] + cn \\
        &= d n \lg \left( \frac{n}{2} \right) + cn \\
        &= d n (\lg n - 1) + cn \\
        &= d n \lg n - d n + cn
    \end{align*}

    For a sufficiently large $d$, we can choose $d \geq c$, so:
    \begin{equation*}
    T(n) \leq d n \lg n
    \end{equation*}

    Thus, $T(n) = O(n \log n)$, which matches our guess.
\end{frame}

\begin{frame}{Substitution Method}{Conclusion}
    By using substitution and induction, we confirmed that:
    \begin{equation*}
    T(n) = O(n \lg n)
    \end{equation*}
\end{frame}

\begin{frame}{Another Example}
    An example of a recurrence relation that \textbf{cannot} be solved using the \textbf{Master Theorem} but can be solved using the \textbf{substitution method} is:

    $$
        T(n) = T(n-1) + O(n)
    $$
\end{frame}

\begin{frame}{Step 1: Expand the Recurrence}
    Expanding the recurrence iteratively:

    \begin{equation*}
        \begin{align*}
            T(n) &= T(n-1) + O(n) \\
                 &= (T(n-2) + O(n-1)) + O(n) \\
                 &= T(n-2) + O(n-1) + O(n) \\
                 &= T(n-3) + O(n-2) + O(n-1) + O(n) \\
        \end{align*}
    \end{equation*}

    Repeating this expansion until we reach the base case $T(1)$, we get:
    $$
        T(n) = T(1) + O(2) + O(3) + \dots + O(n)
    $$
\end{frame}

\begin{frame}{Step 2: Approximate the Summation}
    The summation of the first $n$ natural numbers is:
    $$
        \sum_{k=1}^{n} O(k) = O(1 + 2 + 3 + \dots + n) = O\left(\frac{n(n + 1)}{2}\right) = O(n^2)
    $$
    Thus, the recurrence simplifies to:
    $$
        T(n) = O(n^2)
    $$
\end{frame}

\begin{frame}{Why Can't Master Theorem Be Used?}
    The \textbf{Master Theorem} applies to recurrences of the form:
    $$
        T(n) = aT\left(\frac{n}{b}\right) + O(n^d)
    $$
    where $a$, $b$, and $d$ are constants. However, our recurrence $T(n) = T(n-1) + O(n)$ does \textbf{not} fit this form because:

    \begin{itemize}
        \item There is \textbf{no division} of $n$ (i.e., no factor like $\frac{n}{b}$).
        \item The recurrence is \textbf{not a divide-and-conquer} structure.
    \end{itemize}
    Thus, the \textbf{Master Theorem does not apply}, and we must use the \textbf{substitution method} or an iterative expansion approach.
\end{frame}

\begin{frame}{Final Answer}
    $$
        T(n) = O(n^2)
    $$

    This example demonstrates a \textbf{linear recurrence} that grows quadratically, and it is a case where the \textbf{Master Theorem fails}, but substitution (expanding and summing the terms) works effectively.
\end{frame}

\begin{frame}{Why $T(n) = T(n-1) + O(n)$ is Not Divide \& Conquer}
    \textbf{Key Differences:}
    \begin{enumerate}
        \item \textbf{No Division of the Problem Size}
        \begin{itemize}
            \item In divide \& conquer, we break the problem into smaller parts of size $\frac{n}{b}$, where $b$ is usually a constant.
            \item Here, we only reduce $n$ by \textbf{one} in each step ($n - 1$ instead of $\frac{n}{b}$). This means we are reducing the problem by a fixed amount rather than dividing it into subproblems of proportional size.
        \end{itemize}
    \end{enumerate}
\end{frame}

\begin{frame}{Why $T(n) = T(n-1) + O(n)$ is Not Divide\& Conquer}
    \textbf{Key Differences:}
    \begin{enumerate}
        \item[2] \textbf{Only One Subproblem ($a = 1$)}
        \begin{itemize}
            \item In divide \& conquer, there are typically multiple recursive calls (e.g., \textbf{Merge Sort} has two recursive calls, so $a = 2$).
            \item Here, there is \textbf{only one recursive call} to $T(n - 1)$, so it follows a linear recurrence pattern rather than a branching recursive structure.
        \end{itemize}
    \end{enumerate}
\end{frame}

\begin{frame}{Why $T(n) = T(n - 1) + O(n)$ is Not Divide \& Conquer}
    \textbf{Key Differences:}
    \begin{enumerate}
        \item[3] \textbf{Linear Reduction Instead of Exponential}
        \begin{itemize}
            \item In divide \& conquer, the problem size shrinks \textbf{exponentially} (e.g., $\frac{n}{2}$, $\frac{n}{4}$, etc.), leading to logarithmic depth recursion trees.
            \item Here, the problem size decreases \textbf{linearly} ($n - 1, n - 2, n - 3, \dots$), leading to a \textbf{deep recursion tree of depth $O(n)$}.
        \end{itemize}
    \end{enumerate}
\end{frame}

\begin{frame}{Conclusion}
    Since this recurrence follows a \textbf{linear} reduction pattern instead of an \textbf{exponential} divide \& conquer structure, it \textbf{does not} fit into the Master Theorem framework, which applies to problems that \textbf{divide} into multiple subproblems. Instead, it is better solved using expansion (substitution method) or summation techniques.
\end{frame}

\begin{frame}{ }
    \centering
    \includegraphics[width=0.35\textwidth]{figures/red_wedding.jpeg} \\
    ADA' students reading the questions of the exam... \href{https://youtu.be/TZZpdHMo4UQ?si=RGrIhu2Hzv55pb5d}{\textcolor{blue}{\textbf{click me!}}}
\end{frame}


\end{document}
