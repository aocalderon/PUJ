\documentclass{beamer}
\usepackage{graphicx}
\usepackage{booktabs}
\usepackage{tcolorbox}
\usepackage{amsmath, esint}
\usepackage{tikz}
\usetikzlibrary{backgrounds, positioning, arrows, scopes, shapes, shapes.misc, shapes.multipart}
\tikzset{
    cross/.style={cross out, draw=black, minimum size=2*(#1-\pgflinewidth), inner sep=0pt, outer sep=0pt},
    cross/.default={10pt},
    split/.style={rectangle split, rectangle split parts=7, draw, inner sep=0ex, rectangle split horizontal, minimum size=4ex},
    textstyle/.style={text height=1.5ex, text depth=.25ex}
}

\usepackage{amsthm}
\usepackage{amstext}
\usepackage{amssymb}
\usepackage[plain]{algorithm}
\usepackage{algpseudocode}

\usepackage{xcolor} 
\definecolor{LightGray}{gray}{0.975}

\renewcommand{\qed}{\\ \hfill $\blacksquare$}

%\usetheme{Warsaw}
\usefonttheme{serif} 

\title[Induction]{Introduction to Algorithms \\ Bonus Lecture: Proof by Induction}
\author{et al.}
\date{\today}

\setbeamertemplate{navigation symbols}{}%remove navigation symbols

\defbeamertemplate*{footline}{shadow theme}{%
    \leavevmode%
    \hbox{
        \begin{beamercolorbox}[wd=.5\paperwidth,ht=2.5ex,dp=1.125ex,leftskip=.3cm plus1fil,rightskip=.3cm]{author in head/foot}%
            \usebeamerfont{author in head/foot} Introduction to Algorithms: \hfill \insertshorttitle
        \end{beamercolorbox}%
        \begin{beamercolorbox}[wd=.5\paperwidth,ht=2.5ex,dp=1.125ex,leftskip=.3cm,rightskip=.3cm plus1fil]{title in head/foot}%
            \usebeamerfont{title in head/foot} \hfill \insertframenumber\,/\,\inserttotalframenumber%
        \end{beamercolorbox}
    }%
    \vskip0pt%
}

\AtBeginSection[]{
    \begin{frame}<beamer>
    \frametitle{Plan}
    \tableofcontents[currentsection]
    \end{frame}
}

\begin{document}

\frame{\titlepage}

\begin{frame}{Proof by Induction}
    \begin{itemize}
        \item A powerful mathematical technique.
        \item Prove that a statement is true for all natural numbers (or some sequence of numbers). \item It's like knocking over a line of dominoes...
    \end{itemize}
\end{frame}

\begin{frame}{How Induction Works}
    \textbf{Principle of Mathematical Induction:}
    \begin{itemize}
        \item \textbf{Base Case:} Show the statement holds for the first value (usually $n = 1$).
        \item \textbf{Inductive Hypothesis:} Assume the statement holds for some arbitrary $n = k$.
        \item \textbf{Inductive Step:} Prove it holds for $n = k+1$.
    \end{itemize}
    \pause
    \textbf{If those steps hold, the statement is true for all $n$.}
\end{frame}

\begin{frame}{Example}
    \begin{exampleblock}{ }
        Prove that:
        $$
            1 + 2 + 3 + \cdots + n = \frac{n(n + 1)}{2}
        $$
        for all $n \geq 1$.
    \end{exampleblock}
\end{frame}

\begin{frame}{Example}
    \begin{exampleblock}{Step 1: Base Case}
        For $n = 1$:
        $$
            1 = \frac{1(1 + 1)}{2} = \frac{2}{2} = 1
        $$ \checkmark
        \textcolor{olive}{\textbf{True!}}
    \end{exampleblock}
\end{frame}

\begin{frame}{Example}
    \begin{exampleblock}{Step 2: Inductive Hypothesis}
        Assume that for some $n = k$, the formula holds:
        $$
            1 + 2 + 3 + \cdots + k = \frac{k(k + 1)}{2}
        $$
        (This is our assumption or ``\textit{inductive hypothesis}''.)
    \end{exampleblock}
\end{frame}

\begin{frame}{Example}
    \begin{exampleblock}{Step 3: Inductive Step}
        We must prove it holds for $n = k + 1$, meaning:
        $$
            1 + 2 + 3 + \cdots + k + (k + 1) = \frac{(k + 1)(k + 2)}{2}
        $$ \pause
        Using the inductive hypothesis:
        $$
            \left( \frac{k (k + 1)}{2} \right) + (k + 1)
        $$ \pause
        Factor $k + 1$ out:
        $$
            \frac{k (k + 1) + 2(k + 1)}{2} = \frac{(k + 1)(k + 2)}{2}
        $$ \pause
    \end{exampleblock}
    This matches the formula for $n = k + 1$, so the statement holds!
    \qed
\end{frame}

\begin{frame}{Conclusion}
    By induction, the formula is true for all natural numbers n.\\
    \vspace{5mm}

    \textsc{Why Does This Work?}\\
    \vspace{2mm}
    Think of induction like climbing an infinite ladder:

    \begin{itemize}
        \item The \textbf{base case} puts your foot on the first rung.
        \item The \textbf{inductive hypothesis} and the \textbf{inductive step} shows that if you can reach one step, you can reach the next.
    \end{itemize}

    Since they are true, you can climb forever!
    \qed \footnote{$\blacksquare =$ Q.E.D. which means ``\textit{quod erat demonstrandum}''.}
\end{frame}


\begin{frame}{Example}
    \begin{exampleblock}{ }
        Proof by Induction:
        $$
            2^n \geq n + 1
        $$
        for all $n \geq 1$.
    \end{exampleblock}
\end{frame}

\begin{frame}{Step 1: Base Case}
    For $n = 1$:
    \begin{align*}
        2^1 &= 2,   \text{\hspace{7mm} left side} \\
        1 + 1 &= 2. \text{\hspace{8mm} right side}
    \end{align*}
    Since $2 \geq 2$, the base case holds. \checkmark
\end{frame}

\begin{frame}{Step 2: Inductive Hypothesis}
    Assume the statement is true for $n = k$:
    \begin{align*}
        2^k \geq k + 1.
    \end{align*}
    This assumption is the \textit{inductive hypothesis}.
\end{frame}

% Inductive Step
\begin{frame}{Step 3: Inductive Step}
    We need to prove the statement holds for $n = k+1$:
    \begin{align*}
        2^{k+1} &\geq (k+1) + 1.
    \end{align*}
    \pause
    Start with the left-hand side:
    \begin{align*}
        2^{k+1} &= 2 \cdot 2^k.
    \end{align*}
    \pause
    Using the inductive hypothesis $2^k \geq k+1$:
    \begin{align*}
        2^{k+1} &\geq 2 \cdot (k+1) = 2k + 2.
    \end{align*}
    \pause
    Since $2k + 2 \geq k + 2$, the statement holds for $n = k+1$.
    \qed
\end{frame}

% Conclusion
\begin{frame}{Conclusion}
    By mathematical induction, we have proven that:
    \begin{align*}
        2^n \geq n + 1 \quad \text{for all } n \geq 1.
    \end{align*}
    \textbf{Induction helps us prove statements for infinitely many cases!}
\end{frame}

\begin{frame}{Another Example}
    We will prove that the sum of the first $n$ odd numbers is given by:
    \begin{align*}
        1 + 3 + 5 + \dots + (2n - 1) = n^2.
    \end{align*}
\end{frame}

\begin{frame}{Step 1: Base Case}
    For $n = 1$:
    \begin{align*}
        1 = 1^2.
    \end{align*}
    Since both sides are equal, the base case holds. \checkmark
\end{frame}

\begin{frame}{Step 2: Inductive Hypothesis}
  Assume the statement is true for $n = k$:
  \begin{align*}
    1 + 3 + 5 + \dots + (2k - 1) = k^2.
  \end{align*}
  The \textit{inductive hypothesis}.
\end{frame}

\begin{frame}{Step 3: Inductive Step}
    We need to prove the statement holds for $n = k+1$:
    \begin{align*}
        1 + 3 + 5 + \dots + (2k - 1) + (2(k+1) - 1) = (k+1)^2.
    \end{align*}
    \pause
    Using the \textit{inductive hypothesis}:
    \begin{align*}
        k^2 + (2(k+1) - 1).
    \end{align*}
    \pause
    Expanding the term:
    \begin{align*}
        k^2 + (2k + 1) = (k+1)^2.
    \end{align*}
    \pause
    Since both sides match, the statement holds for $n = k+1$.
    \qed
\end{frame}

\begin{frame}{One More}
    Prove that:
    \begin{align*}
        \sum_{i = 0}^n 3^i &= \frac{3^{n + 1} - 1}{2} \\
    \end{align*}
    for all non negative integer $n$.
\end{frame}

\begin{frame}{Step 1: Base Case}
    For $n = 0$:
    \begin{align*}
        \sum_{i = 0}^0 3^i &= \frac{3^{0 + 1} - 1}{2} \\
                       3^0 &= \frac{3^1 - 1}{2} \\
                         1 &= \frac{3 - 1}{2} \\
                         1 &= \frac{2}{2} \\
                         1 &= 1 \\
    \end{align*}
    Since both sides are equal, the base case holds. \checkmark
\end{frame}

\begin{frame}{Step 1: Base Case}
    For $n = 1$:
    \begin{align*}
        \sum_{i = 0}^1 3^i &= \frac{3^{1 + 1} - 1}{2} \\
                 3^0 + 3^1 &= \frac{3^2 - 1}{2} \\
                     1 + 3 &= \frac{9 - 1}{2} \\
                         4 &= \frac{8}{2} \\
                         4 &= 4 \\
    \end{align*}
    Since both sides are equal, the base case holds. \checkmark
\end{frame}

\begin{frame}{Step 2: Inductive Hypothesis}
    Assume the statement is true for $n = k$:
    \begin{align*}
        \sum_{i = 0}^k 3^i &= \frac{3^{k + 1} - 1}{2} \\
    \end{align*}
    The \textit{inductive hypothesis}.
\end{frame}

\begin{frame}{Step 3: Inductive Step}
    \footnotesize
    We need to prove the statement holds for $n = k + 1$:
    \begin{align*}
        \sum_{i = 0}^{k + 1} 3^i &= \frac{3^{(k + 1) + 1} - 1}{2} \\
    \end{align*}
    \pause
    By $\sum$ definition:
    \begin{align*}
        \sum_{i = 0}^k 3^i + 3^{k + 1} &= \frac{3^{(k + 1) + 1} - 1}{2} \\
    \end{align*}
    \pause
    Using the \textit{inductive hypothesis}:
    \begin{align*}
        \frac{3^{k + 1} - 1}{2} + 3^{k + 1} &= \frac{3^{(k + 1) + 1} - 1}{2} \\
    \end{align*}
    \pause
    Expanding the term:
    \begin{align*}
        \frac{3^{k + 1}}{2} - \frac{1}{2} + \frac{2 \cdot 3^{k + 1}}{2} &= \frac{3^{(k + 1) + 1} - 1}{2} \\
        \frac{3^{k + 1} - 1 + 2 \cdot 3^{k + 1}}{2} &= \frac{3^{(k + 1) + 1} - 1}{2} \\
    \end{align*}
\end{frame}

\begin{frame}{Step 3: Inductive Step}
    \footnotesize
    Using the \textit{inductive hypothesis}:
    \begin{align*}
        \frac{3^{k + 1} - 1}{2} + 3^{k + 1} &= \frac{3^{(k + 1) + 1} - 1}{2} \\
    \end{align*}
    Expanding the term:
    \begin{align*}
        \frac{3^{k + 1}}{2} - \frac{1}{2} + \frac{2 \cdot 3^{k + 1}}{2} &= \frac{3^{(k + 1) + 1} - 1}{2} \\
        \frac{3^{k + 1} - 1 + 2 \cdot 3^{k + 1}}{2} &= \frac{3^{(k + 1) + 1} - 1}{2} \\
        \frac{3 \cdot 3^{k + 1} - 1}{2} &= \frac{3^{k + 1 + 1} - 1}{2} \\
        \frac{3^{k + 1 + 1} - 1}{2} &= \frac{3^{k + 1 + 1} - 1}{2} \\
        \frac{3^{k + 2} - 1}{2} &= \frac{3^{k + 2} - 1}{2} \\
    \end{align*}
    \pause
    Since both sides match, the statement holds for $n = k + 1$.
    \qed
\end{frame}

\end{document}
