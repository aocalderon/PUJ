\documentclass{beamer}

\usepackage{hyperref}
\usepackage{qrcode}

\title{Análisis de Algoritmos}
\author{Andrés Oswaldo Calderón Romero, Ph.D.}
\date{\today}

\begin{document}

\frame{\titlepage}

\section{Contenido}

\begin{frame}{Información}
\begin{itemize}
    \item Profesor: Andrés Oswaldo Calderón Romero, Ph.D.
    \item \textbf{Correo}: \href{mailto:acald013@ucr.edu}{acald013@ucr.edu} (Iniciar el asunto con ``[ADA]'').
    \item \textbf{Web}: \href{https://www.cs.ucr.edu/~acald013/}{https://www.cs.ucr.edu/$\sim$acald013/}.
    \item Páginas importantes: TBA.
    \item Plataformas: 
    \begin{itemize}
        \item Teams (Anuncios y reuniones).
        \item BrightSpace (Entregas: quizzes, talleres, proyecto).
    \end{itemize}
    \item Horario de atención: 
    \begin{itemize}
        \item Jueves: 02:00pm a 04:00pm (Por orden de llegada).
    \end{itemize}
\end{itemize}
\end{frame}

\section{Los Estudiantes}

\begin{frame}{Los Estudiantes}
\begin{itemize}
    \item ¿Nombre? 
    \item ¿Experiencia en programación? ¿Lenguajes? 
    \item ¿Experiencia en bases de datos? ¿DBMSs?
    \item ¿Algo puntual que espera aprender en este curso?
\end{itemize}
\end{frame}

\section{Sobre el Curso}

\begin{frame}{Sobre el Curso}
\begin{itemize}
    \item ¿Qué sabemos?
    \item ¿Qué nos han dicho?
    \item ¿Qué hemos oído?
\end{itemize}

    \centering
    \vspace{10mm}
    {\small \href{https://www.mentimeter.com/app/presentation/alwg84st66fh5c49yxg9r7k3i5jqhuwn/edit?question=uah1wimn5n7w}{Veámos:}}\\
    \vspace{2mm}
  
    \qrcode[hyperlink,height=0.75in]{https://www.menti.com/al5p1ccqsfpv}\\
    \vspace{2mm}
    \footnotesize  o ingrese a \href{https://www.menti.com/}{menti.com}\\ con el código \textbf{3228 4239}
\end{frame}

\section{Objetivos}

\begin{frame}{Objetivos}
\begin{itemize}
    \item Presentar el lenguaje formal de diseño de problemas algorítmicos.
    \item Mostrar los diferentes tipos y clases de problemas algorítmicos.
    \item Exponer a los estudiantes a situaciones de formalización de problemas mal condicionados.
    \item Presentar las principales estrategias de solución de problemas polinomiales (P).
    \item Solucionar algunos problemas no-determinísticos polinomiales (NP-completos) con algoritmos de aproximación bien conocidos en la literatura.
    \item Presentar estrategias para verificar y probar implementaciones.
\end{itemize}
\end{frame}

\section{Reglas de Juego}

\begin{frame}{Reglas de Juego}
\begin{itemize}
    \item Trabajo Individual: Quizzes y Parciales.
    \item Trabajo en Equipo: Talleres y Proyecto.
    \item Grupos de 3-4 personas, definidos en la primera semana.
    \item Entregas fuera del tiempo establecido: Nota 0.0.
\end{itemize}
\end{frame}

\section{Monitor}

\begin{frame}{Monitor}
    \begin{itemize}
        \item Monitor del curso: TBA
        \item Apoyo con preguntas puntuales y tips sobre programación o SQL.
    \end{itemize}
\end{frame}

\end{document}
