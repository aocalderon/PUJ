\documentclass{article}
\usepackage[utf8]{inputenc}
\usepackage{wasysym}
\usepackage{qrcode}
\usepackage[colorlinks]{hyperref}
\usepackage{lmodern}
\usepackage{graphicx}
\usepackage{xcolor}
\usepackage[left=2cm, top=3cm, right=2cm]{geometry}
\usepackage{booktabs}
\usepackage{svg}
\usepackage{minted}
\usepackage{xcolor}
\definecolor{LightGray}{gray}{0.975}

%setup new colors
\hypersetup{
%linkcolor=blue
%,citecolor=
%,filecolor=
urlcolor=blue
%,menucolor=
%,runcolor=
%,linkbordercolor=
%,citebordercolor=
%,filebordercolor=
%,urlbordercolor=
%,menubordercolor=
%,runbordercolor=
}

\title{Database Administration \\ Lab 07: Backup \& Restoring.}
\author{Andrés Calderón, Ph.D.}
\date{\today}

\begin{document}

\maketitle

\section{Introduction}

In today's data-driven world, ensuring the availability and integrity of data is a critical aspect of database administration. Data loss can occur due to various factors, including accidental deletions, system failures, or catastrophic events. To mitigate these risks, organizations implement Backup \& Restore (B\&R) strategies to recover lost data and maintain operational continuity.

This lab focuses on applying real-time database management concepts in a simulated environment where students will:
\begin{itemize}
    \item Utilize real-time data from the SUMO (Simulation of Urban MObility) traffic simulator.
    \item Set up a database to store and manage location data.
    \item Simulate a catastrophic event leading to data loss.
    \item Implement a Backup \& Restore tool to recover lost data and assess its effectiveness.
\end{itemize}

Through this hands-on experience, students will gain a deeper understanding of data resilience techniques, best practices for database recovery, and the importance of periodic backups in real-world applications.

By the end of this lab, students will deliver a presentation on their approach, findings, and insights, with peer evaluation as part of the assessment process.  This lab serves as a practical exercise in database administration, reinforcing theoretical knowledge with real-world implementation.

\section{Getting Real Data}
So, to obtain real-time data, we will use \href{https://eclipse.dev/sumo/}{SUMO}—Simulation of Urban MObility. It is an Eclipse Open Source project that provides a highly portable, microscopic, and continuous multi-modal traffic simulator.

In particular, we will use a scenario created a couple of years ago by DLR-VF\footnote{The German Aerospace Center - Institute of Transport Research} using TAPAS (Travel and Activity PAtterns Simulation) and SUMO for the city of Cologne in Germany. We will refer to this scenario as the \href{https://sumo.dlr.de/docs/Data/Scenarios/TAPASCologne.html}{TAPAS Cologne Scenario} throughout this document. You can download the scenario from this \href{https://sourceforge.net/projects/sumo/files/traffic_data/scenarios/TAPASCologne/}{link} [\texttt{TAPASCologne-0.32.0.7z, 2018-04-10, 52.3 MB}].

To understand the details of the scenario and how we will use it in this project, you should watch this \href{https://drive.google.com/file/d/1Deg6SjXcAUJahbNQ_YmvbhmbWuaVtD6y/view?usp=sharing}{video}.

\section{Setting Up a Database for Location Data}
As mentioned in the video, using the data generated by the TAPAS Cologne Scenario and SUMO, we will populate a database in the DBMS of your choice. You can accomplish this using a script written in your preferred programming language, Pentaho Data Integration, or any other tool suitable for handling the XML file that SUMO updates in real time.

\section{Simulating a Catastrophic Data Event to Use a B\&R Tool}
The goal of this project is to test the use of a Backup \& Restore (B\&R) tool to recover data and mitigate the damage caused by a hypothetical data loss. First, you need to choose a tool (either Open Source or Commercial) from this \href{https://drive.google.com/file/d/1089znTsPLuTGaxKYkdJkx7ZIOl4tV1QX/view?usp=sharing}{list}. During the execution of the scenario and the database population process, you will simulate a catastrophic event—whatever form it may take—resulting in data loss.

Specifically, you will inject a `\texttt{DELETE}' statement into the database to remove a section of the data. For example, around the 8000th second, you could execute an \texttt{SQL} sentence that removes all tuples recorded between seconds 4000 and 6000. It is expected that you configure the Backup \& Restore tool to operate periodically within the same time window in which the scenario is running.

Finally, you will investigate and demonstrate how your chosen B\&R tool can recover the lost data (or part of it) and mitigate any potential damage.

\section{What We Expect}
You will choose a DBMS and a B\&R tool and prepare a presentation to share with the class how you handled real-time data ingestion, simulated data loss, and properly utilized the B\&R tool of your choice.

We expect you to present your work in class on \textbf{April 7, 2025}. The presentation will be co-evaluated by your classmates.

\vspace{5mm}
Happy Hacking \includesvg[width=4mm]{figures/sunglasses}!

\end{document}

