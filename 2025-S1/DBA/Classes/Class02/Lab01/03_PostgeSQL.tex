\documentclass{article}
\usepackage[T1]{fontenc}
\usepackage{lmodern}
\usepackage{graphicx}
\usepackage{xcolor}
\usepackage{listings}
\usepackage{hyperref}

\lstset{
    basicstyle=\ttfamily\footnotesize,
    breaklines=true,
    frame=single,
    backgroundcolor=\color{lightgray},
    keywordstyle=\bfseries\color{blue},
}

\title{Gu\'ia para Compilar PostgreSQL en Ubuntu Server 24.04}
\author{}
\date{}

\begin{document}

\maketitle

\section{Actualizar el Sistema}
Antes de empezar, es recomendable actualizar el sistema para asegurarnos de tener las últimas versiones de los paquetes.

\begin{lstlisting}[language=bash]
sudo apt update && sudo apt upgrade -y
\end{lstlisting}

\section{Instalar Dependencias Necesarias}
PostgreSQL requiere varias bibliotecas y herramientas para compilarse correctamente.

\begin{lstlisting}[language=bash]
sudo apt install -y build-essential libreadline-dev zlib1g-dev \
                    flex bison libxml2-dev libxslt1-dev \
                    libssl-dev libperl-dev libpam0g-dev \
                    libldap2-dev python3-dev uuid-dev
\end{lstlisting}

\section{Descargar la Última Versión de PostgreSQL}
Obtenemos la versión más reciente desde el sitio oficial de PostgreSQL.

\begin{lstlisting}[language=bash]
cd /usr/local/src
wget https://ftp.postgresql.org/pub/source/v16.1/postgresql-16.1.tar.gz
\end{lstlisting}
(Sustituye `16.1` por la última versión disponible si hay una más reciente).

\section{Extraer el Código Fuente}
Descomprimimos el archivo descargado.

\begin{lstlisting}[language=bash]
tar -xvzf postgresql-16.1.tar.gz
cd postgresql-16.1
\end{lstlisting}

\section{Configurar la Compilación}
Ejecutamos el script de configuración con las opciones necesarias.

\begin{lstlisting}[language=bash]
./configure --prefix=/usr/local/pgsql --with-openssl
\end{lstlisting}
Opcionalmente, puedes añadir otras opciones como `--with-python`, `--with-libxml`, etc.

\section{Compilar PostgreSQL}
Ahora compilamos PostgreSQL utilizando `make`.

\begin{lstlisting}[language=bash]
make -j$(nproc)
\end{lstlisting}

\section{Instalar PostgreSQL}
Si la compilación fue exitosa, instalamos PostgreSQL en el sistema.

\begin{lstlisting}[language=bash]
sudo make install
\end{lstlisting}

\section{Crear el Usuario y Configurar Directorios}
Creamos un usuario específico para PostgreSQL y configuramos los directorios.

\begin{lstlisting}[language=bash]
sudo useradd -m -d /usr/local/pgsql -s /bin/bash postgres
sudo mkdir -p /usr/local/pgsql/data
sudo chown postgres:postgres /usr/local/pgsql/data
\end{lstlisting}

\section{Inicializar la Base de Datos}
Iniciamos PostgreSQL con el usuario `postgres`.

\begin{lstlisting}[language=bash]
sudo -i -u postgres
/usr/local/pgsql/bin/initdb -D /usr/local/pgsql/data
exit
\end{lstlisting}

\section{Configurar el Servicio de PostgreSQL}
Creamos un archivo de servicio para `systemd`.

\begin{lstlisting}[language=bash]
sudo nano /etc/systemd/system/postgresql.service
\end{lstlisting}

Pegamos el siguiente contenido:

\begin{lstlisting}[language=bash]
[Unit]
Description=PostgreSQL database server
After=network.target

[Service]
User=postgres
Group=postgres
ExecStart=/usr/local/pgsql/bin/postgres -D /usr/local/pgsql/data
ExecReload=/bin/kill -HUP $MAINPID
KillMode=process
KillSignal=SIGINT
Type=forking

[Install]
WantedBy=multi-user.target
\end{lstlisting}

Guardamos y salimos (`Ctrl + X`, luego `Y` y `Enter`).

\section{Iniciar y Habilitar PostgreSQL}
Recargamos `systemd` y habilitamos PostgreSQL para que se inicie automáticamente.

\begin{lstlisting}[language=bash]
sudo systemctl daemon-reload
sudo systemctl enable postgresql
sudo systemctl start postgresql
\end{lstlisting}

\section{Verificar el Estado del Servidor}
Para confirmar que PostgreSQL está corriendo correctamente, usamos:

\begin{lstlisting}[language=bash]
sudo systemctl status postgresql
\end{lstlisting}

Si PostgreSQL está funcionando bien, deberíamos ver un mensaje indicando que está activo.

\section{Acceder a PostgreSQL}
Iniciamos sesión en PostgreSQL con el usuario `postgres`.

\begin{lstlisting}[language=bash]
sudo -i -u postgres
/usr/local/pgsql/bin/psql
\end{lstlisting}

Para salir de la sesión de `psql`, usamos:

\begin{lstlisting}[language=sql]
\q
\end{lstlisting}

Y para salir del usuario `postgres`:

\begin{lstlisting}[language=bash]
exit
\end{lstlisting}

\section{¡PostgreSQL está listo para usarse!}
Ahora puedes configurar usuarios, bases de datos y tunear PostgreSQL según tus necesidades.

\end{document}
