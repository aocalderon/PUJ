\documentclass{article}
\usepackage[utf8]{inputenc}
\usepackage{hyperref}
\usepackage{listings}
\usepackage{xcolor}

\definecolor{codebg}{rgb}{0.95, 0.95, 0.95}
\lstset{
  backgroundcolor=\color{codebg},
  basicstyle=\ttfamily\small,
  frame=single,
  breaklines=true,
  showstringspaces=false
}

\title{Guía de Instalación de Ubuntu Server 24.04}
\author{}
\date{}

\begin{document}

\maketitle

\section{Descarga de Ubuntu Server 24.04}
\begin{enumerate}
    \item \textbf{Visita la página oficial de Ubuntu Server}: \href{https://ubuntu.com/download/server}{https://ubuntu.com/download/server}
    \item Descarga la imagen ISO de Ubuntu Server 24.04 (64-bit).
    \item Verifica la integridad del archivo comparando el hash SHA256 (opcional).
\end{enumerate}

\section{Creación del medio de instalación}
\subsection{Opción 1: Crear un USB Booteable (recomendado)}
\begin{enumerate}
    \item Conectar una memoria USB (mínimo 4 GB).
    \item Usar una herramienta para crear el USB de arranque:
    \begin{itemize}
        \item \textbf{Windows}: \href{https://rufus.ie}{Rufus}
        \item \textbf{Linux}: \texttt{dd} o \href{https://www.balena.io/etcher/}{balenaEtcher}
        \item \textbf{macOS}: \texttt{dd} o balenaEtcher
    \end{itemize}
    \item Grabar la imagen ISO en la USB:
    \begin{lstlisting}[language=bash]
    sudo dd if=ubuntu-24.04-live-server-amd64.iso of=/dev/sdX bs=4M status=progress
    \end{lstlisting}
    Reemplaza \texttt{/dev/sdX} con tu unidad USB.
\end{enumerate}

\subsection{Opción 2: Grabar en un DVD (menos recomendado)}
\begin{enumerate}
    \item Usa \textbf{Brasero}, \textbf{ImgBurn} o cualquier grabador de discos.
    \item Quema la ISO en un DVD.
\end{enumerate}

\section{Configuración del BIOS/UEFI}
\begin{enumerate}
    \item Accede a la BIOS/UEFI (normalmente con \texttt{F2}, \texttt{F12}, \texttt{DEL} o \texttt{ESC} al encender).
    \item Habilita el modo UEFI (si es compatible).
    \item Deshabilita Secure Boot si tienes problemas al iniciar la instalación.
    \item Cambia el orden de arranque para que el USB/DVD sea la primera opción.
\end{enumerate}

\section{Instalación de Ubuntu Server 24.04}
\begin{enumerate}
    \item Inicia desde el USB/DVD y selecciona \textbf{Install Ubuntu Server}.
    \item Selecciona el idioma (Español o Inglés recomendado).
    \item Configura la distribución del teclado.
    \item Configura la red:
    \begin{itemize}
        \item Si tienes DHCP, la red se configura automáticamente.
        \item Para IP estática, selecciona "Configuración manual" e introduce los valores correctos.
    \end{itemize}
    \item Elige el tipo de instalación:
    \begin{itemize}
        \item \textbf{Ubuntu Server (sin interfaz gráfica)} es la opción estándar.
        \item Puedes elegir una instalación mínima si deseas un sistema más ligero.
    \end{itemize}
    \item Configura el almacenamiento:
    \begin{itemize}
        \item Opción recomendada: \textbf{Usar disco entero y configurar LVM} (permite ampliar particiones en el futuro).
        \item Si deseas RAID o particionado manual, elige la opción correspondiente.
    \end{itemize}
    \item Crea un usuario y contraseña.
    \item Opcional: Instala OpenSSH marcando la opción "Install OpenSSH server".
    \item Selecciona software adicional (Docker, Kubernetes, etc.).
    \item Confirma la instalación y espera a que finalice.
\end{enumerate}

\section{Primer arranque y configuración post-instalación}
\begin{enumerate}
    \item Reinicia el sistema y retira el medio de instalación.
    \item Inicia sesión con tu usuario y contraseña.
    \item Actualiza el sistema:
    \begin{lstlisting}[language=bash]
    sudo apt update && sudo apt upgrade -y
    \end{lstlisting}
    \item Configura la zona horaria (si no se configuró correctamente):
    \begin{lstlisting}[language=bash]
    sudo timedatectl set-timezone America/Bogota
    \end{lstlisting}
    \item Configura el firewall (opcional pero recomendado):
    \begin{lstlisting}[language=bash]
    sudo ufw allow OpenSSH
    sudo ufw enable
    \end{lstlisting}
    \item Verifica la dirección IP asignada:
    \begin{lstlisting}[language=bash]
    ip a
    \end{lstlisting}
\end{enumerate}

\section{Configuración adicional}
\begin{itemize}
    \item Crear un usuario adicional con permisos sudo:
    \begin{lstlisting}[language=bash]
    sudo adduser nuevo_usuario
    sudo usermod -aG sudo nuevo_usuario
    \end{lstlisting}
    \item Configurar acceso remoto con SSH:
    \begin{lstlisting}[language=bash]
    ssh usuario@ip_del_servidor
    \end{lstlisting}
    \item Instalar paquetes esenciales:
    \begin{lstlisting}[language=bash]
    sudo apt install htop net-tools curl git -y
    \end{lstlisting}
    \item Configurar un hostname personalizado:
    \begin{lstlisting}[language=bash]
    sudo hostnamectl set-hostname mi-servidor
    \end{lstlisting}
\end{itemize}

\section{Conclusión}
Con estos pasos, Ubuntu Server 24.04 estará listo para usarse. A partir de aquí, puedes instalar servicios como Apache, Nginx, PostgreSQL, Docker, Kubernetes, etc., según tus necesidades. \textbf{¡Éxito en tu instalación!}
\end{document}
