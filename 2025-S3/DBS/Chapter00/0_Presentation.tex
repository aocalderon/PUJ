\documentclass{beamer}

\usepackage{hyperref}
\usepackage{qrcode}

\title{Bases de Datos}
\author{Andrés Oswaldo Calderón Romero, Ph.D.}
\date{\today}

\begin{document}

\frame{\titlepage}

\section{Contenido}

\begin{frame}{Información}
\begin{itemize}
    \item Profesor: Andrés Oswaldo Calderón Romero, Ph.D.
    \item \textbf{Correo}: \href{mailto:acald013@ucr.edu}{acald013@ucr.edu} (Iniciar el asunto con ``[DBS]'')\\
    \item \textbf{Web}: \href{https://www.cs.ucr.edu/~acald013/}{https://www.cs.ucr.edu/$\sim$acald013/} \\
    \item Páginas importantes: \url{https://db-book.com/}
    \item Plataformas: 
    \begin{itemize}
        \item BrightSpace (Entregas, quices, talleres, etc.)
        \item GitHub [\url{https://github.com/aocalderon/PUJ/}] (Material de clase)
    \end{itemize}
    \item Herramientas: PgAdmin (PostgreSQL), Compass (MongoDB)
    \item Horario de atención: 
    \begin{itemize}
        \item Lunes: 9:00 a 10:00am (FIFO)
    \end{itemize}
\end{itemize}
\end{frame}

\section{Los Estudiantes}

\begin{frame}{Los Estudiantes}
\begin{itemize}
    \item ¿Nombre? 
    \item ¿Experiencia en programación? 
    \item ¿Lenguajes? 
    \item ¿Algo puntual que espera aprender en este curso?
\end{itemize}
\end{frame}

\section{Sobre el Curso}

\begin{frame}{Sobre el Curso}
\begin{itemize}
    \item ¿Qué sabemos?
    \item ¿Qué nos han dicho?
    \item ¿Qué hemos oído?
\end{itemize}

    \centering
    \vspace{10mm}
    {\small \href{https://www.mentimeter.com/app/presentation/aljzuorbyosmeh7v9vvh4ik9e9y3zdcn/edit?question=fvszuicwkyw2}{Veámos:}}\\
    \vspace{2mm}
  
    \qrcode[hyperlink,height=0.75in]{https://www.menti.com/alqgaa5yhyd4}\\
    \vspace{2mm}
    \footnotesize  o ingrese a \href{https://www.menti.com/}{menti.com}\\ con el código \textbf{5540 4513}
\end{frame}

\section{Objetivos}

\begin{frame}{Objetivos}
\begin{itemize}
    \item Brindar conceptos para diseño y manipulación de Bases de Datos relacionales.
    \item Explicar tipos de bases de datos y criterios para elegir según el problema.
    \item Mejorar el desempeño de Bases de Datos y discutir sus consecuencias.
\end{itemize}
\end{frame}

\section{Contenido}

\begin{frame}{Contenido}
\begin{itemize}
    \item Modelo relacional
    \item SQL
    \item Diseño de Bases de Datos relacionales
    \item Bases de Datos no relacionales
\end{itemize}
\end{frame}

\section{Evaluación y Fechas}

\begin{frame}{Evaluación y Fechas}
\textbf{Parciales (50\%)} 
\begin{itemize}
    \item 20\%: TBA
    \item 15\%: TBA
    \item 15\%: TBA
\end{itemize}
\textbf{Quizzes y Talleres (20\%)} \\
\textbf{Proyecto (30\%)} 
\begin{itemize}
    \item Primera Entrega: 10\%
    \item Segunda Entrega: 10\%
    \item Final y Sustentación: 10\%
\end{itemize}
\end{frame}

\section{Reglas de Juego}

\begin{frame}{Reglas de Juego}
\begin{itemize}
    \item Trabajo Individual: Tareas, Quizzes y Ejercicios.
    \item Trabajo en Equipo: Talleres y Proyecto.
    \item Grupos de 2 personas para los talleres y de 4 personas para el proyecto.
    \item Entregas fuera del tiempo establecido: Nota 0.0.
\end{itemize}
\end{frame}

\end{document}
