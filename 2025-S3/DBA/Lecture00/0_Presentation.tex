\documentclass{beamer}

\usepackage{hyperref}
\hypersetup{
    colorlinks=true,
    linkcolor=blue,
    filecolor=magenta,
    urlcolor=cyan
}
\usepackage{qrcode}

\title{Administración de Bases de Datos}
\author{Andrés Oswaldo Calderón Romero, Ph.D.}
\date{\today}

\begin{document}

\frame{\titlepage}

\section{Contenido}

\begin{frame}{Información}
\begin{itemize}
    \item Profesor: Andrés Oswaldo Calderón Romero, Ph.D.
    \item \textbf{Correo}: \href{mailto:andrescalderonr@javeriana.edu.co}{andrescalderonr@javeriana.edu.co} (Iniciar el asunto con ``[DBA]'').
    \item \textbf{Web}: \href{https://www.cs.ucr.edu/~acald013/}{https://www.cs.ucr.edu/$\sim$acald013/}.
    \item Páginas importantes: \url{https://github.com/aocalderon/PUJ/} $\longrightarrow$ \texttt{2025-S3}.
    \item Plataformas: 
    \begin{itemize}
        \item BrightSpace (Anuncios y entregas).
        \item GitHub (Material).
    \end{itemize}
    \item Herramientas: \href{https://www.pgadmin.org/}{PgAdmin} (PostgreSQL), Compass (MongoDB).
    \item Horario de atención (Office Hours):
    \begin{itemize}
        \item Lunes: 05:00 a 06:00pm (FIFO).
    \end{itemize}
\end{itemize}
\end{frame}

\section{Los Estudiantes}

\begin{frame}{Los Estudiantes}
\begin{itemize}
    \item ¿Nombre? 
    \item ¿Experiencia en programación? ¿Lenguajes? 
    \item ¿Experiencia en bases de datos? ¿DBMSs?
    \item ¿Algo puntual que espera aprender en este curso?
\end{itemize}
\end{frame}

\section{Sobre el Curso}

\begin{frame}{Sobre el Curso}
\begin{itemize}
    \item ¿Qué sabemos?
    \item ¿Qué nos han dicho?
    \item ¿Qué hemos oído?
\end{itemize}

    \centering
    \vspace{10mm}
    {\small \href{https://www.mentimeter.com/app/presentation/alfaisiivs1g37v36evz9fa49zpqobru/edit?question=wgh3ewvzk1pe}{Veámos:}}\\
    \vspace{2mm}
  
    \qrcode[hyperlink,height=0.75in]{https://www.menti.com/alqyayfy5j4g}\\
    \vspace{2mm}
    \footnotesize  o ingrese a \href{https://www.menti.com/}{menti.com}\\ con el código \textbf{1745 6240}
\end{frame}

\section{Objetivos}

\begin{frame}{Objetivos}
\begin{itemize}
    \item Consolidar el ciclo de vida de una base de datos, complementando el curso básico de bases de datos.
    \item Presentar los aspectos que implican la administración de una base de datos.
    \item Establecer las estructuras que ofrece SQL para el Análisis de Datos.
    \item Presentar estrategias de rendimiento para bases de datos relacionales
    \item Presentar estrategias avanzadas de persistencia mediante bases de datos no relacionales (NoSQL).
\end{itemize}
\end{frame}

\section{Contenido}

\begin{frame}{Contenido}
\begin{itemize}
    \item Administración de base de datos.
    \item Tópicos avanzados de bases de datos relacionales.
    \item SQL Para Análisis.
    \item Pruebas de carga y estrés para base de datos.
    \item Bases de datos No Relacionales (NOSQL).
\end{itemize}
\end{frame}

\section{Evaluación y Fechas}

\begin{frame}{Evaluación y Fechas}
    \begin{itemize}
        \item 15\% → Examen teórico-práctico 1.
        \item 15\% → Examen teórico-práctico 2.
        \item 30\% → Talleres, quizzes, tareas.
        \item 20\% → Proyecto Midterm.
        \item 20\% → Proyecto final.
    \end{itemize}
\end{frame}

\section{Reglas de Juego}

\begin{frame}{Reglas de Juego}
\begin{itemize}
    \item Trabajo Individual: Quizzes y Parciales.
    \item Trabajo en Equipo: Talleres y Proyecto.
    \item Grupos de 2 personas para talleres y 4 personas para el proyecto.
    \item No se permiten entregas fuera del tiempo establecido.
\end{itemize}
\end{frame}

\end{document}
