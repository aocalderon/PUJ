\documentclass{article}
\usepackage[utf8]{inputenc}
\usepackage{wasysym}
\usepackage{qrcode}
\usepackage[colorlinks]{hyperref}
\usepackage{lmodern}
\usepackage{graphicx}
\usepackage{xcolor}
\usepackage[left=2cm, top=3cm, right=2cm]{geometry}
\usepackage{booktabs}
\usepackage{svg}
\usepackage{minted}
\usepackage{xcolor}
\definecolor{LightGray}{gray}{0.975}

\title{Database Administration \\ Lab 10: NoSQL and MongoDB Primer.}
\author{Andrés Calderón, Ph.D.}
\date{\today}

\begin{document}

\maketitle

\section{Introduction}
In this lab, we will continue our exploration of NoSQL technologies by focusing on MongoDB, one of the most widely used NoSQL databases. Through selected readings and hands-on labs, you will gain a foundational understanding of NoSQL types, use cases, and essential MongoDB operations. Although no formal submission is required, developing familiarity with MongoDB is crucial for your upcoming project work. This lab is designed to help you grasp basic concepts and commands that will be vital for your success in group activities and real-world applications.

\section{Reading Material}
To complement what we have covered about NoSQL in class, we will explore additional readings from the Coursera MOOC \textit{``Introduction to NoSQL Databases''} by IBM, available \href{https://www.coursera.org/learn/introduction-to-nosql-databases}{here}. The materials are:

\begin{enumerate}
    \item NoSQL Types and Use Cases (\href{https://drive.google.com/file/d/1dgG_1JSRWCmeIBABfW83ykd6I1L0tCJJ/view?usp=sharing}{en}, \href{https://drive.google.com/file/d/1rqhSVxo7ci1MEieJIbGGbL3s-qxI1Ltr/view?usp=sharing}{sp}).
    \item NoSQL Database Deployment Operations (\href{https://drive.google.com/file/d/11wvBhvxNRG0HttqZ9p_SgdIaQMlDE9nt/view?usp=sharing}{en}, \href{https://drive.google.com/file/d/1gMjSwO69xOUUbY7pJ_Rs10uJ15oQYUUA/view?usp=sharing}{sp}).
    \item NoSQL Glossary (\href{https://drive.google.com/file/d/1XmvBUvBvYdzN6l4gS-Q6eavlJZm6m4DH/view?usp=sharing}{en}).
\end{enumerate}

\section{Introduction to MongoDB}
We will now focus on the second module of the course. For this activity, you must log in to Coursera and register for the course in order to gain access (\href{https://www.coursera.org/learn/introduction-to-nosql-databases/home/module/2}{link to Module 2}). In this module, we will concentrate on the Labs, particularly:

\begin{itemize}
    \item Lab: Getting Started with MongoDB (Basics of MongoDB section).
    \item Lab: MongoDB CRUD (Getting Started with MongoDB section).
    \item Lab: MongoDB Indexing (Getting Started with MongoDB section).
\end{itemize}

\section{What We Expect}
Actually, you do not have to submit any evidence of your work, but you will need to become familiar with MongoDB to work on the project. It is important for your group that you grasp the basic concepts of its use and fundamental commands.

\vspace{5mm}
Happy Hacking \includesvg[width=4mm]{figures/sunglasses}!

\end{document}

