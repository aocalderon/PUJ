\documentclass{article}
\usepackage[utf8]{inputenc}
\usepackage{qrcode}
\usepackage[colorlinks]{hyperref}
\usepackage{listings}
\usepackage{lmodern}
\usepackage{graphicx}
\usepackage{xcolor}
\usepackage[left=2cm, top=3cm, right=2cm]{geometry}

\definecolor{codebg}{rgb}{0.95, 0.95, 0.95}
\lstset{
  backgroundcolor=\color{codebg},
  basicstyle=\ttfamily\small,
  frame=single,
  breaklines=true,
  keywordstyle=\bfseries\color{blue},
  showstringspaces=false
}

%setup new colors
\hypersetup{
%linkcolor=blue
%,citecolor=
%,filecolor=
urlcolor=blue
%,menucolor=
%,runcolor=
%,linkbordercolor=
%,citebordercolor=
%,filebordercolor=
%,urlbordercolor=
%,menubordercolor=
%,runbordercolor=
}

\title{Bases de Datos \\ Laboratorio 1}
\author{Andrés Calderón, Ph.D.}
\date{\today}

\begin{document}

\maketitle

%\begin{center}
%    \qrcode[height=1in]{https://drive.google.com/file/d/1nVtNfBqhExLhbEMYybH8Mfrl3exG5J3M/view?usp=sharing}
%\end{center}

\section{Introducción}
La idea de este laboratorio en aprender la instalación y configuración de las herramientas básicas que usaremos a lo largo del curso para poner en práctica los conocimientos impartidos durante las sesiones teóricas.  Dadas las condiciones del laboratorio, las limitaciones de instalación de nuevo software y para no entorpecer los recursos ya instalados vamos a trabajar en una máquina virtual GNU/Linux.  El aplicativo que usaremos para la creación de la máquina virtual será \href{https://www.virtualbox.org/wiki/Downloads}{VirtualBox}.  Puedes descargar el instalador para el sistema operativo que utilizas en tu equipo de computo o usar la versión instalada en los equipos del laboratorio. 

El laboratorio consta de tres secciones: 
\begin{enumerate}
    \item Guía paso a paso para la creación de una máquina virtual en VirtualBox para la instalación de Ubuntu Server 24.04.
    \item Guia paso a paso para la instalación de Ubuntu Server 24.04 en la máquina virtual previamente creada.
    \item Guia paso a paso para la compilación de PostgreSQL 17.2 desde su código fuente.
\end{enumerate}

A continuación describimos el contenido de dichas secciones.

\section{Introducción a VirtualBox}
VirtualBox es un software de virtualización de código abierto que permite ejecutar múltiples sistemas operativos en un mismo equipo. Es compatible con Windows, macOS y Linux, y es una herramienta ideal para probar sistemas operativos, entornos de desarrollo y configuraciones sin afectar el sistema anfitrión.

\subsection{Controles Básicos en VirtualBox}

\subsubsection{Creación y gestión de máquinas virtuales}
\begin{itemize}
    \item \textbf{Nuevo}: Permite crear una nueva máquina virtual.
    \item \textbf{Configuración}: Modifica los parámetros de la máquina virtual antes de iniciarla.
    \item \textbf{Iniciar}: Enciende una máquina virtual seleccionada.
    \item \textbf{Apagar}: Detiene la ejecución de una máquina virtual.
    \item \textbf{Instantáneas}: Guarda el estado actual de la máquina virtual para restaurarlo en cualquier momento.
    \item \textbf{Exportar/Importar}: Permite compartir máquinas virtuales entre equipos.
\end{itemize}

\subsubsection{Controles dentro de una máquina virtual}
\begin{itemize}
    \item \textbf{Modo Pantalla Completa}: Visualiza la máquina en pantalla completa.
    \item \textbf{Modo Escalado}: Ajusta la resolución de la ventana de la máquina virtual.
    \item \textbf{Integración del ratón y teclado}: Permite que el ratón y el teclado interactúen sin restricciones.
\end{itemize}

\subsection{Atajos de Teclado en VirtualBox}

\begin{table}[h]
    \centering
    \begin{tabular}{|c|l|}
        \hline
        \textbf{Atajo} & \textbf{Función} \\
        \hline
        Host + F & Activar/desactivar pantalla completa \\
        Host + L & Bloquear la sesión de la máquina virtual \\
        Host + C & Habilitar/deshabilitar portapapeles compartido \\
        Host + P & Habilitar/deshabilitar vista previa \\
        Host + G & Ajustar ventana de la VM al tamaño de la pantalla \\
        Host + D & Abrir configuración de la máquina virtual \\
        Host + R & Reiniciar la máquina virtual \\
        Host + H & Apagar la máquina virtual \\
        Ctrl + Alt + Supr & Enviar la combinación a la VM \\
        \hline
    \end{tabular}
    \caption{Atajos de teclado más usados en VirtualBox}
    \label{tab:atajos}
\end{table}

\subsection{Guía Paso a Paso: Creación de una Máquina Virtual para Ubuntu Server 24.04}

\subsubsection{Descargar Ubuntu Server 24.04}
Antes de iniciar, descarga la imagen ISO desde:

\begin{center}
    \url{https://ubuntu.com/download/server}
\end{center}

\subsubsection{Creación de la Máquina Virtual}

\textbf{Paso 1: Crear una nueva máquina virtual}
\begin{enumerate}
    \item Abrir VirtualBox y hacer clic en \textbf{"Nueva"}.
    \item Asignar un nombre (ej. \texttt{UbuntuServer24.04}).
    \item Tipo: \texttt{Linux}, Versión: \texttt{Ubuntu (64-bit)}.
    \item Asignar al menos \textbf{4096 MB} de RAM.
    \item Crear un disco virtual (\texttt{VDI}, \texttt{Reservado dinámicamente}, 10GB mínimo, 20GB recomendado).
\end{enumerate}

\textbf{Paso 2: Configurar la máquina virtual}
\begin{enumerate}
    \item \textbf{Procesadores}: Asignar al menos 2 núcleos.
    \item \textbf{Red}: Elegir \texttt{NAT} o \texttt{Adaptador Puente}.
    \item \textbf{Montar la imagen ISO}: Ir a \texttt{Almacenamiento} y agregar la ISO descargada.
\end{enumerate}

\subsection{Conclusión}
Siguiendo estos pasos, habrás creado una máquina virtual y estarás listo para la instalación y configuración de Ubuntu Server 24.04 en VirtualBox. 

\section{Instalación y Configuración de Ubuntu Server 24.04}

\subsection{Descarga de Ubuntu Server 24.04}
\begin{enumerate}
    \item \textbf{Visita la página oficial de Ubuntu Server}: \href{https://ubuntu.com/download/server}{https://ubuntu.com/download/server}
    \item Descarga la imagen ISO de Ubuntu Server 24.04 (64-bit).
    \item Verifica la integridad del archivo comparando el hash SHA256 (opcional).
\end{enumerate}

\subsection{Creación del medio de instalación (Opcional)}
Esta sección es únicamente como referencia en el caso de que quieras instalar Ubuntu en una partición de tu disco duro.  Para nuestro caso, vamos a trabajar en una máquina virtual así que puedes saltar a la sección \ref{sec:instalacion}

\subsubsection{Opción 1: Crear un USB Booteable (recomendado)}
\begin{enumerate}
    \item Conectar una memoria USB (mínimo 4 GB).
    \item Usar una herramienta para crear el USB de arranque:
    \begin{itemize}
        \item \textbf{Windows}: \href{https://rufus.ie}{Rufus}
        \item \textbf{Linux}: \texttt{dd} o \href{https://www.balena.io/etcher/}{balenaEtcher}
        \item \textbf{macOS}: \texttt{dd} o balenaEtcher
    \end{itemize}
    \item Grabar la imagen ISO en la USB:
    \begin{lstlisting}[language=bash]
    sudo dd if=ubuntu-24.04-live-server-amd64.iso of=/dev/sdX bs=4M status=progress
    \end{lstlisting}
    Reemplaza \texttt{/dev/sdX} con tu unidad USB.
\end{enumerate}

\subsubsection{Opción 2: Grabar en un DVD (menos recomendado)}
\begin{enumerate}
    \item Usa \textbf{Brasero}, \textbf{ImgBurn} o cualquier grabador de discos.
    \item Quema la ISO en un DVD.
\end{enumerate}

\subsubsection{Configuración del BIOS/UEFI}
\begin{enumerate}
    \item Accede a la BIOS/UEFI (normalmente con \texttt{F2}, \texttt{F12}, \texttt{DEL} o \texttt{ESC} al encender).
    \item Habilita el modo UEFI (si es compatible).
    \item Deshabilita Secure Boot si tienes problemas al iniciar la instalación.
    \item Cambia el orden de arranque para que el USB/DVD sea la primera opción.
\end{enumerate}

\subsection{Instalación de Ubuntu Server 24.04}\label{sec:instalacion}
\begin{enumerate}
    \item Inicia desde el USB/DVD y selecciona \textbf{Install Ubuntu Server}.
    \item Selecciona el idioma (Español o Inglés recomendado).
    \item Configura la distribución del teclado.
    \item Configura la red:
    \begin{itemize}
        \item Si tienes DHCP, la red se configura automáticamente.
        \item Para IP estática, selecciona "Configuración manual" e introduce los valores correctos.
    \end{itemize}
    \item Elige el tipo de instalación:
    \begin{itemize}
        \item \textbf{Ubuntu Server (sin interfaz gráfica)} es la opción estándar.
        \item Puedes elegir una instalación mínima si deseas un sistema más ligero.
    \end{itemize}
    \item Configura el almacenamiento:
    \begin{itemize}
        \item Opción recomendada: \textbf{Usar disco entero y configurar LVM} (permite ampliar particiones en el futuro).
        \item Si deseas RAID o particionado manual, elige la opción correspondiente.
    \end{itemize}
    \item Crea un usuario y contraseña.
    \item Opcional: Instala OpenSSH marcando la opción "Install OpenSSH server".
    \item Selecciona software adicional (Docker, Kubernetes, etc.).
    \item Confirma la instalación y espera a que finalice.
\end{enumerate}

\subsection{Primer arranque y configuración post-instalación}
\begin{enumerate}
    \item Reinicia el sistema y retira el medio de instalación.
    \item Inicia sesión con tu usuario y contraseña.
    \item Actualiza el sistema:
    \begin{lstlisting}[language=bash]
    sudo apt update && sudo apt upgrade -y
    \end{lstlisting}
    \item Configura la zona horaria (si no se configuró correctamente):
    \begin{lstlisting}[language=bash]
    sudo timedatectl set-timezone America/Bogota
    \end{lstlisting}
    \item Configura el firewall (opcional pero recomendado):
    \begin{lstlisting}[language=bash]
    sudo ufw allow OpenSSH
    sudo ufw enable
    \end{lstlisting}
    \item Verifica la dirección IP asignada:
    \begin{lstlisting}[language=bash]
    ip a
    \end{lstlisting}
\end{enumerate}

\subsection{Configuración adicional}\label{sec:user}
\begin{itemize}
    \item Crear un usuario adicional con permisos sudo. Ten presente el nombre que utilizarás como \texttt{nuevo\_usuario} y su contraseña porque la necesitaremos más adelante.  Nos vamos a referir a el como nuestro usuario linux:
    \begin{lstlisting}[language=bash]
    sudo adduser nuevo_usuario
    sudo usermod -aG sudo nuevo_usuario
    \end{lstlisting}
    \item Configurar acceso remoto con SSH:
    \begin{lstlisting}[language=bash]
    ssh usuario@ip_del_servidor
    \end{lstlisting}
    \item Instalar paquetes esenciales:
    \begin{lstlisting}[language=bash]
    sudo apt install htop net-tools curl git emacs-nox terminator -y
    \end{lstlisting}
    \item Configurar un hostname personalizado, puedes utilizar el nombre que prefieras como valor para \texttt{mi\_servidor}:
    \begin{lstlisting}[language=bash]
    sudo hostnamectl set-hostname mi_servidor
    \end{lstlisting}
\end{itemize}

\subsection{Conclusión}
Con estos pasos, Ubuntu Server 24.04 estará listo para usarse. A partir de aquí, puedes instalar servicios como Apache, Nginx, PostgreSQL, Docker, Kubernetes, etc., según tus necesidades. \textbf{¡Éxito en tu instalación!}

\section{Compilación de PostgreSQL}

\subsection{Actualizar el Sistema}
Antes de empezar, es recomendable actualizar el sistema para asegurarnos de tener las últimas versiones de los paquetes.

\begin{lstlisting}[language=bash]
sudo apt update && sudo apt upgrade -y
\end{lstlisting}

\subsection{Instalar Dependencias Necesarias}
PostgreSQL requiere varias bibliotecas y herramientas para compilarse correctamente.  Ten en cuenta que usamos el caracter `\textbackslash' como indicador de salto de línea.  Si lo utilizas en la terminal deberas digitar \texttt{Enter} y continuar escribiendo en la siguiente línea de la terminal.  Si prefieres puedes obviar el `\textbackslash' y escribir todo en una sola línea.

\begin{lstlisting}[language=bash]
sudo apt install -y build-essential libreadline-dev zlib1g-dev flex bison libxml2-dev \
    libxslt1-dev libssl-dev libperl-dev libpam0g-dev libldap2-dev python3-dev uuid-dev
\end{lstlisting}

\subsection{Descargar la Última Versión de PostgreSQL}
Obtenemos la versión más reciente desde el sitio oficial de PostgreSQL.

\begin{lstlisting}[language=bash]
cd /usr/local/src
wget https://ftp.postgresql.org/pub/source/v17.2/postgresql-17.2.tar.gz
\end{lstlisting}
(Sustituye `17.2' por la última versión disponible si hay una más reciente).

\subsection{Extraer el Código Fuente}
Descomprimimos el archivo descargado.

\begin{lstlisting}[language=bash]
tar -xvzf postgresql-17.2.tar.gz
cd postgresql-17.2
\end{lstlisting}

\subsection{Configurar la Compilación}
Ejecutamos el script de configuración con las opciones necesarias.

\begin{lstlisting}[language=bash]
./configure --prefix=/usr/local/pgsql --with-ssl=openssl --with-python --with-perl \
    ICU_CFLAGS='-I/usr/include' ICU_LIBS='-L/usr/lib -licui18n -licuuc -licudata'
\end{lstlisting}
Opcionalmente, puedes añadir otras opciones como \lstinline{--with-libxml}, \lstinline{--with-systemd}, etc.  Puedes consultar más opciones en \url{https://www.postgresql.org/docs/current/install-make.html#CONFIGURE-OPTIONS}.

\subsection{Compilar PostgreSQL}
Ahora compilamos PostgreSQL utilizando `make'.

\begin{lstlisting}[language=bash]
make -j$(nproc)
\end{lstlisting}

\subsection{Instalar PostgreSQL}
Si la compilación fue exitosa, instalamos PostgreSQL en el sistema.

\begin{lstlisting}[language=bash]
sudo make install
\end{lstlisting}

\subsection{Crear el Superusuario Postgres y Configurar Directorios}
Creamos un usuario específico para PostgreSQL y configuramos los directorios.  Recuerda memorizar la contraseña del usuario `postgres' porque la utilizaremos más adelante.

\begin{lstlisting}[language=bash]
sudo useradd postgres
sudo mkdir -p /usr/local/pgsql/data
sudo chown postgres:postgres /usr/local/pgsql/data
\end{lstlisting}

\subsection{Inicializar la Base de Datos}
Iniciamos PostgreSQL con el usuario `postgres'.

\begin{lstlisting}[language=bash]
su - postgres
/usr/local/pgsql/bin/initdb -D /usr/local/pgsql/data
/usr/local/pgsql/bin/pg_ctl -D /usr/local/pgsql/data -l logfile start
/usr/local/pgsql/bin/createdb test
/usr/local/pgsql/bin/psql test
\end{lstlisting}

Creamos un rol de superusuario para nuestro usuario linux (el que configuramos en la sección \ref{sec:user}):

\begin{lstlisting}[language=sql]
CREATE ROLE nuevo_usuario WITH LOGIN SUPERUSER PASSWORD 'password_nuevo_usuario';
\end{lstlisting}

Para salir de la sesión de `psql' usamos:

\begin{lstlisting}[language=sql]
\q
\end{lstlisting}

\section{Accesos directos a los Binarios de PostgreSQL}
Si continuamos logeados como el usuario `postgres' salimos pulsando:

\begin{lstlisting}[language=bash]
exit
\end{lstlisting}

Después de confirmar que estamos logeados como nuestro usuario linux, editamos el siguiente archivo:

\begin{lstlisting}[language=bash]
nano ~/.bashrc
\end{lstlisting}

y adicionamos las siguientes líneas al final del archivo:

\begin{lstlisting}[language=bash]
PATH=/usr/local/pgsql/bin:$PATH
export PATH
\end{lstlisting}

Recargamos el fichero con:

\begin{lstlisting}[language=bash]
source ~/.bashrc
\end{lstlisting}

y ya no necesitaremos digitar la ruta completa para ejecutar los comandos de PostgreSQL.



\subsection{Configurar el Servicio de PostgreSQL}
Creamos un archivo de servicio para `\texttt{systemd}'.

\begin{lstlisting}[language=bash]
sudo nano /etc/systemd/system/postgresql.service
\end{lstlisting}

Pegamos el siguiente contenido:

\begin{lstlisting}[language=bash]
[Unit]
Description=PostgreSQL database server
Documentation=man:postgres(1)
After=network-online.target
Wants=network-online.target
[Service]
User=postgres
ExecStart=/usr/local/pgsql/bin/postgres -D /usr/local/pgsql/data
ExecReload=/bin/kill -HUP $MAINPID
KillMode=mixed
KillSignal=SIGINT
TimeoutSec=infinity

[ Install ]
WantedBy=multi-user.target
\end{lstlisting}

Guardamos y salimos (`\texttt{Ctrl + X}', luego `\texttt{Y}' y `\texttt{Enter}').

\subsection{Iniciar y Habilitar PostgreSQL}
Recargamos `\texttt{systemd}' y habilitamos PostgreSQL para que se inicie automáticamente.

\begin{lstlisting}[language=bash]
sudo systemctl daemon-reload
sudo systemctl enable postgresql
sudo systemctl start postgresql
\end{lstlisting}

\subsection{Verificar el Estado del Servidor}
Para confirmar que PostgreSQL está corriendo correctamente, usamos:

\begin{lstlisting}[language=bash]
sudo systemctl status postgresql
\end{lstlisting}

Si PostgreSQL está funcionando bien, deberíamos ver un mensaje indicando que está activo.

\subsection{Acceder a PostgreSQL desde el Usuario Linux}
Desde ahora podremos iniciar sesión en PostgreSQL con nuestro usuario linux (`\texttt{nuevo\_usuario}').  Para listar las bases de datos presentes en el sistema usamos:

\begin{lstlisting}[language=bash]
psql -l
\end{lstlisting}

\subsection{¡PostgreSQL está listo para usarse!}
Ahora puedes configurar usuarios, bases de datos y tunear PostgreSQL según tus necesidades.

\end{document}

