\documentclass{article}
\usepackage{amsmath}
\usepackage{amssymb}
\usepackage{graphicx}
\usepackage{hyperref}
\usepackage{minted}
\usepackage{xcolor}
\usepackage{booktabs}
\usepackage{fancyhdr}
\usepackage{geometry}
\hypersetup{
  colorlinks=true,
  linkcolor=blue,
  filecolor=magenta,
  urlcolor=blue,
  citecolor=blue
}

\geometry{margin=2.5cm}
\setlength{\parindent}{0pt}
\setlength{\parskip}{1em}

\title{Study Guide: Triggers in SQL}
\author{Prepared by Andrés Calderón}
\date{\today}

\begin{document}
\maketitle

\section{Introduction to Triggers}
A \textbf{trigger} is a database object that automatically executes a predefined action in response to certain events on a table or view. It is a type of stored procedure that is triggered automatically when a specified database operation occurs.

\subsection*{Key Features}
\begin{itemize}
    \item Automated execution
    \item Event-driven
    \item Typically used for enforcing business rules, maintaining audit trails, and validating data
\end{itemize}

\section{Why Use Triggers?}
Triggers are particularly useful for:
\begin{itemize}
    \item \textbf{Data Integrity:} Enforcing complex constraints that cannot be captured by primary keys, unique constraints, or check constraints.
    \item \textbf{Audit Trails:} Automatically recording changes to critical data.
    \item \textbf{Business Rules:} Implementing complex logic at the database level.
    \item \textbf{Synchronization:} Keeping related tables consistent.
    \item \textbf{Security:} Validating data before changes are made.
\end{itemize}

\section{Types of Triggers}
Triggers can be classified based on the timing and event that triggers their execution:

\subsection*{Timing}
\begin{itemize}
    \item \textbf{BEFORE Trigger:} Executes before the triggering event.
    \item \textbf{AFTER Trigger:} Executes after the triggering event.
    \item \textbf{INSTEAD OF Trigger:} Replaces the triggering event (typically for views).
\end{itemize}

\subsection*{Event}
\begin{itemize}
    \item \textbf{INSERT Trigger:} Fires when a new row is inserted.
    \item \textbf{UPDATE Trigger:} Fires when an existing row is updated.
    \item \textbf{DELETE Trigger:} Fires when a row is deleted.
\end{itemize}

\section*{Nested Triggers}

A \textbf{nested trigger} is a trigger that is fired as a result of another trigger. This occurs when the execution of one trigger causes a data modification (INSERT, UPDATE, or DELETE) that activates a second trigger on the same or a different table.

\subsubsection*{Key Characteristics}
\begin{itemize}
    \item \textbf{Chained Execution:} Triggers can form a chain where one trigger's action initiates another.
    \item \textbf{Depth Control:} Some database systems, like SQL Server, allow you to limit the nesting level to prevent infinite loops.
    \item \textbf{Performance Impact:} Excessive nesting can lead to performance issues and complex debugging.
    \item \textbf{Transactional Context:} Nested triggers operate within the same transaction, meaning a failure in a nested trigger can roll back the entire chain.
\end{itemize}

\textbf{Example: Nested Trigger in PostgreSQL}\\
This example demonstrates how a trigger on one table can cause a second trigger on a related table to activate.

\begin{minted}[fontsize=\small, frame=lines, linenos, bgcolor=gray!10]{sql}
-- Table for tracking customer status changes
CREATE TABLE customer_status_changes (
    change_id SERIAL PRIMARY KEY,
    customer_id INT,
    old_status VARCHAR(50),
    new_status VARCHAR(50),
    change_time TIMESTAMP DEFAULT CURRENT_TIMESTAMP
);

-- Table for logging audit history
CREATE TABLE audit_log (
    log_id SERIAL PRIMARY KEY,
    message TEXT,
    logged_at TIMESTAMP DEFAULT CURRENT_TIMESTAMP
);

-- Trigger function to log status changes
CREATE OR REPLACE FUNCTION log_status_changes()
RETURNS TRIGGER AS $$
BEGIN
    INSERT INTO customer_status_changes (customer_id, old_status, new_status)
    VALUES (OLD.customer_id, OLD.status, NEW.status);
    RETURN NEW;
END;
$$ LANGUAGE plpgsql;

-- Trigger function for nested logging
CREATE OR REPLACE FUNCTION log_audit_message()
RETURNS TRIGGER AS $$
BEGIN
    INSERT INTO audit_log (message)
    VALUES ('Customer status updated for customer ID: ' || NEW.customer_id);
    RETURN NEW;
END;
$$ LANGUAGE plpgsql;

-- Primary trigger on customer status updates
CREATE TRIGGER customer_status_trigger
AFTER UPDATE ON customers
FOR EACH ROW
EXECUTE FUNCTION log_status_changes();

-- Nested trigger on customer status change table
CREATE TRIGGER nested_audit_trigger
AFTER INSERT ON customer_status_changes
FOR EACH ROW
EXECUTE FUNCTION log_audit_message();
\end{minted}

\section{Basic Syntax of Triggers (PostgreSQL)}
\begin{minted}[fontsize=\small, frame=lines, linenos, bgcolor=gray!10]{sql}
CREATE TRIGGER trigger_name
{ BEFORE | AFTER | INSTEAD OF } { INSERT | UPDATE | DELETE }
ON table_name
FOR EACH { ROW | STATEMENT }
WHEN (condition)
EXECUTE FUNCTION function_name();
\end{minted}

\section{Practical Examples}
\subsection*{Audit Log Trigger}
Tracking changes in an \texttt{employees} table:

\begin{minted}[fontsize=\small, frame=lines, linenos, bgcolor=gray!10]{sql}
CREATE TABLE employees_audit (
    audit_id SERIAL PRIMARY KEY,
    employee_id INT,
    operation VARCHAR(10),
    change_time TIMESTAMP DEFAULT CURRENT_TIMESTAMP
);

CREATE OR REPLACE FUNCTION log_employee_changes()
RETURNS TRIGGER AS $$
BEGIN
    INSERT INTO employees_audit (employee_id, operation)
    VALUES (NEW.employee_id, TG_OP);
    RETURN NEW;
END;
$$ LANGUAGE plpgsql;

CREATE TRIGGER employee_audit_trigger
AFTER INSERT OR UPDATE OR DELETE
ON employees
FOR EACH ROW
EXECUTE FUNCTION log_employee_changes();
\end{minted}

\subsection*{Data Validation Trigger}
Preventing negative salaries in the \texttt{employees} table:

\begin{minted}[fontsize=\small, frame=lines, linenos, bgcolor=gray!10]{sql}
CREATE OR REPLACE FUNCTION validate_salary()
RETURNS TRIGGER AS $$
BEGIN
    IF NEW.salary < 0 THEN
        RAISE EXCEPTION 'Salary cannot be negative';
    END IF;
    RETURN NEW;
END;
$$ LANGUAGE plpgsql;

CREATE TRIGGER salary_check
BEFORE INSERT OR UPDATE
ON employees
FOR EACH ROW
EXECUTE FUNCTION validate_salary();
\end{minted}

\section{Transition Tables in Triggers}

A \textbf{transition table} in PostgreSQL is a special table used in triggers to capture the full set of affected rows during bulk operations. Unlike the \texttt{OLD} and \texttt{NEW} row-level references, transition tables provide a set-based view of all modified rows, making them ideal for performance tuning and bulk data processing. They are available for \texttt{AFTER} triggers on \texttt{UPDATE} and \texttt{DELETE} operations.

\begin{itemize}
    \item Useful for auditing large changes without per-row overhead.
    \item Enables set-based processing within triggers for better efficiency.
    \item Can be referenced in the trigger function using the \texttt{REFERENCING} clause.
\end{itemize}

\subsection*{Example: Using Transition Tables for Bulk Auditing}

This example demonstrates how to use transition tables to log bulk updates to an \texttt{employees} table:

\begin{minted}[fontsize=\small, frame=lines, linenos, bgcolor=gray!10]{sql}
CREATE TABLE employees_bulk_audit (
    audit_id SERIAL PRIMARY KEY,
    employee_id INT,
    old_salary NUMERIC,
    new_salary NUMERIC,
    change_time TIMESTAMP DEFAULT CURRENT_TIMESTAMP
);

CREATE OR REPLACE FUNCTION log_bulk_salary_changes()
RETURNS TRIGGER AS $$
BEGIN
    INSERT INTO employees_bulk_audit (employee_id, old_salary, new_salary)
    SELECT OLD.employee_id, OLD.salary, NEW.salary
    FROM OLD_TABLE AS OLD, NEW_TABLE AS NEW
    WHERE OLD.employee_id = NEW.employee_id;
    RETURN NULL;
END;
$$ LANGUAGE plpgsql;

CREATE TRIGGER bulk_salary_audit
AFTER UPDATE ON employees
REFERENCING OLD TABLE AS OLD_TABLE NEW TABLE AS NEW_TABLE
FOR EACH STATEMENT
EXECUTE FUNCTION log_bulk_salary_changes();
\end{minted}

\section{Best Practices for Using Triggers}
\begin{itemize}
    \item Use triggers sparingly to avoid performance issues.
    \item Document the purpose of each trigger clearly.
    \item Avoid complex logic that can make debugging difficult.
    \item Use triggers for auditing and validation, but not as a primary business logic layer.
    \item Test triggers extensively before deploying to production.
    \item Use transition tables for bulk operations to reduce overhead.
    \item Monitor trigger performance regularly to avoid bottlenecks.
\end{itemize}

\section{Additional Content}

\begin{itemize}
    \item\textit{``SQL Triggers: A Beginner's Guide''}. Oluseye Jeremiah. 2024. Link in  \href{https://www.datacamp.com/tutorial/sql-triggers}{DataCamp}. \cite{datacamp_sql_triggers}.
    \item \textit{``Triggers In SQL| Triggers In Database, SQL Triggers Tutorial For Beginners''}. Edureka!, 2023. Link in \href{https://www.youtube.com/watch?v=MhsL8ke7IJY}{YouTube}. \cite{edureka_sql_triggers}.
    \item \textit{``¿Qué diablos es un Trigger? Ejemplo sencillo en Sql Server''}. Héctor de León. 2018. In Spanish. Link in  \href{https://www.youtube.com/watch?v=uTx7xd4ojkk}{YouTube}. \cite{hdeleon_trigger_sqlserver}.
\end{itemize}

\section{Exercises}
\begin{enumerate}
    \item Create a trigger to prevent the deletion of VIP customers.
    \item Implement a trigger that logs every update to a \texttt{products} table, including the old and new price.
    \item Write a trigger that automatically updates a stock table whenever a sale is made.
    \item Create a trigger that validates email format before inserting a customer record.
    \item Optimize a trigger for bulk data updates using transition tables.
    \item Design a trigger that automatically archives deleted rows into a history table.
    \item Implement a performance monitoring trigger to track long-running updates.
\end{enumerate}

\newpage

\section*{Solutions}

\subsection*{Exercise 1: Preventing the Deletion of VIP Customers}
To prevent the deletion of VIP customers, you can create a trigger that raises an exception if an attempt is made to delete a customer marked as VIP.

\begin{minted}[fontsize=\small, frame=lines, linenos, bgcolor=gray!10]{sql}
CREATE OR REPLACE FUNCTION prevent_vip_deletion()
RETURNS TRIGGER AS $$
BEGIN
    IF OLD.is_vip THEN
        RAISE EXCEPTION 'Cannot delete VIP customers.';
    END IF;
    RETURN OLD;
END;
$$ LANGUAGE plpgsql;

CREATE TRIGGER vip_deletion_blocker
BEFORE DELETE ON customers
FOR EACH ROW
EXECUTE FUNCTION prevent_vip_deletion();
\end{minted}

\subsection*{Exercise 2: Logging Product Price Updates}
To log every price update in a \texttt{products} table, you can use a trigger that captures both the old and new prices.

\begin{minted}[fontsize=\small, frame=lines, linenos, bgcolor=gray!10]{sql}
CREATE TABLE product_price_log (
    log_id SERIAL PRIMARY KEY,
    product_id INT,
    old_price NUMERIC,
    new_price NUMERIC,
    change_time TIMESTAMP DEFAULT CURRENT_TIMESTAMP
);

CREATE OR REPLACE FUNCTION log_price_changes()
RETURNS TRIGGER AS $$
BEGIN
    INSERT INTO product_price_log (product_id, old_price, new_price)
    VALUES (OLD.product_id, OLD.price, NEW.price);
    RETURN NEW;
END;
$$ LANGUAGE plpgsql;

CREATE TRIGGER price_update_logger
AFTER UPDATE ON products
FOR EACH ROW
EXECUTE FUNCTION log_price_changes();
\end{minted}

\subsection*{Exercise 3: Stock Update on Sale}
To automatically update the stock of a product when a sale is made, you can use a trigger to adjust the inventory.

\begin{minted}[fontsize=\small, frame=lines, linenos, bgcolor=gray!10]{sql}
CREATE OR REPLACE FUNCTION update_stock_on_sale()
RETURNS TRIGGER AS $$
BEGIN
    UPDATE products
    SET stock = stock - NEW.quantity
    WHERE product_id = NEW.product_id;
    RETURN NEW;
END;
$$ LANGUAGE plpgsql;
CREATE TRIGGER stock_update_trigger
AFTER INSERT ON sales
FOR EACH ROW

EXECUTE FUNCTION update_stock_on_sale();
\end{minted}

\subsection*{Exercise 4: Email Format Validation}
To validate email formats before inserting customer records, you can create a trigger to enforce proper formatting.

\begin{minted}[fontsize=\small, frame=lines, linenos, bgcolor=gray!10]{sql}
CREATE OR REPLACE FUNCTION validate_email_format()
RETURNS TRIGGER AS $$
BEGIN
    IF NOT NEW.email ~* '^[A-Za-z0-9._%+-]+@[A-Za-z0-9.-]+\.[A-Za-z]{2,}$' THEN
        RAISE EXCEPTION 'Invalid email format: %', NEW.email;
    END IF;
    RETURN NEW;
END;
$$ LANGUAGE plpgsql;

CREATE TRIGGER email_validation_trigger
BEFORE INSERT OR UPDATE ON customers
FOR EACH ROW
EXECUTE FUNCTION validate_email_format();
\end{minted}

\subsection*{Exercise 5: Bulk Data Updates Using Transition Tables}
To efficiently handle bulk data updates, you can use transition tables to capture the old and new state of rows during a single operation.

\begin{minted}[fontsize=\small, frame=lines, linenos, bgcolor=gray!10]{sql}
CREATE TABLE bulk_update_audit (
    audit_id SERIAL PRIMARY KEY,
    product_id INT,
    old_price NUMERIC,
    new_price NUMERIC,
    change_time TIMESTAMP DEFAULT CURRENT_TIMESTAMP
);

CREATE OR REPLACE FUNCTION log_bulk_price_changes()
RETURNS TRIGGER AS $$
BEGIN
    INSERT INTO bulk_update_audit (product_id, old_price, new_price)
    SELECT OLD.product_id, OLD.price, NEW.price
    FROM OLD_TABLE AS OLD
    JOIN NEW_TABLE AS NEW
    ON OLD.product_id = NEW.product_id;
    RETURN NULL;
END;
$$ LANGUAGE plpgsql;

CREATE TRIGGER bulk_price_update_trigger
AFTER UPDATE ON products
REFERENCING OLD TABLE AS OLD_TABLE NEW TABLE AS NEW_TABLE
FOR EACH STATEMENT
EXECUTE FUNCTION log_bulk_price_changes();
\end{minted}

\subsection*{Exercise 6: Archiving Deleted Rows}
To automatically archive deleted rows, you can create a trigger that stores deleted data in a history table.

\begin{minted}[fontsize=\small, frame=lines, linenos, bgcolor=gray!10]{sql}
CREATE TABLE customer_archive (
    archive_id SERIAL PRIMARY KEY,
    customer_id INT,
    name VARCHAR(255),
    email VARCHAR(255),
    deleted_at TIMESTAMP DEFAULT CURRENT_TIMESTAMP
);

CREATE OR REPLACE FUNCTION archive_deleted_customers()
RETURNS TRIGGER AS $$
BEGIN
    INSERT INTO customer_archive (customer_id, name, email)
    VALUES (OLD.customer_id, OLD.name, OLD.email);
    RETURN OLD;
END;
$$ LANGUAGE plpgsql;

CREATE TRIGGER archive_on_delete
BEFORE DELETE ON customers
FOR EACH ROW
EXECUTE FUNCTION archive_deleted_customers();
\end{minted}

\subsection*{Exercise 7: Performance Monitoring}
To track long-running updates, you can create a trigger to log slow transactions for performance analysis.

\begin{minted}[fontsize=\small, frame=lines, linenos, bgcolor=gray!10]{sql}
CREATE TABLE slow_query_log (
    log_id SERIAL PRIMARY KEY,
    table_name VARCHAR(255),
    operation VARCHAR(10),
    duration_ms INT,
    logged_at TIMESTAMP DEFAULT CURRENT_TIMESTAMP
);

CREATE OR REPLACE FUNCTION log_slow_updates()
RETURNS TRIGGER AS $$
DECLARE
    start_time TIMESTAMP;
    end_time TIMESTAMP;
    duration INT;
BEGIN
    start_time := clock_timestamp();
    -- Simulate some processing delay for demonstration
    PERFORM pg_sleep(0.5);
    end_time := clock_timestamp();
    duration := EXTRACT(MILLISECOND FROM end_time - start_time);

    IF duration > 500 THEN
        INSERT INTO slow_query_log (table_name, operation, duration_ms)
        VALUES (TG_TABLE_NAME, TG_OP, duration);
    END IF;

    RETURN NEW;
END;
$$ LANGUAGE plpgsql;

CREATE TRIGGER slow_update_logger
AFTER UPDATE OR DELETE ON customers
FOR EACH ROW
EXECUTE FUNCTION log_slow_updates();
\end{minted}

\bibliographystyle{plain}  % or choose a different style, like ieeetr, abbrv, unsrt
\bibliography{references}  % name of your .bib file (without the .bib extension)

\end{document}
