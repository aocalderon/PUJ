\documentclass{beamer}
\usepackage{graphicx}
\usepackage{booktabs}
\usepackage{tcolorbox}
\usepackage{amsmath, esint}
\usepackage{tikz}
\usetikzlibrary{backgrounds, positioning, arrows, scopes, shapes, shapes.misc, shapes.multipart}
\tikzset{
    cross/.style={cross out, draw=black, minimum size=2*(#1-\pgflinewidth), inner sep=0pt, outer sep=0pt},
    cross/.default={10pt},
    split/.style={rectangle split, rectangle split parts=7, draw, inner sep=0ex, rectangle split horizontal, minimum size=4ex},
    textstyle/.style={text height=1.5ex, text depth=.25ex}
}

\usepackage{amsthm}
\usepackage{amstext}
\usepackage{amssymb}
\usepackage[plain]{algorithm}
\usepackage{algpseudocode}

\usepackage{xcolor} 
\definecolor{LightGray}{gray}{0.975}

%\usetheme{Warsaw}
\usefonttheme{serif} 

\title[L4 Quicksort]{Introduction to Algorithms \\ Lecture 4: Quicksort}
\author{Prof. Charles E. Leiserson and Prof. Erik Demaine \\ Massachusetts Institute of Technology}
\date{\today}

\setbeamertemplate{navigation symbols}{}%remove navigation symbols

\defbeamertemplate*{footline}{shadow theme}{%
    \leavevmode%
    \hbox{
        \begin{beamercolorbox}[wd=.5\paperwidth,ht=2.5ex,dp=1.125ex,leftskip=.3cm plus1fil,rightskip=.3cm]{author in head/foot}%
            \usebeamerfont{author in head/foot} Introduction to Algorithms: \hfill \insertshorttitle
        \end{beamercolorbox}%
        \begin{beamercolorbox}[wd=.5\paperwidth,ht=2.5ex,dp=1.125ex,leftskip=.3cm,rightskip=.3cm plus1fil]{title in head/foot}%
            \usebeamerfont{title in head/foot} \hfill \insertframenumber\,/\,\inserttotalframenumber%
        \end{beamercolorbox}
    }%
    \vskip0pt%
}

\AtBeginSection[]{
    \begin{frame}<beamer>
    \frametitle{Plan}
    \tableofcontents[currentsection]
    \end{frame}
}


\begin{document}

\frame{\titlepage}

\begin{frame}{Introduction to Algorithms}
    \centering
    \includegraphics[width=0.45\textwidth]{figures/book_cover.jpg} \\
    \vspace{5mm}{
        \tiny
        Content has been extracted from \textit{Introduction to Algorithms}, Fourth Edition, by Cormen, Leiserson, Rivest, and Stein. MIT Press. 2022.\\
        Visit \url{https://mitpress.mit.edu/9780262046305/introduction-to-algorithms/}.\\
        Original slides from \textit{Introduction to Algorithms 6.046J/18.401J}, Fall 2005 Class by Prof. Charles Leiserson and Prof. Erik Demaine. MIT OpenCourseWare Initiative available at \url{https://ocw.mit.edu/courses/6-046j-introduction-to-algorithms-sma-5503-fall-2005/}.\\
    }
\end{frame}

\begin{frame}{Quicksort}
    \begin{itemize}
            \item Proposed by C.A.R. Hoare in 1962.
            \item Divide-and-conquer algorithm.
            \item Sorts `in place' (like insertion sort, but not like merge sort).
            \item Very practical (with tuning).
    \end{itemize}
\end{frame}

\begin{frame}{Divide and Conquer}
    Quicksort an $n$-element array:
    \begin{enumerate}
        \item \textbf{\Large Divide:} Partition the array into two subarrays around a \textbf{pivot} $x$ such that elements in lower subarray $\leq x \leq$ elements in upper subarray.
            \includegraphics[width=\textwidth, trim={2cm 7.25cm 2cm 9.75cm}, clip]{pages/lec4_3} 
        \item \textbf{\Large Conquer:} Recursively sort the two subarrays.
        \item \textbf{\Large Combine:} Trivial.
    \end{enumerate}
    \begin{alertblock}{Key:}
        Linear-time partitioning subroutine.
    \end{alertblock}
\end{frame}

\begin{frame}{Partitioning Subroutine}
    \begin{minipage}{0.58\textwidth}
        \begin{algorithm}[H]
            \begin{algorithmic}[1]
                %\State \hrulefill
                \Procedure{\textsc{Partition}}{$A$, $p$, $q$}
                    \State $x \leftarrow A[p]$
                    \State $i \leftarrow p$
                    \For{$j \leftarrow p + 1$ \textbf{to} $q$}
                        \If{$A[j] \leq x$}
                            \State $i \leftarrow i + 1$
                            \State exchange $A[i] \leftrightarrow A[j]$
                        \EndIf
                    \EndFor
                    \State exchange $A[p] \leftrightarrow A[i]$
                    \State \Return $i$
                \EndProcedure
            \end{algorithmic}
        \end{algorithm}
    \end{minipage}
    \hfill
    \begin{minipage}{0.38\textwidth}
        \begin{alertblock}{Running time:}
            $O(n)$ for $n$ elements.
        \end{alertblock}
        \vspace{4.5cm}
    \end{minipage}
    \includegraphics[width=\textwidth, trim={1.75cm 1.50cm 2.50cm 14.30cm}, clip]{pages/lec4_4} 
\end{frame}

\begin{frame}{Example of partitioning}
    \centering
    \includegraphics[width=\textwidth, trim={2.75cm 1.80cm 2.75cm 5.00cm}, clip]{pages/lec4_5}
\end{frame}
\begin{frame}{Example of partitioning}
    \centering
    \includegraphics[width=\textwidth, trim={2.75cm 1.80cm 2.75cm 5.00cm}, clip]{pages/lec4_6}
\end{frame}
\begin{frame}{Example of partitioning}
    \centering
    \includegraphics[width=\textwidth, trim={2.75cm 1.80cm 2.75cm 5.00cm}, clip]{pages/lec4_7}
\end{frame}
\begin{frame}{Example of partitioning}
    \centering
    \includegraphics[width=\textwidth, trim={2.75cm 1.80cm 2.75cm 5.00cm}, clip]{pages/lec4_8}
\end{frame}
\begin{frame}{Example of partitioning}
    \centering
    \includegraphics[width=\textwidth, trim={2.75cm 1.80cm 2.75cm 5.00cm}, clip]{pages/lec4_9}
\end{frame}
\begin{frame}{Example of partitioning}
    \centering
    \includegraphics[width=\textwidth, trim={2.75cm 1.80cm 2.75cm 5.00cm}, clip]{pages/lec4_10}
\end{frame}
\begin{frame}{Example of partitioning}
    \centering
    \includegraphics[width=\textwidth, trim={2.75cm 1.80cm 2.75cm 5.00cm}, clip]{pages/lec4_11}
\end{frame}
\begin{frame}{Example of partitioning}
    \centering
    \includegraphics[width=\textwidth, trim={2.75cm 1.80cm 2.75cm 5.00cm}, clip]{pages/lec4_12}
\end{frame}
\begin{frame}{Example of partitioning}
    \centering
    \includegraphics[width=\textwidth, trim={2.75cm 1.80cm 2.75cm 5.00cm}, clip]{pages/lec4_13}
\end{frame}
\begin{frame}{Example of partitioning}
    \centering
    \includegraphics[width=\textwidth, trim={2.75cm 1.80cm 2.75cm 5.00cm}, clip]{pages/lec4_14}
\end{frame}
\begin{frame}{Example of partitioning}
    \centering
    \includegraphics[width=\textwidth, trim={2.75cm 1.80cm 2.75cm 5.00cm}, clip]{pages/lec4_15}
\end{frame}
\begin{frame}{Example of partitioning}
    \centering
    \includegraphics[width=\textwidth, trim={2.75cm 1.80cm 2.75cm 5.00cm}, clip]{pages/lec4_16}
\end{frame}

\begin{frame}{Pseudocode for Quicksort}
        \begin{algorithm}[H]
            \begin{algorithmic}[1]
                \Procedure{\textsc{Quicksort}}{$A$, $p$, $r$}
                    \If{$p < r$}
                        \State $q \leftarrow$ \textsc{Partition}($A$,$p$, $r$)
                        \State \textsc{Quicksort}($A$, $p$, $q - 1$)
                        \State \textsc{Quicksort}($A$, $q + 1$, $r$)
                    \EndIf
                \EndProcedure
            \end{algorithmic}
        \end{algorithm}
        \begin{alertblock}{Initial call:}
            \textsc{Quicksort}($A$, $1$, $n$)
        \end{alertblock}
\end{frame}

\begin{frame}{Analysis of Quicksort}
    \begin{itemize}
        \item Assume all input elements are distinct.
        \item In practice, there are better partitioning algorithms for when duplicate input elements may exist.
        \item Let $T(n) =$ worst-case running time on an array of $n$ elements.
    \end{itemize}
\end{frame}

\begin{frame}{Worst-case of Quicksort}
    \begin{itemize}
        \item Input sorted or reverse sorted.
        \item Partition around min or max element.
        \item One side of partition always has no elements.
    \end{itemize}
    \begin{equation*}
        \begin{split}
            T(n) =& T(0) + T(n - 1) + \Theta(n) \\
                    =& \Theta(1) + T(n - 1) + \Theta(n) \\
                    =& T(n - 1) + \Theta(n) \\
                    =& \Theta(n^2)
        \end{split}
    \end{equation*}
    \pause
    \centering
    \textcolor{red}{\textbf{Arithmetic Series!}}
\end{frame}

\begin{frame}{Worst-case Recursion Tree}
    \begin{equation*}
        \begin{split}
            T(n) =& T(0) + T(n - 1) + cn \\
        \end{split}
    \end{equation*}
    \centering
    \includegraphics[width=\textwidth, trim={1.00cm 1.80cm 0.50cm 5.25cm}, clip]{pages/lec4_20}
\end{frame}

\begin{frame}{Worst-case Recursion Tree}
    \begin{equation*}
        \begin{split}
            T(n) =& T(0) + T(n - 1) + cn \\
        \end{split}
    \end{equation*}
    \centering
    \includegraphics[width=\textwidth, trim={1.00cm 1.80cm 0.50cm 5.25cm}, clip]{pages/lec4_21}
\end{frame}
\begin{frame}{Worst-case Recursion Tree}
    \begin{equation*}
        \begin{split}
            T(n) =& T(0) + T(n - 1) + cn \\
        \end{split}
    \end{equation*}
    \centering
    \includegraphics[width=\textwidth, trim={1.00cm 1.80cm 0.50cm 5.25cm}, clip]{pages/lec4_22}
\end{frame}
\begin{frame}{Worst-case Recursion Tree}
    \begin{equation*}
        \begin{split}
            T(n) =& T(0) + T(n - 1) + cn \\
        \end{split}
    \end{equation*}
    \centering
    \includegraphics[width=\textwidth, trim={1.00cm 1.80cm 0.50cm 5.25cm}, clip]{pages/lec4_23}
\end{frame}
\begin{frame}{Worst-case Recursion Tree}
    \begin{equation*}
        \begin{split}
            T(n) =& T(0) + T(n - 1) + cn \\
        \end{split}
    \end{equation*}
    \centering
    \includegraphics[width=\textwidth, trim={1.00cm 1.80cm 0.50cm 5.25cm}, clip]{pages/lec4_24}
\end{frame}
\begin{frame}{Worst-case Recursion Tree}
    \begin{equation*}
        \begin{split}
            T(n) =& T(0) + T(n - 1) + cn \\
        \end{split}
    \end{equation*}
    \centering
    \includegraphics[width=\textwidth, trim={1.00cm 1.80cm 0.50cm 5.25cm}, clip]{pages/lec4_25}
\end{frame}
\begin{frame}{Worst-case Recursion Tree}
    \begin{equation*}
        \begin{split}
            T(n) =& T(0) + T(n - 1) + cn \\
        \end{split}
    \end{equation*}
    \centering
    \includegraphics[width=\textwidth, trim={1.00cm 1.80cm 0.50cm 5.25cm}, clip]{pages/lec4_26}
\end{frame}

\begin{frame}{Best-case Analysis}{For intuition only!}
    If we're lucky, \textsc{Partition} splits the array evenly:
    \begin{equation*}
        \begin{split}
            T(n) =& 2T\left(\frac{n}{2}\right) + \Theta(n) \\
                 =& \Theta(n \ln n) \text{\hspace{20mm} (same as merge-sort)} \\
        \end{split}
    \end{equation*}
    What if the split is always $\frac{1}{10}:\frac{9}{10}$?
    \begin{equation*}
        \begin{split}
            T(n) =& T\left(\frac{1}{10} n \right) + T\left(\frac{9}{10} n \right) + \Theta(n) \\
        \end{split}
    \end{equation*}
    What is the solution to this recurrence?
\end{frame}

\begin{frame}{Analysis of ``Almost-best'' Case}
    \centering
    \includegraphics[width=\textwidth, trim={1.00cm 1.50cm 1.00cm 4.25cm}, clip]{pages/lec4_28}
\end{frame}
\begin{frame}{Analysis of ``Almost-best'' Case}
    \centering
    \includegraphics[width=\textwidth, trim={1.00cm 1.50cm 1.00cm 4.25cm}, clip]{pages/lec4_29}
\end{frame}
\begin{frame}{Analysis of ``Almost-best'' Case}
    \centering
    \includegraphics[width=\textwidth, trim={1.00cm 1.50cm 1.00cm 4.25cm}, clip]{pages/lec4_30}
\end{frame}
\begin{frame}{Analysis of ``Almost-best'' Case}
    \centering
    \includegraphics[width=\textwidth, trim={1.00cm 1.50cm 1.00cm 4.25cm}, clip]{pages/lec4_31}
\end{frame}
\begin{frame}{Analysis of ``Almost-best'' Case}
    \centering
    \includegraphics[width=\textwidth, trim={1.00cm 1.50cm 1.00cm 4.25cm}, clip]{pages/lec4_32}
\end{frame}

\begin{frame}{More Intuition}
    Suppose we alternate lucky, unlucky, lucky, unlucky, lucky, ...
    \begin{equation*}
        \begin{split}
            L(n) =& 2U\left(\frac{n}{2}\right) + \Theta(n) \text{\hspace{17mm} luckcy} \\
            U(n) =& L\left(n - 1\right) + \Theta(n) \text{\hspace{15mm} unluckcy} \\
        \end{split}
    \end{equation*}
    Solving:
    \begin{equation*}
        \begin{split}
            L(n) =& 2\left(L\left(\frac{n}{2} - 1 \right) + \Theta(\frac{n}{2})\right) + \Theta(n) \\
                 =& 2L\left(\frac{n}{2} - 1 \right) + \Theta(n) \\
                 =& \Theta(n \ln n) \text{\hspace{15mm} \textbf{lucky!}}
        \end{split}
    \end{equation*}
    How can we make sure we are usually lucky?
\end{frame}








\begin{frame}{}
\end{frame}

\begin{frame}{}
    \centering
    \Huge End of Lecture 4.
\end{frame}

\section*{Takeaways}

% Tim Duncan's Top 5 Fundamental Takeaways of the Today's Class
\begin{frame}{TDT5FTOTTC}
    \centering
    \includegraphics[width=0.75\textwidth]{figures/tim.png}
\end{frame}

\begin{frame}{Top 5 Fundamental Takeaways}
    \small
    \begin{enumerate} \pause
        \item[5] \pause

    \end{enumerate}
\end{frame}

\begin{frame}{Introduction to Algorithms}
    \centering
    \includegraphics[width=0.45\textwidth]{figures/book_cover.jpg} \\
    \vspace{5mm}{
        \tiny
        Content has been extracted from \textit{Introduction to Algorithms}, Fourth Edition, by Cormen, Leiserson, Rivest, and Stein. MIT Press. 2022.\\
        Visit \url{https://mitpress.mit.edu/9780262046305/introduction-to-algorithms/}.\\
        Original slides from \textit{Introduction to Algorithms 6.046J/18.401J}, Fall 2005 Class by Prof. Charles Leiserson and Prof. Erik Demaine. MIT OpenCourseWare Initiative available at \url{https://ocw.mit.edu/courses/6-046j-introduction-to-algorithms-sma-5503-fall-2005/}.\\
    }
\end{frame}

\end{document}
